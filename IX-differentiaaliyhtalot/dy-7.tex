\section[Differentiaaliyhtälöiden numeeriset ratkaisumenetelmät]{Differentiaaliyhtälöiden
numeeriset \\ ratkaisumenetelmät} 
\label{DYn numeeriset menetelmät}
\sectionmark{DY:n numeeriset ratkaisumenetelmät}
\alku
\index{differentiaaliyhtälön!i@numeerinen ratkaiseminen|vahv}

Edellisissä luvuissa on kehitelty differentiaaliyhtälöiden ratkaisemistaitoa lähinnä nk.\
'tarkan' ratkaisemisen näkökulmasta. Tämä ratkaisuoppi on hyvin perinteistä, ja sillä on oma
pysyvä arvonsa. --- Etenkin silloin kun ratkaisun voi esittää alkeisfunktioden avulla 
suhteellisen yksinkertaisena lausekkeena, on perinteistä ratkaisutapaa vaikea sivuuttaa. 
Sovelluksissa törmätään kuitenkin hyvin usein differentiaaliyhtälöihin (ja etenkin 
-systeemeihin), joihin mikään tähän asti tarjotuista resepteistä ei tehoa. Silloin on tyydyttävä
etsimään ratkaisu likimäärin \kor{numeerisesti}, käytännössä tietokoneen tai tehokkaan 
laskimen avustuksella.\footnote[2]{Differentiaaliyhtälöitä ratkottiin numeerisesti jo ennen 
tietokoneiden aikaa, mutta ratkaiseminen rajoittui vain hyvin tärkeinä pidettyihin kohteisiin,
kuten tykinammusten lentoratojen määrittämiseen. Tietokoneiden aikakaudella 
differentiaaliyhtälöiden numeerisesta ratkaisemisesta on tullut arkipäivää mitä erilaisimmilla
tieteen ja tekniikan aloilla. Samalla on ratkaisumenetelmistä kehittynyt laaja menetelmätiede,
josta on kirjoitettu paksuja kirjoja.} Numeerinen ratkaiseminen siis poikkeaa melkoisesti
perinteisistä menetelmistä, joilla ratkaisu yleensä saadaan (jos saadaan) 'käsipelillä'. 
Mitään periaatteellista eroa ei numeerisen ja 'tarkan' ratkaisemisen välillä silti ole, sillä
myös numeerisin keinoin saadaan tarkka ratkaisu, kun ajatellaan algoritmin tuottama 
likimääräisratkaisujen jono 'loppuun asti' lasketuksi. --- Itse asiassa juuri näin meneteltiin
Luvussa \ref{määrätty integraali}, kun etsittiin yleispätevää ratkaisutapaa 
differentiaaliyhtälölle $y'=f(x)$.

Jatkossa tarkastellaan lyhyesti \pain{alkuarvotehtävien} numeerisessa ratkaisussa käytettävien
\index{askelmenetelmä}%
nk.\ \kor{askelmenetelmien} perusiedoita. Otetaan tarkastelun kohteeksi jälleen ensimmäisen
kertaluvun normaalimuotoinen alkuarvotehtävä
\begin{equation} \label{dy-7: ivp}
\begin{cases}
\,y'=f(x,y),\quad x>x_0, \\
\,y(x_0)=y_0.
\end{cases}
\end{equation}
Kaikissa askelmenetelmissä on ideana lähteä alkuarvopisteestä $x_0$ ja edetä lyhyin 
\pain{askelin}, ensin pisteeseen $x_1$, sitten pisteeseen $x_2$, jne. Jokaisella askeleella
$x_k \kohti x_{k+1}$ lasketaan luku $y_{k+1}$, joka pyrkii olemaan likiarvo tarkan ratkaisun
$y(x)$ arvolle pisteessä $x_{k+1}$:
\[
y_{k+1}\approx y(x_{k+1}),\quad k=0,1,\ldots
\]
Lukua $y_{k+1}$ laskettaessa tunnetaan jo luvut $y_0,y_1, \ldots y_{k}$ (alkuehdosta ja 
aikaisemmilta askelilta), joten $y_{k+1}$ voidaan laskea p\pain{alautuvasti}. Jos laskukaavassa
esiintyy vain edeltävä luku $y_k$, on kyseessä 
\index{yksiaskelmenetelmä} \index{moniaskelmenetelmä}%
\kor{yksiaskelmenetetelmä}, muussa tapauksessa \kor{moniaskelmenetelmä}.
Erilaiset askelmenetelmät erottaa siis lopulta toisistaan vain se,
millaista palautuskaavaa tai yleisempää algoritmia luvun $y_{k+1}$ laskemisessa käytetään.
Tarkkuutta voidaan säädellä sekä menetelmän valinnalla että
\index{askelpituus}%
\kor{askelpituuksilla} $x_{k+1}-x_k$ (vrt.\ numeerisen integroinnin menetelmät Luvussa
\ref{numeerinen integrointi}). Seuraavassa tarkastellaan ensin askelmenetelmistä kaikkein
yksinkertaisinta --- myös vanhinta.

\subsection*{Eulerin menetelmä}
\index{Eulerin!d@menetelmä|vahv}

\kor{Eulerin menetelmä}\footnote[2]{Sveitsiläinen matemaatikko \hist{Leonhard Euler} 
(1707-1783) on entuudestaan tuttu jo kompleksisesta eksponenttifunktiosta (Eulerin kaava) ja
hänen mukaansa nimetystä differentiaaliyhtälöstä. \kor{Eulerin yhtälöt} (vallan toisessa 
merkityksessä) ovat keskeisiä matematiikan lajissa nimeltä \kor{variaatiolaskenta} ja 
(jälleen toisessa merkityksessä) virtausmekaniikassa. Euler tutki aikansa matematiikkaa ja 
fysiikkaa hyvin laajasti, myös tähtitiedettä, musiikkia, merenkulkua, ym. Euler oli 1700-luvun
merkittävimpiä ja kaikkien aikojen monipuolisimpia matemaatikkoja. \index{Euler, L.|av}} on
yksiaskelmenetelmä, jossa luku $y_{k+1}$ lasketaan palautuskaavasta 
\[
\boxed{\kehys\quad y_{k+1}=y_k+hf(x_k,y_k),\quad h=x_{k+1}-x_k, \quad k=0,1,\ldots \quad}
\]
Kaavan voi johtaa esim.\ seuraavasti: Integroidaan differentiaaliyhtälö $y'=f(x,y)$ puolittain
välin $[x_k,\,x_{k+1}]$ yli, jolloin seuraa
\[
y(x_{k+1})=y(x_k)+\int_{x_k}^{x_{k+1}} f(x,y(x))dx.
\]
Oletetaan, että $f(x,y(x))=y'(x)$ on likimain vakio välillä $[x_k,x_{k+1}]$ ja tehdään tähän
nojaten approksimaatio
\[
\int_{x_k}^{x_{k+1}} f(x,y(x))dx \approx hf(x_k,y(x_k)), \quad h=x_{k+1}-x_k.
\]
(Kyse on yksinkertaisesta numeerisesta integroinnista, vrt.\ Luku 
\ref{numeerinen integrointi}.) Eulerin menetelmään päädytään, kun tulos
\begin{align*}
&y_{k+1} \approx y_k+hf(x_k,y_k), \quad y_k=y(x_k),\ y_{k+1}=y(x_{k+1})
\intertext{'murjotaan' muotoon}
&y_{k+1} = y_k+hf(x_k,y_k), \quad y_k \approx y(x_k),\ y_{k+1} \approx y(x_{k+1}).
\end{align*}
Jos askelpituus $h$ oletetaan tunnetuksi (esim.\ vakio), niin Eulerin menetelmän askel koostuu
yhdestä funktioevaluaatiosta, yhdestä kertolaskusta ja kahdesta yhteenlaskusta:
\[
x_k=x_{k-1}+h, \quad t_k=f(x_k,y_k), \quad y_{k+1}=y_k+ht_k.
\]
\begin{Exa} Laske Neperin luvulle $e$ likiarvo $y_n$ ratkaisemalla alkuarvotehtävä 
$y'=y,\ y(0)=1$ Eulerin menetelmällä ja vakioaskelpituudella $h=1/n,\ n\in\N$. Arvioi virhe.
\end{Exa}
\ratk Eulerin menetelmällä laskien saadaan
\begin{align*}
y_0=1, \quad &y_k=y_{k-1}+hy_{k-1}=(1+h)y_{k-1}, \quad k=1,2 \ldots \\
      \qimpl &y_n = (1+h)^n = \underline{\underline{\left(1+\frac{1}{n}\right)^n}}.
\intertext{Virheen arvioimiseksi tarkastellaan lauseketta}
             &\ln\frac{y_n}{e} = n\,\ln\left(1+\frac{1}{n}\right)-1.
\end{align*}
Taylorin lauseen (ks.\ Luku \ref{taylorin lause}) mukaan on 
$\,\ln(1+x)=x-\frac{1}{2}x^2+\ordoO{\abs{x}^3}$, joten seuraa
\begin{align*}
\ln\frac{y_n}{e}\,=\,-\frac{1}{2n}+\ordoO{n^{-2}} 
                  &\qimpl \frac{y_n}{e}\,=\,e^{-\frac{1}{2n}+\ordoO{n^{-2}}}
                                       \,=\,1-\frac{1}{2n}+\ordoO{n^{-2}} \\
                  &\qimpl e-y_n = \underline{\underline{\frac{e}{2n}+\ordoO{n^{-2}}}}. \loppu
\end{align*}

\subsection*{Eulerin menetelmän virhe}

Edellä esimerkissä saatu virhearvio $\,y_n-y(x_n)=\ordoO{h}\,$ pätee Eulerin menetelmälle melko
yleisin edellytyksin. Yleisemmän virhearvion johtamiseksi olkoon askelpituus $h=$ vakio ja 
kirjoitetaan $Y_k=y(x_k)=y(x_0+kh)$, missä $y(x)$ on alkuarvotehtävän \eqref{dy-7: ivp} 
\index{konsistenssivirhe}%
ratkaisu. Arvioidaan ensin Eulerin menetelmän nk.\ \kor{konsistenssivirhe}, eli luku $\delta_k$
yhtälössä
\[
Y_{k+1}=Y_k+hf(x_k,Y_k)+\delta_k.
\]
Kun merkitään $\,u(x)=f(x,y(x))=y'(x)$, niin osittain integroimalla seuraa (ks.\ myös Eulerin
menetelmän johto edellä)
\begin{align*}
\delta_k &= \int_{x_k}^{x_{k+1}} u(x)\,dx - hu(x_k) \\
         &=\sijoitus{x_k}{x_{k+1}} (x-x_{k+1})u(x) - \int_{x_k}^{x_{k+1}} (x-x_{k+1})u'(x)\,dx
                                                   - hu(x_k) \\
         &= \int_{x_k}^{x_{k+1}} (x_{k+1}-x)y''(x)\,dx.
\end{align*}
Oltetetaan nyt (Oletus \#1), että $y$ on kahdesti jatkuvasti derivoituva ja pätee arvio
$\abs{y''(x)} \le M$ (kyseeseen tulevilla $x$:n arvoilla). Tällöin konsistenssivirhe on enintään
\[
\abs{\delta_k} \le \int_{x_k}^{x_{k+1}} (x_{k+1}-x)\,M\,dx = \frac{1}{2}\,Mh^2.
\]
Käyttäen tätä arviota ja kolmioepäyhtälöä voidaan nyt päätellä:
\begin{align*}
&\begin{cases} 
\, Y_{k+1}=Y_k+hf(x_k,Y_k)+\delta_k, \\ \,y_{k+1}=y_k+hf(x_k,y_k)
 \end{cases} \\
&\qimpl \abs{Y_{k+1}-y_{k+1}} \le \abs{Y_k-y_k}+h\abs{f(x_k,Y_k)-f(x_k,y_k)}+\frac{1}{2}\,Mh^2.
\end{align*}
Oletetaan (Oletus \#2), että oikealla puolella voidaan käyttää arviota
\[
\abs{f(x,y_1)-f(x,y_2)} \le L\abs{y_1-y_2}
\]
($x=x_k,\,y_1=Y_k,\,y_2=y_k$), missä $L$ on vakio. Tällöin seuraa
\[
\abs{Y_{k+1}-y_{k+1}} \le (1+hL)\,\abs{Y_k-y_k} + \frac{1}{2}\,Mh^2.
\]
Soveltamalla tätä epäyhtälöä palautuvasti, kun $k=0,1 \ldots\,$ seuraa (induktio!)
\[
\abs{Y_n-y_n}\,\le\,\frac{1}{2}\,Mh^2 \sum_{k=0}^{n-1} (1+hL)^k\,
                 =\,\frac{M}{2L}\bigl[(1+hL)^n-1\bigr]\,h, \quad n=1,2 \ldots
\]
Käyttämällä tässä vielä arviota 
$\,(1+hL)^n \le \bigl(e^{hL}\bigr)^n = e^{Lnh} = e^{L(x_n-x_0)}\,$ saadaan Eulerin menetelmän 
virhearvioksi tehdyin oletuksin
\begin{equation} \label{Eulerin virhe}
\boxed{\quad
\abs{y(x_n)-y_n} \le \frac{M}{2L}\bigl(e^{L(x_n-x_0)}-1\bigr)\,h, \quad x_n=x_0+nh. \quad}
\end{equation}
\begin{Exa} Alkuarvotehtävä
\[
\begin{cases} \,y'=e^{-x}+x\sin y, \quad x>0, \\ \,y(0)=1 \end{cases}
\]
ei ratkea perinteisin menetelmin, joten on tyytyminen numeeriseen ratkaisuun.
Differentiaaliyhtälöstä nähdään, että ratkaisulle pätee $\abs{y'(x)} \le 1+x$, kun $x \ge 0$.
Derivoimalla implisiittisesti ja käyttämällä tätä arviota seuraa
\[
y''(x)=-e^{-x}+\sin y+xy'\cos y \qimpl \abs{y''(x)} \le 2+x+x^2,\,\ x \ge 0.
\]
Koska edelleen on
\[
\abs{f(x,y_1)-f(x,y_2)}\,=\,x\abs{\sin y_1-\sin y_2}\,\le\,x\abs{y_1-y_2}, \quad x \ge 0,
\]
niin nähdään, että Eulerin menetelmän virhearvio \eqref{Eulerin virhe} on pätevä, kun arviossa
asetetaan
\[
L=x_n=nh, \quad M=2+x_n+x_n^2 \quad (n \ge 1).
\]
Esimerkiksi jos halutaan ratkaisu välillä $[0,2]$, niin voidaan valita $L=2,\ M=8$, jolloin
virhe on enintään
\[
\abs{y(x_n)-y_n} < 2e^4 h < 110 h \quad (x_n \le 2\,).
\]
Todennäköisesti virhe on huomattavasti tätä arviota pienempi. \loppu
\end{Exa}

\subsection*{Differentiaaliyhtälösysteemit}
\index{differentiaaliyhtälö!g@--systeemi|vahv}

Eulerin menetelmän, samoin kuin muidenkin askelmenetetelmien (ks.\ esimerkit jäljempänä) vahva
puoli sovellusten kannalta on, että menetelmät soveltuvat sellaisenaan myös normaalimuotoisen 
differentiaaliyhtälösysteemin ratkaisuun, sikäli kuin kyseessä on alkuarvotehtävä. Myös 
korkeamman kertaluvun differentiaaliyhtälöiden alkuarvotehtävät ratkeavat tällä tavoin 
kirjoittamalla tehtävä ensin systeemimuotoon, vrt.\ Luku \ref{DY-käsitteet}. Esimerkki
valaiskoon asiaa.
\begin{Exa} Esitä Eulerin menetelmään perustuva algoritmi, jolla voidaan ratkaista numeerisesti
alkuarvotehtävä
\[
\begin{cases} \,y'''=x+yy'y'', \quad x>0, \\ \,y(0)=1,\ y'(0)=y''(0)=0. \end{cases}
\]
\end{Exa}
\ratk Alkuehdoista ja ratkaistavan differentiaaliyhtälön systeemimuodosta
\[
\begin{cases} \,y'=u, \\ \,u'=v, \\ \,v'=x+yuv \end{cases}
\]
saadaan algoritmiksi
\[
y_0=1,\ u_0=v_0=0, \quad
\begin{cases}
\,y_{k+1}=y_k+hu_k, \\ 
\,u_{k+1}=u_k+hv_k, \\ 
\,v_{k+1}=v_k+h(x_k+y_ku_kv_k), \quad k=0,1 \ldots \loppu
\end{cases}
\]

\subsection*{*Muita askelmenetelmiä}
\index{askelmenetelmä|vahv}

Tavallisten differentiaaliyhtälöiden alkuarvotehtävissä käytettävien 'mahdollisten menetelmien
luettelo' on nykyisin tavattoman pitkä. Seuraavassa vain muutamia esimerkkejä usein käytetyistä
tai muuten hyvin tunnetuista askelmenetelmistä.
\index{implisiittinen Euler} \index{trapetsi} \index{keskipistesääntö}
\index{klassinen Runge-Kutta} \index{Runge-Kutta (klassinen)}%
\vspace{5mm}\newline
\underline{Implisiittinen Euler} \vspace{2mm}\newline
$y_{k+1}=y_k+hf(x_{k+1},y_{k+1})$ \vspace{5mm}\newline
\underline{Trapetsi} \vspace{2mm}\newline
$y_{k+1}=y_k+\frac{h}{2}\bigl[f(x_k,y_k)+f(x_{k+1},y_{k+1})\bigr]$ \vspace{5mm}\newline
\underline{Keskipistesääntö} \vspace{2mm}\newline
$y_{k+1}=y_k+hf\left(x_k+\frac{1}{2}h,\,\frac{1}{2}y_k+\frac{1}{2}y_{k+1}\right)$
\vspace{5mm}\newline
\underline{BDF-2} \vspace{2mm}\newline
$y_{k+1}=\frac{4}{3}y_k-\frac{1}{3}y_{k-1}+\frac{2}{3}hf(x_{k+1},y_{k+1})$ \vspace{5mm}\newline
\underline{BDF-3} \vspace{2mm}\newline
$y_{k+1}=\frac{18}{11}y_k-\frac{9}{11}y_{k-1}+\frac{2}{11}y_{k-2}
                         +\frac{6}{11}hf(x_{k+1},y_{k+1})$ \vspace{5mm}\newline
\underline{BDF-4} \vspace{2mm}\newline
$y_{k+1}=\frac{48}{25}y_k-\frac{36}{25}y_{k-1}+\frac{16}{25}y_{k-2}-\frac{3}{25}y_{k-3}
                         +\frac{12}{25}hf(x_{k+1},y_{k+1})$ \vspace{5mm}\newline
\underline{Klassinen Runge-Kutta} \vspace{2mm}\newline
$y_{k+1}=y_k+\frac{h}{6}\bigl(V_1+2V_2+2V_3+V_4\bigr),$ \vspace{2mm}\newline
$\phantom{y_{k+1}=\,} 
V_1=f(x_k,y_k),\ \ 
V_2=f\left(x_k+\frac{1}{2}h,y_k+\frac{1}{2}hV_1\right)$ \vspace{2mm}\newline
$\phantom{y_{k+1}=\,}
V_3=f\left(x_k+\frac{1}{2}h,y_k+\frac{1}{2}hV_2\right),\ \ V_4=f\left(x_k+h,y_k+hV_3\right)$
\vspace{5mm}\newline
\index{kertaluku!b@tarkkuuden}
Näistä implisiittinen (samoin kuin tavallinen) Eulerin menetelmä on \kor{ensimmäisen kertaluvun}
menetelmä, eli virhe on suotuisissa oloissa luokkaa $\ordoO{h}$. Trapetsi, keskipistesääntö ja
BDF-2 ovat \kor{toisen kertaluvun} menetelmiä (virhe $\ordoO{h^2}$ riittävän säännöllisille 
ratkaisuille), BDF-3 on kertalukua $3$ ja BDF-4 ja klassinen Runge-Kutta molemmat kertalukua 
$4$. Esimerkeistä BDF-2, BDF-3 ja BDF-4 edustavat \pain{moniaskelmenetelmiä} (askelpituus 
oletettu vakioksi). Näiden käynnistämiseksi on laskettava ensin erillisellä algoritmilla 
$y_1 \ldots y_{m-1}$ (BDF-$m$). --- Nimi BDF on lyhenne sanoista Backward Differentiation
Formula (ks. Harj.Teht.\,\ref{H-dy-7: BDF}). 

\index{implisiittinen (askel)menetelmä}%
Kaikissa esimerkeissä, viimeistä lukuun ottamatta, on kyse \kor{implisiittisestä} menetelmästä.
Tällä tarkoitetaan, että laskettava luku $y_{k+1}$ esiintyy laskukaavassa myös funktion $f$ 
'alla'. Tällöin lukua ei saada selville pelkästään funktioevaluaatioilla, vaan on ratkaistava
yhtälö (systeemin tapauksessa yhtälöryhmä) muotoa $y_{k+1}=F(y_{k+1})$. Käytännössä tämä 
ratkaistaan likimäärin, esimerkiksi muutamalla (1--3) Newtonin iteraatiolla tai 
yksinkertaisemmin kiintopisteiteraatiolla: 
\[
t_{j+1}=F(t_j), \quad j=0 \ldots m\,; \quad y_{k+1}=t_{m+1}.
\]
(Vrt.\ Luku \ref{kiintopisteiteraatio} --- yhtälöryhmien iteratiivisista ratkaisumenetelmistä
tulee puhe myöhemmin.) Iteraation alkuarvoksi voidaan ottaa esim.\ Eulerin menetelmän antama
$\,t_0=y_k+hf(x_k,y_k)$. 

Implisiittiset menetelmät kuten implisiittinen Euler ja BDF-$m$ ovat differentiaaliyhtälöiden
'black box'-ratkaisijoina hyvin suosittuja, syystä että ne ovat vaihtelevissa tilanteissa 
suhteellisen varmoja. --- Todettakoon tässä ainoastaan, että varmuus tulee esiin erityisesti
sellaisissa nk.\
\index{kankeus (DY:n)}%
\kor{kankeissa} (engl.\ stiff) tehtävissä, joissa ratkaisu sisältää nopeasti
(eksponentiaalisesti) vaimenevia transientteja. Tällaisessa tehtävässä esimerkiksi 
implisiittinen Euler on tavallista Eulerin menetelmää huomattavasti varmempi vaihtoehto, vaikka
menetelmien muodollisessa tarkkuudessa ei ole eroa (ks.\ Harj.teht.\,\ref{H-dy-7: kankea DY}).
Sen sijaan tehtävissä, joissa mainitun tyyppisiä erityisongelmia ei esiinny, ovat 
yksinkertaisemmat
\index{eksplisiittinen (askel)menetelmä}%
\kor{eksplisiittiset} menetelmät (kuten Euler ja klassinen Runge-Kutta)
edelleen varteenotettavia vaihtoehtoja ja myös käytössä.

\Harj
\begin{enumerate}

\item \label{H-dy-7: Euler-kokeita} 
Laske seuraavissa alkuarvotehtävissä Eulerin menetelmällä likiarvot luvuille 
$\,y(x_k),\ k=1 \ldots 5$, missä $x_k=k/5$ (askelpituus $h=0.2$). Vertaa tarkkaan ratkaisuun!
\[
\text{a)}\ \ \begin{cases} \,y'=-y^2, \\ \,y(0)=1 \end{cases}
\text{b)}\ \ \begin{cases} \,y'=y^2, \\ \,y(0)=1 \end{cases}
\text{c)}\ \ \begin{cases} \,y'=-2xy^2, \\ \,y(0)=1 \end{cases}
\text{d)}\ \ \begin{cases} \,y'=2xy^2, \\ \,y(0)=1 \end{cases}
\]

\item
Alkuarvotehtävä
\[
y(0)=1, \quad y'=\frac{x}{\sqrt{y^2+1}}+\frac{2x^2}{x^2+1}
\]
ratkaistaan numeerisesti välillä $[0,10]$ käyttäen Eulerin menetelmää ja vakioaskelpituutta $h$.
Arvioi vakiot $L$ ja $M$ Eulerin menetelmän virhearviossa ko.\ välillä. Millä askelpituuksilla
voidaan taata, että numeerisen ratkaisun suhteellinen virhe ei ylitä yhdessäkään
laskentapisteessä arvoa $10^{-4}\,$?

\item
Alkuarvotehtävä $y'=-1/y,\ y(0)=3$ ratkaistaan numeerisesti välillä $[0,1]$ käyttäen
askelpituutta $h=0.2$. Laske seuraavilla menetelmillä approksimaatio $y_5$ luvulle
$y(1)\,$: Euler, Trapetsi, BDF-2, BDF-3, Klassinen Runge-Kutta. Vertaa tarkkaan arvoon!

\item
Alkuarvotehtävässä $y'=f(x),\ y(0)=0$ klassinen Runge-Kutta-menetelmä pelkistyy erääksi
tunnetuksi numeerisen integroinnin säännöksi. Mistä säännöstä on kyse?

\item \label{H-dy-7: BDF}
Differentiaaliyhtälön $y'=f(x,y)$ ratkaisumenetelmä BDF-$m$ johdetaan kirjoittamalla yhtälö
pisteessä $x=x_{k+1}$, tekemällä approksimaatio
\[
y'(x_{k+1}) \approx \frac{1}{h} \sum_{j=0}^m \alpha_j y_{k+1-j}, \quad y_i=y(x_i) 
\]
ja vaatimalla, että tämä on tarkka polynomeille astetta $\le m$. Tarkista laskemalla kertoimet
$\alpha_j$, että tuloksena on todella menetelmä BDF-$2$, kun $m=2$. Millainen menetelmä on 
BDF-$1\,$?

\item \label{H-dy-7: kankea DY}
Olkoon $\tau = 0.02$. Halutaan ratkaista numeerisesti alkuarvotehtävä
\[
\begin{cases} \,\tau y'+y=\sin x, \quad x>0, \\ \,y(0)=0.1. \end{cases}
\]
Vertaa tavallisen ja implisiittisen Eulerin menetelmän antamia ratkaisuja laskemalla molemmilla
menetelmillä $\,y_k \approx y(kh),\ k=1 \ldots 5\,$ askelpituuksilla $h=0.01$, $h=0.04$ ja 
$h=0.08$. Hahmottele kaikki ratkaisut graafisesti ja vertaa myös tarkkaan ratkaisuun! 
(Ks.\ sovellusesimerkki Luvussa \ref{lineaarinen 1. kertaluvun DY}.)

\item
Tasossa liikkuvaan avaruusalukseen vaikuttaa origossa olevan maapallon vetovoima
$\vec F=-(k/r^2)\,\vec e_r$, missä $k$ on vakio (napakoordinaatisto). Hetkellä $t$ alus on 
paikassa $(x(t),y(t))$ ja sen nopeus on $u(t)\vec i+v(t)\vec j$. Kirjoita aluksen liikeyhtälö
$m\vec r\,''=\vec F$ karteesisessa koordinaatistossa normaalimuotoiseksi
differentiaaliyhtälösysteemiksi ja esitä Eulerin menetelmään perustuva algoritmi yhtälöiden
numeeriseksi ratkaisemiseksi. Oletetaan tunnetuksi alkuarvot $x_0,y_0,u_0,v_0$ hetkellä $t=0$.

\item (*)
Ratkaistaessa Eulerin menetelmällä alkuarvotehtävää
\[
\begin{cases} \,y'=\dfrac{y^2}{x+y+1}+4x+2, \quad x>0, \\[3mm] \,y(0)=A \end{cases}
\]
askelpituudella $h=0.1$ havaitaan yllättäen, että saadaan tarkka ratkaisu, ts.\
$y_k=y(x_k)\ \forall k\ (x_k=kh)$. Mikä on $A$:n arvo?

\end{enumerate}