\section{Käyrän kaarevuus} \label{käyrän kaarevuus}
\alku
\index{kaarevuus (käyrän)|vahv}
\index{kzyyrzy@käyrä|vahv}
\index{parametrinen käyrä|vahv}

Tarkastellaan tason tai avaruuden (parametrisoitua) käyrää
\[
S=\{P(t) \in \R^d \mid P(t) \vastaa \vec r\,(t),\,\ t\in [a,b]\},
\]
missä
\[
\vec r\,(t) = \begin{cases}
x(t)\vec i + y(t)\vec j, &(\Rkaksi) \\
x(t)\vec i + y(t)\vec j + z(t)\vec k. &(\Rkolme)
\end{cases}
\]
Jatkossa oletetaan, että $x(t)$, $y(t)$ ja $z(t)$ ovat kahdesti jatkuvasti derivoituvia
välillä $[a,b]$. Merkitään lisäksi $\,\abs{\dvr(t)}=v(t)\,$ ja asetetaan 'pysähtymiskielto'
(vrt.\ edellinen luku)
\begin{equation} \label{derivaattaehto}
v(t)=\abs{\dvr(t)}>0 \quad\forall t\in [a,b].
\end{equation}
Käyttöön tulevat myös seuraavat derivoimissäännöt vektoriarvoisille funktioille.
\begin{Prop} \label{vektorifunktioiden tulon derivointi} Jos $f$ on derivoituva pisteessä
$t$ ja
\[ 
\vec u(t) = u_1(t)\vec i + u_2(t)\vec j + u_3(t)\vec k, \quad 
\vec v(t) = v_1(t)\vec i + v_2(t)\vec j + v_3(t)\vec k, 
\]
missä funktiot $u_i$ ja $v_i$ ovat derivoituvia pisteessä $t$, niin pätee
\[ \boxed{ \begin{aligned}
\quad \frac{\ykehys d}{dt}\left[f(t)\vec u(t)\right] \quad\, 
            &=\,f'(t)\,\vec u(t) + f(t)\,\vec u\,'(t), \\
\frac{d}{dt}\left[\vec u(t)\cdot\vec v(t)\right]\,\          
            &=\,\vec u\,'(t)\cdot\vec v(t) + \vec u(t)\cdot\vec v\,'(t), \\
\frac{d}{\akehys dt}\left[\vec u(t)\times\vec v(t)\right]    
            &=\,\vec u\,'(t)\times\vec v(t) + \vec u(t)\times\vec v\,'(t). \quad
\end{aligned} } \]
\end{Prop}
\tod Kyse on tulon derivoimissäännön (Luku \ref{derivaatta}) yleistyksistä. Esimerkkinä
johdettakoon säännöistä toinen (muut perustellaan vastaavalla tavalla):
\begin{align*} 
\frac{d}{dt}\,\vec u(t)\cdot\vec v(t)\ 
               &=\ \frac{d}{dt}\,\sum_{i=1}^3 u_i(t)\,v_i(t)\ 
                =\ \sum_{i=1}^3 \frac{d}{dt}\,u_i(t)\,v_i(t) \\
               &=\ \sum_{i=1}^3\left[\,u_i'(t)v_i(t)+u_i(t)v_i'(t)\,\right] \\
               &=\ \sum_{i=1}^3 u_i'(t)\,v_i(t) + \sum_{i=1}^3 u_i(t)\,v_i'(t)
                =\ \vec u\,'(t)\cdot\vec v(t) + \vec u(t)\cdot\vec v\,'(t). \loppu
\end{align*}

Kerrattakoon edellisestä luvusta, että kun merkitään
\[
\Delta\vec r=\vec r\,(t+\Delta t)-\vec r\,(t) \approx \dvr(t)\Delta t,
\]
niin ehdon \eqref{derivaattaehto} ollessa voimassa voidaan käyrän $S$ yksikkötangenttivektori
$\tv(t)$ pisteessä $t \in [a,b]$ määrätä toispuolisina raja-arvoina
\[
\tv_\pm = \lim_{\Delta t\kohti 0^\pm} \frac{\Delta\vec r}{\abs{\Delta\vec r\,}}, 
                                                             \quad t \in [a,b].
\]
(Välin päätepisteissä on vain toinen raja-arvoista mahdollinen.) Funktion $t \map \vec r\,(t)$ 
ollessa kahdesti jatkuvasti derivoituva välillä $[a,b]$ voidaan edelleen laskea 
yksikkötangenttivektorin derivaatta $\tv\,'(t)$, kun $t \in [a,b]$ (toispuolinen derivaatta
välin päätepisteissä). Jos käyrä $S$ on jana, niin $\tv(t)$ on vakio, jolloin 
$\tv\,'(t) = \vec 0,\ t \in [a,b]$. Yleisemmin itseisarvo $\abs{d\tv/dt}$ kertoo, kuinka
\pain{kaareva} käyrä on ko.\ pisteessä. Kun tangenttivektorin $\tv$  muutosta merkitään
\[
\Delta\tv=\tv(t+\Delta t)-\tv(t),
\]
niin käyrän \kor{kaarevuus} (engl.\ curvature) määritellään raja-arvona
\begin{equation} \label{kaarevuuden peruskaava} \boxed{
\quad \kappa(t)=\frac{\ykehys 1}{\akehys R(t)}
               =\lim_{\Delta t\kohti 0} \frac{\abs{\Delta\tv\,}}{\abs{\Delta\vec r\,}}
               =\frac{1}{v(t)} \left|\frac{d\tv}{dt}\right| \quad \text{(kaarevuus).} \quad }
\end{equation}
Tässä siis $\,v(t)=\abs{\dvr(t)}$, ja $\,R=1/\kappa\,$ on nimeltään 
\index{kaarevuussäde}%
\kor{kaarevuussäde} (engl.\ radius of curvature).
\begin{figure}[H]
\setlength{\unitlength}{1cm}
\begin{center}
\begin{picture}(8,5)
\put(0,3){\vector(3,-2){3}}
\put(0,3){\vector(1,0){7}}
\put(3,1){\vector(2,1){4}}
\curve(1,0.8,3,1,5,1.7,7,3,8,4)
\put(3,1){\vector(4,1){3.5}}
\put(7,3){\vector(4,3){3}}
\put(6,3.2){$\scriptstyle{\vec r(t+\Delta t)}$}
\put(2.6,1.4){$\scriptstyle{\vec r(t)}$}
\put(6.3,1.5){$\scriptstyle{\tv}$}
\put(9.9,4.7){$\scriptstyle{\tv + \Delta\tv}$}
\put(5.5,2.5){$\scriptstyle{\Delta\vec r}$}
\put(2.95,0.95){$\scriptstyle{\bullet}$}
\put(6.94,2.93){$\scriptstyle{\bullet}$}
\end{picture}
\end{center}
\end{figure}
%\end{multicols}
Samoin kuin yksikkötangenttivektori, myös kaarevuus on vain käyrän geometriasta, ei 
parametrisoinnista riippuva.
\begin{Exa} Partikkeli liikkuu tasossa pitkin origokeskistä $R$-säteistä ympyrärataa siten, 
että napakulma hetkellä $t$ on $\varphi(t)$. Määritä liikeradan kaarevuus pisteessä 
$P(t) \vastaa \vec r\,(t)$, jossa $\varphi'(t) \neq 0$. \end{Exa}
\ratk Tässä on (vrt.\ Esimerkki \ref{ympyräliike} edellisessä luvussa)
\[ 
\vec r\,(t) = R[\cos\varphi(t)\vec i + \sin\varphi(t)\vec j\,], \quad  
\tv(t) = \pm[-\sin\varphi(t)\vec i + \cos\varphi(t)\vec j\,], 
\]
joten $v(t)=\abs{\dvr(t)}=R\abs{\varphi'(t)}$ ja $\abs{d\tv/dt}=\abs{\varphi'(t)}$. Kaavan
\eqref{kaarevuuden peruskaava} mukaan kaarevuus on $\kappa(t)=1/R=$ vakio, ja kaarevuussäde
siis $=R$, kuten odottaa sopikin. \loppu

Vektorin $d\tv/dt$ suuntaista yksikkövektoria $\vec n$ sanotaan käyrän 
\index{pzyzy@päänormaalivektori}%
\kor{päänormaalivektoriksi}. Tämä on todella käyrän normaalivektori, eli tangenttivektoria 
vastaan kohtisuora. Nimittäin koska $\abs{\tv(t)} = 1\ \forall t$, niin 
Proposition \ref{vektorifunktioiden tulon derivointi} säännöillä päätellään
\[ 
\tv(t)\cdot\tv(t) = 1 \qimpl \frac{d}{dt}\,\tv(t)\cdot\tv(t) 
                              = 2\,\tv(t)\cdot\tv\,'(t) = \frac{d}{dt}\,1 = 0. 
\]
(Yleisemmin on y\pain{ksikkövektorin} derivaatta aina vektoria vastaan kohtisuora.) Kaava 
\eqref{kaarevuuden peruskaava} huomioiden on siis päädytty vektorimuotoiseen kaarevuuden
määritelmään
\begin{equation} \label{kaarevuuskaava a}
\boxed{\quad \frac{\ykehys 1}{\akehys v(t)}\frac{d\tv}{dt}
              =\frac{1}{R}\vec n,\qquad \begin{cases} 
                                       \,1/R=\text{kaarevuus}, \\
                                       \,\vec n=\text{päänormaalivektori}. \quad 
                                       \end{cases}}
\end{equation}
\jatko\begin{Exa} (jatko) Tässä saadaan kaavasta \eqref{kaarevuuskaava a} odotusten
mukaisesti
\[
\vec n(t) = \frac{R}{v(t)}\frac{d\tv}{dt} 
          = -\cos\varphi(t)\vec i-\sin\varphi(t)\vec j = -\vec r\,(t)/R. \loppu
\]
\end{Exa}
Päänormaalivektori $\vec n$ on siis käyrän normaalivektoreista (kaksi vaihtoehtoa!) se, joka
osoittaa käyrän kaareutumissuuntaan. Kun derivoidaan puolittain yhtälöt
$\vec n(t)\cdot\vec n(t)=1$ ja $\vec n(t)\cdot\tv(t)=0$ ja sovelletaan kavaa
\eqref{kaarevuuskaava a}, niin seuraa
\[
\frac{d\vec n(t)}{dt}\cdot\vec n(t)=0, \quad
\frac{d\vec n(t)}{dt}\cdot\tv(t) = -\vec n(t)\cdot\frac{d\tv(t)}{dt}=-\frac{v(t)}{R}\,.
\]
Tämän mukaan kaarevuuden määritelmän \eqref{kaarevuuskaava a} voi esittää myös muodossa
\begin{equation} \label{kaarevuuskaava b}
\boxed{\quad \frac{\ykehys 1}{\akehys v(t)}\frac{d\vec n}{dt}
              =-\frac{1}{R}\tv. \quad }
\end{equation}
Johdetaan vielä kaarevuudelle lauseke, jonka arvo määräytyy suoraan derivaatoista $\dvr$ ja 
$\ddvr$. Määritelmän ja Proposition \ref{vektorifunktioiden tulon derivointi} ensimmäisen
säännön mukaan on
\[
\frac{1}{R}\vec n\ =\ \frac{1}{\abs{\dvr}}\frac{d}{dt}\left(\frac{1}{\abs{\dvr}}\,\dvr\right)\
             =\ \frac{1}{\abs{\dvr}^2}\,\ddvr+\frac{d}{dt}\left(\frac{1}{\abs{\dvr}}\right)\tv.
\]
Kertomalla tämä ristiin vektorilla $\tv$ ja ottamalla puolittain itseiarvot saadaan
\[
\frac{1}{R}\ =\ \frac{\abs{\ddvr\times\tv\,}}{\abs{\dvr}^2}\,.
\]
Kun tähän vielä sijoitetaan $\tv=\dvr/\abs{\dvr}$, tulee laskukaavaksi
\begin{equation} \label{kaarevuuden laskukaava}
\boxed{\quad \kappa = \frac{\ykehys 1}{\akehys R}
                    =\frac{\abs{\dvr\times\ddvr}}{\abs{\dvr}^3}\,. \quad}
\end{equation}
Tulkitaan tämä vielä tasokäyrälle $y=f(x)$. Kun parametrina on $t=x$, niin
$\,\vec r\,(x) = x\vec i + f(x)\vec j$, $\,\dvr = \vec i + f'(x)\vec j$,
$\,\ddvr=f''(x)\vec j$, joten $\,\dvr\times\ddvr=f''(x)\vec k$. Sijoittamalla tämä kaavaan
\eqref{kaarevuuden laskukaava} todetaan, että tasokäyrän kaarevuus pisteessä $(x,f(x))$ on
\begin{equation} \label{tasokäyrän kaarevuus}
\boxed{\quad \kappa=\frac{\ykehys 1}{\akehys R}
     =\frac{\abs{f''(x)}}{[1+(f'(x))^2\,]^{3/2}} \quad (\text{tasokäyrä $y=f(x)$}). \quad}
\end{equation}
\index{merkkinen kaarevuus}%
Ilman itseisarvomerkkejä tätä sanotaan \kor{merkkiseksi} kaarevuudeksi.
\begin{Exa} Käyrän $S:\ y = x^2$ kaarevuussäde pisteessä $(x,x^2)$ on kaavan 
\eqref{tasokäyrän kaarevuus} mukaan
\[ R(x) = \frac{1}{2}\,(1+4x^2)^{3/2}. \loppu \]
\end{Exa}

\subsection*{Kaarevuuskeskiö. Evoluutta}
\index{kaarevuuskeskiö|vahv} \index{evoluutta|vahv}

Kun parametrisen käyrän pisteestä $P(t)\vastaa\vec r\,(t)$ kuljetaan kaarevuussäteen $R$
pituinen matka päänormaalivektorin $\vec n$ suuntaan, tullaan \kor{kaarevuuskeskiöön}. Tämän
paikkavektori on siis
\[
\boxed{\kehys\quad \vec r_0(t)=\vec r\,(t)+R(t)\vec n(t) \quad\text{(kaarevuuskeskiö)}. \quad}
\]
Geometrisesti voidaan tulkita niin, että pisteen $P(t)\vastaa\vec r\,(t)$ ympäristössä käyrä 
on likimain ympyräviiva, jonka säde $=\text{kaarevuussäde}\ R$ ja keskipiste =
kaarevuuskeskiö. Lisäksi tämä nk.\
\index{kaarevuusympyrä}%
\kor{kaarevuusympyrä} on tasossa, jonka suuntavektoreina
ovat $\tv$ ja $\vec n$. Kaarevuusympyrän 'vieriessä' pitkin käyrää piirtää kaarevuuskeskiö
toisen käyrän, jota sanotaan alkuperäisen käyrän \kor{evoluutaksi}.
\begin{Exa} \label{evoluutta} Määritä käyrän $\,y=1/x$, $x>0\,$ evoluutta.
\end{Exa}
\ratk Koska $f'(x)=-1/x^2$, niin käyrän (pää)normaalivektori pisteessä $(x,1/x)$ on
\[
\vec n \,=\, (1+x^{-4})^{-1/2}(x^{-2}\vec i + \vec j)
       \,=\, (x^4+1)^{-1/2}(\vec i + x^2\vec j).
\]
Kaarevuus ko. pisteessä on
\[
\frac{1}{R}=\frac{2x^{-3}}{(1+x^{-4})^{3/2}}=\frac{2x^3}{(x^4+1)^{3/2}}.
\]
Käyrän pistettä $(t,1/t)$ vastaava kaarevuuskeskiö $(x(t),y(t))$ on näin ollen
\begin{align*}
x(t) &\,=\, t+\frac{(t^4+1)^{3/2}}{2t^3}\cdot (t^4+1)^{-1/2} 
      \,=\, \frac{1}{2}(3t+t^{-3}), \\
y(t) &\,=\, t^{-1}+\frac{(t^4+1)^{3/2}}{2t^3}\cdot t^2(t^4+1)^{-1/2} 
      \,=\,\frac{1}{2}(3t^{-1}+t^3).
\end{align*}
Tämä on evoluutan parametriesitys ($t>0$). Suurilla ja pienillä $t$:n arvoilla on likimain
\[ 
\begin{cases}
\,t\gg 1\ \impl\ y\approx\frac{4}{27}x^3, \\
\,t\ll 1\ \impl\ x\approx\frac{4}{27}y^3.
\end{cases}
\]
Pisteessä $(x,y)=(2,2)$ ($t=1$) evoluutalla on nk.\
\index{kzyzy@kääntymispiste}%
\kor{kääntymispiste} (engl.\ turn\-ing point). Kääntymispisteessä on $x'(t)=y'(t)=0$, joten
kyseessä on myös 'pysähtymispiste'. \loppu
\begin{figure}[H]
\setlength{\unitlength}{1cm}
\begin{center}
\begin{picture}(8,8)(-0.5,0)
\put(-0.5,0){\vector(1,0){8}} \put(7.3,-0.5){$x$}
\put(0,-0.5){\vector(0,1){8}} \put(0.2,7.3){$y$}
\curve(
     2,      2,
2.0894,  2.114,
2.2822,2.44343,
2.5221, 2.9855,
2.7857,3.74933,
3.0625,   4.75,
 3.347,6.00582,
3.6362,  7.537)
\curve(
      2,     2,
  2.114,2.0894,
2.44343,2.2822,
 2.9855,2.5221,
3.74933,2.7857,
   4.75,3.0625,
6.00582, 3.347,
  7.537,3.6362)
\curvedashes[0.1cm]{0,1,2}
\curve(
  1,      1,
1.2,0.83333,
1.4,0.71429,
1.6,  0.625,
1.8,0.55556,
  2,    0.5,
2.2,0.45455,
2.4,0.41667,
2.6,0.38462,
2.8,0.35714,
  3,0.33333,
3.2, 0.3125,
3.4,0.29412,
3.6,0.27778,
3.8,0.26316,
  4,   0.25)
\curve(
      1,  1,
0.83333,1.2,
0.71429,1.4,
  0.625,1.6,
0.55556,1.8,
    0.5,  2,
0.45455,2.2,
0.41667,2.4,
0.38462,2.6,
0.35714,2.8,
0.33333,  3,
 0.3125,3.2,
0.29412,3.4,
0.27778,3.6,
0.26316,3.8,
   0.25,  4)
\put(1.1,1.1){$y=1/x$}
\put(1,0){\line(0,-1){0.1}}
\put(0,1){\line(-1,0){0.1}}
\put(0.93,-0.4){$\scriptstyle{1}$}
\put(-0.4,0.9){$\scriptstyle{1}$}
\end{picture}
\end{center}
\end{figure}

\pagebreak
\subsection*{Kaarevuus fysiikassa: kiihtyvyys}
\index{kiihtyvyys|vahv}

Kun parametrisessa käyrässä on kyse liikkuvan pisteen $P(t) \vastaa \vec r(t)$ paikasta ajan 
funktiona, niin $\vec v(t)=\dvr\,(t)=$ nopeusvektori. Tällöin vektoria
\[ 
\vec a(t) = \vec v\,'(t) = \ddvr(t) = x''(t)\vec i + y''(t)\vec j + z''(t)\vec k 
\]
sanotaan \pain{kiiht}y\pain{v}yy\pain{deksi} hetkellä $t$. Kiihtyvyysvektori voidaan kaartuvalla
ratakäyrällä aina esittää muodossa
\[ 
\vec a(t) = a_\tau\,\tv + a_n\,\vec n, 
\]
missä $a_\tau$ on liikeradan suuntainen \pain{tan}g\pain{entiaalikiiht}y\pain{v}yy\pain{s} ja $a_n$
on päänormaalivektorin suuntainen \pain{normaalikiiht}y\pain{v}yy\pain{s}. Tähän tulokseen 
päädytään, kun kirjoitetaan
\[
\vec v(t)=v(t)\,\tv(t),\quad v(t)=\abs{\vec v(t)}=\abs{\dvr(t)}
\]
ja käytetään Proposition \ref{vektorifunktioiden tulon derivointi} ensimmäistä sääntöä sekä
kaavaa \eqref{kaarevuuskaava a}\,:
\begin{align*}
\vec a(t) &= v'(t)\,\tv(t)+v(t)\,\frac{d\tv}{dt} \\
          &=v'(t)\,\tv(t)+\frac{[v(t)]^2}{R}\,\vec n(t).
\end{align*}
Tämän mukaisesti on siis
\[ \boxed{\kehys\quad a_\tau(t) 
          = v'(t), \quad a_n(t) 
          = \frac{[v(t)]^2}{R} \quad \text{(tangentiaali- ja normaalikiihtyvyys)}. \quad} 
\]
Kiihtyvyyden kannalta voidaan siis liike ajatella hetkellisesti ympyräliikkeeksi, jossa 
liikerata yhtyy kaarevuusympyrään.
\index{zza@\sov!Vapaa putoamisliike}%
\begin{Exa}:\ \vahv{Vapaa putoamisliike}. Maan vetovoimakentässä hetkellä $t=0$ käynnistyvän
vapaan putoamisliikkeen (Newtonin) liikeyhtälö on
\[ 
\vec a(t) = \ddvr(t) = -g\vec k, \quad t>0, 
\]
missä $g \approx 9.81$ m/s$^2$. Yhtälö sisältää kolme skalaarista differentiaaliyhtälöä:
\[ 
x''(t) = 0, \quad y''(t) = 0, \quad z''(t) = -g. 
\]
Näiden mukaisesti on ensinnäkin oltava
\[ 
x'(t) = \alpha, \quad y'(t) = \beta, \quad z'(t) = -gt + \gamma, 
\]
missä $\alpha,\beta,\gamma$ ovat (määräämättömiä) vakioita. Tästä nähdään, että on edelleen
oltava (vrt.\ Luku \ref{väliarvolause 2})
\[ 
x(t) = \alpha\,t + x_0, \quad y(t) = \beta\,t + y_0, \quad 
                              z(t) = -\tfrac{1}{2}\,gt^2 + \gamma\,t + z_0, 
\]
missä $x_0,y_0,z_0$ ovat jälleen määräämättömiä vakioita. Siis
\begin{align*}
\vec r\,(t)&= \vr_0 + \vec v_0\,t - \tfrac{1}{2}\,gt^2\,\vec k, \quad t>0, \quad \text{missä} \\
\vec r_0   &= x_0\vec i + y_0\vec j + z_0\vec k = \vr(0^+), \quad 
   \vec v_0 = \alpha\vec i + \beta\vec j + \gamma\vec k = \dvr(0^+) = \vec v(0^+). 
\end{align*}
Jos raja-arvot $\vec r\,(0^+)=\vr_0$ ja $\vec v(0^+) = \vec v_0$ tunnetaan alkuehtoina, on
$\vec r\,(t)$ yksikäsitteisesti määrätty, kun $t \ge 0$. Ratakäyrä on tällöin paraabeli
avaruustasossa, jonka suuntavektorit ovat $\vec v_0$ ja $\vec k$
(vrt.\ Esimerkki \ref{parametriset käyrät}:\,\ref{heittoparaabeli}). \loppu 
\end{Exa}
\index{zza@\sov!Irtoaminen}%
\begin{Exa}:\ \vahv{Irtoaminen}. Kappale, johon vaikuttaa painovoima
\[
\vec G=-mg\vec j \quad (\text{$m=$ massa, $g=$ maan vetovoiman kiihtyvyys})
\]
on levossa origossa ja lähtee siitä kitkattomaan liukuun pitkin käyrää 
\[
y=-\frac{1}{3}\,x^3,\quad x\geq 0.
\]
Määritä kappaleen liikerata muodossa $y=f(x),\ x\ge0$.
\end{Exa}
\ratk Rata noudattaa aluksi käyrää $y=-x^3/3$, mutta irtoaa siitä, kun tämän radan mukainen 
normaalikiihtyvyys ylittää maan vetovoiman kiihtyvyyden normaalin suunnalla. Irtoamisehto on 
siis
\[
\frac{v^2}{R}=-g\vec j\cdot\vec n,
\]
missä $\vec n$ on (pää)normaalivektori
\[
\vec n=-\frac{1}{\sqrt{1+x^4}}(x^2\vec i + \vec j).
\]
Energiaperiaatteen mukaan on
\[
\frac{1}{2}m[v(x)]^2 = mg\cdot \frac{1}{3}\,x^3,
\]
ja kaarevuus pisteessä $x$ on
\[
\frac{1}{R}=\frac{2x}{(1+x^4)^{3/2}}\,,
\]
joten irtoaminen tapahtuu, kun
\begin{align*}
\frac{2x}{(1+x^4)^{3/2}}\cdot\frac{2}{3}gx^3 &= \frac{g}{(1+x^4)^{1/2}} \\
\ekv \ x^4=3 \ &\ekv \ \underline{\underline{x=\sqrt[4]{3}=1.316074..}}
\end{align*}
Irtoamisen jälkeen lentorata on paraabeli (vapaa putoamisliike), joten koko liikerata on
muotoa
\[
y=f(x)=\begin{cases}
-x^3/3,     &\text{kun}\ 0\leq x\leq\sqrt[4]{3}, \\
Ax^2+Bx+C,  &\text{kun}\ x>\sqrt[4]{3}.
\end{cases}
\]
Irtoamiskohdassa on maan vetovoima ainoa kappaleeseen vaikuttava ulkoinen voima. Koska tämä 
voima on jatkuva, ja myös kappaleeseen vaikuttavan tukivoiman voi olettaa olevan jatkuva 
irtoamiskohdassa (irtoamisesta eteenpäin tukivoima $=0$), niin päätellään, että 
kiihtyvyysvektori on jatkuva. Tämä merkitsee, että funktio $f$ on kahdesti jatkuvasti 
derivoituva myös irtoamiskohdassa, eli on oltava
\[
f^{(k)}(x_0^+)=f^{(k)}(x_0^-),\quad k=0,1,2 \ , \ x_0=\sqrt[4]{3}.
\]
Näistä ehdoista voidaan ratkaista vakiot $A,B,C$. Tulos (liikerata) on
\[
y=f(x)=\begin{cases}
-x^3/3,                                         &x\in [0,\sqrt[4]{3}], \\
-\sqrt[4]{3}\,x^2+\sqrt{3}\,x-1/\sqrt[4]{3},\ \ &x\in (\sqrt[4]{3},\infty).
\end{cases}
\]
Kuvassa liuku/lentorata on piirretty yhtenäisellä viivalla. Toisen asteen käyrä ennen
irtoamiskohtaa ja käyrä $y=-x^3/3$ irtoamiskohdan jälkeen on merkitty katkoviivalla. \loppu
\begin{figure}[H]
\setlength{\unitlength}{1cm}
\begin{center}
\begin{picture}(8,7)(-0.5,-6)
\put(-0.5,0){\vector(1,0){8}} \put(7.3,-0.5){$x$}
\put(0,-6){\vector(0,1){7}} \put(0.2,0.8){$y$}
\curve(
  0,        -0,
0.4,-0.0026667,
0.8, -0.021333,
1.2,    -0.072,
1.6,  -0.17067,
  2,  -0.33333,
2.4,    -0.576,
2.8,  -0.91467)
\curvedashes[0.1cm]{0,1,2}
\curve(
2.8,  -0.91467,
3.2,   -1.3653,
3.6,    -1.944,
  4,   -2.6667,
4.4,   -3.5493,
4.8,    -4.608,
5.2,   -5.8587)
%5.6,   -7.3173,
%  6,        -9)
\curvedashes[0.1cm]{0,1,2}
\curve(
  0,-0.75984,
0.4,-0.46607,
0.8,-0.27759,
1.2,-0.19439,
1.6,-0.21648,
  2,-0.34386,
2.4,-0.57652,
2.8,-0.91447)
\curvedashes{}
\curve(
2.8,-0.91447,
3.2, -1.3577,
3.6, -1.9062,
  4,   -2.56,
4.4, -3.3191,
4.8, -4.1835,
5.2, -5.1532)
%5.6, -6.2281,
%  6, -7.4083)
\put(2,0){\line(0,1){0.1}}
\put(0,-1){\line(-1,0){0.1}}
\put(1.9,0.2){$1$}
\put(-0.7,-1.1){$-1$}
\put(2.62,-0.885){$\scriptstyle{\bullet}\,\ \text{Irtoamiskohta}$}
\end{picture}
\end{center}
\end{figure}

\Harj
\begin{enumerate}

\item
Olkoon $\vec u(t)$ ja $\vec v(t)$ derivoituvia vektoriarvoisia funktioita. Todista: \newline
a) \ $\vec u(t)\ ||\ \vec v(t)\ \forall t \,\qimpl \vec u\,'\times\vec v=\vec v\,'\times\vec u$ 
\newline
b) \ $\vec u(t) \perp \vec v(t)\ \forall t  \qimpl \vec u\,'\cdot\vec v=-\vec v\,'\cdot\vec u$
 
\item
Määritä seuraavien tasokäyrien karevuusympyrä (säde $R$ sekä kaarevuuskeskiö) annetussa
käyrän pisteessä $P$: \newline
a) \ $y=x^2,\quad P=(1,1)$ \newline
b) \ $y=x^3-2x^2,\quad P=(2,0)$ \newline
c) \ $y=e^x,\quad P=(0,1)$ \newline
d) \ $\vec r=t\cos t\,\vec i+t\sin t\,\vec j,\quad P=(-\pi,0)$ \newline
e) \ $x=t-\sin t,\ y=1-\cos t,\quad P=(\pi,2)$ \newline
f) \ $2x^2+3y^2=5,\quad P=(1,-1)$ \newline
g) \ $x^3-y^3+y^2-3x+2=0,\quad P=(2,2)$ \newline
h) \ $r=e^\varphi,\quad P=(r,\varphi)=(1,0)$

\item
Määritä seuraavien avaruuskäyrien kaarevuusssäde käyrän pisteessä, joka vastaa annettua
parametrin arvoa. Määritä myös sen avaruustason yhtälö, jossa kaarevuusympyrä sijaitsee, sekä
kaarevuuskeskiö. \newline
a) \ $\vec r=t\,\vec i+t^2\,\vec j+t^3\,\vec k,\quad t=1$ \newline
b) \ $x=\cos t,\ y=\sin t,\ z=(4/\pi)\,t,\quad t=\pi/4$ \newline
c) \ $x=e^t\cos t,\ y=e^t\sin t,\ z=e^t,\quad t=0$

\item \index{oskuloivat käyrät}
Sanotaan, että tasokäyrät $S_1$ ja $S_2$ \kor{oskuloivat} (suom.\ suutelevat) pisteessä 
$(x_0,y_0)$, jos käyrillä on yhteinen kaarevuusympyrä ko.\ pisteessä. \ a) Päättele, että jos 
käyrien yhtälöt ovat $S_1: y=f(x)$ ja $S_2: y=g(x)$, niin oskulointiehdot ovat
\[
f(x_0)=g(x_0)=y_0\,, \quad f^{(k)}(x_0)=g^{(k)}(x_0),\,\ k=1,2.
\]
b) Etsi sellainen toisen asteen polynomikäyrä $S_1:\ y=ax^2+bx+c$, joka oskuloi käyrää
$S_2:\ 2x^3+6y^3+xy=0$ pisteessä $(3,-2)$.

\item
Määritä käyrän $y=x^2$ evoluutta. Piirrä kuva!

\item
Pistemäinen kappale liikkuu pitkin ruuviviivaa
\[
x=\cos \varphi,\ \ y=\sin \varphi,\ \ z=\varphi, \quad\varphi \in [0,\infty),
\]
siten, että sen ratanopeus (vauhti) on vakio $v_0$. Määritä kappaleen nopeus
$\vec{v}$ ja kiihtyvyys $\vec{a}$ ajan $t$ funktiona ($t\geq 0$), kun kappale on pisteessä 
$(1,0,0)$ hetkellä $t=0$. Määritä myös radan kaarevuussäde $R$ ja tarkista, että pätee:
$|\vec{a}|=|\vec{v}|^2/R$.

\item
Maaston korkeusprofiili tunturimaastossa on
\[
h(x,y)=40-0.005xy,
\]
missä pituusyksikkö = m. Retkeilijä heittää pisteessä $(x,y)=(0,0)$ olevasta leiripaikastaan
pilaantuneen tomaatin siten, että tomaatin lähtönopeus on 
\[
\vec v_0=(10\,\text{m}/\text{s})(\vec i-2\vec j+2\vec k).
\]
Missä pisteessä tomaatti törmää maahan ja mikä on tällöin sen vauhti? 
(Oletetaan $\,g=10\text{m}/\text{s}^2$, ei ilmanvastusta eikä tuulikorjausta.)

\item (*)
Näytä, että sykloidin $S: x=R(t-\sin t),\ y=R(1-\cos t),\ t\in\R$ evoluutta on toinen, $S$:n
kanssa yhtenevä sykloidi, joka saadaan siirtämällä $S\,$ vektorin
$-\pi R\vec i-2R\vec j$ verran. Kuva!

\item (*)
Origossa oleva kappale lähtee levosta liukumaan kitkattomasti pitkin käyrää 
$y=1-\cosh x,\ x \ge 0$, painovoiman vaikuttaessa suunnassa $-\vec j$. Määritä kappaleen 
liikerata.

\item (*) \label{H-dif-3: kierevyys}
\index{Frenet'n kanta} \index{kierevyys (käyrän)} \index{sivunormaalivektori}
Avaruuskäyrän pisteeseen $P(t)\vastaa\vec r\,(t)$ liittyvä \kor{Frenet'n kanta} on
vektorisysteemi $\{\tv,\vec n,\vec\nu\}$, missä $\tv$ ja $\vec n$ ovat yksikkö\-tangentti- ja 
päänormaalivektorit ko.\ pisteessä ja $\vec\nu=\tv\times\vec n$ on \kor{sivunormaalivektori}. 
Avaruuskäyrän \kor{kierevyys} $\omega$ (engl.\ torsion) määritellään tällöin kaavalla
\[
\frac{1}{v(t)}\frac{d\vec\nu}{dt}=-\omega(t)\vec n, \quad v(t)=\abs{\dvr(t)}.
\]
a) Näytä, että $d\vec\nu/dt$ todella on päänormaalivektorin suuntainen. \newline
b) Päättele, että jos avaruuskäyrä on tasokäyrä jollakin avaruustasolla, niin sen kierevyys 
$=0$. \newline
c) Laske ruuviviivan $S:\ x=a\cos t,\ y=a\sin t,\ z=bt$ kierevyys käyrän pisteessä $P(t)$.

\item (*) \index{zzb@\nim!Sotaharjoitus 2}
(Sotaharjoitus 2) Tykinammus laukaistaan origosta lähtönopeudella 
$\vec v_0=(150\,\text{m}/\text{s})(\vec i+2\vec j+\vec k)$. Lentoradalla ammukseen vaikuttaa 
painovoiman lisäksi tuuli ja nopeuteen verrannollinen vastusvoima siten, että liikeyhtälöt
ovat
\[
\vec v\,'=c\vec i-g\vec k-k\vec v, \quad \dvr=\vec v,
\]
missä $\vec r\,(t)$ ja $\vec v(t)$ ovat ammuksen paikka- ja nopeusvektorit hetkellä $t$,
$\,g=10$ m/s$^2$, $k=0.01$ s$^{-1}$ ja $c=0.10$ m/s$^2$. Laske, mihin $xy$-tason pisteeseen
($10$ metrin tarkkuus!) ammus putoaa. \kor{Vihje}: Ratkaise liikeyhtälöt erikseen suunnissa
$\vec i,\,\vec j,\,\vec k$ (ensin $\vec v$, sitten $\vec r\,$). Aloita pystysuunnasta lentoajan
selville saamiseksi! 

\end{enumerate}