\chapter{Yhden muuttujan differentiaalilaskenta} 
\label{yhden muuttujan differentiaalilaskenta}
\chaptermark{Differentiaalilaskenta}

Kun derivaattaa käytetään laskennassa välineenä, puhutaan \kor{differentiaalilaskennasta}
(kirjaimellisesti 'pienten erotusten laskennasta', engl.\ differential calculus). Derivaatan
toivat matematiikkaan toisistaan riippumatta englantilainen fyysikko-matemaatikko
\index{Newton, I.} \index{Leibniz, G. W.}%
\hist{Isaac Newton} (1642-1727) ja saksalainen filosofi-matemaatikko 
\hist{Gottfried Wilhelm Leibniz} (1646-1716) 1600-luvun lopulla. Leibniz on vaikuttanut
huomattavasti vielä nykyisinkin käytössä oleviin differentiaalilaskun
(myös myöhemmin tarkasteltavan integraalilaskun) merkintöihin. 

Koska derivaatta oli alunperinkin fysiikan motivoima käsite (etenkin Newtonin tutkimuksissa),
ei derivaatan soveltuvuudessa fysiikkaan ole ihmettelemistä. Pitkälle 1800-luvulle 
differentiaalilaskennan ja sen pohjalta nousevien matematiikan alojen kehitys olikin 
voimakkaasti sidoksissa fysiikkaan, ja vielä nykyisinkin on yhteys säilynyt vahvana monissa 
matematiikan lajeissa.

Tässä luvussa tarkastellaan yhden muuttujan differentiaalilaskennan soveltamista
käyräteoriassa, fysiikassa (mm.\ liikeopissa) ja funktioiden approksimoinnissa. Funktion
approksimoinnin keskeinen tulos on Luvussa \ref{taylorin lause} esitettävä ja todistettava
\kor{Taylorin lause}. Tässä on kyse linearisoivan approksimaation yleistämisestä
\kor{Taylorin polynomeihin} perustuvaksi yleisemmäksi polynomiapproksimaatioksi. Suotuisissa
oloissa funktio voidaan esittää myös tarkasti \kor{Taylorin sarjana}. Taylorin sarjat ovat
potenssisarjoja --- ja potenssisarjojen teorian kauniiksi lopuksi osoittautuukin, että
jokainen potenssisarja, jonka suppenemissäde on positiivinen, on itse asiassa sarjan summana
määritellyn funktion Taylorin sarja. Viimeisessä osaluvussa esitellään vielä numeerisissa
laskentamenetelmissä yleisesti käytettyjen \kor{interpolaatiopolynomien} teoriaa ja
käyttötapoja.