\section{Determinantti} \label{determinantti} 
\alku
\index{determinantti|vahv}

Olkoon $\ma_1,\ldots,\ma_n\in\R^n$ $n$ kappaletta pystyvektoreita kokoa $n$. Asetetaan
\begin{Def} \label{determinantin määritelmä} \index{determinanttifunktio|emph}
Reaaliarvoinen funktio $V(\ma_1,\ldots,\ma_n), \ \ma_i\in\R^n$, on (normeerattu)
\kor{determinanttifunktio}, jos pätee:
\begin{itemize}
\item[(i)]   $V$ on lineaarinen jokaisen vektorin $\ma_i$ suhteen.
\item[(ii)]  Jos järjestetty indeksijoukko $(i_1,\ldots,i_n)$ on saatu joukosta $(1,2,\ldots,n)$
             vaihtamalla kahden alkion paikkaa, niin
             \[
             V(\ma_{i_1},\ldots,\ma_{i_n})=-V(\ma_1,\ldots,\ma_n).
             \]
\item[(iii)] Jos $[\me_i]_j=\delta_{ij}$, niin $V(\me_1,\ldots,\me_n)=1$.
\end{itemize}
\end{Def}
\pain{Selityksiä}:
\begin{itemize}
\item[(i)]   Lineaarisuus tarkoittaa, että jokaisella 
             $k\in\{1,\ldots,n\}$ ja $\forall\lambda,\mu\in\R$ pätee
             \begin{align*}
             V(\ma_1,\ldots,\lambda\ma_k &+ \mu\mb_k,\ldots,\ma_n)\\
             &=\lambda V(\ma_1,\ldots,\ma_k,\ldots,\ma_n)
              +\mu V(\ma_1,\ldots,\mb_k,\ldots,\ma_n).
             \end{align*}
\item[(ii)]  \index{vaihtoszyzy@vaihtosääntö!b@determinantin}%
             Tämä \kor{vaihtosääntö} tarkoittaa, että determinantin arvo vaihtuu vastaluvukseen,
             kun järjestetyssä joukossa $(\ma_1,\ldots,\ma_n)$ kahden vektorin paikka 
             vaihdetaan (Luvusta \ref{inverssi} tuttu parivaihto). Jos vaihdettavat vektorit
             ovat samat, ei determinantin arvo luonnollisesti muutu, joten vaihtosäännöstä
             \index{nollasääntö!b@determinantin}%
             seuraa \kor{nollasääntö}
             \[
             \ma_i=\ma_j\,,\,\ i \neq j \qimpl V(\ma_1,\ldots,\ma_n)=0.
             \]
             Jos yleisemmin $p=(i_1,\ldots,i_n)$ on indeksijoukon $(1,\ldots,n)$ permutaatio,
             niin $p$ on joko parillinen tai pariton, eli $p$ saadaan joko parilllisella tai
             parittomalla määrällä parivaihtoja (vrt.\ Luku \ref{inverssi}). Vaihtosääntö
             yleistyy siis säännöksi: Parillisessa permutaatiossa determinantin arvo säilyy,
             parittomassa vaihtuu vastaluvukseen.
\item[(iii)] Tämä on normeerausehto. Ilman tätä ehtoa determinanttifunktio voitaisiin kertoa
             mielivaltaisella vakiolla, jolloin seuraava väittämä ei olisi tosi.
\end{itemize}
\begin{Prop}
Determinanttifunktio on funktio, ts.\ $V(\ma_1,\ldots\ma_n)\in\R$ on yksikäsitteisesti määrätty,
kun $\ma_1,\ldots,\ma_n\in\R^n$.
\end{Prop}
\tod Lasketaan determinanttifunktion arvo annetuilla säännöillä (i)--(iii). Merkitään 
$[\ma_j]_i=a_{ij}$, jolloin
\[
\ma_j=\sum_{i=1}^n a_{ij}\me_i
\]
ja siis
\[
V(\ma_1,\ldots,\ma_n)=V\Bigl(\,\sum_{i=1}^n a_{i1}\me_i,\ldots,\sum_{i=1}^n a_{in}\me_i\Bigr).
\]
Käyttämällä toistuvasti lineaarisuusominaisuutta (i) nähdään, että tämä purkautuu summaksi
\[
V(\ma_1,\ldots,\ma_n)
=\sum_{i_1=1}^n\cdots\sum_{i_n=1}^n a_{i_11} \cdots a_{i_nn} V(\me_{i_1},\ldots,\me_{i_n}).
\]
Tässä on $n^n$ termiä, mutta nollasäännön perusteella summasta voidaan jättää pois kaikki ne 
termit, joissa sama indeksi $i_k$ esiintyy kahdesti. Näin muodoin
\[
V(\ma_1,\ldots,\ma_n)=\sum_p a_{i_11} \cdots a_{i_nn}V(\me_{i_1},\ldots,\me_{i_n}),
\]
missä summaus käy läpi kaikki joukon $(1,\ldots,n)$ eri permutaatiot $p=(i_1,\ldots,i_n)$
($n!$ kpl). Säännöistä (ii) ja (iii) seuraa edelleen, että 
$V(\me_{i_1},\ldots,\me_{i_n})=\sigma_p=\pm 1$ riippuen siitä, onko permutaatio $p$ parillinen
vai pariton. Siis
\begin{equation} \label{determinantin laskukaava}
\boxed{\kehys\Ykehys \quad V(\ma_1,\ldots,\ma_n)
                           =\sum_p\sigma_p\,a_{i_11}a_{i_22} \cdots a_{i_nn} \quad}
\end{equation}
ja näin muodoin $V(\ma_1,\ldots,\ma_n)$ on yksikäsitteisesti määrätty. \loppu

Determinanttifunktion ominaisuudet (i)--(iii) voi päätellä käänteisesti laskukaavasta
\eqref{determinantin laskukaava}, joten tämä kaava on käypä myös determinanttifunktion 
määritelmänä. Kaavalle saadaan hieman selkeämpi muoto, kun merkitään (vrt.\ Luku \ref{inverssi})
\[
\mI_p = [\me_{i_1} \ldots \me_{i_n}], \quad \Lambda_p = \{(i,j) \mid [\mI_p]_{ij}=1\}, \quad
                                                        p=(i_1, \ldots,i_n).
\]
Tällöin nähdään, että kaava \eqref{determinantin laskukaava} on sama kuin
\begin{equation} \label{determinantin laskukaava 2}
V(\ma_1,\ldots,\ma_n) = \sum_p \sigma_p \prod_{(i,j)\in\Lambda_p} a_{ij}.
\end{equation}
Determinantin arvoa käytännössä laskettaessa kaavat 
\eqref{determinantin laskukaava}--\eqref{determinantin laskukaava 2} ovat petollisia, sillä
pelkkiä kertolaskuja tarvitaan niitä käyttäen peräti $(n-1)n!$ kpl. --- Kaavoja käytetäänkin
algoritmeina yleensä vain tapauksissa $n=2$ ja $n=3$. Teoreettista käyttöä kaavoilla 
\eqref{determinantin laskukaava}--\eqref{determinantin laskukaava 2} on sen sijaan 
yleisemminkin, kuten nähdään jatkossa.

\subsection*{Matriisin determinantti}
\index{matriisin ($\nel$neliömatriisin)!ha@$\nel$determinantti|vahv}

\kor{Matriisin determinantti} määritellään yksinkertaisesti ajattelemalla, että 
determinanttifunktiossa $V(\ma_1,\ldots,\ma_n)$ vektorit $\ma_j$ ovat (neliö)matriisin $\mA$ 
sarakkeet. Determinanttia merkitään $\det\mA$ (toisinaan $\abs{\mA}$), ja määritelmä on siis
\[
\det\mA=V(\ma_1,\ldots,\ma_n), \quad \mA=[\ma_1 \ldots \ma_n].
\]
Tämän mukaisesti on kaavassa \eqref{determinantin laskukaava 2} $\,\sigma_p = \det\mI_p$.

Determinantti on siis määritelty vain neliömatriiseille. Jos $\mA$ on annettu taulukkona, niin
determinanttia merkitään taulukkoa rajoittavin pystyviivoin:
\[
\det\mA=\begin{vmatrix}
a_{11} & \ldots & a_{1n} \\
\vdots &  & \vdots \\
a_{n1} & \ldots & a_{nn}
\end{vmatrix}, \quad \mA=(a_{ij}).
\]
Normeeraussäännön (iii) mukaan yksikkömatriisin determinantti on
\[
\det\mI=1.
\]
Kaavasta \eqref{determinantin laskukaava} seuraa myös helposti sääntö
\[
\det(\lambda\mA)=\lambda^n\det\mA, \quad \lambda \in \R.
\]
Myös seuraavat kaksi determinantin ominaisuutta ovat määritelmästä johdettavissa, mutta nämä
ovat vähemmän ilmeisiä. --- Kyseessä ovat determinanttiopin keskeisimmät väittämät.
\begin{Lause} Matriisin determinantille pätee
\[
\boxed{\kehys \quad \det\mA\mB = \det\mA\,\det\mB, \quad\ \det\mA^T = \det\mA. \quad}
\]
\end{Lause}
\tod Olkoon $\mA=[\ma_1 \ldots \ma_n]$ ja $\mB=[\mb_1 \ldots \mb_n]=(b_{ij})$. Käyttämällä Luvun
\ref{matriisialgebra} purkukaavoja voidaan kirjoittaa
\[
\mA\mB 
= [\mA\mb_1 \ldots \mA\mb_n] 
= \Bigl[\,\sum_{i_1=1}^n b_{i_11}\ma_{i_1}\,\ldots\,\sum_{i_n=1}^n b_{i_nn}\ma_{i_n}\Bigr],
\]
jolloin determinantin lineaarisuussäännön (i) perusteella seuraa
\[
\det\mA\mB 
= \sum_{i_1=1}^n\cdots\sum_{i_n=1}^n b_{i_11}\ldots b_{i_nn}\det\,[\ma_{i_1} \ldots \ma_{i_n}].
\]
Tästä summasta voidaan nollasäännön perusteella jälleen jättää pois kaikki termit, joissa sama
indeksi $i_k$ esiintyy kahdesti. Jäljelle jäävissä termeissä on $p=$ $(i_1,\ldots,i_n)$ joukon 
$(1,\ldots,n)$ permutaatio, jolloin vaihtosäännön (ii) mukaan
\[
\det\,[\ma_{i_1} \ldots \ma_{i_n}] = \sigma_p\det\,[\ma_1 \ldots \ma_n] 
                                  = \sigma_p\det\mA, \quad p=(i_1,\ldots,i_n).
\]
Tässä $\sigma_p$ on määritelty kuten kaavassa \eqref{determinantin laskukaava}, joten
\[
\det\mA\mB = \det\mA\,\sum_p \sigma_p\,b_{i_11}\ldots b_{i_nn} = \det\mA\det\mB.
\]

Determinantin transponointisäännön todistamiseksi toetaan ensinnäkin, että tämä pätee kaikille
permutaatiomatriiseille $\mI_p$. Nimittäin koska $\mI_p^T\mI_p=\mI$, niin jo todistetun tulon
determinanttisäännön perusteella on
\[
\det\mI_p^T\det\mI_p = 1.
\]
Koska tässä on $\det\mI_p=\pm 1$ ja $\det\mI_p^T=\det\mI_q=\pm 1$, niin päätellään, että on
oltava $\det\mI_p^T=\det\mI_p$. Kun huomioidaan tämä ja kirjoitetaan $\mI_p^T=\mI_q$, niin
laskukaavan \eqref{determinantin laskukaava 2} perusteella on
\[
\det\mA^T = \sum_p \det\mI_p \prod_{(i,j)\in\Lambda_p} a_{ji}
          = \sum_q \det\mI_q \prod_{(i,j)\in\Lambda_q} a_{ij} = \det\mA. \loppu
\]

Säännöstä $\det\mA^T=\det\mA$ voidaan päätellä, että determinantin vaihtösääntö ja nollasääntö
pätevät myös muodossa: $\det\mA$ vaihtuu vastaluvukseen, jos $\mA$:n kaksi riviä vaihdetaan 
keskenään, ja $\det\mA=0$, jos $\mA$:n kaksi riviä ovat samat.
  
Jos $\mA$ on säännöllinen matriisi, niin soveltamalla tulon determinanttisääntöä tuloon 
$\inv{\mA}\mA=\mI$ nähdään, että
\[
\boxed{\kehys\quad \det\inv{\mA}=\inv{(\det\mA)}. \quad}
\]
Nähdään myös, että pätee
\[
\mA \text{ säännöllinen } \impl \ \det\mA \neq 0,
\]
mikä on loogisesti sama kuin
\[
\det\mA=0 \ \impl \ \mA \text{ singulaarinen.}
\]
Tämä implikaatio pätee käänteiseenkin suuntaan, ja tässä onkin determinanttiteorian tärkein 
tulos sovellutusten kannalta. Itse tulos on suora seuraus edellisen luvun tulohajotelmista, 
tulon determinanttisäännöstä ja kolmiomatriisin determinanttisäännöstä, joka esitetään 
jäljempänä (Propositio \ref{kolmiomatriisin determinantti}).
\begin{Lause} \label{determinanttilause} \index{determinanttikriteeri|emph}
\vahv{(Determinanttikriteeri)} Neliömatriisille pätee
\[
\mA \text{ singulaarinen } \ekv \ \det\mA=0.
\]
\end{Lause}

\subsection*{Determinantin laskeminen}

Laskukaavaa \eqref{determinantin laskukaava} käytetään yleensä vain tapauksissa $n=2,3\,$:
\begin{align*}
&n=2: \qquad \begin{vmatrix}
a_{11} & a_{12} \\
a_{21} & a_{22} \\
\end{vmatrix}\ =\ a_{11}a_{22}-a_{21}a_{12}\,. \\[2mm]
&n=3: \qquad \begin{vmatrix}
a_{11} & a_{12} & a_{13} \\
a_{21} & a_{22} & a_{23} \\
a_{31} & a_{32} & a_{33}
\end{vmatrix}\
=\ \begin{array}{l} \\ a_{11}a_{22}a_{33}+a_{31}a_{12}a_{23}+a_{21}a_{32}a_{13} \\
                       -a_{31}a_{22}a_{13}-a_{11}a_{32}a_{23}-a_{21}a_{12}a_{33}\,.
\end{array}
\end{align*}

Nämä voi helposti johtaa myös suoraan  Määritelmästä \ref{determinantin määritelmä}. Esimerkiksi:
\begin{align*}
\begin{vmatrix} 
a_{11} & a_{12} \\ a_{21} & a_{22} 
\end{vmatrix}\ 
       &=\ V\left(\begin{bmatrix} a_{11}\\a_{21} \end{bmatrix}\,,\,
                  \begin{bmatrix} a_{12}\\a_{22} \end{bmatrix}\right) \\[2mm]
       &=\ V\left(a_{11}\begin{bmatrix} 1\\0 \end{bmatrix}
                 +a_{21}\begin{bmatrix} 0\\1 \end{bmatrix}\,,\
                  a_{12}\begin{bmatrix} 1\\0 \end{bmatrix}
                 +a_{22}\begin{bmatrix} 0\\1 \end{bmatrix}\right) \\[2mm]
       &=\ a_{11}a_{12}\begin{vmatrix} 1&1\\0&0 \end{vmatrix}
          +a_{11}a_{22}\begin{vmatrix} 1&0\\0&1 \end{vmatrix}
          +a_{21}a_{12}\begin{vmatrix} 0&1\\1&0 \end{vmatrix}
          +a_{21}a_{22}\begin{vmatrix} 0&0\\1&1 \end{vmatrix} \\[4mm]
       &=\ a_{11}a_{12}\cdot 0 + a_{11}a_{22}\cdot 1 + a_{21}a_{12}\cdot (-1) 
         + a_{21}a_{22}\cdot 0 
        =\ a_{11}a_{22}-a_{21}a_{12}\,.   
\end{align*} 
Tapauksessa $\,n=3\,$ voi laskukaavan muistaa ryhmittelemällä yhteenlaskettavat termit
kaavan \eqref{determinantin laskukaava 2} mukaisesti seuraavasti
\index{Sarrus'n sääntö}%
(nk. \kor{Sarrus'n sääntö})\,:
\[
\begin{aligned}
\text{etumerkki } + \qquad
\begin{vmatrix}
\bullet & \cdot & \cdot \\ 
\cdot & \bullet & \cdot \\
\cdot & \cdot & \bullet
\end{vmatrix} \quad
\begin{vmatrix}
\cdot & \bullet & \cdot \\ 
\cdot & \cdot & \bullet \\
\bullet & \cdot & \cdot
\end{vmatrix} \quad
\begin{vmatrix}
\cdot & \cdot & \bullet \\ 
\bullet & \cdot & \cdot  \\
\cdot & \bullet & \cdot
\end{vmatrix} \\[5mm]
\text{etumerkki } - \qquad
\begin{vmatrix}
\cdot & \cdot & \bullet \\ 
\cdot & \bullet & \cdot \\
\bullet & \cdot & \cdot
\end{vmatrix} \quad
\begin{vmatrix}
\bullet & \cdot & \cdot \\ 
\cdot & \cdot & \bullet \\
\cdot & \bullet & \cdot
\end{vmatrix} \quad
\begin{vmatrix}
\cdot & \bullet & \cdot \\ 
\bullet & \cdot & \cdot  \\
\cdot & \cdot & \bullet
\end{vmatrix} \\
\end{aligned}
\]
\begin{Exa}
\[
\begin{detmatrix} 1&1&3 \\ 2&0&2 \\ 3&-1&1 \end{detmatrix} \quad = 
         \quad \begin{array}{l} 
               \\ 1\cdot 0\cdot 1+1\cdot 2\cdot 3+2\cdot (-1)\cdot 3 \\
                  -3\cdot 0\cdot 3-1\cdot (-1)\cdot 2-2\cdot  1\cdot 1 \ =\ 0. 
               \end{array}
\]
Matriisina tämä on siis singulaarinen. \loppu
\end{Exa}
Sarrus'n sääntö rajoittuu determinantteihin kokoa $3 \times 3$. Hieman isompia determinantteja
käsivoimin purettaessa on seuraava nk. \kor{alideterminanttisääntö} laskukaavoja 
\eqref{determinantin laskukaava}--\eqref{determinantin laskukaava 2} huomattavasti
helppokäyttöisempi. Sääntö on yleispätevä, ja se on myös käypä determinanttifunktion
määritelmänä (ks.\ Harj.teht.\,\ref{H-m-5: alidet-tod}).
\begin{Lause} \label{alideterminanttisääntö} \index{alideterminanttisääntö|emph}
\vahv{(Alideterminanttisääntö)} Jos $\mA$ on neliömatriisi kokoa $n \times n$, niin jokaisella 
$k \in \{1, \ldots, n\}$ pätee
\begin{align*}
\det\mA &= \sum_{i=1}^n (-1)^{k+i} a_{ik} \det\mA^{(i,k)} \\
        &= \sum_{j=1}^n (-1)^{k+j} a_{kj} \det\mA^{(k,j)},
\end{align*}
missä $\mA^{(i,j)}$ on matriisi kokoa $(n-1) \times (n-1)$, joka saadaan poistamalla $\mA$:sta
$i$:s rivi ja $j$:s sarake.
\end{Lause}
\tod Tarkastellaan väitetyistä purkusäännöistä ensimmäistä, kun $k=1$. Determinantin
lineaarisuussäännön (i) perusteella voidaan kirjoittaa
\[
\det\mA = \sum_{i=1}^n a_{i1} \det\mA_i,
\]
missä $\mA_i$ on saatu matriisista $\mA$ korvaamalla ensimmäinen sarake yksikkövektorilla
$\me_i$. Vaihdetaan matriisissa $\mA_i$ ensin ensimmäinen ja $i$:s rivi keskenään ja 
permutoidaan tämän jälkeen rivit n:o $2 \ldots i$ siten, että $i$:s rivi siirtyy toiseksi
riviksi muiden rivien järjestyksen muuttumatta. Jälkimmäisessä operaatiossa tarvitaan $i-2$
parittaista rivin vaihtoa, joten vaihtoja kertyy kaikkiaan $i-1$ kpl. Jos vaihtojen jälkeen
saatavaa matriisia merkitään $\tilde{\mA}_i$, niin on siis 
$\,\det\mA_i=(-1)^{i-1}\det\tilde{\mA}_i$. Vaihtojen seurauksena on $\tilde{\mA}_i\,$:n 
ensimmäinen sarake $=\me_1$ ja $\tilde{\mA}_i^{(1,1)}=\mA^{(i,1)}$. Determinantin laskukaavasta
\eqref{determinantin laskukaava} nähdään tällöin, että 
$\det\tilde{\mA}_i=\det\tilde{\mA}_i^{(1,1)}=\det\mA^{(i,1)}$. Yhdistämällä päätelmät seuraa
\[ 
\det\mA = \sum_{i=1}^n (-1)^{i-1} a_{i1}\det\mA^{(i,1)} 
        = \sum_{i=1}^n (-1)^{i+1} a_{i1}\det\mA^{(i,1)}.
\]
Väitetyistä purkusäännöistä ensimmäinen on näin todistettu tapauksessa $k=1$. Muut väitetyt
säännöt palautuvat tähän tapaukseen vaihto- ja transponointisääntöjen avulla. \loppu

Determinantteja $\det\mA^{(i,j)}$ sanotaan determinantin $\det\mA$ \kor{alideterminanteiksi} 
--- siitä Lauseen \ref{alideterminanttisääntö} purkusääntöjen nimi.
\begin{Exa}
Kun valitaan $k=1$ ja kehitetään determinantti ensimmäisen sarakkeen mukaan, niin saadaan
\[
\begin{detmatrix}
1&2&-2&1 \\
2&-1&1&1 \\
0&1&0&1 \\
1&0&2&2
\end{detmatrix}=
1 \cdot \begin{detmatrix}
-1&1&1 \\
1&0&1 \\
0&2&2
\end{detmatrix}
-2 \cdot \begin{detmatrix}
2&-2&1 \\
1&0&1 \\
0&2&2
\end{detmatrix}
-1 \cdot \begin{detmatrix}
2&-2&1 \\
-1&1&1 \\
1&0&1
\end{detmatrix}.
\]
Alideterminantit voidaan kehittää Sarrus'n säännöllä tai alideterminanttisäännöllä. Sovelletaan 
jälkimmäistä siten, että kaksi ensimmäistä alideterminanttia kehitetään ensimmäisen sarakkeen 
mukaan $(k=1)$ ja viimeinen kolmannen rivin mukaan $(k=3)$:
\begin{align*}
&\begin{detmatrix} -1&1&1\\1&0&1\\0&2&2\end{detmatrix} 
   \,=\,(-1)\,\cdot \begin{detmatrix} 0&1\\2&2 \end{detmatrix}
          -1  \cdot \begin{detmatrix} 1&1\\2&2 \end{detmatrix}
   \,=\, (-1) \cdot (-2) -1 \cdot 0 \,\,= 2, \\[3mm]
&\begin{detmatrix} 2&-2&1\\1&0&1\\0&2&2 \end{detmatrix} 
   \,=\,   2 \cdot \begin{detmatrix} 0&1\\2&2 \end{detmatrix}
          -1 \cdot \begin{detmatrix} -2&1\\2&2 \end{detmatrix}
   \,=\, 2 \cdot (-2) - 1 \cdot (-6) \,=\, 2, \\[3mm]
&\begin{detmatrix} 2&-2&1\\-1&1&1\\1&0&1 \end{detmatrix} 
   \,=\,   1 \cdot \begin{detmatrix} -2&1\\1&1\end{detmatrix}
          +1 \cdot \begin{detmatrix} 2&-2\\-1&1\end{detmatrix}
   \,=\, 1 \cdot (-3) + 1 \cdot 0 \,=\, -3.
\end{align*}
Determinantin arvoksi tulee $1\cdot 2-2\cdot 2-1\cdot (-3)=1$. \loppu
\end{Exa}
Alideterminanttisäännöllä voidaan yleinen, kokoa $n\times n$ oleva determinantti purkaa lopulta
alideterminanteiksi kokoa $2\times 2$. Tarvittava laskuoperaatioiden määrä on kuitenkin
edelleen suuruusluokkaa $W \sim n!$ (ks.\ Harj.teht.\,\ref{H-m-5: alidet-työ}), joten suurilla
$n$:n arvoilla on tämäkin menetelmä kelvoton --- Miten siis ylipäänsä on mahdollista laskea
esimerkiksi determinantti kokoa $n=100$\,? Vastaukseen johtaa seuraava tulos, joka on helposti
todistettavissa alideterminanttisäännöllä.
\begin{Prop} \label{kolmiomatriisin determinantti}
Kolmiomatriisin $\mA=\{a_{ij}\}$ determinantti on matriisin lävistäjäalkioiden tulo:
\[
\det\mA=\prod_{i=1}^n a_{ii}.
\]
\end{Prop}
\tod Kun alideterminanttisäännössä valitaan $k=1$ ja kehitetään determinantti joko ensimmäisen
rivin (alakolmio) tai ensimmäisen sarakkeen (yläkolmio) mukaan, niin $a_{11}$ on ko.\ 
rivin/sarakkeen ainoa nollasta poikkeava alkio, joten Lauseen \ref{alideterminanttisääntö}
mukaan
\[
\det\mA=a_{11}\det\mA^{(1,1)}.
\]
Tässä $\mA^{(1,1)}$ on jälleen kolmiomatriisi, joten sama sääntö soveltuu yhä uudelleen ja 
johtaa väitettyyn lopputulokseen. \loppu

Propositioon \ref{kolmiomatriisin determinantti} perustuen voidaan determinantin arvo laskea
--- kuinkas muuten --- Gaussin algoritmilla. Nimittäin jos algoritmi menee läpi ilman 
tukioperaatioita (rivien/sarakkeiden vaihtoja), niin se johtaa $LU$-hajotelmaan
\[
\mA=\mL\mU,
\]
missä $[\mL]_{kk}=1$ ja $[\mU]_{kk}=a_{kk}^{(k-1)},\ k=1 \ldots n$ (vrt. edellinen luku). 
Tällöin on tulon determinanttisäännön ja Proposition \ref{kolmiomatriisin determinantti} mukaan
\[
\det\mA = \det\mL\,\det\mU = \prod_{k=1}^n a_{kk}^{(k-1)}.
\]
Determinantin arvo on siis yksinkertaisesti tukialkioiden tulo Gaussin algoritmin 
eliminaatiovaiheessa (!). Jos eliminaatiovaiheessa tehdään tukioperaatioita, tulee determinantin
arvoksi
\[
\det\mA=(-1)^m\prod_{k=1}^n[\mU]_{kk},
\]
missä m on tehtyjen rivien/sarakkeiden vaihtojen kokonaismäärä.

Gaussin algoritmia käyttäen tulee siis determinantin laskemisen työmääräksi 
$W \sim \frac{1}{3}n^3$ (kerto- ja yhteenlaskua) --- eli tämä on suurilla $n$:n arvoilla
huomattavasti tehokkaampi menetelmä kuin mikään edellä esitetyistä vaihtoehdoista. Puhtaasti
numeerisissa laskuissa, ja suurilla $n$, determinantilla ei olekaan juuri käytännön merkitystä
--- selvittäähän Gaussin algoritmi matriisin säännöllisyyskysymyksen muutenkin. Determinantin
käsite on kuitenkin käyttökelpoinen erinäisissä lineaarialgebran teoreettisissa tarkasteluissa,
ja myös symbolisessa laskennassa determinantilla on käyttöä. Symbolisen laskennan ongelma voi
olla esimerkiksi sellainen, että tarkasteltava matriisi riippuu jostakin parametrista
(muuttujasta) $s$, ts.\ $\mA=(a_{ij}(s))=\mA(s)$, ja halutaan vaikkapa selvittää, millä $s$:n
arvoilla $\mA(s)$ on singulaarinen. Jos $\mA$ ei ole kooltaan kovin suuri, niin 
determinanttiehto
\[
\det\mA(s) = 0
\]
on ratkaisun lähtökohtana luonnollinen ja (etenkin käsinlaskussa) usein käytetty.

\subsection*{Cramerin sääntö}

Determinantin muista käyttömuodoista on syytä vielä mainita (ilman todistusta, ks.\ 
Harj.teht.\,\ref{H-m-5: Cramer}) seuraavat kaksi laskusääntöä.

\begin{Lause} (\vahv{Cramerin\footnote[2]{Sveitsiläinen matemaatikko \hist{Gabriel Cramer}
(1704--1752) oli determinanttiteorian uran\-uurtajia teoksellaan ``Introduction
$\grave{\text{a}}$ l'analyse des lignes courbes alg$\acute{\text{e}}$brigues'' (1750).
\index{Cramer, G.|av}} sääntö}) \label{Cramerin sääntö} \index{Cramerin sääntö|emph}
Jos $\mA$ on säännöllinen neliömatriisi kokoa $n \times n$, niin yhtälöryhmän $\mA \mx = \mb$
ratkaisu on
\[
\mx=(x_i), \quad x_i=\frac{\det\mA^{(i)}}{\det\mA}\,, \quad i=1\,\ldots\,n,
\]
missä $\mA^{(i)}$ saadaan $\mA$:sta korvaamalla $i$:s sarake $\mb$:llä.
\end{Lause}
\begin{Lause} \label{käänteismatriisin determinanttikaava} Säännöllisen matriisin $\mA$ 
käänteismatriisi on 
\[
\mA^{-1}=\mB^T, \quad [\mB]_{ij}= (-1)^{i+j}\,\frac{\det\mA^{(ij)}}{\det\mA}\,.
\]
\end{Lause}
Cramerin säännöllä on pienikokoisia yhtälöryhmiä ratkaistaessa edelleen jonkin verran käyttöä 
käsinlaskussa. Etenkin jos kerroinmatriisin alkiot ovat kokonaislukuja, voidaan säännöllä 
minimoida (käsinlaskussa vaivalloiset) jakolaskut. Myös symbolisessa (käsin)laskennassa
\mbox{Cramerin} säännöllä on käyttöä samaan tapaan kuin determinantilla yleensä. 
--- Numeerisessa matriisilaskennassa sen sijaan \linebreak Cramerin säännöllä on kyseenalainen
kunnia esiintyä 'maailman huonoimpana' lineaarisen yhtälöryhmän ratkaisualgoritmina.

\Harj
\begin{enumerate}

\item \label{H-m-5: ortogonaalisen matriisin determinantti}
Näytä, että ortogonaalisen matriisin determinantilla on vain kaksi mahdollista arvoa:
Joko $\det\mA=1$ tai $\det\mA=-1$.

\item
Laske Sarrus'n säännöllä:
\[
\text{a)}\ \ \begin{detmatrix} 2&3&-1\\-1&2&0\\1&4&3 \end{detmatrix} \qquad
\text{b)}\ \ \begin{detmatrix} 0&-2&-3\\1&0&-2\\3&4&0 \end{detmatrix} \qquad
\text{c)}\ \ \begin{detmatrix} -3&7&2\\-5&4&0\\9&-1&-6 \end{detmatrix}
\]

\item
Olkoon $\alpha,\beta,\gamma\in\R$. Laske
\[
\text{a)}\ \ \begin{detmatrix}
             1&\alpha&\alpha^2\\1&\beta&\beta^2\\1&\gamma&\gamma^2
             \end{detmatrix} \qquad
\text{b)}\ \ \left| \begin{array}{lll}
             \alpha+\beta&\alpha+2\beta&\alpha+3\beta \\
             \alpha+3\beta&\alpha+\beta&\alpha+2\beta \\
             \alpha+2\beta&\alpha+3\beta&\alpha+\beta
             \end{array} \right| \qquad
\text{c)}\ \ \left| \begin{array}{lll}
             2&\alpha&\alpha^2 \\ 1&\alpha^2&\alpha \\ \alpha&\alpha^3&1
             \end{array} \right|
\]

\item
Laske a) alideterminanttikehitelmällä, b) ja c) Gaussin algortimilla, d) molemmilla:
\begin{align*}
&\text{a)}\ \ \begin{detmatrix} 1&-1&0&2\\2&1&0&0\\1&1&2&2\\0&0&1&1 \end{detmatrix} \qquad\,\
 \text{b)}\ \ \begin{detmatrix} 5&4&2&1\\2&3&1&-2\\-5&-7&-3&9\\1&-2&-1&4 \end{detmatrix} \\[3mm]
&\text{c)}\ \ \begin{detmatrix}
              2&1&0&0&4\\0&1&1&0&0\\0&0&1&1&0\\0&0&0&1&1\\3&0&0&0&5 
              \end{detmatrix} \qquad
 \text{d)}\ \ \begin{detmatrix} 1&1&4&5\\1&1&5&4\\2&4&1&1\\4&2&1&1 \end{detmatrix}
\end{align*}

\item
Laske $\det\mB$, kun
$\displaystyle{\
\mB=(-5\mA\mA^T)^7 \quad\text{ja} \quad  
\mA=\begin{rmatrix} 13&8&6\\-13&-8&-4\\8&5&5 \end{rmatrix}.
}$

\item
Määritä kaikki reaaliset tai kompleksiset $\lambda$:n arvot, joilla seuraavat matriisit ovat
singulaarisia.
\begin{align*}
&\text{a)}\ \ \begin{bmatrix} 2-\lambda&-4 \\ 2&6-\lambda \end{bmatrix} \qquad
 \text{b)}\ \ \begin{bmatrix} 
              \lambda-2&3&1 \\ \lambda-4&3&2 \\ \lambda-6&\lambda&3 
              \end{bmatrix} \\[3mm]
&\text{c)}\ \ \begin{bmatrix}
              1&3-\lambda&4 \\ 4-\lambda&2&-1 \\ 1&\lambda-6&2
              \end{bmatrix} \qquad
 \text{d)}\ \ \begin{bmatrix}
              1-\lambda&-1&2 \\ 1&2-\lambda&-13 \\ -2&1&1+\lambda
              \end{bmatrix}
\end{align*}

\item
Olkoon $\ma_1, \ldots, \ma_n\in\R^n$, ja määritellään
\[
\mb_k=\ma_k, \quad \mb_j=\ma_j+\beta_j\ma_k, \quad j=1 \ldots n,\ j \neq k,
\]
missä $1 \le k \le n$ ja $\beta_j\in\R$. Näytä determinanttifunktion $V$ ominaisuuksien 
perusteella, että pätee $\,V(\mb_1, \ldots, \mb_n)=V(\ma_1, \ldots, \ma_n)$. Miten tämä tulos 
liittyy Gaussin algoritmiin sovellettuna matriisiin $\mA=[\ma_1 \ldots \ma_n]^T\,$?

\item \index{neliömatriisi!f@tridiagonaalinen} \index{tridiagonaalinen matriisi}
Neliömatriisi $\mA=(a_{ij})$ olkoon kokoa $n \times n$ ja \kor{tridiagonaalinen}, ts.\
$a_{ij}=0$, kun $\abs{i-j} \ge 2$. Edelleen olkoon $a_{ii}=1,\ i=1 \ldots n$, ja 
$a_{ij}=\lambda$, kun $\abs{i-j}=1$. \ a) Näytä, että determinantille $D_n=\det\mA$ pätee
palautuskaava $D_n=D_{n-1}-\lambda^2 D_{n-2}$. \ b) Laske $D_n,\ n=2 \ldots 10$, kun
$\lambda=2$. c) Jos $\lambda=1$, niin millä $n$:n arvoilla $\mA$ on singulaarinen?

\item (*) \label{H-m-5: alidet-työ}
Näytä, että jos determinantti kokoa $n \ge 3$ lasketaan alideterminanttisäännöllä, niin
tarvittava kertolaskujen lukumäärä on
\[
W_n = n!\sum_{k=1}^{n-1} \frac{1}{k!}\,.
\]

\item (*) \label{H-m-5: alidet-tod}
Määritellään determinanttifunktio $V(\ma_1,\ldots,\ma_n)=\det[\ma_1 \ldots \ma_n]$ palautuvasti
käyttäen Lauseen \ref{alideterminanttisääntö} ensimmäistä purkusääntöä ($k=1$) sekä sääntöä
$\,\det a=a\,$ determinantille kokoa $n=1$. \vspace{1mm}\newline
a) Näytä induktiolla, että mainitulla tavalla määritellyllä funktiolla on determinanttifunktion
ominaisuudet (i)--(iii). \vspace{1mm}\newline
b) Näytä, että jos $p=(i_1,\ldots,i_n)$ on joukon $(1,\ldots,n)$ permutaatio ja
$\mI_p=[\me_{i_1} \ldots \me_{i_n}]$, niin a-kohdan määritelmän mukaisesti on joko
$\det\mI_p=1$ tai $\det\mI_p=-1$. Päättele, että edellisessä tapauksessa permutaatio $p$ on
parillinen (eli saavutettavissa parillisella määrällä parivaihtoja) ja jälkimmäisessä pariton
--- siis jokainen permutaatio on jompaa kumpaa tyyppiä.

\item (*) \label{H-m-5: Cramer}
Valitsemalla $\mb=b_j\me_j,\ j=1 \ldots n\,$ johda Cramerin sääntö alideterminanttisäännöstä. 
Todista edelleen Lause \ref{käänteismatriisin determinanttikaava} lähtien Cramerin säännöstä.

\end{enumerate} 