\chapter{Matriisit} \label{matriisit}

\begin{quote}
''Jos yhtälöryhmässä on yli $50$ tuntematonta, on ratkaisemisessa syytä käyttää
tietokonetta''.\footnote[2]{Lentokoneen siiven lujuuslaskentaan perehtyneen insinöörin toteamus
alan julkaisussa 1950-luvulla.}
\end{quote}

Monilla insinöörialoilla on ollut pitkä \kor{lineaaristen yhtälöryhmien} ratkaisemisen perinne
jo ennen tietokoneiden aikaa. Myös lineaaristen yhtälöryhmien matematiikassa,
\index{lineaarialgebra}%
\kor{lineaarialgebrassa}, perinteet ovat pitkät. Ne ulottuvat 1700-luvulle, jolloin kehitettiin
\kor{determinantin} käsitteeseen perustuva yhtälöryhmien ratkeavuusteoria ja ratkaisukaavat.
Determinanteilla on vielä nykyäänkin käyttöä teoreettisissa tarkasteluissa, erinäisissä
laskukaavoissa (vrt.\ Luku \ref{ristitulo}) ja yleisemminkin käsinlaskussa silloin, kun
yhtälöryhmän koko on pieni. Muuten determinanttioppia on pidettävä vanhentuneena, syystä että
tämän opin mukaiset laskukaavat eivät sovellu laajamittaiseen numeeriseen käyttöön.
Tietokoneiden aikakaudella lineaarialgebra onkin pitkälti irtaantunut determinanttiperinteestä
ja vanhoista käsinlaskun menetelmistä. 

Tässä luvussa tutkimuksen kohteena on lineaarialgebran avainkäsite, \kor{matriisi}.
Matriiseilla laskemista eli \kor{matriisialgebraa} tarkastellaan ensin Luvussa 
\ref{matriisialgebra}. Tämän jälkeen Luvuissa \ref{inverssi}--\ref{tuettu Gauss} tutkitaan 
lineaarisia yhtälöryhmiä, niiden ratkaisemista \kor{Gaussin algoritmilla} ja ratkaisemiseen 
keskeisesti liittyvää \kor{käänteismatriisin} käsitettä. Luvussa \ref{determinantti} käydään
lyhyesti läpi vanha determinanttioppi laskukaavoineen. Luvuissa 
\ref{lineaarikuvaukset}---\ref{affiinikuvaukset} ovat tutkimuskohteena matriisialgebraan 
perustuvat \kor{lineaarikuvaukset} ja \kor{affiinikuvaukset} sekä näiden geometriset 
sovellutukset, \kor{geometriset kuvaukset}. Viimeisessä osaluvussa esitellään joitakin 
lineaaristen yhtälöryhmien sovellusesimerkkejä perinteisestä insinöörimatematiikasta.
