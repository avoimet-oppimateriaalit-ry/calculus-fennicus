\section{Pintaintegraalit} \label{pintaintegraalit}
\alku
\index{pintaintegraali|vahv}


\kor{Pintaintegraalilla} tarkoitetaan integraalia muotoa
\[
I(f,A,\mu)=\int_A f\,d\mu,
\]

missä $A$ on pinnan $S \subset \R^ 3$ osa ja $\mu$ 
\index{pinta-alamitta!b@kaarevan pinnan}
\kor{pinta-alamitta pinnalla} $S$.
Jatkossa käytetään kaarevan pinnan pinta-alamitan yhteydessä merkintää $d\mu=dS$. Tämän mitan
suhde tason (Jordanin) pinta-alamittaan on samantyyppinen kuin kaarenpituusmitan suhde $\R$:n
pituusmittaan. Pintaintegraali on näin ajatellen tasointegraalin yleistys, ja etenkin
sovelluksissa termiä saatetaan käyttää tasointegraaleistakin puhuttaessa
(vrt.\ Luku \ref{pinta- ja tilavuusintegraalit}).

Kuten käyrän kaarenpituusmitta, voidaan myös kaarevan pinnan pinta-alamitta määritellä pintaa
'oikaisevien' approksimaatioiden kautta. Luonteva menettely on approksimoida pintaa 
\pain{kolmion} \pain{muotoisilla} \pain{taso}p\pain{innoilla}. Kun kolmiot sijoitetaan verkoksi,
jonka solmupisteet ovat pinnalla, niin verkkoa tihennettäessä tulee kolmioiden yhteenlasketun
pinta-alan lähestyä $A$:n pinta-alamittaa $\mu(A)$. Jos tämä sovitaan mitallisuuden 
määritelmäksi, niin pinnan kolmioapproksimaatioita voi käyttää pinta-alan numeeriseen 
laskemiseen 'suoraan määritelmästä'. Tällaisilla approksimaatioilla on muutakin käyttöä, 
esimerkiksi esitettäessä pintoja graafisesti (tietokonegrafiikka) tai jopa konstruoitaessa 
todellisia pintoja (esim.\ kaarevat peilipinnat).
\begin{figure}[H]
\begin{center}
\import{kuvat/}{kuvaUint-37.pstex_t}
\end{center}
\end{figure}
Suoran 'numeronmurskauksen' vaihtoehtona pinta-alaa ja pintaintegraaleja on jälleen mahdollista
käsitellä myös differentiaalilaskennan keinoin, jolloin pintaintegraalille saadaan laskukaava
tasointegraalina. Fubinin lauseen avulla tämä palautuu edelleen yksiulotteisiksi integraaleiksi,
jotka suotuisissa (käytännössä kylläkin harvinaisissa) oloissa voidaan laskea suljetussa
muodossa. Pintaintegraalin laskukaavaa johdettaessa on luonnollinen lähtökohta pinnan 
\index{parametrinen pinta}%
parametrisointi eli esittäminen \kor{parametrisena pintana}
(vrt.\ Luku \ref{parametriset käyrät}). Jos parametrisointi esitetään vektorimuodossa 
\[
\vec r = \vec r\,(u,v) = x(u,v)\vec i + y(u,v)\vec j + z(u,v)\vec k
\]
ja oletetaan, että $(x(u,v),y(u,v),z(u,v)) \in A\ \ekv\ (u,v) \in B \subset \R^2$, niin 
pintaintegraali on muunnettavissa muotoon
\[
\int_A f\,dS=\int_B g\,d\mu', \quad g(u,v)=f(x(u,v),y(u,v),z(u,v)),
\]
missä $\mu'$ on pinnan $S$ pinta-alamitan vastine parametritasossa. Muunnos parametritasoon on
jälleen muotoa
\[
d\mu'=J\,dudv,
\]
missä $J=J(u,v)$ on (yleensä laskettavissa oleva) muuntosuhde. Kun muuntosuhde tunnetaan, on
laskukaava valmis:
\[
\int_A f\,dS=\int_B gJ\,dudv.
\]

Muuntosuhteen laskemiseksi tutkitaan suorakulmion 
\[
\Delta B=[u_0,u_0+\Delta u]\times [v_0,v_0+\Delta v] \subset B
\]
muuntumista. Oletetaan funktiot $x(u,v), y(u,v), z(u,v)$ riittävän säännöllisiksi niin, että 
voidaan käyttää linearisoivaa approksimaatiota
\[
\vec r\,(u,v) \approx \vec r\,(u_0,v_0)+(u-u_0)\frac{\partial\vec r}{\partial u}(u_0,v_0)
                                       +(v-v_0)\frac{\partial\vec r}{\partial v}(u_0,v_0).
\]
Tämän mukaisesti $\Delta B$ kuvautuu pintaa pisteessä $P_0\vastaa\vec r\,(u_0,v_0)$ sivuavalle
tangenttitasolle. Kuva on tämän tason suunnikas, jonka virittävät vektorit
\[
\vec v_1=\Delta u\,\partial_u\vec r,\quad \vec v_2=\Delta v\,\partial_v\vec r.
\]
Suunnikaan pinta-ala on 
$|\vec v_1\times\vec v_2|=|\partial_u\vec r\times\partial_v\vec r\,||\Delta u||\Delta v|$,
joten päätellään, että muuntosuhde on\footnote[2]{Muuntosuhteen laskukaava on tarkemmin
perusteltavissa olettaen, että funktioiden $x(u,v), y(u,v), z(u,v)$ osittaisderivaatat ovat
jatkuvia perussuorakulmiossa $T \supset B$,}
\[
J=\left|\frac{\partial\vec r}{\partial u}\times\frac{\partial\vec r}{\partial v}\right|.
\]
Jos pinta on annettu muodossa $z=f(x,y)$, eli
$\,\vec r=\vec r\,(x,y)=x\vec i+y\vec j+f(x,y)\vec k$, niin
\begin{align*}
\partial_x\vec r\times\partial_y\vec r 
               &= \begin{vmatrix} 
                  \vec i & \vec j & \vec k \\ 1 & 0 & f_x \\ 0 & 1 & f_y 
                  \end{vmatrix} 
                = -f_x\vec i-f_y\vec j+\vec k \\
\impl \ J(x,y) &= \sqrt{1+[f_x(x,y)]^2+[f_y(x,y)]^2}.
\end{align*}
Koska tässä $\vec n=(-f_x\vec i-f_y\vec j+\vec k\,)/J=n_x\vec i+n_y\vec j+n_z\vec k$ on pinnan
yksikkönormaalivektori, niin tuloksen voi esittää myös muodossa $J(x,y)=1/\abs{n_z(x,y)}$.

$R$-säteisellä pallopinnalla on
\begin{align*}
\vec r = \vec r\,(\theta,\varphi)
      &= R\sin\theta\cos\varphi\,\vec i+R\sin\theta\sin\varphi\,\vec j+R\cos\theta\,\vec k \\
\impl\ J(\theta,\varphi) 
      &= \abs{\partial_\theta\vec r\times\partial_\varphi\vec r\,}
       = R^2\sin\theta.
\end{align*}
Yleisemmällä pyörähdyspinnalla, joka syntyy käyrän $K:\,y=f(x),\ x\in[a,b]$ pyörähtäessä
$x$-akselin tai käyrän $K:\,y=f(z),\ z\in[a,b]$ $z$-akselin ympäri (ks.\ kuviot alla) voidaan
muuntosuhde laskea vastaavaan tapaan
(Harj.teht.\,\ref{H-Uint-9: pyörähdyspintojen muuntosuhteet}). Tulokset koottuna:
\index{muuntosuhde integraalissa!e@pintaintegraalissa}
\index{kzyyrzy@käyräviivaiset koordinaatistot!d@--integraalit}%
\vspace{2mm}
\begin{center}
\begin{tabular}{lll}
Pinta & Muuttujat & $\quad$Muuntosuhde \\ \hline \\
$\vec r=\vec r\,(u,v)$ & $u,v$ & $\quad \abs{\partial_u\vec r \times \partial_v\vec r\,}$ \\ \\
$z=f(x,y)$ & $x,y$ & $\quad \sqrt{1+[f_x(x,y)]^2+[f_y(x,y)]^2}$ \\ \\
pallopinta $\,r=R$& $\theta,\varphi$ & $\quad R^2\sin\theta$ \\ \\
\parbox{3.5cm}{pyörähdyspinta: 
$y=f(x) \hookrightarrow\ \curvearrowleft \negthickspace\negthickspace \longrightarrow_x$} & 
$x,\varphi$ & $\quad |f(x)|\sqrt{1+[f'(x)]^2}$ \\ \\
\parbox{3.5cm}{pyörähdyspinta: 
$z=f(r) \hookrightarrow$\ \raisebox{-0.2cm}{$\circlearrowleft$}$\negthinspace \negthinspace 
\negmedspace \negthinspace \uparrow^z$} & $r,\varphi$ & $\quad r\sqrt{1+[f'(r)]^2}$

\end{tabular}
\end{center}
\begin{multicols}{2}
\begin{figure}[H]
\vspace{1cm}
\begin{center}
\import{kuvat/}{kuvaDD-2.pstex_t}
\end{center}
\end{figure}
\begin{figure}[H]
\begin{center}
\import{kuvat/}{kuvaDD-4.pstex_t}
\end{center}
\end{figure}
\end{multicols}
\begin{Exa}
$R$-säteiselle pallopinnalle on jakautunut tasaisesti massa $m$. Laske hitausmomentti 
$z$-akselin suhteen.
\end{Exa}
\ratk
\begin{align*}
I &= \int_A \rho_0(x^2+y^2)\,dS \\
&= \int_B \rho_0 R^2\sin^2\theta\cdot R^2\sin\theta\,d\theta d\varphi 
   \qquad \left(\,B=[0,\pi]\times[0,2\pi]\,\right) \\
&= \rho_0 R^4 \int_0^{2\pi} d\varphi \cdot \int_0^\pi \sin^3 \theta\,d\theta \\
&= 2\pi\rho_0R^4\sijoitus{\theta=0}{\theta=\pi} (-\cos\theta+\frac{1}{3}\cos^3\theta) \\
&= \frac{8\pi}{3}\rho_0 R^4.
\end{align*}
Koska
\begin{align*}
m =\int_A \rho_0\,dS
 &= \rho_0 R^2 \int_0^{2\pi} d\varphi \cdot \int_0^{\pi} \sin\theta\,d\theta \\
 &= \rho_0\cdot 4\pi R^2,
\end{align*}
niin $\displaystyle{I=\underline{\underline{\frac{2}{3}mR^2}}}$. \loppu
\begin{Exa}
Laske $\int_A f\,dS$, kun $f(x,y,z)=xy\ $ ja
\[
A = \left\{(x,y,z)\in\R^3 \mid z=x^2+\frac{1}{2}y^2 \; \ja \; (x,y)\in B\right\},
    \quad B=[0,1]\times [0,1].
\]
\end{Exa}
\ratk
\begin{align*}
\int_A f\,dS &= \int_B xy\sqrt{1+4x^2+y^2}\,dxdy \\
&=\int_0^1 x\left(\int_0^1 y\sqrt{1+4x^2+y^2}\,dy\right)dx \\
&=\int_0^1 x\left[\sijoitus{y=0}{y=1}\frac{1}{3}(1+4x^2+y^2)^{3/2}\right]dx \\
&=\int_0^1 \left[\frac{1}{3}x(2+4x^2)^{3/2}-\frac{1}{3}x(1+4x^2)^{3/2}\right]\,dx
\end{align*}
\begin{align*}
\quad &=\ \sijoitus{0}{1}\left[\frac{1}{60}(2+4x^2)^{5/2}-\frac{1}{60}(1+4x^2)^{5/2}\right] \\
      &= \underline{\underline{\frac{1}{60}\bigl(36\sqrt{6}-25\sqrt{5}-4\sqrt{2}+1\bigr)}}.
\loppu \\
\end{align*}

\begin{multicols}{2} \raggedcolumns
\begin{Exa}
Katkaistun suoran ympyräkartion korkeus $=h$, pohjan säde $=R$ ja katkaisukohdassa säde $=r$.
Laske vaipan ala $\mu(A)$.
\end{Exa}
\begin{figure}[H]
\begin{center}
\import{kuvat/}{kuvaUint-38.pstex_t}
\end{center}
\end{figure}
\end{multicols}
\ratk Tässä on kyseessä pyörähdyspinta, joka syntyy, kun jana
\[
K:\,y=f(x)=r+(R-r)\frac{x}{h}, \quad x\in [0,h]
\]
pyörähtää $x$-akselin ympäri. Parametreilla $(x,\varphi) \in B=[0,1]\times[0,2\pi]$ laskien
saadaan
\begin{align*}
\mu(A) &= \int_B f(x)\sqrt{1+[f'(x)]^2}\,dxd\varphi \\
&= 2\pi \int_0^h\sqrt{1+\left(\frac{R-r}{h}\right)^2}\,\left[\,r+(R-r)\frac{x}{h}\,\right]dx \\
&=\pi(R+r)h\,\sqrt{1+\left(\frac{R-r}{h}\right)^2} \\[3mm]
&=\underline{\underline{\pi(R+r)\sqrt{h^2+(R-r)^2}}}. \Akehys \loppu
\end{align*}

\begin{Exa}
Laske viivoitinpinnan
\[
\vec r\,(u,v)=v[\,\cos (\omega u)\vec i+\sin (\omega u)\vec j\,]+u\vec k, \quad
(u,v)\in [0,H]\times [0,L]
\]
pinta-ala.
\end{Exa}
\begin{multicols}{2} \raggedcolumns
\ratk
\begin{align*}
\partial_u\vec r\times\partial_v\vec r
          &= \begin{detmatrix} 
             \vec i & \vec j & \vec k \\
             -\omega v\sin (\omega u) & \omega v\cos (\omega u) & 1 \\
             \cos (\omega u) & \sin (\omega u) & 0 
             \end{detmatrix} \\
          &= -\sin (\omega u)\vec i+\cos (\omega u)\vec j -\omega v\vec k \\[2mm]
\impl \ J &= \abs{\partial_u\vec r\times\partial_v\vec r\,}=\sqrt{1+\omega^2 v^2}.
\end{align*}
\begin{figure}[H]
\begin{center}
\import{kuvat/}{kuvaUint-39.pstex_t}
\end{center}
\end{figure}
\end{multicols}

\begin{align*}
\mu(A) &= \int_0^H\left(\int_0^L \sqrt{1+\omega^2 v^2}\,dv\right)du \\
&= H\int_0^L \sqrt{1+\omega^2 v^2}\,dv \\
&\qquad [\ \text{sijoitus}\ \ \omega v=\sinh t,\quad dv=\inv{\omega}\cosh t\,dt,\quad 
                              \alpha=\text{arsinh}\, (\omega L)\ ] \\
&=H\inv{\omega}\int_0^\alpha \cosh^2 t\,dt \\
&=H\inv{\omega}\sijoitus{0}{\alpha} \left(\frac{1}{2}\sinh t\cosh t+\frac{1}{2}\,t\right) \\
&=\underline{\underline{
    \frac{1}{2}\left[\ln (\omega L +\sqrt{\omega^2 L^2+1})
                                   +\omega L\sqrt{\omega^2 L^2+1}\right]H\inv{\omega}}}. \loppu
\end{align*}

\subsection*{Avaruuskulma}
\index{avaruuskulma|vahv}

Olkoon $V\subset\R^3$ avaruuden osajoukko, käytännössä esim.\ kiinteä kappale, pinta tai pinnan
osa. Määritellään pallokoordinaatistossa joukko $B\subset[0,\pi]\times[0,2\pi]$ ehdolla
\[
B=\{\,(\theta,\varphi)\ | \ P=(r,\theta,\varphi) \in V\ \text{jollakin}\ r>0\,\}.
\]
Tällöin $V$ näkyy origosta suuntaan $\vec e_r(\theta,\varphi)$ täsmälleen kun
$(\theta,\varphi) \in B$. Olkoon $A$ yksikköpallon $S$ osa, joka vastaa
pallonpintakoordinaattien $(\theta,\varphi)$ joukkoa $B$. Tällöin sanotaan, että $V$ näkyy
origosta \kor{avaruuskulmassa} (engl.\ solid angle)
\[
\Omega=\int_A dS=\int_B \sin\theta\,d\theta d\varphi.
\]
Tämän suurin arvo (kun $A=S$) on $4\pi$.
\begin{Exa}
Missä avaruuskulmassa lieriöpinta
\[
A=\{(x,y,z)\in\R^2 \; | \; x^2+y^2=R^2, \; 0\leq z\leq H\}
\]
näkyy origosta?
\end{Exa}
\ratk Kyseisellä pinnalla pallonpintakoordinaatit saavat arvot
\[
(\theta,\varphi) \in B=[\theta_0,\pi/2]\times[0,2\pi],\quad 
                 \cos\theta_0=\frac{R}{\sqrt{R^2+H^2}}\,,
\]
joten
\[
\Omega = \int_B \sin\theta\,d\theta d\varphi
       = \int_0^{2\pi} d\varphi\cdot\int_{\theta_0}^{\pi/2} \sin\theta\,d\theta
       = 2\pi\cos\theta_0
       =\underline{\underline{2\pi\,\frac{H}{\sqrt{R^2+H^2}}}}\,. \loppu
\]

\Harj
\begin{enumerate}

\item \label{H-Uint-9: pyörähdyspintojen muuntosuhteet} 
Halutaan laskea pinta-alamitan muuntosuhde $J$ pyörähdyspinnalle, joka syntyy, kun
a) $xy$-tason käyrä $K: y=f(x)$ pyörähtää $x$-akselin ympäri, b) $yz$-tason käyrä
$K: y=f(z)$ pyörähtää $z$-akselin ympäri. Laske muuntosuhde
parametrisaatioille \vspace{1mm}\newline
a) \ $y=f(x)\cos\varphi,\,\ z=f(x)\sin\varphi$, \newline 
b) \ $x=r\cos\varphi,\,\ y=r\sin\varphi,\,\ z=f(r)$.

\item
Laske sen pinnan ala, jonka lieriö $\,L:\, x^2+y^2=a^2$ ($a>0$) leikkaa \vspace{1mm}\newline
a) kartiosta $\,z^2=x^2+y^2$, \newline
b) satulapinnasta $\,az=xy$, \newline
c) paraboloidista $\,az=x^2+y^2$. 

\item
Laske pinta-ala $\mu(A)$, kun $A$ on \vspace{1mm}\newline
a) joukon $V=\{(x,y,z)\in\R^3 \mid x^2+y^2 \le z \le \sqrt{x^2+y^2}\,\}$ 
   reunapinta $\partial V$, \newline
b) parametrinen pinta $\,x=8u^2,\ y=v^2,\ z=4uv,\ (u,v)\in[0,1]\times[0,3]$.

\item
a) Laske puolipallon (pinnan) keskiö. \vspace{1mm}\newline
b) Pinnalle $S:\,x^2+y^2=R^2,\ 0 \le z \le H$ on jakautunut tasaisesti massa $m$. Laske
lieriökoordinaateilla hitausmomentit koordinaattiakselien suhteen.

\item \index{Guldinin sääntö}
Olkoon $f(x) \ge 0$, kun $x\in[a,b]$ ja $S$ pinta, joka syntyy, kun käyrä
$K:\,y=f(x)\ \ja\ x\in[a,b]$ pyörähtää $x$-akselin ympäri. Todista \kor{Guldinin sääntö}:
$S$:n pinta-ala = $K$:n pituus kertaa $K$:n keskiön pyörähdyksessä kulkema matka.

\item
Laske, missä avaruuskulmassa kohde näkyy origosta: \vspace{1mm}\newline
a) \ $R$-säteinen kiekko tasolla $z=a>0$, keskipiste $z$-akselilla. \newline
b) \ $R$-säteinen kuula, jonka keskipisteen etäisyys origosta $=a \ge R$. \newline
c) \ Puolikartion rajaama joukko $A:\,z\ge\sqrt{x^2+y^2}+a,\ a \ge 0$. \newline
d) \ Pararaboloidi $\,z=x^2+y^2+a, \ a\in\R$.

\item (*) \index{Vivian'in ikkuna} 
\kor{Vivian'in ikkunaksi} sanotaan sitä pintaa $A,$ jonka lieriö $x^2+y^2=Rx$ erottaa pallosta
$x^2 + y^2 + z^2 = R^2$. Laske ko.\ pinnan ala.

\item (*)
Olkoon $A\subset\R^3$ lieriöiden
\[
S_1:\ x^2+y^2=R^2, \quad S_2:\ y^2+z^2=R^2, \quad S_3:\ x^2+z^2=R^2
\]
sisään jäävä joukko. Laske $A$:n reunapinnan $\partial A$ pinta-ala.

\item (*) \index{zzb@\nim!N:s ydinvoimala} 
(N:s ydinvoimala) Voimalaitoksen jäähdytystornin vaipan ulkopinnan parametriesitys on
(vrt.\ Esimerkki \ref{parametriset käyrät}:\,\ref{jäähdytystorni})
\[
\begin{cases}
\,x=a[(2-2v)\cos u-v\sin u], \\
\,y=a[v\cos u +(2-2v)\sin u], \\
\,z=3av,
\end{cases}
\]
missä $a$=50 m ja $(u,v)\in B= [0,2\pi]\times [0,1]$. Vaippa on valmistettu betonista ja sen
paksuus on 20 cm. Laske tarvittavan betonin määrälle likiarvo käyttäen pintaintegraalia.

\item (*)
a) Levy
\[
A=\{\,(x,y,z) \ | \ \frac{x^2}{a^2}+\frac{y^2}{b^2} \le 1,\ z=c\,\} \quad (a,b,c>0)
\]
näkyy origosta avaruuskulmassa $\Omega$. Johda $\Omega$:lle laskukaava muotoa
$\Omega=\int_0^{2\pi} f(\varphi)\,d\varphi$. \ b) Laske, missä avaruuskulmassa avaruusneliö
\[
K=\{\,(x,y,z) \ | \ 0 \le x,y \le 1,\ z=1\,\}
\]
näkyy origosta.

\end{enumerate}