\section[Taso- ja avaruusintegraalien muuntaminen]{Taso- ja avaruusintegraalien \\ 
muuntaminen} 
\label{muuttujan vaihto integraaleissa}
\sectionmark{Integraalien muuntaminen}
\alku
\index{muuttujan vaihto (sijoitus)!b@integraalissa|vahv}

Olkoon $A,B\subset\R^n$, $n\in\N$ ja olkoon $\mpu:B \kohti A$ bijektio (tai 'melkein bijektio',
ks.\ huomautukset jäljempänä).
\begin{figure}[H]
\begin{center}
\import{kuvat/}{kuvaUint-26.pstex_t}
\end{center}
\end{figure}
Halutaan laskea integraali $\int_A f\,d\mu = \int_A f\,dx_1\ldots dx_n$ muunnetussa muodossa
\[
\int_A f\,d\mu=\int_B g\,d\mu', \quad g(\mt)=f(\mpu(\mt)), \quad \mt \in B.
\]
Tässä $g$ on kuvauksen $\mpu$ välittämä $f$:n vastine $B$:ssä, ja $\mu'$ on toinen, toistaiseksi
tuntematon $\R^n$:n mitta. Kyse on siis muuttujan vaihdosta eli \pain{si}j\pain{oituksesta}
\[
\mx=\mpu(\mt),\quad \mt\in B.
\]
Muunnoksessa on ensinnäkin määrättävä funktio $g$. Sikäli kuin lähtökohtana on
(niinkuin yleensä) tunnettu kuvaus $\mpu$, tämä on suoraviivainen toimenpide. Myös mitan
$\mu'$ formaali määrittely on helppoa. Nimittäin jos $\mpu(\Delta B)=\Delta A \subset A$ ja
valitaan $f(\mx)=1,\ \mx\in \Delta A$ (jolloin $g(\mt)=1,\ \mt\in \Delta B$), on mitan ja
integraalin yhteyden (vrt. Luku \ref{tasointegraalit}) ja oletetun integraalien välisen
yhteyden perusteella
\[
\mu'(\Delta B)=\int_{\Delta B} d\mu'=\int_{\Delta A} d\mu=\mu(\Delta A).
\]
Siis $\mu'(\Delta B)=\mu(\Delta A)$, eli $\Delta B$:n mitta määräytyy vastinjoukon
$\Delta A=\mpu(\Delta B)$ Jordan-mittana. --- Tässä on huomattava, että kuvauksen $\mpu$ on
oltava (ainakin lähes) injektio, jotta $\mu'$ olisi todella mitta. Nimittäin koska mitta on
additiivinen, niin $B$:n mitallisille osajoukoille on oltava voimassa
\[
\Delta B_1\cap\Delta B_2=\emptyset \qimpl
       \mu'(\Delta B_1\cup\Delta B_2)=\mu'(\Delta B_1)+\mu'(\Delta B_2).
\]
Jos $\mpu$ ei olisi injektio, niin joillakin $\mx_1,\mx_2 \in B$, $\mx_1\neq\mx_2$ olisi
$\mpu(\mx_1)=\mpu(\mx_2)$, tai yleisemmin $\mpu(\Delta B_1)=\mpu(\Delta B_2)=\Delta A$, missä
$\Delta B_1,\Delta B_2 \subset B$ ja $\Delta B_1 \cap \Delta B_2 =\emptyset$. Tällöin
$\mu'(\Delta B_1\cup\Delta B_2)=\mu(\Delta A)$ ja
$\mu'(\Delta B_1)+\mu'(\Delta B_2)=2\mu(\Delta A)\neq\mu(\Delta A)$, ellei ole
$\mu(\Delta A)=0$. Päätellään siis, että $\mpu$:n epäinjektiivisyys voidaan sallia
\pain{enintään} (mitan $\mu'$ suhteen) \pain{nollamittaisessa} j\pain{oukossa}.

Sikäli kuin $\mpu$ on injektio, tai mainitulla tavalla 'melkein', on integraalin
muuntamiskysymys siis periaatteessa ratkaistu. Toistaiseksi ei kuitenkaan ole selvää,
millä tavoin mitan $\mu'$ avulla voidaan käytännössä laskea (muuten kuin palaamalla 
alkuperäisiin muuttujiin). Tämä laskutekninen kysymys onkin kaikkein keskeisin, sikäli 
kuin muunnoksesta halutaan jotakin hyötyä.

\subsection*{Muuntosuhde}

Tutkitaan muunnettua integraalia $\int_B g\,d\mu'$. Oletetaan, että $B$ on rajoitettu ja
$g(\mt)$ rajoitettu $B$:ssä, jolloin on
(vrt.\ Luvut \ref{tasointegraalit} ja \ref{avaruusintegraalit})
$\int_B g\,d\mu'=\int_T g_0\,d\mu'$, missä $T \supset B$ on $n$-ulotteinen suorakulmainen
perussärmiö ja $g_0=g$:n nollajatko $B$:n ulkopuolelle. Oletetaan jatkossa, että
$\mpu:\ T \kohti \mpu(T)$ on (lähes) injektio ja että $\mpu(T')$ on Jordan-mitallinen aina
kun $T' \subset T$ on Jordan-mitallinen. Tällöin jos $\mathcal{T}_h=\{T_k\}\,$ on $T$:n jako
osasärmiöihin, joiden särmien pituus on enintään $h$, niin vastaavasti kuin Jordan-mitan suhteen
integroitaessa pätee summakaava
\[
\int_T g_0\,d\mu' \,=\, \Lim_{h \kohti 0} \sum_k g_0(\mt_k)\mu'(T_k)
                  \,=\, \Lim_{h \kohti 0} \sum_k g_0(\mt_k)\mu(\mpu(T_k)),
\]
missä $\mt_k \in T_k$.
\begin{figure}[H]
\begin{center}
\import{kuvat/}{kuvaUint-27.pstex_t}
\end{center}
\end{figure}
Oletetaan nyt, että kuvaukselle $\mpu$ on määriteltävissä $T$:ssä jatkuvana funktiona
$J(\mt)$ nk.\ \kor{muuntosuhde} (mittasuhde, suurennussuhde) siten, että jokaisella
$T_k\in\mathcal{T}_h$ ja jokaisella $\mt_k \in T_k$ pätee\footnote[2]{Muuntosuhteen
määrittely voidaan yleisemmin rajoittaa särmiöihin, joille pätee $T_k \cap U=\emptyset$,
missä $\mu'(U)=0$. Tällöin muuntosuhde ei välttämättä ole koko $T$:ssä jatkuva tai edes
määritelty.}
\[
\left|\frac{\mu(\mpu(T_k))}{\mu(T_k)}-J(\mt_k)\right| \,\le\, \eps(h) \kohti 0 \quad 
                                                             \text{kun}\,\ h \kohti 0,
\]
missä $\eps(h)$ riippuu vain tiheysparametrista $h$ (ei $k$:sta eikä yleisemmin
$\mathcal{T}_h$:sta).

Tämän perusteella voidaan em.\ summakaavassa kirjoittaa likimäärin
$\mu(\mpu(T_k) \approx J(\mt_k)\mu(T_k)$. Olettaen, että $|g_0(\mt)| \le C,\ \mt \in T$
($g$ oli rajoitettu), saadaan tälle approksimatiolle tehtyjen oletusten perusteella virhearvio
\[
\left|\sum_k g_0(\mt_k)\mu(\mpu(T_k))-\sum_k g_0(\mt_k)J(\mt_k)\mu(T_k)\right|
\le C\eps(h)\sum_k\mu(T_k) = C\eps(h)\mu(T).
\]
Oletukseen $\,\lim_{h \kohti 0}\eps(h)=0\,$ perustuen ja integraalin summakaavaa uudelleen
soveltaen seuraa
\[
\Lim_{h \kohti 0} \sum_k g_0(\mt_k)\mu(\mpu(T_k)) 
          \,=\, \Lim_{h \kohti 0} \sum_k g_0(\mt_k)J(\mt_k)\mu(T_k)
          \,=\, \int_T g_0(\mt)J(\mt)\,d\mu.
\]
Tässä on edelleen $\int_T g_0(\mt)J(\mt)\,d\mu = \int_B g(\mt)J(\mt)\,d\mu$
(koska $g_0J=(gJ)_0$), joten on päätelty:
\[
\int_A f\,d\mu = \int_B g\,d\mu' = \int_B g(\mt)J(\mt)\,d\mu.
\]
Integraalia $\int_A f\,d\mu$ muunnettaessa on siis muunnosten $f(\mx)=f(\mpu(\mt))=g(\mt)$
ja $A \ext B$, $\mpu(B)=A$, lisäksi suoritettava paikalliseen muuntosuhteeseen $J(\mt)$
perustuva \pain{mittamuunnos} $d\mu'=J(\mt)\,d\mu$, eli muunnoskaava on
\begin{equation} \label{muuntokaava}
\boxed{\kehys\quad \int_A f(\mx)\,dx_1\ldots dx_n=\int_B g(\mt)J(\mt)\,dt_1\ldots dt_n. \quad}
\tag{$\star$}
\end{equation}
%Siis jos muuntosuhde $J(\mt)$ pystytään määräämään (laskutekniikasta jäljempänä), niin
%integraali $\int_A f\,d\mu,\ A\subset\R^n$ muunnetaan muodollisesti sijoituksilla
%\[
%f(\mx) = f(\mpu(\mt)) = g(\mt), \quad d\mu = J(\mt)\,dt_1\cdots dt_n, \quad 
%                                      A \ext B, \quad \mpu(B)=A. \Akehys
%\]
\begin{Exa} Olkoon $A=\{(x,y)\in\R^2 \mid x\in[1,2]\ \ja\ x \le y \le 2x\}$. Laske
$\int_A xy\,dxdy\,$ käyttäen sijoitusta $(x,y)=\mpu(\mt)=\mpu(t,s)=(t,st)$.
\end{Exa}
\ratk Ensinnäkin todetaan, että $\,A=\mpu(B)$, missä $\,B=[1,2]\times[1,2]$, ja että 
$\mpu:\,B \kohti A$ on bijektio. Muuntosuhteen laskemiseksi tarkastellaan suorakulmiota 
$\Delta T=[t,t+\Delta t]\times[s,s+\Delta s]$, missä $t,s \ge 1$ ja 
$0 < \Delta t,\Delta s \le h$. Tällöin on
\begin{align*}
\Delta A\,=\,\mpu(\Delta T)\,
         &=\{(x,y)\in\R^2 \mid t \le x \le t+\Delta t\ \ja\ sx \le y \le (s+\Delta s)x\}\\[1mm]
\impl\quad\mu(\Delta A) 
         &= \int_t^{t+\Delta t}\left(\int_{sx}^{(s+\Delta s)x}\,dy\right)dx\,
          =\,\int_t^{t+\Delta t} (\Delta s)x\,dx \\
         &=\,\frac{1}{2}\Delta s\left[(t+\Delta t)^2-t^2\right]\,
          =\,t\,\Delta t\Delta s + \frac{1}{2}(\Delta t)^2\Delta s.
\end{align*}
Koska $\mu(\Delta A)/\mu(\Delta T)=t+\frac{1}{2}\Delta t = t + \Ord{h}$, niin muuntosuhteen
määritelmän perusteella on $J(t,s)=t$. Siis kaavan \eqref{muuntokaava} mukaan
\[
\int_A f(x,y)\,dxdy = \int_B f(t,st)\,t\,dtds = \int_1^2 t^3\,dt\cdot\int_1^2 s\,ds 
                    = \frac{15}{4}\cdot\frac{3}{2} = \frac{45}{8}\,.
\]
Tarkistus (ilman muuttujan vaihtoa):
\[
\int_A xy\,dxdy = \int_1^2\left[\int_x^{2x} xy\,dy\right]dx
                = \int_1^2\left[\sijoitus{y=x}{y=2x} \frac{1}{2}xy^2\right]dx
                = \int_1^2 \frac{3}{2}x^3\,dx = \frac{45}{8}\,. \loppu
\]
Kuten esimerkissä, integraalin muuntamisen keskeisin laskutekninen ongelma on muuntosuhteen
määrääminen. Jatkossa ratkaistaan tämä ongelma differentiaalilaskennan keinoin. Aloitetaan
yksiulotteisesta integraalista.


\subsection*{Muuntosuhde $\R$:ssä}
\index{muuntosuhde integraalissa!a@$\R$:ssä|vahv}

Olkoon $A=[a,b]$ ja $B=[c,d]$ suljettuja välejä ja $u:B \kohti A$ bijektio. Oletetaan, että
$u$ on jatkuvasti derivoituva välillä $B$. Tällöin jos $T_k=[t_{k-1},t_k]\subset B$, missä 
$t_k-t_{k-1} \le h$, niin jollakin $\xi_k\in T_k$ on
(Lause \ref{toinen väliarvolause}).
\[ 
\mu(u(T_k)) = |u(t_k)-u(t_{k-1})| = |u'(\xi_k)|(t_k-t_{k-1}) = |u'(\xi_k)|\mu(T_k).
\]
Tässä $u'$ (ja näin ollen myös $|u'|$) on jatkuva, joten $|u'(\xi_k)|-|u'(t)|=\ord{1}$
$\forall t \in T_k$, kun $h \kohti 0$.\footnote[2]{Päättely nojaa tässä jatkuvuuden
syvällisempään logiikkaan: Tehdyin oletuksin $|u'|$ on tasaisesti jatkuva välillä $[a,b]$
(Lause \ref{kompaktissa joukossa jatkuva on tasaisesti jatkuva}, ks.\ myös Lause
\ref{tasaisen jatkuvuuden käsite}).} Näin ollen jokaisella $t \in T_k$
\[
\mu(u(T_k)) \,=\, |u'(t)|\mu(T_k)+\ord{1}\mu(T_k), \quad \text{kun}\ h \kohti 0.
\]
Muuntosuhteen määritelmän mukaan on siis $J(t)=\abs{u'(t)}$, jolloin muunnoskaava 
\eqref{muuntokaava} saa muodon
\begin{equation} \label{muuntokaava-R1a}
\int_a^b f(x)\,dx=\int_B f(u(t))\abs{u'(t)}\,dt, \quad u(B)=[a,b]. \tag{a}
\end{equation}
Koska oletettiin, että $u:\ B \kohti [a,b]$ on jatkuva bijektio, niin $u$ on välillä $B$ joko
aidosti kasvava tai aidosti vähenevä. Tällöin jos $u(\alpha)=a$ ja $u(\beta)=b$, niin
\begin{alignat*}{2}
&u\text{ kasvava } (u'(x)\geq 0) \ &&\impl \ B=[\alpha,\beta], \\
&u\text{ vähenevä } (u'(x)\leq 0) \ &&\impl \ B=[\beta,\alpha].
\end{alignat*}
Kummassakin tapauksessa kaava \eqref{muuntokaava-R1a} voidaan kirjoittaa muotoon
\begin{equation} \label{muuntokaava-R1b}
\int_a^b f(x)\,dx=\int_B f\bigl(u(t)\bigr)\abs{u'(t)}\,dt=\int_\alpha^\beta f(u(t))u'(t)\,dt.
\tag{b}
\end{equation}
Tässä (määrätyn integraalin vaihtosääntöön perustuvassa) laskukaavassa on kyse
\pain{si}j\pain{oituksesta} \pain{määrät}y\pain{ssä} \pain{inte}g\pain{raalissa}, vrt.\ Luku
\ref{analyysin peruslause}.
Kaava \eqref{muuntokaava-R1b} on (tietyin edellytyksin, ks.\ mainittu luku) pätevä, vaikka $u$
ei olisikaan bijektio. Sen sijaan kaava \eqref{muuntokaava-R1a} \pain{ei} ole pätevä, jos $u'$
vaihtaa merkkinsä välillä $B$.

\begin{multicols}{2} \raggedcolumns
\begin{Exa} Jos integraalissa $\int_0^1 x\,dx$ tehdään sijoitus $x=u(t)=3t-2t^2$,
niin $u'(t)=3-4t$, $u(0)=0$, $u(1)=1$, ja määrätyn integraalin sijoituskaava
\eqref{muuntokaava-R1b} on pätevä:
\begin{align*}
\int_0^1 x\,dx &= \int_0^1 f(u(t))u'(t)\,dt \\
               &= \int_0^1 (9t-18t^2+8t^3)\,dt \\
               &= \sijoitus{0}{1}(\tfrac{9}{2}t^2-6t^3+2t^4) = \frac{1}{2}\,.
\end{align*}
\end{Exa}
\begin{figure}[H]
\setlength{\unitlength}{1cm}
\begin{center}
\begin{picture}(4,5)(-1,-0.5)
\put(0,0){\vector(1,0){3}} \put(2.8,-0.5){$t$}
\put(0,0){\vector(0,1){3}} \put(0.2,2.8){$x$}
\curve(0,0,1,2,2,2)
\dashline{0.1}(0,2)(2,2) \dashline{0.1}(1,0)(1,2)
\put(2,0){\line(0,-1){0.1}} \put(1.9,-0.5){$1$} \put(0.9,-0.5){$\tfrac{1}{2}$}
\put(0,2){\line(-1,0){0.1}} \put(-0.4,1.85){$1$}
\put(1.5,2.4){$x=u(t)$}
\end{picture}
\end{center}
\end{figure}
\end{multicols}
Sen sijaan kaava \eqref{muuntokaava-R1a} antaa väärän tuloksen:
\[
\int_0^1 f(u(t))\abs{u'(t)}\,dt
     =\int_0^{3/4}(9t-18t^2+8t^3)\,dt - \int_{3/4}^1(9t-18t^2+8t^3)\,dt 
     = \frac{49}{64}\,.
\]
Kummallakin tavalla saadaan oikea tulos, jos muunnetuksi integroimisväliksi valitaan 
$B=[0,\tfrac{1}{2}]$, jolloin $u:B\kohti[0,1]$ on bijektio (vrt.\ kuvio). \loppu

Esimerkistä nähdään, että määrätyn integraalin muunnoskaava \eqref{muuntokaava} antaa
yleisesti väärän tuloksen, jos muunnoskuvaus ei ole injektiivinen. Ongelma on hankalampi
useammassa dimensiossa, missä epäinjektiivisyyttä ei aina ole helppo havaita. Useammassa
dimensiossa ei ongelman poistamiseksi myöskään ole mitään yksinkertaista, määrätyn integraalin
sijoituskaavaan verrattavaa oikotietä.

\subsection*{Muuntosuhde $\R^n$:ssä}
\index{muuntosuhde integraalissa!b@$\R^n$:ssä|vahv}

Tarkastellaan ensin kaksiulotteista muunnosta
\[
\begin{cases}
\,x=u(t,s), \\
\,y=v(t,s), &\mt=(t,s)\in T, \ \mpu=(u,v),
\end{cases}
\]
missä $T$ on perussuorakulmio. Oletetaan, että $u$ ja $v$ ovat jatkuvasti derivoituvia $T$:ssä
(osittaisderivaatat olemassa ja jatkuvia $T$:ssä, Määritelmä
\ref{jatkuvuus kompaktissa joukossa - Rn}). Tällöin suorakulmion
$\Delta T=[t_0+\Delta t]\times [s_0+\Delta s]\subset T$ kuvautumista voi tutkia likimäärin
linearisoivalla approksimaatiolla (vrt. Luku \ref{jacobiaani})
\[
\begin{cases}
\,x(t,s)\approx u(t_0,s_0)+u_t(t_0,s_0)(t-t_0)+u_s(t_0,s_0)(s-s_0), \\
\,y(t,s)\approx v(t_0,s_0)\,+v_t(t_0,s_0)(t-t_0)+v_s(t_0,s_0)(s-s_0).
\end{cases}
\]
Tämä on affiinikuvaus, jonka mukaisesti suorakulmio kuvautuu suunnikkaaksi. Suunnikkaan
virittävät vektorit
\begin{align*}
\vec a &= \Delta t\bigl[u_t(t_0,s_0)\vec i + v_t(t_0,s_0)\vec j\,\bigr], \\
\vec b &= \Delta s\bigl[u_s(t_0,s_0)\vec i + v_s(t_0,s_0)\vec j\,\bigr].
\end{align*}
\begin{figure}[H]
\begin{center}
\import{kuvat/}{kuvaUint-28.pstex_t}
\end{center}
\end{figure}
Kun määritellään kuvauksen $\mpu$ Jacobin matriisi (vrt.\ Luku \ref{jacobiaani})
\[
\mJ\mpu(t,s)=\begin{bmatrix} u_t & u_s \\ v_t & v_s \end{bmatrix},
\]
niin nähdään, että suunnikkaan pinta-ala on
\[
\mu(\Delta A)=\abs{\vec a \times \vec b}=\abs{\det[\mJ\mpu(t_0,s_0)]}\,|\Delta t||\Delta s|.
\]
Koska $\mu(\Delta T)=|\Delta t||\Delta s|$, niin muuntosuhde on tämän perusteella
$J(t,s)=|\det[\mJ\mpu(t,s)]|$. Oletetuin säännöllisyyehdoin tämä osoittautuu oikeaksi
tulokseksi (sivuutetaan tarkemmat perustelut), joten tasointegraalin muuntosuhde saadaan
\index{Jacobin determinantti}%
\kor{Jacobin determinantin} $\det[\mJ\mpu(t,s)]$ avulla:
\[
\boxed{\kehys\quad 
       \text{Muuntosuhde = kuvauksen $\mpu$ Jacobin determinantin itseisarvo.} \quad}
\]

Em.\  tulos on pätevä myös yleisemmin $n$-ulotteiselle avaruusintegraalille. Nimittäin jos
$\mpu:\,T \kohti \mpu(T)$ on epälineaarinen kuvaus ($T\subset\R^n$ perussärmiö), niin tämän
linearisaatio pisteessä $\mt$ on $\R^n$:n affiinikuvaus, joka kuvaa suorakulmaisen särmiön
suuntaissärmiöksi. Ko.\ suuntaissärmiön särmävektoreista muodostettu determinantti = $\mpu$:n
Jacobin (matriisin) determinantti pisteessä $\mt$. Koska determinantin itseisarvo on särmiön
tilavuus (Propositio \ref{suuntaissärmiön tilavuuskaava}), niin ym.\ sääntö on pätevä. Siis
yleistä $n$-ulotteista integraalia muunnettaessa mitan muunnoskaava on
\[
\boxed{\kehys\quad d\mu=\abs{\det(\mJ\mpu)}\,dt_1\cdots dt_n. \quad}
\]
\begin{Exa}
$A=\text{suunnikas}$, jonka kärjet ovat $(0,0)$, $(1,0)$, $(2,1)$ ja $(3,1)$. 
Laske $\,\int_A xy\,dxdy$.
\end{Exa}
\begin{multicols}{2} \raggedcolumns
\ratk Tehdään muuttujan vaihto
\[
\begin{cases}
\,x=u(t,s)=t+2s, \\
\,y=v(t,s)=s,
\end{cases}
\]
\begin{figure}[H]
\begin{center}
\import{kuvat/}{kuvaUint-29.pstex_t}
\end{center}
\end{figure}
\end{multicols}
jolloin $\mpu: B \kohti A$, $B=[0,1]\times [0,1]$ on bijektio ja
\begin{align*}
\mJ=\begin{bmatrix} 1 & 2 \\ 0 & 1 \end{bmatrix} \ &\impl \ J=\abs{\det\mJ}=1 \\
\impl \ \int_A xy\,dxdy &= \int_0^1\left[\int_0^1 (t+2s)s\,ds\right]dt \\
                        &= \int_0^1 \left(\frac{1}{2}t+\frac{2}{3}\right)\,dt
                         = \underline{\underline{\frac{11}{12}}}\,. \loppu
\end{align*}
\begin{Exa} Laske integraali
\[ 
\int_A e^{-x-y-z}\,dxdydz, \quad 
         A = \{(x,y,z) \in \R^3 \mid x+y \ge 0\ \ja\ y+z \ge 0\ \ja\ x+z \ge 0\,\}. 
\]
\end{Exa}
\ratk Tehdään $A$:n muotoon sopiva muuttujan vaihdos
\[
\begin{cases} \,u = x+y \\ \,v = y+z \\ \,w = x+z \end{cases} 
    \ekv\quad \begin{bmatrix} x\\y\\z \end{bmatrix} =
              \frac{1}{2} \begin{rmatrix} 1&-1&1\\1&1&-1\\-1&1&1 \end{rmatrix}
              \begin{bmatrix} u\\v\\w \end{bmatrix},
\]
jolloin muuntosuhde on
\[
J = \left(\frac{1}{2}\right)^3 \begin{detmatrix} 1&-1&1\\1&1&-1\\-1&1&1 \end{detmatrix} 
  = \frac{1}{2}\,.
\] 
Kuvaus $(u,v,w) \in B \map (x,y,z) \in A$ on ilmeinen bijektio, kun
$B = [0,\infty)\times[0,\infty)\times[0,\infty)$, joten muunnoskaavaa \eqref{muuntokaava}
ja Fubinin sääntöä soveltaen on tulos
\begin{align*} 
\int_A &e^{-x-y-z}\,dxdydz = \int_B e^{-\frac{1}{2}(u+v+w)}\,\tfrac{1}{2}\,dudvdw \\ 
       &= \frac{1}{2}\int_0^\infty\int_0^\infty\int_0^\infty 
           e^{-\frac{1}{2}u}e^{-\frac{1}{2}v}e^{-\frac{1}{2}w}\,dudvdw
        = \frac{1}{2}\left[\int_0^\infty e^{-\frac{1}{2}u}\,du\right]^3 
        = \underline{\underline{4}}. \loppu
\end{align*}

\subsection*{Integraalit käyräviivaisissa koordinaatistoissa}
\index{tasointegraali!c@käyräv.\ koordinaateissa|vahv}
\index{avaruusintegraali!b@käyräv.\ koordinaateissa|vahv}
\index{muuntosuhde integraalissa!c@käyräv.\ koordinaatistoissa|vahv}
\index{kzyyrzy@käyräviivaiset koordinaatistot!d@--integraalit|vahv}

Integraalien muunnoskaavat tulevat fysikaalisissa sovelluksissa käyttöön useimmin silloin, kun
karteesisesta koordinaatistosta halutaan siirtyä käyräviivaiseen napa-, lieriö- tai 
pallokoordinaatistoon. Jos tasossa siirrytään napakoordinaatteihin, niin muunnoksen
\[
x=r\cos\varphi=u(r,\varphi), \quad y=r\sin\varphi=v(r,\varphi)
\]
Jacobin matriisi on
\[
\mJ=\begin{rmatrix} \cos\varphi & -r\sin\varphi \\ \sin\varphi & r\cos\varphi \end{rmatrix}.
\]
Tämän determinantin itseisarvo on $J=r$, joten muunnoskaavaksi tulee
\[
\boxed{\kehys\quad \int_A f(x,y)\,dxdy=\int_B g(r,\varphi)\,rdrd\varphi. \quad}
\]
Esimerkiksi jos $A=$ origokeskinen, $R$-säteinen kiekko, niin kaava on pätevä, kun 
$B=[0,R]\times[0,2\pi]$. Muunnoksen lievästä epäinjektiivisyydestä ei tässä ole haittaa, koska
se rajoittuu nollamittaiseen osajoukkoon ($r=0$, $\varphi=0$ tai $\varphi=2\pi$).
\begin{Exa} Laske $\displaystyle{\int_{\R^2} e^{-x^2-y^2}\,dxdy}$.
\end{Exa}
\ratk Siirrytään napkoordinaatistoon. Koordinaattimuunnos $\mpu: B \kohti \R^2$ on (melkein)
bijektio, kun valitaan $B=\{(r,\varphi)\in\R^2 \mid r \ge 0,\ 0 \le \varphi \le 2\pi\}$, joten
\begin{align*}
\int_{\R^2} e^{-x^2-y^2}\,dxdy 
&= \int_B e^{-r^2}r\,dr d\varphi \\
&= \int_0^{2\pi} \left[\int_0^\infty e^{-r^2}r\,dr\right]d\varphi
 = 2\pi \sijoitus{0}{\infty} \left(-\frac{1}{2}e^{-r^2}\right)
 = \underline{\underline{\pi}}. \loppu
\end{align*}
Esimerkin integraali laskettuna karteesisessa koordinaatistossa on
\begin{align*}
\int_{\R^2} e^{-x^2-y^2}\,dxdy &= \int_{\R^2} e^{-x^2}\cdot e^{-y^2}\,dxdy \\
&= \left[\int_{-\infty}^\infty e^{-x^2}\,dx\right]
   \left[\int_{-\infty}^\infty e^{-y^2}\,dy\right] 
 = \left[\int_{-\infty}^\infty e^{-x^2}\,dx\right]^2.
\end{align*}
Saatiin siis hauska tulos:
\begin{Prop} \label{Gamma(1/2)} $\ \int_{-\infty}^\infty e^{-x^2}\,dx=\sqrt{\pi}$.
\end{Prop}
Siirryttäessä kolmessa dimensiossa lierökoordinaatistoon on muuntosuhde sama kuin tasossa
siirryttäessä polaarikoordinaatistoon, eli $J(r,\varphi,z)=r$. Pallokoordinaatistoon
siirryttäessä saadaan muuntosuhteeksi
\[ 
J=\abs{\det\mJ\,}=\left|\begin{array}{ccc} 
                  \sin\theta\cos\varphi & r\cos\theta\cos\varphi & -r\sin\theta\sin\varphi \\
                  \sin\theta\sin\varphi & r\cos\theta\sin\varphi & r\sin\theta\cos\varphi \\
                  \cos\theta            & -r\sin\theta           & 0
                  \end{array}\right| = r^2\sin\theta. 
\]
Integraalin muunnoskaavat lieriö- ja pallokoordinaatistoon ovat näin ollen                                     
\[
\boxed{\begin{aligned}
\ykehys\quad &\text{Lieriökoord.\,:} \qquad 
              \int_A f(x,y,z)\,dxdydz = \int_B g(r,\varphi,z)\,r\,drd\varphi dz \\
             &\text{Pallokoord.\,:} \qquad\    
              \int_A f(x,y,z)\,dxdydz 
                = \int_B g(r,\theta,\varphi)\,r^2\sin\theta\,drd\theta d\varphi \quad\akehys 
\end{aligned}}
\]
\begin{Exa} \label{integraali yli pallokuoren} Laske $I=\int_A(x^2+y^2)\,dxdydz$, kun $A$ on
\index{pallokuori}%
\kor{pallokuori} eli kahden pallopinnan väliin jäävä alue:
\[
A=\{\,(x,y,z)\in\R^3\ |\ R_1^2 \le x^2+y^2+z^2 \le R_2^2\,\}.
\]
\end{Exa} 
\ratk Lasku käy helpoiten pallokoordinaattien avulla:
\begin{align*}
I &= \int_0^{2\pi}\left\{\int_0^\pi\left[\int_{R_1}^{R_2}
         r^2\sin^2\theta\cdot r^2\sin\theta\,dr\right]d\theta\right\}d\varphi \\
  &= \int_0^{2\pi}d\varphi\cdot\int_0^\pi\sin^3\theta\,d\theta\cdot\int_{R_1}^{R_2}r^4\,dr
   = 2\pi\cdot\frac{4}{3}\cdot\frac{1}{5}(R_2^5-R_1^5)
   = \underline{\underline{\frac{8\pi}{15}(R_2^5-R_1^5)}}. \loppu
\end{align*}

\Harj
\begin{enumerate}

\item
Muunna integraali $\int_A f(x,y)\,dxdy$ annettua muunnosta $\mpu$ käyttäen laskemalla ensin 
tarkasti suhde $\mu(\Delta A)/\mu(\Delta T)$, missä $\Delta T=[t,t+\Delta t,s,s+\Delta s]$ ja
$\Delta A=\mpu(\Delta T)$, ja muuntosuhde tämän raja-arvona, kun 
$\max\{\Delta t,\Delta s\} \kohti 0\,$: 
a) \ $A:\ x\in[1,2]\ \ja\ -x^2 \le y \le 3x^2,\,\ \mpu(t,s)=(t,st^2)$ \newline
b) \ $A:\ x\in[1,\infty)\ \ja\ 1/x^2 \le y \le 2/x^2,\,\ \mpu(t,s)=(t,s/t^2)$

\item
Laske sopivalla muunnoksella: \vspace{1mm}\newline
a) \ $\int_A (x+2y)^4(x-2y)^6\,dxdy,\,\ A:\ \abs{x+2y} \le 1\ \ja\ \abs{x-2y} \le 2$ \newline
b) \ $\int_A (2x+3y)^2(x-5y)^2\,dxdy,\,\ A:\ \abs{2x+3y} \le 4\ \ja\ \abs{x-5y} \le 3$
 
\item
Laske seuraavat integraalit sijoituksella muotoa $x=as\,\cos t,\ y=bs\,\sin t$.
\begin{align*}
&\text{a)}\ \ \int_{\R^2} e^{-3x^2-4y^2}\,dxdy \qquad
 \text{b)}\ \ \int_{\R^2} \frac{1}{(1+x^2+4y^2)^2}\,dxdy \\
&\text{c)}\ \ \int_A \ln\left(1+\frac{x^2}{4}+\frac{y^2}{9}\right)dxdy, \quad
              A:\ x,y \ge 0\ \ja\ 9x^2+4y^2 \le 36
\end{align*}

\item
a) Joukko $A\subset\R^2$ sijaitsee koordinaattineljänneksessä $x,y>0$ ja rajoittuu käyriin
$y^3=ax^2$, $y^3=bx^2$, $x^4=cy^3$ ja $x^4=dy^3$, missä $0<a<b$ ja $0<c<d$. Laske pinta-ala
$\mu(A)$ sijoituksella $y^3=ux^2,\ x^4=vy^3$. \vspace{1mm}\newline
b) Ratkaise Harj.teht.\,\ref{avaruusintegraalit}:\ref{H-uint-3: ellipsoidin tilavuus} muuttujan
vaihdolla $(x,y,z)=(au,bv,cw)$.

%\item 
%Kappaletta, jonka tiheys $\rho$ on vakio, rajoittavat $xy$-taso, lieriö $x^2+y^2=a^2$ ja 
%paraboloidi $x^2+y^2=az$ ($a>0$). Laske kappaleen massa ja hitausmomentit koordinaatiakselien 
%suhteen käyttäen lieriökoordinaatteja.

\item 
Laske integraali $\int_A f\,dxdydz$ lieriö- tai pallokoordinaatteihin
siirtymällä: \vspace{1mm}\newline
a)\ \ $A:\ x^2+y^2 \le z \le \sqrt{2-x^2-y^2}, \quad f(x,y,z)=z$ \newline
b)\ \ $A:\ x^2+y^2+z^2 \le 12,\ z \ge x^2+y^2, \quad f(x,y,z)=1$ \newline
c)\ \ $A:\ x^2+y^2+z^2 \le R^2,\ x,y,z \ge 0, \quad f(x,y,z)=x$ \newline
d)\ \ $A:\ x^2 + y^2 + z^2 \le R^2,\ x,y,z\ge 0, \quad f(x,y,z=xyz$ \newline
e)\ \ $A:\ 1 \le x^2+y^2 \le 4,\ 0 \le y \le x,\ 0 \le z \le 1, \quad f(x,y,z)=x^2+y^2$ \newline
f)\,\ \ $A:\ 4 \le x^2+y^2+z^2 \le 9,\ z \ge 0, \quad f(x,y,z)=z$ \newline
g)\ \ $A:\ x^2+y^2 \ge R^2,\ x^2+y^2+z^2 \le R^2, \quad f(x,y,z)=x^2+y^2$

\item 
Puolikartio $z=\sqrt{x^2+y^2}$ jakaa origokeskiset pallot kahteen osaan. Laske näiden
osien tilavuuksien suhde pallokoordinaattien avulla.

\item (*)
Olkoon $f(x,y)=e^{(y-x)/(y+x)}$ ja $A$ kolmio, jonka kärjet ovat $(0,0)$, $(1,0)$ ja $(0,1)$.
Laske integraali $\int_A f\,dxdy$ \ a) polaarikoordinaattien avulla, \linebreak
b) sijoituksella $u=y-x,\ v=y+x$.

\item (*)
Laske tasointegraali $\D \int_0^\infty\left[\int_0^x \frac{e^{-x-2y}}{x+2y}\,dy\right]dx$.

\item (*)
Laske tilavuus $\mu(A)$, kun $A\subset\R^3$ määritellään ehdoilla
\[
\frac{x^2}{a^2}+\frac{y^2}{b^2}+\frac{z^2}{c^2} \le 1\ \ja\ z+y \ge b \quad (a,b,c>0).
\]

\item (*)
Olkoon $K$ suuntaissärmiö, jonka yksi kärki on origossa ja origosta lähtevien särmien toiset
päätepisteet ovat $(2,1,1)$, $(-1,2,2)$ ja $(0,-2,1)$. Laske $\int_K xye^z\,dxdydz\,$ sopivalla
muuttujan vaihdolla.

\end{enumerate} 