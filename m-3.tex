\section{Gaussin algoritmi} \label{Gaussin algoritmi}
\alku
\index{Gaussin algoritmi (eliminaatio)|vahv}

Lineaarisia yhtälöryhmiä, jotka eivät ole kooltaan aivan hirmuisia (esim. $n<10^4$) ratkotaan
käytännössä usein \kor{Gaussin algoritmilla} eli \kor{Gaussin eliminaatiolla}. Algoritmilla on
monia variaatioita, ja niillä monia nimiä, mutta pääidea on kaikissa sama.

Lähdetään ratkaisemaan yhtälöryhmää
\[
\left\{
\begin{alignedat}{3}
a_{11} &x_1 + a_{12} &x_2 + \ldots + a_{1n}         &x_n  \ &= \ &b_1,    \\
a_{21} &x_1 + a_{22} &x_2 + \ldots + a_{2n}         &x_n  \ &= \ &b_2,    \\
       &\vdots       & \vdots \qquad \qquad \quad \ &\vdots &    &\vdots \\
a_{i1} &x_1 + a_{i2} &x_2 + \ldots + a_{in}         &x_n  \ &= \ &b_i,    \\
       &\vdots       & \vdots \qquad \qquad \quad \ &\vdots &    &\vdots \\
a_{n1} &x_1 + a_{n2} &x_2 + \ldots + a_{nn}         &x_n  \ &= \ &b_n. \
\end{alignedat}
\right.
\]
Gaussin algoritmissa oletetaan, että yhtälöryhmällä on ratkaisu $(x_i)$ ja toimitaan ikäänkuin
ratkaisu olisi sijoitettu tuntemattomien paikalle. Ratkaisun ei tarvitse olla yksikäsitteinen
--- tämä selviää algoritmin kuluessa. Samoin selviää, onko ratkaisua ylipäänsä olemassa.

Gaussin algoritmissa yhtälöryhmä muunnetaan askelittain yksinkertaisempaan muotoon. Muunnetut
yhtälöryhmät ovat alkuperäisen kanssa ekvivalenttisia (eli muunnokset voidaan tehdä molempiin 
suuntiin), joten mitään informaatiota ei laskentaprosessin kuluessa menetetä.

Olettaen, että $a_{11} \neq 0$ (tapaukseen $a_{11} = 0$ palataan myöhemmin), Gaussin algoritmin
ensimmäinen nk.\ 
\index{eliminaatioaskel (Gaussin alg.)}%
\kor{eliminaatioaskel} on: Ratkaistaan 1. yhtälöstä
\[
x_1=-\sum_{j=2}^n \frac{a_{1j}}{a_{11}}\,x_j + \frac{b_1}{a_{11}}
\]
ja sijoitetaan tämä lauseke muihin yhtälöihin $x_1$:n paikalle. Sievennysten jälkeen on 
tuloksena alkuperäisen kanssa ekvivalenttinen yhtälöryhmä 
\[
\left\{
\begin{alignedat}{2}
a_{11} x_1 + a_{12} &x_2 + \ldots +       a_{1n}  &x_n   &= b_1, \\
       a_{22}^{(1)} &x_2 + \ldots + a_{2n}^{(1)}  &x_n   &= b_2^{(1)}, \\
                    & \vdots \qquad \qquad  \quad \ \vdots & &  \quad \vdots \\
       a_{i2}^{(1)} &x_2 + \ldots + a_{in}^{(1)}  &x_n  &=  b_i^{(1)}, \\
                    & \vdots \qquad \qquad  \quad \ \vdots & &  \quad \vdots \\
       a_{n2}^{(1)} &x_2 + \ldots + a_{nn}^{(1)}  &x_n   &=  b_n^{(1)}. \
\end{alignedat}
\right.
\]
Tässä ensimmäinen yhtälö on jätetty alkuperäiseen muotoonsa ja on merkitty
\begin{align*}
a_{ij}^{(1)} &= a_{ij} - a_{i1}a_{1j}/a_{11}, \quad i,j = 2 \ldots n, \\
b_i^{(1)} &= b_i - a_{i1}b_1/a_{11}, \quad i = 2 \ldots n.
\end{align*}
Taulukkomuodossa laskien tämä muunnos toteutuu, kun kerrotaan 1. yhtälö puolittain luvulla 
$-a_{i1}/a_{11}$, lasketaan tulos puolittain yhteen $i$:nnen yhtälön kanssa, ja korvataan 
tuloksella aiempi $i$:s yhtälö, $i=2 \ldots n$. 

Ensimmäisen eliminaatioaskeleen jälkeinen yhtälöryhmä on matriisimuodossa
\[
\begin{bmatrix}
a_{11} & a_{12}       & \ldots & a_{1n} \\
     0 & a_{22}^{(1)} & \ldots & a_{2n}^{(1)} \\
\vdots & \vdots       &        & \vdots \\
     0 & a_{n2}^{(1)} & \ldots & a_{nn}^{(1)}
\end{bmatrix}
\begin{bmatrix}
x_1 \\ x_2 \\ \vdots \\ x_n
\end{bmatrix} =
\begin{bmatrix}
b_1 \\ \ b_2^{(1)} \\ \vdots \\ \ b_n^{(1)}
\end{bmatrix}.
\]
Koska tässä yhtälöryhmässä $x_1$ esiintyy vain ensimmäisessä yhtälössä, voidaan tämä yhtälö 
jättää jatkossa omaan rauhaansa ja tarkastella supistettua yhtälöryhmää
\[
\begin{bmatrix}
a_{22}^{(1)} & a_{23}^{(1)} & \ldots & a_{2n}^{(1)} \\
a_{33}^{(1)} & a_{33}^{(1)} & \ldots & a_{3n}^{(1)} \\
\vdots & \vdots      &      & \vdots \\
a_{n2}^{(1)} & a_{n3}^{(1)} & \ldots & a_{nn}^{(1)}
\end{bmatrix}
\begin{bmatrix}
x_2 \\ x_3 \\ \vdots \\ x_n
\end{bmatrix} =
\begin{bmatrix}
b_2^{(1)} \\ b_3^{(1)} \\ \vdots \\ b_n^{(1)}
\end{bmatrix}.
\]
Jos $a_{22}^{(1)} \neq 0$, suoritetaan tähän yhtälöryhmään samanlainen eliminaatiomuunnos kuin
edellä, jolloin se muuntuu muotoon
\[
\begin{bmatrix}
a_{22}^{(1)} & a_{23}^{(1)} & \ldots & a_{2n}^{(1)} \\
0 & a_{33}^{(2)} & \ldots & a_{3n}^{(2)} \\
\vdots & \vdots      &      & \vdots \\
0 & a_{n3}^{(2)} & \ldots & a_{nn}^{(2)}
\end{bmatrix}
\begin{bmatrix}
x_2 \\ x_3 \\ \vdots \\ x_n
\end{bmatrix} =
\begin{bmatrix}
b_2^{(1)} \\ b_3^{(2)} \\ \vdots \\ b_n^{(2)}
\end{bmatrix}.
\]
Tässä $x_2$ esiintyy vain ensimmäisessä yhtälössä, joten yhtälöryhmä supistuu jälleen. 
Yleisesti on $k-1$ eliminaatioaskeleen jälkeen koko yhtälöryhmä saatu muotoon
\[
\left[\begin{array}{llllll}
a_{11}  &  a_{12} & \ldots & a_{1k} & \ldots & a_{1n} \\
0       &  a_{22}^{(1)} & \ldots & a_{2k}^{(1)} & \ldots & a_{2n}^{(1)} \\
\vdots  & \vdots        & \ddots \\
0       & 0             & \ldots & a_{kk}^{(k-1)} & \ldots & a_{kn}^{(k-1)} \\
\vdots  & \vdots        &        & \vdots \\
0       & 0             & \ldots & a_{nk}^{(k-1)} & \ldots & a_{nn}^{(k-1)}
\end{array}\right]
\begin{bmatrix}
x_1 \\ x_2 \\ \vdots \\ x_k \\ \vdots \\ x_n
\end{bmatrix} =
\left[\begin{array}{l}
b_1 \\ b_2^{(1)} \\ \vdots \\ b_k^{(k-1)} \\ \vdots \\ b_n^{(k-1)}
\end{array}\right].
\]
Edellytyksenä tähän muotoon pääsemiseksi on siis, että
\[
a_{ll}^{(l-1)} \neq 0, \quad l=1 \ldots k-1,
\]
missä $a_{11}^{(0)}=a_{11}$. Alkiota $a_{kk}^{(k-1)}$ sanotaan seuraavan eli $k$:nnen 
eliminaatioaskeleen
\index{tukialkio (Gaussin alg.)}%
\kor{tukialkioksi} (engl.\ pivot element). Mikäli kaikki tukialkiot ovat
nollasta poikkeavia ja myös $a_{nn}^{(n-1)} \neq 0$, päädytään $n-1$ askeleen jälkeen
yhtälöryhmään, jonka kerroinmatriisi on säännöllinen kolmiomatriisi:
\[
\left[\begin{array}{llllll}
a_{11} & a_{22} & \ldots & a_{1n} \\
0      & a_{22}^{(1)} & \ldots & a_{2n}^{(1)} \\
\vdots & \vdots & \ddots & \vdots \\
0 & 0 & \ldots  & a_{nn}^{(n-1)}
\end{array}\right]
 \begin{bmatrix}
x_1 \\ x_2 \\ \vdots \\ x_n
\end{bmatrix} =
\left[\begin{array}{l}
b_1 \\ b_2^{(1)} \\ \vdots \\ b_n^{(k-1)}
\end{array}\right].
\]
Kuten pian nähdään, algoritmin läpimeno tällä tavoin osoittaa, että alkuperäisen yhtälöryhmän
kerroinmatriisi $\mA$ on säännöllinen. Joka tapauksessa muunnettu yhtälöryhmä ratkeaa helposti
purkamalla se lopusta lukien, vrt.\ Lauseen \ref{kolmiomatriisi} todistus edellä. Tätä 
algoritmin vaihetta kutsutaan
\index{takaisinsijoitus (Gaussin alg.)}%
\kor{takaisinsijoitukseksi}.
\begin{Exa} \label{ristikkoesimerkki} (Ks.\ Luku \ref{yhtälöryhmät}, Esimerkki \ref{ristikko}) 
Ratkaise Gaussin eliminaatiolla yhtälöryhmä
\[
\begin{rmatrix} 3&-1&0&0&-1&1\\1&-3&0&2&-1&1\\0&0&3&1&-1&-1\\0&-2&1&3&-1&-1\\1&-1&1&1&-4&0\\
                -1&1&\ \ 1&\ \ 1&0&-4\end{rmatrix}
\begin{rmatrix} x_1\\x_2\\x_3\\x_4\\x_5\\x_6 \end{rmatrix} 
\ =\ F \begin{rmatrix} 0\\1\\0\\0\\0\\0 \end{rmatrix} 
   + G \begin{rmatrix} 0\\0\\0\\1\\0\\0 \end{rmatrix},
\]
missä $F,G\in\R$. 
\end{Exa}
\ratk Ratkaisu on $\mx=F\mx_1+G\mx_2$, missä $\mA\mx_1=\mb_1=[0\ 1\ 0\ 0\ 0\ 0]^T$ ja 
$\mA\mx_2=\mb_2=[0\ 0\ 0\ 1\ 0\ 0]^T$, joten on ratkaistava kaksi lineaarista yhtälöryhmää. 
Gaussin algoritmilla voidaan molemmat yhtälöryhmät käsitellä samanaikaisesti: Kirjoitetaan 
yhtälöryhmät taulukkomuotoon
\[ 
       \left[\begin{array}{rrrrrr} 
             3&-1&0&0&-1&1\\1&-3&0&2&-1&1\\0&0&3&1&-1&-1\\
             0&-2&1&3&-1&-1\\ 1&-1&1&1&-4&0\\-1&1&\ \ 1&\ \ 1&0&-4 
             \end{array} \right.
\left. \left|\begin{array}{rr} 
             \ \ 0&0\\1&0\\0&0\\0&1\\0&0\\0&\ \ 0 
             \end{array} \right. \right]
\]
ja ulotetaan eliminaatio koko taulukkoon. Seuraavassa muunnettu taulukko kunkin 
eliminaatioaskeleen jälkeen.
\begin{align*}
&(k=1) \qquad \left[\begin{array}{rrrrrr}
                    3&-1&0&0&-1&1\\0&-\frac{8}{3}&0&2&-\frac{2}{3}&\frac{2}{3}\\
                    0&0&3&1&-1&-1\\0&-2&1&3&-1&-11 \\
                    0&-\frac{2}{3}&1&1&-\frac{11}{3}&-\frac{1}{3} \\[0.5mm]
                    0&\frac{2}{3}&1&1&-\frac{1}{3}&-\frac{11}{3} 
                    \end{array} \right.
       \left. \left|\begin{array}{rr} 
                    \ \ 0&0\\1&0\\0&0\\0&1\\0&0\\0&\ \ 0 
                    \end{array} \right. \right] \\[5mm]
&(k=2) \qquad \left[\begin{array}{rrrrrr} 
                    3&-1&0&0&-1&1\\0&-\frac{8}{3}&0&2&-\frac{2}{3}&\frac{2}{3}\\0&0&3&1&-1&-1\\
                    0&0&1&\frac{3}{2}&-\frac{1}{2}&\ \ -\frac{3}{2}\\[0.5mm]
                    0&0&1&\frac{1}{2}&-\frac{7}{2}&-\frac{1}{2}\\[0.5mm]
                    0&0&1&\frac{3}{2}&-\frac{1}{2}&-\frac{7}{2} 
                    \end{array} \right.
       \left. \left|\begin{array}{rr}
                    0&0\\1&0\\0&0\\-\frac{3}{4}&1\\-\frac{1}{4}&0\\ \frac{1}{4} & \ \ 0
                    \end{array} \right. \right] \\[5mm]
&(k=3) \qquad \left[\begin{array}{rrrrrr} 
                    3&-1&0&0&-1&1\\0&-\frac{8}{3}&0&2&-\frac{2}{3}&\frac{2}{3}\\0&0&3&1&-1&-1\\
                    0&0&0&\frac{7}{6}&-\frac{1}{6}&-\frac{7}{6}\\[0.5mm]
                    0&0&0&\frac{1}{6}&-\frac{19}{6}&-\frac{1}{6}\\[0.5mm]
                    0&0&0&\frac{7}{6}&-\frac{1}{6}&-\frac{19}{6} 
                    \end{array} \right.
       \left. \left|\begin{array}{rr}
                    0&0\\1&0\\0&0\\-\frac{3}{4}&1\\-\frac{1}{4}&0\\ \frac{1}{4} & \ \ 0
                    \end{array} \right. \right] \\[5mm]
&(k=4) \qquad \left[\begin{array}{rrrrrr} 
                   3&-1&0&0&-1&1\\0&-\frac{8}{3}&0&2&-\frac{2}{3}&\frac{2}{3}\\0&0&3&1&-1&-1\\
                   0&0&0&\frac{7}{6}&-\frac{1}{6}&-\frac{7}{6}\\[0.5mm]
                   0&0&0&0&-\frac{22}{7}&0\\0&0&0&0&0&\ -2
                   \end{array} \right.
       \left. \left|\begin{array}{rr}
                    0&0\\1&0\\0&0\\-\frac{3}{4}&1\\-\frac{1}{7}&-\frac{1}{7}\\1&-1
                    \end{array} \right. \right]
\end{align*}
Askelta $k=5$ ei tässä tapauksessa tarvita, koska on $a_{65}^{(4)}=0$.

Ratkaisu saadaan tästä takaisinsijoituksella. --- Käytetään tässä kuitenkin toista, laskutyön
kannalta samanarvoista mahdollisuutta, jossa eliminaatioalgoritmia käytetään uudelleen 
'takaperin'. Koska kerroinmatriisi on yläkolmio, niin takaisin päin eliminoitaessa muuttuu
matriisissa kullakin eliminaatioaskeleella vain tukialkion kohdalla oleva sarake, tai tarkemmin
tämän lävistäjän yläpuolinen osa. Eliminaatiovaiheet takaisin päin ovat seuraavat.
\begin{align*}
&(k=5) \qquad \left[\begin{array}{rrrrrr}
                    3&-1&0&0&-1&0\\[0.5mm] 0&-\frac{8}{3}&0&2&-\frac{2}{3}&0\\[0.5mm]
                    0&0&3&1&-1&0\\[0.5mm] 0&0&0&\frac{7}{6}&-\frac{1}{6}&0\\[0.5mm]
                    0&0&0&0&-\frac{22}{7}&0\\0&0&\ \ 0&\ \ 0&0&-2 
                    \end{array} \right.
       \left. \left|\begin{array}{rr}
                    \frac{1}{2}&-\frac{1}{2}\\[0.5mm]\frac{4}{3}&-\frac{1}{3}\\[0.5mm]
                    -\frac{1}{2}&\frac{1}{2}\\[0.5mm]-\frac{4}{3}&\frac{19}{12}\\[0.5mm]
                    \ \ \ -\frac{1}{7}&\ -\frac{1}{7}\\ 1&-1
                    \end{array} \right. \right] \\[5mm]
&(k=4) \qquad \left[\begin{array}{rrrrrr}
                  3&-1&0&0&0&0\\[0.5mm] 0&-\frac{8}{3}&0&2&0&0\\[0.5mm] 0&0&3&1&0&0\\[0.5mm]
                  0&0&0&\frac{7}{6}&0&0\\[0.5mm] 0&0&0&0&-\frac{22}{7}&0\\0&0&\ \ 0&\ \ 0&0&-2 
                  \end{array} \right.
       \left. \left|\begin{array}{rr}
                    \frac{6}{11}&-\frac{5}{11}\\[0.5mm]\frac{15}{11}&-\frac{10}{33}\\[0.5mm]
                    -\frac{5}{11}&\frac{6}{11}\\[0.5mm]\ -\frac{175}{132}&\frac{35}{22}\\[0.5mm]
                    -\frac{1}{7}&-\frac{1}{7}\\ 1&-1
                    \end{array} \right. \right] \\[5mm]
&(k=3) \qquad \left[\begin{array}{rrrrrr} 
                  3&-1&0&0&0&0\\[0.5mm] 0&-\frac{8}{3}&0&0&0&0\\[0.5mm] 0&0&3&0&0&0\\[0.5mm]
                  0&0&0&\frac{7}{6}&0&0\\[0.5mm] 0&0&0&0&-\frac{22}{7}&0\\0&0&\ \ 0&\ \ 0&0&-2 
                  \end{array} \right.
       \left. \left|\begin{array}{rr}
                    \frac{6}{11}&-\frac{5}{11}\\[0.5mm]\frac{40}{11}&-\frac{100}{33}\\[0.5mm]
                    \frac{15}{22}&-\frac{9}{11}\\[0.5mm]-\frac{175}{132}&\frac{35}{22}\\[0.5mm]
                    -\frac{1}{7}&-\frac{1}{7}\\ 1&-1
                    \end{array} \right. \right] \\[5mm]
&(k=2) \qquad \left[\begin{array}{rrrrrr} 
                     3&0&0&0&0&0\\[0.5mm] 0&-\frac{8}{3}&0&0&0&0\\[0.5mm] 0&0&3&0&0&0\\[0.5mm]
                     0&0&0&\frac{7}{6}&0&0\\[0.5mm] 0&0&0&0&-\frac{22}{7}&0\\
                     0&0&\ \ 0&\ \ 0&0&-2 
                    \end{array} \right.
       \left. \left|\begin{array}{rr}
                    -\frac{9}{11}&\frac{15}{22}\\[0.5mm]\frac{40}{11}&-\frac{100}{33}\\[0.5mm]
                    \frac{15}{22}&-\frac{9}{11}\\[0.5mm]-\frac{175}{132}&\frac{35}{22}\\[0.5mm]
                    -\frac{1}{7}&-\frac{1}{7}\\1&-1
                    \end{array} \right. \right]
\intertext{Lopuksi diagonaalisen systeemin ratkaisu:}
&\phantom{(k=2)} \qquad 
              \left[\begin{array}{rrrrrr} 
                    1&\,\ \ 0&\,\ \ 0&\,\ \ 0&\,\ \ 0&\,\ \ 0\\[0.5mm] 0&1&0&0&0&0\\[0.5mm] 
                    0&0&1&0&0&0\\[0.5mm]
                    0&0&0&1&0&0\\[0.5mm] 0&0&0&0&1&0\\[0.5mm] 0&0&0&0&0&1 
                    \end{array} \right.
       \left. \left|\begin{array}{rr}
                    -\frac{3}{11}&\frac{5}{22}\\[0.5mm]-\frac{15}{11}&\frac{25}{22}\\[0.5mm]
                    \frac{5}{22}&\ -\frac{3}{11}\\[0.5mm]\ -\frac{25}{22}&\frac{30}{22}\\[0.5mm]
                    \frac{1}{22}&\frac{1}{22}\\[0.5mm]-\frac{1}{2}&\frac{1}{2}
                    \end{array} \right. \right].
\end{align*}

Kysytty yhtälöryhmän ratkaisu on näin muodoin
\[ 
\begin{bmatrix} x_1\\x_2\\x_3\\x_4\\x_5\\x_6 \end{bmatrix}\ =\
               \frac{F}{22} \begin{rmatrix} -6\\-30\\5\\-25\\1\\-11 \end{rmatrix}
              +\frac{G}{22} \begin{rmatrix} 5\\25\\-6\\30\\1\\11 \end{rmatrix}. \quad \loppu
\]

\subsection*{Gaussin algoritmin työmäärä}
\index{Gaussin algoritmi (eliminaatio)!c@työmäärä|vahv}

Gaussin algoritmi koostuu siis kahdesta vaiheesta, eliminaatiovaiheesta ja 
takaisinsijoituksesta. Englannin kielessä käytetään myös termejä 'forward sweep' ja
'backward sweep'. Näistä ensimmäinen 'pyyhkäisy' on työmäärää (laskenta-aikaa) ajatellen 
huomattavasti raskaampi. Tämä nähdään eliminaatiovaiheen yleisestä algoritmimuodosta, joka on
\begin{align*}
a_{ij}^{(k)} &= a_{ij}^{(k-1)} - a_{ik}^{(k-1)}a_{kj}^{(k-1)}/a_{kk}^{(k-1)}, \quad 
                                                                         i,j = k+1 \ldots n, \\
b_i^{(k)} &= b_i^{(k-1)} - a_{ik}^{(k-1)}b_k^{(k-1)}/a_{kk}^{(k-1)}, \quad i = k+1 \ldots n.
\end{align*}
Tässä on jakolaskut $\,a_{ik}^{(k-1)}/a_{kk}^{(k-1)}\,,\ i=k+1 \ldots n\,$ syytä laskea ensin,
jolloin algoritmi on ohjelmointikielellä
\begin{align*}
l_{ik}\ &\leftarrow\ a_{ik}/a_{kk}\,, \qquad\ \ i=k+1 \ldots n, \\
a_{ij}\ &\leftarrow\ a_{ij} - l_{ik}a_{kj}\,, \quad\, i,j=k+1 \ldots n, \\
b_i\    &\leftarrow\ b_i - l_{ik}b_k\,, \qquad i=k+1 \ldots n.
\end{align*}
Työmäärää arvioitaessa pidettäköön työyksikkönä laskuoperaatiota, joka koostuu yhdestä 
kertolaskusta (tai jakolaskusta) ja yhdestä yhteen- tai vähennyslaskusta. Algoritmista nähdään,
että algoritmin työläin osa koostuu muunnoksista $\,a_{ij}^{(k)} \leftarrow a_{ij}^{(k-1)}$.
Näiden vaatima työmäärä koko eliminaatiovaiheessa on mainittuina operaatioina
\[
W = (n-1)^2+(n-2)^2 + \cdots + 1 = \frac{1}{6}(n-1)n(2n-1).
\]
Muut algoritmin osat, kuten muunnokset $b_i^{(k)} \leftarrow b_i^{(k-1)}$ ja takaisinsijoitukset
ovat työmäärältään suuruusluokkaa $\ordoO{n^2}$, joten Gaussin algoritmin työmäärä on
\[
\boxed{\quad W\ =\ \frac{1}{3}\,n^3 + \ordoO{n^2}. \quad}
\]
Tästä nähdään, että esimerkiksi tuhannen yhtälön yhtälöryhmän ($W \approx 0.3 \cdot 10^9$) 
ratkaiseminen tietokoneella ei ole ongelma laskenta-ajan kannalta. Sen sijaan yhtälöryhmä
kokoa $n=10^6$ ($W \approx 0.3 \cdot 10^{18}$) saattaa jo olla ylivoimainen tehtävä. ---
Näinkin suuria yhtälöryhmiä ratkotaan nykyisin, mutta silloin käytetään yleensä hyväksi
kerroinmatriisin erikoisominaisuuksia. Toinen, perinteisesti suosittu vaihtoehto on käyttää
tehtävään sopivia
\kor{iteratiivisia} (likimääräisiä) ratkaisutapoja. Suhteessa iteratiivisiin menetelmiin
sanotaan Gaussin algoritmiin perustuvaa ratkaisua \kor{suoraksi} ratkaisutavaksi.
Kohtuullisilla $n$:n arvoilla yhtälöryhmän suora ratkaisu Gaussin algoritmilla on edelleen
kilpailukykyinen menetelmä, ja tietokoneiden suorituskyvyn kasvaessa se on myös voittanut alaa
iteratiivisilta menetelmiltä. Etenkin silloin kun algoritmi menee läpi em. suoraviivaisella
tavalla se on myös hyvin helppo ohjelmoida.

\subsection*{Neliömatriisin $LU$-hajotelma}
\index{LU@$LU$-hajotelma (neliömatriisin)|vahv}
\index{matriisin ($\nel$neliömatriisin)!f@$\nel$$LU$-hajotelma|vahv}

Gaussin eliminaatiolla, sikäli kuin se onnistuu edellä kuvatulla tavalla, saadaan yhtälöryhmälle
$\mA\mx=\mb$ ratkaisu jokaisella $\mb\in\R^n$, joten Lauseen \ref{säännöllisyyskriteerit}
perusteella kerroinmatriisin on oltava säännöllinen. Ko.\ lauseeseen (jota ei vielä ole 
todistettu) ei kuitenkaan tarvitse vedota, sillä kerroinmatriisin säännöllisyys tulee Gaussin
algoritmilla suoraan todistetuksi. Päättely on seuraava: Ensinnäkin todetaan, että Gaussin
algoritmin eliminaatiovaihe muuntaa alkuperäisen yhtälöryhmän
\[
\mA \mx = \mb
\]
muotoon
\[
\mU \mx = \mc = \mR \mb,
\]
missä $\mU$ on muunnettu kerroinmatriisi, eli yläkolmiomatriisi, jonka lävistäjäalkiot ovat 
$a_{kk}^{(k-1)} \neq 0, \ k=1 \ldots n$. Matriisi $\mR$ ei ole eliminaatiovaiheen jälkeen 
suoraan nähtävissä, mutta se saadaan selville tarkastelemalla yhtälöryhmää
\[
\mL \my = \mb,
\]
missä $\mL$ on alakolmiomatriisi, joka määritellään
\[
[\mL]_{ik} = \begin{cases}
\,0, &\text{jos}\ i<k, \\
\,1, &\text{jos}\ i=k, \\
\,l_{ik}=a_{ik}^{(k-1)}/a_{kk}^{(k-1)}\,, &\text{jos}\ i>k.
\end{cases}
\]
Tässä siis $a_{ik}^{(k-1)}, \ i=k \ldots n$, ovat muunnetun kerroinmatriisin alkioita $k-1$ 
eliminaatioaskeleen jälkeen:
\[
\begin{array}{ccc}
a_{k-1,k-1}^{(k-2)} \\
0 & a_{kk}^{(k-1)} \\
\vdots & \vdots \\
0 & a_{ik}^{(k-1)} & \rightarrow\ l_{ik} = a_{ik}^{(k-1)}/a_{kk}^{(k-1)} \\
\vdots & \vdots \\
0 & a_{nk}^{(k-1)}
\end{array}
\]
Matriisin $\mL$ muodosta nähdään, että jos em.\ yhtälöryhmään sovelletaan Gaussin algoritmia, 
niin jo eliminaatiovaihe antaa ratkaisun $\my=\mL^{-1}\mb$. Toisaalta nähdään vertaamalla 
Gaussin algoritmeja sovellettuna yhtälöryhmiin $\mL\my=\mb$ ja $\mA\mx=\mb$, että 
eliminaatiovaiheen muunnokset $b_i^{(k)} \leftarrow b_i^{(k-1)}$ ovat kummassakin tapauksessa 
täsmälleen samat. Siis on päätelty:
\[
\my = \inv{\mL} \mb = \mR \mb.
\]
Koska tämä pätee $\forall \mb \in \R^n$, on $\mR=\inv{\mL}$ 
(Propositio \ref{matriisien yhtäsuuruus}), ja näin ollen
\[
\mA \mx = \mb \ \ekv \ \mL \mU \mx = \mb
\]
Siis $\mA \mx = \mL \mU \mx \ \forall \mx \in \R^n$, joten
\[
\boxed{\kehys\quad \mA=\mL\mU. \quad}
\]
Tätä sanotaan matriisin \kor{$LU$-hajotelmaksi} (engl. $LU$-decomposition). Sen laskemiseksi ei
siis tarvita muuta kuin Gaussin algoritmin eliminaatiovaihe, kunhan muistetaan tallettaa 
matriisi $\mL$, eli luvut
\[
[\mL]_{ik} = l_{ik} = a_{ik}^{(k-1)}/a_{kk}^{(k-1)}, \quad i=k+1 \ldots n, \ k=1  \ldots n-1.
\]
Nämä tulevat eliminaation kuluessa joka tapauksessa lasketuksi, mutta ilman talletusta $\mL$ ei
jää näkyviin (koska yhtälöryhmän ratkaisualgoritmi laskee valmiiksi vektorin 
$\inv{\mL} \mb = \mR \mb$).
\jatko \begin{Exa} (jatko) Seuraamalla eliminaatiovaiheita ($k=1\,\ldots\,4$) nähdään, että
esimerkin yhtälöryhmässä kerroinmatriisin $LU$-hajotelma on
\[
\mA =\begin{rmatrix} 
     1&0&0&0&0&0 \\ \frac{1}{3}&1&0&0&0&0 \\ 0&0&1&0&0&0 \\ 
     0&\frac{3}{4}&\frac{1}{3}&1&0&0 \\[0.5mm]
     \frac{1}{3}&\frac{1}{4}&\frac{1}{3}&\frac{1}{7}&1&0 \\[0.5mm] 
     -\frac{1}{3}&-\frac{1}{4}&\frac{1}{3}&1&0&1
     \end{rmatrix}
     \begin{rmatrix} 
     3&-1&0&0&-1&1\\0&-\frac{8}{3}&0&2&-\frac{2}{3}&\frac{2}{3} \\[0.5mm]
     0&0&3&1&-1&-1\\0&0&0&\frac{7}{6}&-\frac{1}{6}&-\frac{7}{6} \\[0.5mm] 
     0&0&0&0&-\frac{22}{7}&0\\0&0&0&0&0&-2 
     \end{rmatrix} = \mL\mU. \loppu
\]
\end{Exa} 

Matriisin $LU$-hajotelma siis onnistuu Gaussin eliminaatiolla, jos 
$\,a_{kk}^{(k-1)} \neq 0, \ k=1 \ldots n$. Tällöin $\mL$ ja $\mU$ ovat säännöllisiä 
(Lause \ref{kolmiomatriisi}), joten $\mA$ on säännöllinen ja
\[
\inv{\mA} = \inv{\mU} \inv{\mL}.
\]
Mainittu tukialkioita koskeva ehto on siis riittävä ehto matriisin säännöllisyydelle. 
Seuraavassa luvussa nähdään, että kun Gaussin algoritmia hieman muunnellaan, se ratkaisee 
neliömatriisin säännöllisyyskysymyksen yleisessäkin tapauksessa. 

\subsection*{Käänteismatriisin laskeminen}

Sikäli kuin Gaussin eliminaatio yhtälöryhmälle $\mA \mx = \mb$ onnistuu edellä kuvatulla tavalla
(tukialkiot $\neq 0$), niin kerroinmatriisi on siis säännöllinen, jolloin yhtälöryhmän ratkaisu
voidaan kirjoittaa $\mx=\mA^{-1}\mb$. Jos halutaan laskea myös käänteismatriisi $\mA^{-1}$
eikä vain ratkaisua $\mx$, niin eräs (etenkin käsinlaskussa luonteva) menettely on soveltaa
Gaussin algoritmia \pain{s}y\pain{mbolisesti} siten, että $\mb$:n alkioille $\,b_i$ ei anneta
numeroarvoja. Algoritmia käyttäen ratkaisu saadaan tällöin muotoon $\mx = \mB \mb$, jolloin 
on $\mA^{-1}=\mB$.
\begin{Exa} Laske matriisin
\[ 
\mA\ =\ \begin{rmatrix} 1&1&0 \\ 1&0&1 \\ 1&-1&-1 \end{rmatrix} 
\]
käänteismatriisi. 
\end{Exa}
\ratk Eliminaatiolla ja takaisinsijoituksella saadaan
\begin{align*}
&\begin{rmatrix} 1&1&0 \\ 1&0&1 \\ 1&-1&-1 \end{rmatrix} 
 \begin{rmatrix} x_1 \\ x_2 \\ x_3 \end{rmatrix}\ =\ 
 \begin{rmatrix} b_1 \\ b_2 \\ b_3 \end{rmatrix} \\[5mm] \map \quad 
&\begin{rmatrix} 1&1&0 \\ 0&-1&1 \\ 0&0&-3 \end{rmatrix} 
 \begin{rmatrix} x_1 \\ x_2 \\ x_3 \end{rmatrix}\ =\ 
 \begin{cmatrix}  b_1 \\ -b_1 + b_2 \\ 3b_1 - 2b_2 + b_3 \end{cmatrix} \\[5mm] \map \quad
&\begin{rmatrix} 1&0&0 \\ 0&1&0 \\ 0&0&1 \end{rmatrix}   
 \begin{rmatrix} x_1 \\ x_2 \\ x_3 \end{rmatrix}\ =\ 
 \dfrac{1}{3} \begin{cmatrix}  
              b_1 + b_2 + b_3 \\ 2b_1 - b_2 - b_3 \\ -b_1 + 2b_2 - b_3 
              \end{cmatrix}.
\end{align*}
Tuloksesta voidaan lukea käänteismatriisi:
\[ 
\begin{rmatrix} x_1 \\ x_2 \\ x_3 \end{rmatrix}\
    =\ \dfrac{1}{3} \begin{rmatrix} 1&1&1 \\ 2&-1&-1 \\ -1&2&-1 \end{rmatrix}
                    \begin{rmatrix} b_1 \\ b_2 \\ b_3 \end{rmatrix}
    =\ \mA^{-1}\mb. \loppu
\]
Edellistä suoraviivaisempi tapa laskea käänteismatriisi on käyttää hyväksi matriisialgebran
identieettiä (vrt.\ Luku \ref{matriisialgebra})
\[
\mA^{-1}\ =\ \mA^{-1} \mI\ =\ \mA^{-1}\,[\me_1 \ldots \me_n]\ 
                           =\  [\mA^{-1}\me_1 \ldots \mA^{-1}\me_n].
\]
Tämän mukaisesti $\mA^{-1}$:n sarakkeet voidaan laskea ratkaisemalla lineaarinen yhtälöryhmä
$\mA \mx = \mb$ kun $\mb = \me_i,\ i = 1 \ldots n$. Koska Gaussin algoritmissa voidaan käsitellä
yhtä aikaa useita eri vektoreita $\mb$ (vrt.\ Esimerkki \ref{ristikkoesimerkki}), voidaan myös 
$\mA^{-1}$:n sarakkeet määrätä kaikki yhdellä kertaa. Käänteismatriisin laskeminen voidaan 
tällöin kuvata kaksivaiheisena operaationa (eliminaatio ja takaisinsijoitus)
\[
 [\,\mA \mid \mI\ ]\ \map\ [\,\mU \mid \mL^{-1}\,]\ \map\ [\,\mI \mid \mA^{-1}\ ].
\]
Vastaavalla tavalla menetellään itse asiassa em.\ symbolisessa laskussa, sillä kun kirjoitetaan
$\mb = \sum_{i=1}^n b_i \me_i$, niin pätee (vrt.\ Luku \ref{matriisialgebra})
\[
\mA^{-1} \mb\ =\ \sum_{i=1}^n b_i \mA^{-1} \me_i\ =\ [\mA^{-1} \me_1 \ldots \mA^{-1} \me_n]\,\mb.
\]

\jatko \begin{Exa} (jatko) Esimerkin lasku toisin organisoituna:
\begin{align*}
[\,\mA \mid \mI\ ]\ 
    &=\  \left[ \begin{array}{rrr} 1&1&0 \\ 1&0&1 \\ 1&-1&-1 \end{array} \left\vert 
                 \begin{array}{rrr} 1&0&0 \\ 0&1&0 \\ 0&0&1 \end{array} \right. \right] \quad 
     \map \quad 
         \left[ \begin{array}{rrr} 1&1&0 \\ 0&-1&1 \\ 0&0&-3 \end{array} \left\vert
                \begin{array}{rrr} 1&0&0 \\ -1&1&0 \\ 1&-2&1 \end{array} \right. \right] \\[5mm]
    &\,\map \quad 
         \left[ \begin{array}{rrr} 1&0&0 \\[0.5mm] 0&1&0 \\[0.5mm] 0&0&1 \end{array} \left\vert
                \begin{array}{rrr} \frac{1}{3} & \frac{1}{3} & \frac{1}{3} \\[0.5mm] 
                                   \frac{2}{3} & -\frac{1}{3} & -\frac{1}{3} \\[0.5mm]
                                  -\frac{1}{3} & \frac{2}{3} & -\frac{1}{3} \end{array} \right.
         \right]\ =\ [\,\mI \mid \mA^{-1}\ ]. \quad \loppu
\end{align*}
\end{Exa}
Käänteismatriisi voidaan myös laskea käyttäen $LU$-hajotelmaan perustuvaa kaavaa
$\mA^{-1}=\mU^{-1}\mL^{-1}$, eli laskemalla ensin ym.\ tavalla erikseen käänteismatriisit 
$\mL^{-1}$ ja $\mU^{-1}$ ja sitten näiden tulo. Sekä tällä tavoin että ym.\ algoritmin 
mukaisesti laskien tulee työmääräksi $W=n^3+\ordoO{n^2}$ 
(Harj.teht.\,\ref{H-m-3: työmääriä 2}) --- siis noin kolminkertainen työmäärä verrattuna
$LU$-hajotelman laskemiseen tai yksittäisen yhtälöryhmän ratkaisuun. Osoittautuukin, että 
yhtälöryhmiä numeerisesti ratkaistaessa ei käänteismatriisin laskeminen yksinkertaisesti 
kannata. Nimittäin vaikka ratkaistaisiin useita yhtälöryhmiä kerroinmatriisin 
pysyessä samana (esim.\ vaihteleva kuormitus ristikkorakenteessa, vrt.\ Luku 
\ref{yhtälöryhmät}), niin Gaussin algoritmin eliminaatiovaiheen toistaminen pystytään
välttämään pelkän $LU$-hajotelman avulla: Kun hajotelman matriisit $\mL$ ja $\mU$ tallennetaan, 
niin yhtälöryhmä $\mA \mx = \mb$ voidaan purkaa kahdeksi peräkkäiseksi yhtälöryhmäksi:
\[ 
\mL\mU\mx=\mb \qekv \mL \my = \mb, \quad \mU \mx = \my. 
\]
Nämä ratkeavat eteen- ja takaisinsijoituksilla, jolloin yhtälöryhmän ratkaisemisen työmääräksi
tulee $W \sim n^2$ (ks.\ Harj.teht.\,\ref{H-m-3: työmääriä 1}b). Jos $\mA^{-1}$ olisi
tallennettu, olisi työmäärä sama, joten käänteismatriisin laskemisella ei voiteta mitään (!). 

\Harj
\begin{enumerate}


\item
Kirjoita seuraavat yhtälöryhmät taulukkomuotoon ja ratkaise Gaussin algoritmilla. Laske myös
kerroinmatriisien $LU$-hajotelmat.
\begin{align*}
&\text{a)}\ \ \begin{cases} \,x+y=33 \\ \,2x+4y=100 \end{cases} \qquad\qquad
 \text{b)}\ \ \begin{cases} \,x+ay=1 \\ \,ax+y=2 \end{cases} \quad (a \neq \pm 1) \\[2mm]
&\text{c)}\ \ \begin{cases} 
              \,x_1+3x_2-x_3=4 \\ \,-x_1+x_2+2x_3=7 \\ \,2x_1+6x_2+2x_3=20
              \end{cases} \quad
 \text{d)}\ \ \begin{cases}
              \,x_1+2x_2+3x_3=1 \\ \,x_1+x_2-x_3=2 \\ \,x_1-6x_3=4
              \end{cases} \\[2mm]
&\text{e)}\ \ \begin{cases} 
              \,x_1+x_2+2x_3=-1 \\ \,-x_1+x_2=1 \\ \,3x_1-x_2+3x_3=-3
              \end{cases} \quad\,
 \text{f)}\ \ \begin{cases}
              \,x_1-x_2+4x_3=8 \\ \,x_1-2x_2+3x_3=0 \\ \,-2x_1+4x_2-5x_3=8
              \end{cases} \\[2mm]
&\text{g)}\ \ \begin{cases}
              \,2x_1-x_2=1 \\ \,-x_1+2x_2-x_3=2 \\ \,-x_2+2x_3-x_4=-1 \\ \,-x_3+2x_4=1
              \end{cases} \quad\ \
 \text{h)}\ \ \begin{cases}
              \,2x_1-x_2-x_4=-4 \\ \,-x_1+2x_2-x_3=1 \\ \,-x_2+2x_3-x_4=4 \\ \,-x_1-x_3+3x_4=10
              \end{cases}
\end{align*}

\item
Olkoon
\[
\mA   = \begin{bmatrix} 1&2&3&1\\2&1&1&1\\1&2&1&0\\0&1&1&2 \end{bmatrix}, \quad
\mb_1 = \begin{bmatrix} 1\\1\\1\\1 \end{bmatrix}, \quad
\mb_2 = \begin{bmatrix} 5\\3\\4\\0 \end{bmatrix}
\]
Ratkaise Gaussin algoritmilla yhtälöryhmät $\mA\mx=\mb_1$ ja $\mA\mx=\mb_2$ ja laske matriisin
$\mA$ $LU$-hajotelma.

\item
Olkoon
\[
\mA=\begin{bmatrix} 1&0&0&1\\1&1&0&2\\1&0&1&0\\1&1&2&1 \end{bmatrix}.
\]
Laske käänteismatriisi $\mA^{-1}$ Gaussin algoritmilla \vspace{1mm}\newline 
a) ratkaisemalla yhtälöryhmä $\mA\mx=\mb$ yleisellä $\mb\in\R^4$ (symbolinen lasku), \newline
b) ratkaisemalla taulukkomuodossa yhtälöryhmät $\mA\mx=\me_i,\ i=1 \ldots 4$

\item
Näytä, että jos lineaarisen yhtälöryhmän kerroimatriisi $\mA$ on symmetrinen, niin Gaussin 
algoritmi säilyttää symmetrisenä sen matriisin osan, johon eliminaatio ei ole vielä edennyt,
ts.\ $(k-1)$:n eliminaatioaskeleen jälkeen pätee
\[
a_{ji}^{(k-1)}=a_{ij}^{(k-1)} \quad \forall\,i,j \ge k.
\]
Päättele, että Gaussin algoritmin vaatima laskutyö on symmetrisen kerroinmatriisin tapauksessa
$W=\frac{1}{6}\,n^3+\ordoO{n^2}$.

\item \label{H-m-3: työmääriä 1}
Olkoon $\mA,\mB$ matriiseja kokoa $n \times n$, $\mL,\mU$ ala- ja yläkolmiomatriiseja samaa
kokoa ja $\mb\in\R^n$. Näytä oikeaksi seuraavat laskuoperaatioiden työmääriä koskevat arviot
(työyksikkö = kertolasku + yhteenlasku).
\begin{align*}
\text{a)} \qquad\qquad\qquad &\mA,\mb\,\map\ \mA\mb \qquad\qquad\qquad\, n^2+\ordoO{n} \\
\text{b)} \qquad\qquad\qquad &\mL,\mb\ \map\ 
                              \mL^{-1}\mb \qquad\qquad\quad\ \tfrac{1}{2}\,n^2+\ordoO{n} \\
\text{c)} \qquad\qquad\qquad &\mA,\mB \map\ \mA\mB \qquad\qquad\qquad   n^3+\ordoO{n^2} \\
\text{d)} \qquad\qquad\qquad &\mL,\mU\,\map\ 
                              \mL\mU \qquad\qquad\qquad  \tfrac{1}{3}\,n^3+\ordoO{n^2}
\end{align*}

\item (*) \label{H-m-3: työmääriä 2}
Näytä, että laskuoperaation $\mA\map\mA^{-1}$ työmäärä Gaussin algoritmin eri vaihtoehdoissa on
($\mA$ kokoa $n \times n$, työyksikkö = kertolasku + yhteenlasku)
\begin{align*}
\text{a)} \qquad &[\,\mA \mid \mI\ ]\ \map\ [\,\mI \mid \mA^{-1}\ ] \qquad n^3+\ordoO{n^2} \\
\text{b)} \qquad &\mA\map\mA^{-1}\ = \mU^{-1}\mL^{-1} \qquad\, n^3+\ordoO{n^2}
\end{align*}

\end{enumerate}

