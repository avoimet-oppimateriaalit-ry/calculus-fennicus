\subsection*{Muuntosuhde}

Tutkitaan muunnettua integraalia $\int_B g\,d\mu'$. Oletetaan, että $B$ on rajoitettu ja
$g(\mt)$ rajoitettu $B$:ssä, jolloin integraalin perusmääritelmän mukaan on
$\int_B g\,d\mu'=\int_T g_0\,d\mu'$, missä $T \supset B$ on $n$-ulotteinen suorakulmainen
perussärmiö ja $g_0=g$:n nollajatko $B$:n ulkopuolelle. Olkoon edelleen 
$\mathcal{T}_h=\{T_k\}\,$ $T$:n jako osasärmiöihin, joiden särmien pituus on enintään $h$.
Jatkossa erotetaan $T$:stä nollamittainen osajoukko $S$ ja merkitään
$T'=\{\,\mt \in T\ |\ \mt \notin S\,\}$ ja
$\mathcal{T}_h'= \{\,T_k\in\mathcal{T}_h\ |\ T_k \subset T'\,\}$ (mahdollisesti $S=\emptyset$,
jolloin $T'=T$ ja $\mathcal{T}_h'=\mathcal{T}_h$). Oletetaan $\mpu$ määritellyksi
kuvauksena $\mpu: T \kohti \mpu(T) \supset A$ ja injektiiviseksi osajoukkoon $T'$ rajoitettuna.
Merkitään $\Delta A_k=\mpu(T_k)$.
\begin{figure}[H]
\begin{center}
\import{kuvat/}{kuvaUint-27.pstex_t}
\end{center}
\end{figure}
Jatkossa tehdään vielä osasärmiöitä $T_k\in\mathcal{T}_h'$ (ja vain näitä!) koskien kaksi
oletusta: 

1. Oletetaan, että $g_0$ on särmiöissä $T_k\in\mathcal{T}_h'$ likimain vakio, tarkemmin
\[
\max_{\mt_1,\mt_2 \in T_k}|g_0(\mt_1)-g_0(\mt_2)| \,\le\, \eps(h) \kohti 0, \quad 
                                                        \text{kun}\,\ h \kohti 0. 
\]
Tässä oletetaan, että $\eps(h)$ riippuu vain tiheysparametrista $h$ (ei $k$:sta tai muulla
tavoin $\mathcal{T}_h$:sta). (Kyseessä on tasaisen jatkuvuuden ehto, vrt.\
Luku \ref{jatkuvuuden logiikka}.)

2. Oletetaan, että kuvaukseen $\mpu$ on liitettävissä
\index{muuntosuhde integraalissa}%
nk.\ \kor{muuntosuhde} (mittasuhde, suurennussuhde) funktiona $J(\mt),\ \mt \in T'$, jolle
pätee jokaisella $T_k\in\mathcal{T}_h'$
\[
\max_{\mt \in T_k}\left|\frac{\mu(\Delta A_k)}{\mu(T_k)}-J(\mt)\right| 
   \,\le\,\delta(h) \kohti 0 \quad \text{kun}\,\ h \kohti 0.
\]
Myös tässä oletetaan, että $\delta(h)$ riippuu vain $h$:sta.

Kun valitaan $\mt_k \in T_k$ jokaisella $T_k\in\mathcal{T}_h'$, niin tehtyjen oletusten
perusteella voidaan arvioida
\begin{align}
\int_B g\,d\mu' &= \int_T g_0\,d\mu' \tag{a} \\
                &= \sum_{T_k\in\mathcal{T}_h} \int_{T_k} g_0\,d\mu' \tag{b} \\
                &\approx \sum_{T_k\in\mathcal{T}_h'} \int_{T_k} g_0\,d\mu' \tag{c} \\
                &\approx \sum_{T_k\in\mathcal{T}_h'} g_0(\mt_k)\mu'(T_k) \tag{d} \\
                &\approx \sum_{T_k\in\mathcal{T}_h'} g_0(\mt_k)J(\mt_k)\mu(T_k) \tag{e} \\
                &\approx \sum_{T_k\in\mathcal{T}_h'} \int_{T_k} g_0(\mt)J(\mt)\,d\mu \tag{f} \\
                &\approx \sum_{T_k\in\mathcal{T}_h} \int_{T_k} g_0(\mt)J(\mt)\,d\mu \tag{g} \\
                &= \int_T g_0(\mt)J(\mt)\,d\mu \tag{h} \\
                &= \int_B g(\mt)J(\mt)\,d\mu. \tag{i}
\end{align}
Tässä on vedottu integraalin määritelmään (a,i), integraalin additiivisuuteen (b,h),
oletukseen: $T=T' \cup S$, missä $S$ on nollamittainen (c,g) ja em.\ oletuksiin koskien
funktioita $g_0(\mt)$ ja $J(\mt)$ (d,e,f). Sikäli kuin näiden approksimaatioiden virhe
$\kohti 0$, kun $h \kohti 0$, niin ko.\ rajalla päädytään 
muunnoskaavaan\footnote[2]{Muunnoskaavan täsmälliset perustelut sivuutetaan. Kaava edellyttää
tehtyjen oletuksien lisäksi, että $g(\mt)J(\mt)$ on integroituva yli $B$:n Jordan-mitan $\mu$
suhteen. --- Muuntosuhde on määriteltävissä aina, kun kuvaus $\mpu$ on riittävän säännöllinen,
ks.\ laskukaava jäljempänä.}
\begin{equation} \label{muuntokaava}
\boxed{\kehys\quad \int_A f(\mx)\,dx_1\ldots dx_n=\int_B g(\mt)J(\mt)\,dt_1\ldots dt_n. \quad}
\end{equation}