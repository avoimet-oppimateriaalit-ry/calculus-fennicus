Aiemmin Luvussa \ref{pinta-ala ja kaarenpituus} johdettiin k�yr�n $y=f(x)$ ja $x$-akselin
rajaaman alueen pinta-alalle v�lill� $[a,b]$ laskukaava m��r�tyn integraalin avulla.
Varmistetaan nyt, ett� t�m� kaava on edelleen p�tev�.
\begin{Lause} Jos $f$ on m��ritelty, rajoitettu, ja ei-negatiivinen v�lill� $[a,b]$, niin
$\,A = \{(x,y) \in \R^2 \mid x \in [a,b]\ \ja\ 0 \le y \le f(x)\}$
on Jordan-mitallinen t�sm�lleen kun $f$ on Riemann-integroituva v�lill� $[a,b]$, jolloin
$\,\mu(A)=\int_a^b f(x)\,dx$.
\end{Lause}
\tod Olkoon $T = [a,b] \times [0,c]$, miss� $c \ge f(x)\ \forall x \in [a,b]$, jolloin 
$T \supset A$, ja olkoon $\mathcal{T}_h$ jokin $T$:n jako suorakulmioihin 
$T_{kl} = [x_{k-1},x_k] \times [y_{l-1},y_l]$, $k = 1 \ldots m,\ l = 1 \ldots n$. Olkoon t�ss� 
edelleen jakopisteist� $X = \{x_k\}$ kiinnitetty, ja tutkitaan, miten pisteist�n $Y = \{y_l\}$ 
valinta vaikuttaa ala- ja yl�summiin 
$\underline{\sigma}(\chi_A,\mathcal{T}_h),\ \overline{\sigma}(\chi_A,\mathcal{T}_h)$. N�hd��n
helposti, ett�
\[ 
\sup_Y \underline{\sigma}(\chi_A,\mathcal{T}_h) = \underline{\sigma}(f,X), \quad 
\inf_Y \overline{\sigma}(\chi_A,\mathcal{T}_h) = \overline{\sigma}(f,X), 
\]
miss� $\underline{\sigma}(f,X),\ \overline{\sigma}(f,X)$ ovat funktion $f$ ja pisteist�n $X$
m��r��m��n v�lin $[a,b]$ jakoon liittyv�t Riemannin ala- ja yl�summat
(vrt.\ Luku \ref{riemannin integraali}). T�st� voidaan p��tell�, ett�
\[ 
\underline{\mu}(A) = \underline{\int_a^b} f(x)\,dx, \quad 
\overline{\mu}(A) = \overline{\int_a^b} f(x)\,dx \qimpl \text{v�ite.} \loppu 
\]