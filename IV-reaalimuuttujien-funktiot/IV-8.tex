\section{*Funktioavaruus} \label{funktioavaruus}
\alku

Tarkastellaan \pain{samassa} joukossa $A$ määriteltyjä yhden, kahden tai kolmen reaalimuutujan
reaaliarvoisia funktioita ja merkitään näiden joukkoa $V$:llä:
\[ 
V\ =\ \{\,\text{funktiot}\ f:\ A \map \R\,\}, 
\]
Joukossa $V$ on määritelty funktioiden yhteenlasku $f,g \map f+g$ ja skalaarilla kertominen
$f \map \lambda f$ aiemmin kerrotulla tavalla (Määritelmän \ref{funktioiden yhdistelysäännöt}
säännöt (1) ja (2). kun $A \subset \R$). Näiden laskuoperaatioiden perusteella $V$ on
mahdollista tulkita vektoriavaruudeksi. Samassa joukossa määritellyt funktiot voidaan siis 
mieltää 'vektoreiksi' (!), jolloin puhutaan \kor{funktioavaruudesta} (engl.\ function space). 
Tällaisen avaruuden nolla-alkio on nk.\ (algebrallinen) \kor{nollafunktio}, joka määritellään
\[ 
\mathbf{0}(x) = 0\,\ \forall x \in A .
\]
Äärellinen funktiojoukko $\{ f_i,\ i = 1 \ldots n\} \subset V$ on (albegrallisesti) 
\kor{lineaarisesti riippumaton}, jos pätee
\[ 
\sum_{i=1}^n \lambda_i f_i = \mathbf{0} \qimpl \lambda_i = 0,\ i = 1 \ldots n. 
\]
\begin{Exa} Enintään kolmannen asteen polynomit (määrittelyjoukko $= \R$) muodostavat 
funktioiden algebrallisten yhdistelysääntöjen perusteella funktioavaruuden
\[
V\ =\ \{ f = \lambda_1 f_1 + \lambda_2 f_2 + \lambda_3 f_3 + \lambda_4 f_4 
                                                                   \mid \lambda_i \in \R \},
\]
missä $f_1(x) = 1$ ('ykkösfunktio'), $f_2(x) = x$ (identiteettifunktio), $f_3(x) = x^2$, ja 
$f_4(x) = x^3$. Algebran peruslauseesta (vrt.\ Luku \ref{III-3}) on helposti pääteltävissä, 
että funktiot $f_i$ ovat lineaarisesti riippumattomia, joten $\{f_i,\ i = 1 \ldots 4\}$ on 
$V$:n kanta. Siis $V$ on neliulotteinen vektoriavaruus: dim $V = 4$. \loppu \end{Exa}
\begin{Exa} Funktiot muotoa $f(x)=c_1\sin x+c_2\cos x,\ c_1,c_2\in\R$ muodostavat 2-ulotteisen
vektoriavaruuden, luonnollisena kantana $\{\sin x,\cos x\}$. \loppu
\end{Exa}
\begin{Exa} Funktioavaruus
\[
V=\{f(x)=c_1+c_2\cos^2 x+c_3\sin^2 x,\ c_i\in\R\}
\]
ei ole 3-ulotteinen, sillä funktiosysteemi $\{f_1\,,f_2\,,f_3\}=\{1,\cos^2 x,\sin^2 x\}$ on 
lineaarisesti riippuva:
\[
f_1-f_2-f_3=\mathbf{0}.
\]
Mikä tahansa pari mainitusta kolmesta funktiosta sen sijaan on lineaarisesti riippumaton,
joten $V$ on 2-ulotteinen, kantana esim.\ $\{1,\cos^2 x\}$. Eräs $V$:n 1-ulotteinen aliavaruus
on
\[
W=\{\lambda\cos 2x \mid \lambda\in\R\},
\]
sillä $\,\cos 2x=2f_2-f_1 \in V$. \loppu
\end{Exa}
\begin{Exa} \label{polynomiavaruus}
Funktioavaruus
\[
V=\{f(x,y)=c_1+c_2x+c_3y+c_4x^2+c_5xy+c_6y^2,\ c_i\in\R\}
\]
koostuu kahden muuttujan polynomeista enintään astetta $2$ ja tämän aliavaruus
\[
W=\{f(x,y)=c_1+c_2x+c_3y,\ c_i\in\R\}
\]
enintään ensimmäisen asteen polynomeista. Funktiosysteemi $\{1,x,y,x^2,xy,y^2\}$ on 
lineaarisesti riippumaton (Harj.teht. \ref{H-IV-8: polynomit}), joten dim $V=6$ ja dim $W=3$.
\loppu
\end{Exa} 

\Harj
\begin{enumerate}

\item
Näytä, että funktiot $f_1(x)=x$ ja $f_2(x)=\abs{x}$ ovat lineaarisesti riippumattomat, jos
määrittelyjoukkona on väli $[-1,1]$ ja lineaarisesti riippuvat, jos määrittelyjoukko on
väli $[0,1]$.

\item
Montako alkiota on joukossa $A\subset\R$ oltava, jotta $A$:ssa määritellyt funktiot
$f_i(x)=x^{i-1},\ i=1 \ldots n$ ovat lineaarisesti riippumattomat?

\item
a) Näytä, että $\{1,\sin x,\sin^2 x,\sin^3 x\}$ on erään 4-ulotteisen funktioavaruuden
$V$ kanta (funktioiden määrittelyjoukko $=\R$). \ b) Sikäli kuin 
$f(x)=2+\cos 2x-\sin 3x \in V$, määrää $f$:n koordinaatit (kertoimet) mainitussa kannassa.

\item \label{H-IV-8: polynomit}
Näytä, että Esimerkin \ref{polynomiavaruus} polynomiavaruus $V$ on 6-ulotteinen, ts.\ näytä, että
$c_1+c_2x+c_3y+c_4x^2+c_5xy+c_6y^2=0\,\ \forall\,(x,y)\in\Rkaksi\,\ \impl\,\ c_1 = \ldots 
=c_6=0$.

\item (*) Olkoon $V=\{f(x)=c_1+c_2x+c_3x^2 \mid c_i\in\R\}$ ja määritellään
\[
\scp{f}{g}=f(0)g(0)+f(1)g(1)+f(2)g(2), \quad f,g \in V.
\]
Näytä, että $f,g \map \scp{f}{g}$ on $V$:n skalaaritulo, ja laske funktion $u(x)=2+3x-2x^2$ 
ortogonaaliprojektio $w$ (ko.\ skalaaritulon mielessä) $V$:n aliavaruuteen $W$, jonka kanta
on $\{1,x\}$. Piirrä samaan kuvaan funktioiden $u$ ja $w$ kuvaajat välillä $[0,2]$. Tarkista,
että $u-w \perp W$, ts.\ $\scp{u-w}{v}=0\ \forall v \in W$.

\end{enumerate}