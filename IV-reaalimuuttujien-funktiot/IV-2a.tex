%Joukko-opillinen käänteisfunktio
Reaalifunktioiden algebran operaatioista on edellisessä luvussa esitelty funktioiden
'ketjutus' yhdistetyksi funktioksi sekä yhdistely laskuoperaatioiden avulla. Kolmas
keskeinen reaalifunktioiden algebran operaatio on funktion 'kääntäminen' eli
\kor{käänteisfunktion} (engl.\ inverse function) muodostaminen. Tarkastellaan
käänteisfunktion käsitettä aluksi yleisemmässä joukko-opillisessa yhteydessä.

Olkoon $A$ ja $B$ joukkoja ja $f:\ A \kohti B$ on funktio, ts.\ jokaiseen $x \in A$ liittyy
yksikäsitteinen $f(x)=y \in B$ ($A=\DF_f=$ määrittelyjoukko, $B=$ maalijoukko, vrt.\
Luku \ref{trigonometriset funktiot}). Sanotaan, että $f$ on 1-1 ('yksi yhteen', engl.\
one to one) eli \kor{kääntyvä} (engl.\ invertible) eli \kor{injektio} (injektiivinen),
jos pätee
\[
\forall x_1,x_2 \in A\ [\,x_1 \neq x_2\ \impl\ f(x_1) \neq f(x_2)\,].
\]
Tällä ehdolla jokaiseen $y \in f(A)=\RF_f$ liittyy y\pain{ksikäsitteinen} $x \in A$ siten,
että $f(x)=y$. Kun liittyminen merkitään $y \map x$, niin kyseessä on siis funktio, jonka
määrittelyjoukko on $f(A)=\RF_f$ ja arvojoukko on $A=\DF_f$. Tätä sanotaan $f$:n 
käänteisfunktioksi, merkitään $\inv{f}$ ja luetaan '$f$ miinus 1' (engl $f$ inverse). 
Käänteisfunktio määritellään siis laskusäännöllä (liittämissännöllä) 
\[ 
x=\inv{f}(y)\,\ \ekv\,\ y=f(x) \quad (y \in f(A)).
\]  
\begin{figure}[H]
\begin{center}
\import{kuvat/}{kuvaII-5.pstex_t}
\end{center}
\end{figure}
\begin{Exa} Olkoon $A=\{1,2,3\}\subset\N$ ja $B=\{4,5,6\}\subset\N$ ja määritellään funktio 
$f:\ A \kohti B$ asettamalla
\[
f(1)=6, \quad f(2)=4, \quad f(3)=5.
\]
Tällöin $f$ on kääntyvä ja
\[
\inv{f}(4)=2, \quad \inv{f}(5)=3, \quad \inv{f}(6)=1.
\]
Joukko-opillisesti ilmaistuna on (vrt.\ Luku \ref{trigonometriset funktiot})
\begin{align*}
f\       &=\ \{(1,6),(2,4),(3,5)\}\ \subset\ A \times B \\
\inv{f}\ &=\ \{(6,1),(4,2),(5,3)\}\ =\ \{(4,2),(5,3),(6,1)\}\ \subset\ B \times A \loppu
\end{align*}
\end{Exa}
\jatko \begin{Exa} (jatko) Jos $f: A \kohti B$ määritellään asettamalla
\[
f(1)=4, \quad f(2)=5, \quad f(3)=4,
\]
niin $f$ ei ole kääntyvä, koska $f(1)=f(3)$. Tässä tapauksessa käänteisfunktiota ei siis ole.
\loppu \end{Exa}
\begin{figure}[H]
\begin{center}
\import{kuvat/}{kuvaII-6.pstex_t}
\end{center}
\end{figure}
\begin{Exa} Luonnollisia lukuja koskevien Peanon aksioomien (ks.\ alaviite Luvussa
\ref{ratluvut}) mukaan jokaiseen $x\in\N$ on liitettävissä nk.\ seuraaja $x'\in\N$.
Tässä on itse asiassa kyseessä funktio (seuraajafunktio), sillä jos merkitään $x'=f(x)$, niin
aksioomat (P2)--(P2) (ks.\ Luku \ref{ratluvut}) voidaan tulkita seuraavasti: (P2) $f$ on
funktio (seuraaja yksikäsitteinen) ja $\DF_f=\N$ (jokaisella $x\in\N$ on seuraaja).
(P3) $1\not\in\RF_f$ ($1$ ei ole minkään $x\in\N$ seuraaja). (P4) $f$ on 1-1. 
Aksiooman (P4) mukaan siis seuraajafunktiolla on käänteisfunktio. Tätä voisi kutsua
'edeltäjäfunktioksi'. \loppu
\end{Exa}

\subsection*{Kääntäen yksikäsitteinen vastaavuus}

Jos $f: A \kohti B$ on 1-1 ja $\RF_f=B$, niin sanotaan, että $f$ luo \kor{kääntäen 
yksikäsitteisen vastaavuuden} $A$:n ja $B$:n välille. Tällä tarkoitetaan yksinkertaisesti,
että jokaista $x \in A$ vastaa yksikäsitteinen $y \in B$, kun asetetaan $y=f(x)$, ja kääntäen,
jokaista $y \in B$ vastaa yksikäsitteinen $x \in A$, kun asetetaan $x=\inv{f}(y)$. Tällaista
vastaavuutta on merkitty aiemmin ueassa yhteydessä symbolilla '$\vast$'
(esim.\ $\Ekaksi \vast \Rkaksi$) tai yksittäisten alkioiden tapauksessa symbolilla '$\vastaa$'
(esim.\ $P\in\Ekaksi\,\vastaa\,\vec r=\Vect{OP} \in V$). 