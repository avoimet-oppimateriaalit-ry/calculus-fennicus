\section{Implisiittifunktiot} \label{implisiittifunktiot} 
\alku

Jos $F$ on kahden reaalimuuttujan funktio, niin Euklidisen tason pistejoukko
\begin{equation}  \label{joukko S}
S=\{P \vastaa (x,y) \in \R^2 \mid F(x,y)=0\}
\end{equation}
on yleensä jokin käyrä. Muitakin mahdollisuuksia on: Joukko $S$ voi myös koostua useista 
erillisistä käyristä tai käyrän 'pätkistä'. Se voi myös sisältää eristettyjä pisteitä, tai olla
pelkästään erillisistä pisteistä koostuva, tai $S$ voi olla tyhjä joukko (vrt.\ käyrää koskeva
alaviite edellisessä luvussa).
\begin{Exa}
Jos $a \in \R$, niin joukko
\[ 
S = \{P \vastaa (x,y) \in R^2 \mid x^2+y^2+a = 0\} 
\]
on ympyräviiva jos $a<0$, piste jos $a=0$, ja tyhjä joukko jos $a>0$. \loppu 
\end{Exa}
Olkoon $S$ määritelty kaavalla \eqref{joukko S}. Halutaan selvittää, millainen $S$ on 
rajoitettuna  euklidisen tason 'viipaleeseen'
\[
E_A=\{P \vastaa (x,y) \in \R^2 \mid x \in A\},
\]
missä $A \subset \R$ on jokin \pain{väli}. Oletetaan ensiksi, että $S\cap E_A$ on esitettävissä
muodossa
\[
S\cap E_A =\{P \vastaa (x,y) \in \R^2 \mid x \in A \ \ja \ y=f(x)\},
\]
missä $\,f:A \kohti \R$ on (yhden reaalimuuttujan) funktio. Sanotaan tällöin, että yhtälö 
$F(x,y)=0$ (tai pistejoukko $S$) määrittelee välillä $A$ \kor{implisiittifunktion} $f$. 
Implisiittifunktio $f$ on siis määritelmän mukaan sellainen, että
\[
F(x,f(x))=0 \ \ \forall x \in A, 
\]
ja lisäksi pätee
\[ 
(x,y) \in A \times \R\ \ja\ F(x,y) = 0 \qimpl y = f(x), 
\]
ts.\ yhtälö $F(x,y)=0$ ratkeaa $y$:n suhteen y\pain{ksikäsitteisesti} muotoon $y = f(x)$, 
kun $x\in A$.

Yksinkertainen esimerkki implisiittifunktiosta on Luvussa \ref{käänteisfunktio} käsitelty 
käänteisfunktio, jonka määritelmä voidaan esittää muodossa
\[
F(x,y)=x-f(y)=0 \ \ekv \ y=\inv{f}(x).
\]
--- Itse asiassa myös tavallinen funktio muuttuu 'implisiittiseksi' (suom.\ 'epäsuoraksi', 
'peitetyksi'), kun $y=f(x)$ kirjoitetaan muotoon $F(x,y)=f(x)-y=0$.
\begin{Exa}
Tason suoran perusmuotoinen yhtälö
\[
ax+by+c=0
\]
voidaan tulkita implisiittifunktion $y=f(x)$ määrittelyksi, kun $b\neq 0$. Tällöin on siis 
$f(x) = -(ax+c)/b$, $D_f = R_f = \R$. \loppu
\end{Exa}
\begin{Exa} Yhtälö 
\[
F(x,y) = \sin x + y + y^5 = 0
\]
määrittelee implisiittifunktion, joka voidaan esittää yhdistettynä funktiona
\[
y = f(x) = \inv{g} \circ (-\sin)(x) = -\inv{g}(\sin x),
\]
missä $g(y) = y^5 + y,\ D_g = \R$ (vrt.\ Esimerkki \ref{algebrallinen käänteisfunktio} 
Luvussa \ref{käänteisfunktio}). Funktion $f$ määrittelyjoukko on $D_f = \R$ ja arvojoukko on 
$R_f = [-a,a]$, missä $a = \inv{g}(1)$, ts.\ $\,a^5 + a = 1$. \loppu 
\end{Exa}

\subsection*{Monihaaraiset implisiittifunktiot}

Implisiittifunktion määrittely-yrityksissä varsin tavallinen on tilanne, että yhtälö $F(x,y)=0$
kyllä ratkeaa $y$:n suhteen tarkasteltavalla välillä $A$, mutta ei yksikäsitteisesti. Oletaan
jatkossa tällainen tilanne. Oletetaan tällöin edelleen, että $S\cap E_A$ on purettavissa osiin
\[
S\cap E_A = \bigcup_{i\in \Lambda} S_i,
\]
missä $\Lambda \subset \N$ (tai $\Lambda \subset \Z$) on jokin indeksijoukko (jossa vähintään
kaksi alkiota) ja jokainen $S_i,\ i \in \Lambda$, on käyrä (vrt.\ alaviite edellisessä luvussa)
muotoa
\[
S_i=\{P \vastaa (x,y) \in \R^2 \mid x \in A \ \ja \ y=f_i(x)\},
\]
missä $f_i: A \kohti \R$ on reaalifunktio. Sanotaan tällöin, että funktioj\pain{oukko}
\[
\mathfrak{f}=\{f_i, \ i \in \Lambda \}
\]
on yhtälön $F(x,y)=0$ (tai pistejoukon $S$) määrittelemä \kor{monihaarainen} (moniarvoinen)
\kor{implisiittifunktio}. Monihaarainen implisiittifunktio ei siis ole funktio

\pagebreak 

vaan sellaisten joukko. Ko.\ funktiojoukon alkioita $f_i$ sanotaan $\mathfrak{f}$:n 
\kor{haaroiksi} (engl. branch).\footnote[1]{Monihaarainen funktio $\mathfrak{f}$ voidaan myös
tulkita \kor{joukkoarvoiseksi} määrittelemällä
\[ 
\mathfrak{f}(x) = \{f_i(x),\ i \in \Lambda\},
\]
missä $f_i,\ i \in \Lambda$ ovat $\mathfrak{f}$:n haarat välillä $A$. Koska $\mathfrak{f}(x)$
on (joukkona) yksikäsitteinen jokaisella $x \in A$, niin $\mathfrak{f}$ on funktio tyyppiä 
$\mathfrak{f}: A \kohti \mathcal{B}$, missä maalijoukko $\mathcal{B}$ on 'joukkojen joukko'
\[ 
\mathcal{B} = \{\,\{y_i, \ i \in \Lambda\} \mid y_i \in \R\ \ \forall i \in \Lambda\,\}. 
\]} 
Indeksijoukko $\Lambda$ voi olla ääretön.
\begin{figure}[H]
\begin{center}
\setlength{\unitlength}{1cm}
\begin{picture}(8,8)(-1,-2)
\put(-1,0){\vector(1,0){8}}\put(6.8,-0.4){$x$}
\put(0,-2){\vector(0,1){8}}\put(0.2,5.8){$y$}
\put(1,0){\drawline(0,-0.1)(0,0.1)} \put(6,0){\drawline(0,-0.1)(0,0.1)}
\put(0.9,-0.5){$a$} \put(5.9,-0.5){$b$}
\put(0,-0.4){\curve(
    1.0000,   -1.1137,
    1.2000,   -1.1013,
    1.4000,   -1.0586,
    1.6000,   -0.9883,
    1.8000,   -0.9004,
    2.0000,   -0.8108,
    2.2000,   -0.7383,
    2.4000,   -0.7012,
    2.6000,   -0.7129,
    2.8000,   -0.7791,
    3.0000,   -0.8952,
    3.2000,   -1.0466,
    3.4000,   -1.2100,
    3.6000,   -1.3572,
    3.8000,   -1.4598,
    4.0000,   -1.4947,
    4.2000,   -1.4487,
    4.4000,   -1.3217,
    4.6000,   -1.1282,
    4.8000,   -0.8954,
    5.0000,   -0.6600,
    5.2000,   -0.4619,
    5.4000,   -0.3379,
    5.6000,   -0.3146,
    5.8000,   -0.4034,
    6.0000,   -0.5976)}
\curve(
    1.0000,    3.0000,
    1.2000,    2.8198,
    1.4000,    2.6562,
    1.6000,    2.5080,
    1.8000,    2.3740,
    2.0000,    2.2531,
    2.2000,    2.1440,
    2.4000,    2.0456,
    2.6000,    1.9566,
    2.8000,    1.8760,
    3.0000,    1.8025,
    3.2000,    1.7349,
    3.4000,    1.6720,
    3.6000,    1.6127,
    3.8000,    1.5558,
    4.0000,    1.5000,
    4.2000,    1.4442,
    4.4000,    1.3873,
    4.6000,    1.3280,
    4.8000,    1.2651,
    5.0000,    1.1975,
    5.2000,    1.1240,
    5.4000,    1.0434,
    5.6000,    0.9544,
    5.8000,    0.8560,
    6.0000,    0.7469)
\put(0,-1.5){
\curve(
    1.0000,    5.0000,
    1.2000,    6.1424,
    1.4000,    6.8307,
    1.6000,    7.1511,
    1.8000,    7.1831,
    2.0000,    6.9988,
    2.2000,    6.6630,
    2.4000,    6.2337,
    2.6000,    5.7614,
    2.8000,    5.2895,
    3.0000,    4.8545,
    3.2000,    4.4854,
    3.4000,    4.2042,
    3.6000,    4.0256,
    3.8000,    3.9574,
    4.0000,    4.0000,
    4.2000,    4.1467,
    4.4000,    4.3836,
    4.6000,    4.6898,
    4.8000,    5.0369,
    5.0000,    5.3898,
    5.2000,    5.7058,
    5.4000,    5.9352,
    5.6000,    6.0213,
    5.8000,    5.8999,
    6.0000,    5.5000)}
\put(6,-1.6){$y=f_1(x)$}
\put(6,1.2){$y=f_2(x)$}
\put(6,4.5){$y=f_3(x)$}
\end{picture}
%\caption{Monihaarainen implisiittifunktio}
\end{center}
\end{figure}

\begin{Exa}
Yhtälö $x^2+y^2=1$ määrittelee välillä $[-1,1]$ kaksihaaraisen implisiittifunktion
\[
\mathfrak f=\{f_+,f_-\},
\]
missä
\[
f_\pm = \pm \sqrt{1-x^2}. \loppu
\]
\end{Exa}
\begin{Exa} Millaisen monihaaraisen implisiittifunktion määrittelee yhtälö \newline
$y^4-2y^2+x^2-x=0$\,?
\end{Exa}
\ratk Jos $x\in\R$, niin mahdolliset ratkaisut $y\in\R$ ovat
\[
y=\pm\sqrt{1\pm\sqrt{1+x-x^2}}.
\]
Tässä on
\begin{align*}
1+x-x^2 &\ge 0 \qekv \frac{1}{2}(1-\sqrt{5}) \le x \le \frac{1}{2}(1+\sqrt{5}) \\
1+x-x^2 &\ge 1 \qekv 0 \le x \le 1
\end{align*}
Päätellään, että väleillä $[\tfrac{1}{2}(1-\sqrt{5}),\,0]$ ja $[1,\,\tfrac{1}{2}(1+\sqrt{5})]$
funktio on kaksihaarainen: $\mathfrak{f}=\{f_1\,,\,f_2\}$, missä
\[
f_1(x)=\sqrt{1+\sqrt{1+x-x^2}}\,, \quad f_2(x)=-\sqrt{1+\sqrt{1+x-x^2}}\,.
\]
Välillä $[0,1]$ funktio on nelihaarainen: $\mathfrak{f}=\{f_1\,,\,f_2\,,\,f_3\,,\,f_4\}$,
missä $f_1$ ja $f_2$ ovat samat kuin edellä ja
\[
f_3(x)=\sqrt{1-\sqrt{1+x-x^2}}\,, \quad  f_4(x)=-\sqrt{1-\sqrt{1+x-x^2}}\,.
\]
Jos $x\in(-\infty,\,\tfrac{1}{2}(1-\sqrt{5})$ tai $x\in(\tfrac{1}{2}(1+\sqrt{5},\,\infty)$,
ei yhtälö toteudu millään $y\in\R$, joten näillä väleillä yhtälö ei määrittele mitään
(reaaliarvoista) funktiota. \loppu 

\subsection*{Funktiot $\arcsin$, $\arccos$, $\arctan$}

Funktiomerkinnöillä $\,\arcsin$, $\arccos$ ja $\arctan$ tarkoitetaan yhtälöiden $x-\sin y =0$,
$x-\cos y=0$ ja $x-\tan y=0$ määrittelemiä äärettömän monihaaraisia funktioita,
määrittelyjoukkona joko väli $[-1,1]$ ($\arcsin,\,\arccos$) tai $\R$ ($\arctan$). Esimerkiksi
$\arctan$ määritellään
\[
\arctan = \{f_n, \ n\in\Z\},
\]
missä
\[
f_n(x)= \Arctan x + n\pi.
\]
Tässä $f_0(x)=\Arctan x\,$ on implisiittifunktion $\arctan$ \kor{päähaara} 
(engl.\ principal branch), vrt.\ Luku \ref{käänteisfunktio}.
\begin{Exa} \label{napakulman kaava} Jos $\varphi(x,y)$ on karteesisen koordinaatiston
pistettä $(x,y)$ vastaava napakulma, $\varphi\in[0,2\pi)$, niin funktioriippuvuus 
$(x,y)\map\varphi(x,y)$ on
\[
\varphi(x,y)=
\begin{cases}
\Arctan(y/x),      &\text{jos}\ \ x>0\ \ja\ y\geq 0 \\
\Arctan(y/x)+\pi,  &\text{jos}\ \ x<0 \\
\Arctan(y/x)+2\pi, &\text{jos}\ \ x>0\ \ja\ y<0 \\
\pi/2,             &\text{jos}\ \ x=0\ \ja\ y>0 \\
3\pi/2,            &\text{jos}\ \ x=0\ \ja\ y<0
\end{cases}
\]
Tässä tarvittiin siis peräti kolmea $\arctan$:n haaraa (!).
\loppu
\begin{figure}[H]
\begin{center}
% GNUPLOT: LaTeX picture using EEPIC macros
\setlength{\unitlength}{0.240900pt}
\begin{picture}(1500,900)(0,0)
\footnotesize
\thicklines \path(419,169)(439,169)
\thicklines \path(1433,169)(1413,169)
\put(397,169){\makebox(0,0)[r]{$-\frac{\pi}{2}$}}
\thicklines \path(419,295)(439,295)
\thicklines \path(1433,295)(1413,295)
\put(397,295){\makebox(0,0)[r]{0}}
\thicklines \path(419,421)(439,421)
\thicklines \path(1433,421)(1413,421)
\put(397,421){\makebox(0,0)[r]{$\frac{\pi}{2}$}}
\thicklines \path(419,547)(439,547)
\thicklines \path(1433,547)(1413,547)
\put(397,547){\makebox(0,0)[r]{$\pi$}}
\thicklines \path(419,673)(439,673)
\thicklines \path(1433,673)(1413,673)
\put(397,673){\makebox(0,0)[r]{$\frac{3\pi}{2}$}}
\thicklines \path(419,799)(439,799)
\thicklines \path(1433,799)(1413,799)
\put(397,799){\makebox(0,0)[r]{$2\pi$}}
\thicklines \path(520,135)(520,155)
\thicklines \path(520,856)(520,836)
\put(520,90){\makebox(0,0){-4}}
\thicklines \path(723,135)(723,155)
\thicklines \path(723,856)(723,836)
\put(723,90){\makebox(0,0){-2}}
\thicklines \path(926,135)(926,155)
\thicklines \path(926,856)(926,836)
\put(926,90){\makebox(0,0){0}}
\thicklines \path(1129,135)(1129,155)
\thicklines \path(1129,856)(1129,836)
\put(1129,90){\makebox(0,0){2}}
\thicklines \path(1332,135)(1332,155)
\thicklines \path(1332,856)(1332,836)
\put(1332,90){\makebox(0,0){4}}
\thicklines \path(419,135)(1433,135)(1433,856)(419,856)(419,135)
\put(300,495){\makebox(0,0)[l]{\shortstack{$y$}}}
\put(926,23){\makebox(0,0){$x$}}
\thinlines \path(419,437)(419,437)(429,437)(439,438)(450,438)(460,438)(470,439)(480,439)(491,439)(501,440)(511,440)(521,441)(532,441)(542,442)(552,442)(562,443)(573,443)(583,444)(593,445)(603,445)(614,446)(624,447)(634,448)(644,449)(655,450)(665,451)(675,452)(685,453)(696,454)(706,456)(716,457)(726,459)(737,460)(747,462)(757,464)(767,467)(777,469)(788,472)(798,475)(808,478)(818,482)(829,486)(839,490)(849,495)(859,500)(870,506)(880,513)(890,520)(900,527)(911,535)(921,543)
\thinlines \path(921,543)(931,551)(941,559)(952,567)(962,574)(972,581)(982,588)(993,593)(1003,599)(1013,604)(1023,608)(1034,612)(1044,616)(1054,619)(1064,622)(1075,625)(1085,627)(1095,629)(1105,631)(1115,633)(1126,635)(1136,637)(1146,638)(1156,640)(1167,641)(1177,642)(1187,643)(1197,644)(1208,645)(1218,646)(1228,647)(1238,648)(1249,648)(1259,649)(1269,650)(1279,650)(1290,651)(1300,652)(1310,652)(1320,653)(1331,653)(1341,654)(1351,654)(1361,654)(1372,655)(1382,655)(1392,656)(1402,656)(1413,656)(1423,657)(1433,657)

\thicklines \path(921,291)(931,299)(941,307)(952,315)(962,322)(972,329)(982,336)(993,342)(1003,347)(1013,352)(1023,356)(1034,360)(1044,364)(1054,367)(1064,370)(1075,373)(1085,376)(1095,378)(1105,380)(1115,382)(1126,383)(1136,385)(1146,386)(1156,388)(1167,389)(1177,390)(1187,391)(1197,392)(1208,393)(1218,394)(1228,395)(1238,396)(1249,397)(1259,397)(1269,398)(1279,399)(1290,399)(1300,400)(1310,400)(1320,401)(1331,401)(1341,402)(1351,402)(1361,403)(1372,403)(1382,404)(1392,404)(1402,404)(1413,405)(1423,405)(1433,405)

\Thicklines \path(419,689)(419,689)(429,689)(439,689)(450,690)(460,690)(470,690)(480,691)(491,691)(501,691)(511,692)(521,692)(532,693)(542,693)(552,694)(562,695)(573,695)(583,696)(593,696)(603,697)(614,698)(624,699)(634,700)(644,700)(655,701)(665,702)(675,704)(685,705)(696,706)(706,707)(716,709)(726,710)(737,712)(747,714)(757,716)(767,718)(777,721)(788,723)(798,726)(808,730)(818,733)(829,737)(839,742)(849,747)(859,752)(870,758)(880,764)(890,771)(900,779)(911,787)(921,795)

\put(1150,325){$y=\Arctan{x}$}
\put(1150,425){I} \put(1150,675){III} \put(600,475){II} \put(600,725){IV}
\thinlines
\put(25,700){\vector(1,0){200}} %\put(200,670){$x$}
\put(125,600){\vector(0,1){200}} %\put(135,780){$y$}
\put(150,725){I} \put(150,625){IV} \put(50,725){II} \put(50,625){III}
\end{picture}







\end{center}
\end{figure}
\end{Exa}

\subsection*{Kahden muuttujan implisiittifunktiot}

Jos $F$ on kolmen muuttujan funktio ja yhtälö
\[
F(x,y,z)=0
\]
ratkeaa yksikäsitteisesti $z$:n suhteen, kun $(x,y)\in A\subset\R^2$, tulee määritellyksi
kahden muuttujan implisiittifunktio
\[
z=f(x,y), \quad (x,y)\in A.
\]
Monihaaraiset tilanteet ovat jälleen tavallisia.
\begin{Exa}
Yhtälö $x^2+y^2+z^2=R^2$ määrittelee kaksihaaraisen implisiittifunktion
$z=\pm\sqrt{R^2-x^2-y^2}\,$ joukossa $A=\{(x,y)\in\R^2 \ | \ x^2+y^2\leq R^2\}$. \loppu
\end{Exa}
\begin{Exa} Jos $m\in\N,\ m \ge 2$, niin yhtälö
\[
y,z\in\C: \quad y^m=z
\]
määrittelee $m$-haaraisen kompleksifunktion $y=\sqrt[m]{z}$, vrt.\ Luku \ref{III-3}.
Jos kirjoitetaan $z=x+iy$ ja $y=u(x,y)+iv(x,y)$, niin $u$ ja $v$ ovat $m$-haaraisia
funktioita tyyppiä $u,v:\ \Rkaksi\kohti\R$. \loppu
\end{Exa}

\Harj
\begin{enumerate}

\item
Seuraavassa on määritelty implisiittifunktioita, mahdollisesti useampihaaraisia. Määritä näiden
haarat funktioina $y=f(x),\ D_f\subset\R$ (määritä myös $D_f$). Hahmottele funktiot myös 
graafisesti!
\begin{align*}
&\text{a)}\ \ x\cos x+y^2=0 \qquad\qquad\quad\,
 \text{b)}\ \ x+y^4-2y^2=0 \\
&\text{c)}\ \ x+\tan\,(y^4-16y^2)=0 \qquad
 \text{d)}\ \ \arctan x + \arctan y=\frac{\pi}{4}
\end{align*}

\item
Mitkä ovat monihaaraisen funktion $\arccos(\cos x)$ arvot, kun $x=13\pi/4$\,?

\item
a) Määrittele monihaaraisen funktion $\arcsin=\{f_n\,,\ n\in\Z\}$ haarat päähaaran $f_0$
avulla. \ b) Ilmaise karteesisen koordinaatiston pisteiden ja napakulman välinen riippuvuus
$\varphi(x,y)$ funktioiden $f_n$ avulla.

\item
Yhtälö $x+3xyz^2+z^4=0\ (x,y,z\in\R)$ määrittelee implisiittifunktion $z=f(x,y)$. \ a) Laske 
$f$:n arvot pisteissä $(0,0)$, $(2,-1)$ ja $(1,1)$. \ b)  Millaisia ovat $f$:n haarat 
(määrittelyjoukot ja laskusäännöt) yleisemmin?

\item (*)
Määrittele kaksihaaraiset funktiot $u=\{u_1,u_2\}$ ja $v=\{v_1,v_2\}$ siten, että
\[
u(x,y)+iv(x,y)=\sqrt{z}\,, \quad z=x+iy\in\C.
\]

\end{enumerate}