\chapter{Reaalimuuttujien funktiot}

Matemaattisten funktioiden päätyypit ovat
\begin{itemize}
\item \kor{yhden} reaali\kor{muuttujan} reaaliarvoiset \kor{funktiot}, eli reaalifunktiot
\item \kor{useamman} reaali\kor{muuttujan} reaaliarvoiset \kor{funktiot}
\item yhden tai useamman reaalimuuttujan \kor{vektoriarvoiset funktiot}
\item \kor{kompleksifunktiot}, eli kompleksimuuttujan kompleksiarvoiset funktiot
\end{itemize}
Tässä luvussa aloitetaan funktoiden tutkimus tarkastelemalla yhden tai useamman, toistaiseksi
kahden tai kolmen, reaalimuuttujan reaaliarvoisia funktioita sekä yhden tai kahden muuttujan
vektoriarvoisia funktioita. Tarkasteltaville funktiotyypeille on yhteistä niiden saama 'näkyvä'
muoto, kun luvut, lukuparit ja lukukolmikot muuttujina tai vektorit funktion arvoina ymmärretään
euklidisten pisteavaruuksien tai vastaavien vektoriavaruuksien olioina.

Funktioiden tutkimus on matematiikassa hyvin keskeistä, siksi myös tähän liittyvä käsitteistö
ja keinovalikoima on huomattavan laaja. Tässä luvussa ei koko 'teknologiaa' oteta vielä käyttöön,
vaan rajoitutaan toistaiseksi kaikkein yksinkertaisimpiin algebran ja geometrian keinoihin.
Toisaalta sovelluksia (etenkin fysiikan sovelluksia) ajatellen tässä luvussa tarkasteltavien
funktioiden tyyppivalikoima on jo melko edustava. Tarkoituksena on tämän valikoiman puitteissa
käydä läpi mm.\ sellaiset funktioiden algebran käsitteet kuin funktioiden 
\kor{algebralliset yhdistelyt}, \kor{yhdistetty funktio}, \kor{käänteisfunktio} ja
\kor{implisiittifunktio}. Selvimmin 'geometrisia funktioita' ovat Luvussa
\ref{parametriset käyrät} esiteltävät \kor{parametriset käyrät} ja \kor{parametriset pinnat}.