\section{Pinta-alamitta ja tasointegraalit} \label{tasointegraalit}
\alku \sectionmark{Tasointegraalit}
\index{tasointegraali|vahv}

Tasoalueen p\pain{inta-alan} määräämistä määrätyn integraalin avulla on tarkasteltu aiemmin
Luvussa \ref{pinta-ala ja kaarenpituus}. Tässä luvussa tarkastelun kohteena ovat yleisemmät
\kor{tasointegraalit} muotoa
\begin{equation} \label{integraalimerkintä}
\int_A f\, d\mu=I(f,A,\mu), \tag{$\star$}
\end{equation}
missä $A\subset\R^2$, $f$ on $A$:ssa määritelty (kahden muuttujan) funktio ja $\mu$ on $\R^2$:n
\kor{pinta-alamitta}. Integraali $I(f,A,\mu)$ on tarkemmin \pain{reaaliluku},
joka riippuu funktiosta $f$, joukosta $A$ ja mitasta $\mu$, kuten ilmenee lukutavasta:
$f$\pain{:n} \pain{inte}g\pain{raali} y\pain{li} \pain{$A$:n} \pain{mitan} $\mu$
\pain{suhteen}. --- Kyseessä on ennestään tutun \pain{määrät}y\pain{n} \pain{inte}g\pain{raalin}
yleistys; nimittäin kuten jäljempänä havaitaan, merkintä \eqref{integraalimerkintä} on pätevä
myös, kun $f$ on välillä $A=[a,b]$ määritelty (integroituva) funktio,
$I(f,A,\mu)=\int_a^b f(x)\,dx$ ja $\mu$ on $\R$:n \kor{pituusmitta}.

\subsection*{Jordanin pinta-alamitta}
\index{pinta-alamitta!a@tason|vahv}
\index{Jordan-mitta!a@pinta-alamitta|vahv}

Koska tasointegraali määritellään suhteessa pinta-alamittaan, on tämä ensin määriteltävä.
\index{mitta, mitallisuus}%
\kor{Mitta} (engl.\ measure) on yleisemmin \kor{joukkofunktio}, joka liittää \kor{mitalliseen}
(engl.\ measurable) joukkoon (tässä $\R^2$:n osajoukkoon) reaaliluvun:
\[
\mu:\mathcal{M}\Kohti\R,\quad \mathcal{M}=\{\text{mitalliset joukot $A\subset\R^2$}\}.
\]
Pinta-alamitaksi oletetaan jatkossa \kor{Jordan-mitta}\footnote[2]{Jordan-mittaa sanotaan myös
\kor{Peanon--Jordanin} mitaksi. Termit viittaavat matemaatikoihin \hist{Camille Jordan}
(ransk.\ 1838-1922) ja \hist{Giuseppe Peano} (ital.\ 1858-1932). \index{Jordan, C.|av}
\index{Peano, G.|av}}, joka jatkossa määritellään niin,
että seuraavat kolme aksioomaa toteutuvat. Aksioomista ensimmäisessä asetetaan
\index{perussuorakulmio}%
koordinattiakselien mukaan suunnatun nk.\ \kor{perussuorakulmion} mitta. Muut kaksi ovat
yleisempienkin \kor{positiivisten} mittojen aksioomia (tai aksioomien seurauksia). 
\begin{itemize}
\item[A1.] \kor{Perussuorakulmion mitta}: \ 
           $T=\{(x,y)\in\R^2 \mid x\in[a_1,b_1]\ \ja\ y\in[a_2,b_2]\}$: \\[0.25cm] 
           $\mu(T)=(b_1-a_1)(b_2-a_2)$.
\item[A2.] \kor{Additiivisuus}: \ $A,B\in\mathcal{M} \ \ja\ A \cap B=\emptyset$ \\[0.25cm]
           $\impl \ A \cup B \in \mathcal{M} \ \ja \ \mu(A \cup B)=\mu(A)+\mu(B)$.
           \index{additiivisuus!b@mitan}
\item[A3.] \kor{Positiivisuus}: \ $\mu(A)\geq 0\quad\forall A\in\mathcal{M}$.
           \index{positiivisuus (mitan)}
\end{itemize}
Huomattakoon, että aksiooman A1 mukaisesti mitta $\mu$ on lähtökohtaisesti
valitusta (karteesisesta) koordinaatistosta riippuva. Mittoja käsittelevälle matematiikan
lajille, \kor{mittateorialle}, tällainen lähtökohta on tyypillinen, jolloin mitan
riippumattomuus koordinaatistosta (mikäli tosi) on aina erikseen osoitettava.

Tyhjä joukko $\emptyset$ katsotaan aina mitalliseksi, jolloin aksiooman A2 mukaan on oltava
$\mu(\emptyset)=0$. Yleisemmin jos $A \in \mathcal{M}$ ja $\mu(A)=0$, sanotaan että $A$ on
\index{nollamittaisuus}%
\kor{nollamittainen}. Nollamittaiset joukot ovat mittateoriassa tärkeällä
sijalla.\footnote[2]{Nollamittaisuuteen liittyy erikoinen matematiikan termi
\kor{melkein kaikkialla} (engl.\ almost everywhere), joka tarkoittaa: muualla kuin
nollamittaisessa (osa)joukossa. \index{melkein kaikkialla|av}}
\begin{Exa} Koordinaattiakselien suuntaiset janat $A=\{a\}\times[b,c]$ ja
$B=[a,b]\times\{c\}$ voidaan tulkita suorakulmion erikoistapauksiksi, jolloin aksiooman A1
mukaan $\mu(A)=\mu(B)=0$. \loppu
\end{Exa}
\begin{Exa} Jos $A=[a_1,b_1]\times[a_2,b_2]$ ja $B=[b_1,c_1]\times[a_2,b_2]$, niin
$A \cup B=[a_1,c_1]\times[a_2,b_2]$, joten aksiooman A1 mukaan on
$\mu(A \cup B)=\mu(A)+\mu(B)$. \loppu
\end{Exa}
Jälkimmäisessä esimerkissä additiivisuusaksiooma A2 toteutuu, vaikka $A \cap B\neq\emptyset$.
Tämä johtuu siitä, että (edellisen esimerkin mukaan) $A \cap B$ on nollamittainen.
--- Yleisemminkin voi aksiooomassa A2 ehdon $\,A \cap B = \emptyset\,$ lieventää ehdoksi
$\mu(A \cap B)=0$ (ks.\ Jordan-mitan määrittely jäljempänä ja
Harj.teht.\,\ref{H-uint-1: väittämiä}c).

\subsection*{Integraali yli rajoitetun joukon}
\index{Riemannin!a@integraali|vahv}
\index{Riemann-integroituvuus!c@tasossa|vahv}

Olkoon $A\subset\R^2$ rajoitettu joukko, jolloin se sisältyy johonkin perussuorakulmioon:
\[
A \subset T = [a_1,b_1]\times [a_2,b_2]
\]
(ks.\ kuvio alla). Olkoon $f$ $A$:ssa määritelty rajoitettu funktio ja olkoon $f_0=$ $f$:n
\index{nollajatko (funktion)}%
\kor{nollajatko} $A$:n ulkopuolelle:
\[
f_0(x,y)=\begin{cases} \,f(x,y), &\text{kun}\ (x,y)\in A, \\ \,0, &\text{muulloin}. \end{cases}
\]
Tällöin sovitaan ensinnäkin, että
\[
\boxed{\kehys\quad \int_A f\, d\mu=\int_T f_0\, d\mu, \quad A \subset T. \quad}
\]
\begin{figure}[H]
\begin{center}
\input{kuvat/kuvaUint-1.pstex_t}
\end{center}
\end{figure}
Kuten määrätyssä integraalissa, integraalin $\int_T f_0\,d\mu$ määrittelyn lähtökohtana on
\index{jako (suorakulmioihin)}%
$T$:n \kor{jako} (ositus), tässä tapauksessa perussuorakulmion muotoisiin osajoukkoihin:
\begin{align*}
&\mathcal{T}_h=\{T_{kl}, \ k=1,\ldots, m, \ l=1,\ldots, n\}, \\[1mm]
&T_{kl} = [x_{k-1},x_k]\times [y_{l-1},y_l], \quad
\begin{cases} \,a_1 = x_0<x_1<\ldots x_m=b_1, \\ \,a_2 = y_0<y_1<\ldots < y_n=b_2. \end{cases}
\end{align*}
\begin{figure}[H]
\begin{center}
\input{kuvat/kuvaUint-2.pstex_t}
\end{center}
\end{figure}
Jaon (eli suorakulmiojoukon $\mathcal{T}_h$) indeksoinnissa käytetty parametri $h$ on jälleen
\index{tiheysparametri}%
\kor{tiheysparametri}. Tämän määritelmä kahdessa ulottuvuudessa on
\[
h=\max\{h_x,\,h_y\}, \quad h_x=\max_k\{x_k-x_{k-1}\}, \quad h_y=\max_l\{y_l-y_{l-1}\}.
\]
Integraali $\int_T f_0\,d\mu$ määritellään nyt määrätyn integraalin tapaan eli
\index{Riemannin!b@summa}%
\kor{Riemannin summien} avulla raja-arvoprosessilla
(vrt.\ Määritelmät \ref{Riemannin integraali} ja \ref{raja-arvo Lim})
\[
\int_T f_0\,d\mu = \Lim_{h\kohti 0} \sum_{k=1}^m \sum_{l=1}^n f_0(\xi_{kl},\eta_{kl})\mu(T_{kl}),
\]
missä $(\xi_{kl},\eta_{kl})\in T_{kl}$. Vaihtoehtoinen, etenkin teoreettisissa tarkasteluissa
kätevä integroituvuuden kriteeri saadaan, kun määritellään jakoon $\mathcal{T}_h$ liittyen
\index{Riemannin!c@ylä- ja alasumma}%
\begin{align*}
&-\text{\kor{yläsumma}}: \quad \overline{\sigma}_h(f_0,\mathcal{T}_h)
                          =\sum_{k=1}^m\sum_{l=1}^n M_{kl}\,\mu(T_{kl}), \quad
                             M_{kl}=\sup_{(x,y)\in T_{kl}} f_0(x,y), \\
&-\text{\kor{alasumma}}: \quad \underline{\sigma}_h(f_0,\mathcal{T}_h)
                          =\sum_{k=1}^m\sum_{l=1}^n m_{kl}\,\mu(T_{kl}), \quad
                             m_{kl}=\inf_{(x,y)\in T_{kl}} f_0(x,y).
\end{align*}
Tällöin $f$ on integroituva yli $A$:n täsmälleen kun pätee
(vrt.\ Lause \ref{Riemann-integroituvuus})
\[
\sup_{\mathcal{T}_h} \underline{\sigma}_h(f_0,\mathcal{T}_h) 
                     = \underline{I}(f,\mu,A) = \overline{I}(f,\mu,A)
                     = \inf_{\mathcal{T}_h} \overline{\sigma}_h(f_0,\mathcal{T}_h),
\]
\index{Riemannin!d@ylä- ja alaintegraali}%
ja integraalin arvo = näiden \kor{ala}- ja \kor{yläintegraalien} yhteinen arvo $I(f,A,\mu)$.

Molemmista määrittelytavoista nähdään, että $\int_T f_0\,d\mu$ ei riipu suorakulmion $T$
valinnasta sikäli kuin $T \supset A$ (Harj.teht.\,\ref{H-uint-1: T:n valinta}). --- Tämä tulos
ennakoitiin jo edellä, kun sovittiin: $\,\int_A f\,d\mu = \int_T f_0\,d\mu,\ T \supset A$.

\subsection*{Lineaarisuus. Additiivisuus. Vertailuperiaate}

Jos merkitään rajoitetun joukon A yli integroituvien funktioiden joukkoa symbolilla
$\mathcal{R}_A$, niin seuraavat yleiset ominaisuudet ovat yhteisiä tasointegraaleille ja 
yksiulotteisille määrätyille integraaleille (ks.\ Lauseet
\ref{integraalin additiivisuus}--\ref{integraalien vertailuperiaate}). Nämä ovat voimassa myös
jatkossa määriteltäville muille integraalityypeille.
\index{lineaarisuus!b@integroinnin}
\index{additiivisuus!a@integraalin}
\index{vertailuperiaate!a@integraalien}%
\[ 
\boxed{ \begin{aligned}
\ykehys\quad &\text{Lineaarisuus}\,: \qquad 
f,g \in \mathcal{R}_A\ \impl\ \alpha f + \beta g \in \mathcal{R}_A \quad
                                         \forall\ \alpha,\beta \in \R \quad \text{ja} \\
&\phantom{\text{Lineaarisuus}\,:} \qquad 
\int_A (\alpha f + \beta g)\,d\mu = \alpha\int_A f\,d\mu + \beta\int_A g\,d\mu. \\[2mm]
&\text{Additiivisuus}\,: \quad 
f\in\mathcal{R}_A\ \ja\ f\in\mathcal{R}_B\ \ja\ A \cap B = \emptyset \\
&\phantom{\text{Additiivisuus}\,:} \quad \impl\ \int_{A \cup B} f\,d\mu 
                                              = \int_A f\,d\mu + \int_B f\,d\mu. \\[2mm]
&\text{Vertailuperiaate}\,: \quad 
f,g \in \mathcal{R}_A\ \ja\ f \le g\,\ A\text{:ssa}\ \impl\ \int_A f\,d\mu \le \int_A\,g\,d\mu. 
        \quad\Akehys \end{aligned} } 
\]
Lineaarisuus ja vertailuperiaate seuraavat suoraan määritelmästä, kuten yksi\-ulotteisessa
tapauksessa (ks.\ Luku \ref{riemannin integraali}). Additiivisuus on johdettavissa
yksinkertaisimmin lineaarisuudesta: Kun merkitään $f$:n nollajatkoja joukkojen $A$, $B$ ja
$A \cup B$ suhteen vastaavasti symboleilla $f_A$, $f_B$ ja $f_0$, niin pätee
\[
A \cap B = \emptyset \qimpl f_0=f_A+f_B.
\]
Näin ollen jos $T=[a_1,b_1]\times[a_2,b_2] \supset A \cup B$, niin lineaarisuuden nojalla
\[
\int_T f_0\,d\mu = \int_T f_A\,d\mu + \int_T f_B\,d\mu
   \qekv \int_{A \cup B} f\,d\mu = \int_A f\,d\mu + \int_B f\,d\mu.
\]

\subsection*{Rajoitetun joukon mitta}

Em.\ määritelmissä on integraalin riippuvuus mitasta varsin yksinkertainen, sillä määritelmissä
tarvitaan vain suorakulmioiden $T_{kl}$ mittoja aksiooman A1 mukaisesti. Mutta kun integraali on
kerran määritelty käyttäen hyväksi mitan yksinkertaisimpia ominaisuuksia, voidaankin määritellä
yleisemmän joukon $A$ mitta käyttäen hyväksi integraalia (!).
Menettely on seuraava: Tarkastellaan funktiota
\[
f(x,y)=1,\quad (x,y)\in A.
\]
\index{karakteristinen funktio}%
Tämän nollajatkoa sanotaan $A$:n \kor{karakteristiseksi funktioksi} ja merkitään
\[
\chi_A(x,y)=\begin{cases} 
            \,1, &\text{kun}\ (x,y) \in A, \\ \,0, &\text{kun}\ (x,y) \notin A. 
            \end{cases}
\]
Sikäli kuin $\chi_A$ on Riemann-integroituva yli suorakulmion $T\supset A$, on em.\ 
määritelmien mukaan
\[
\int_A d\mu=\int_T \chi_A\,d\mu, \quad T=[a_1,b_1]\times[a_2,b_2]\supset A.
\]
\begin{Def} (\vahv{Jordan-mitta}) \label{Jordan-mitta} \index{pinta-alamitta!a@tason|emph}
\index{Jordan-mitta!a@pinta-alamitta|emph} \index{mitta, mitallisuus!a@Jordan-mitta|emph} 
Rajoitettu joukko $A\subset\R^2$ on \kor{Jordan-mitallinen} täsmälleen kun $A$:n
karakteristinen funktio $\chi_A$ on Riemann-integroituva yli suorakulmioiden
$T=[a_1,b_1]\times[a_2,b_2]\supset A$ ja $A$:n \kor{Jordan-mitta} on tällöin
\[
\boxed{\kehys\quad \mu(A)=\int_A d\mu. \quad}
\]
\end{Def}
Mitta-aksioomien A1--A3 toteutuminen on tämän määritelmän ja em.\ yleisten integraalin
ominaisuuksien perusteella ilmeistä.

Kaikki rajoitetut joukot eivät ole mitallisia (ks.\ Harj.teht.\,\ref{H-uint-1: esimerkkejä}a),
mutta jokaiselle rajoitetulle joukolle $A\subset\R^2$ voidaan määritellä
\index{ulkomitta (Jordanin)} \index{siszy@sisämitta (Jordanin)}
\begin{align*}
&\text{\kor{ulkomitta}:} \quad 
            \overline{\mu}(A)=\inf_{\mathcal{T}_h}\overline{\sigma}_h(\chi_A,\mathcal{T}_h), \\
&\text{\kor{sisämitta}:} \quad 
            \underline{\mu}(A)=\sup_{\mathcal{T}_h}\underline{\sigma}_h(\chi_A,\mathcal{T}_h),
\end{align*}
missä $\inf$ ja $\sup$ lasketaan yli kaikkien perussuorakulmion $T\supset A$ jakojen. Joukko
$A$ on mitallinen täsmälleen kun $\overline{\mu}(A)=\underline{\mu}(A)$.
\begin{figure}[H]
\begin{center}
\input{kuvat/kuvaUint-3.pstex_t}
\end{center}
\end{figure}
Kuviossa on
\begin{align*}
\mu(\raisebox{-0.1cm}{\epsfig{file=kuvat/kuvaUint-4a.eps}}) 
                      &= \underline{\sigma}_h(\chi_A,\mathcal{T}_h)
                       =\sum_{T_{kl}\in\mathcal{T}_h:\,T_{kl}\subset A} \mu(T_{kl}), \\
\mu(\raisebox{-0.1cm}{\epsfig{file=kuvat/kuvaUint-4b.eps}}) 
                      &= \overline{\sigma}_h(\chi_A,\mathcal{T}_h)
                         -\underline{\sigma}_h(\chi_A,\mathcal{T}_h).
\end{align*}
\begin{Exa} $A=$ jana, jonka päätepisteet ovat $(0,0)$ ja $(2,1)$. Näytä: $\mu(A)=0$.
\end{Exa}
\begin{multicols}{2}
\ratk Jaetaan $T=[0,2]\times[0,1] \supset A$ suorakulmioihin kokoa $h \times h/2$, missä
$2/h=n\in\N$. Tällöin on (vrt.\ kuvio)
\[
\overline{\sigma}_h(\chi_A,\mathcal{T}_h) = \frac{2}{h}\cdot\frac{h^2}{2} = h.
\]
Näin ollen $\overline{\mu}(A) \le h=2/n\ \forall n\in\N$, joten $\overline{\mu}(A)=0$ ja siis
$\mu(A)=0$. \loppu
\begin{figure}[H]
\setlength{\unitlength}{1cm}
\begin{center}
\begin{picture}(4,4)
\put(0,1){\vector(1,0){5}} \put(0,1){\vector(0,1){2.5}}
\put(5.3,0.9){$x$} \put(-0.1,3.8){$y$}
\put(3.95,0.6){$\scriptstyle{2}$} \put(-0.4,2.9){$\scriptstyle{1}$}
\put(0.45,0.6){$\scriptstyle{h}$} \put(-0.6,1.2){$\scriptstyle{h/2}$}
\put(0,3){\line(1,0){4}} \put(4,1){\line(0,1){2}}
\multiput(0,1)(0.5,0.25){8}{\line(1,0){0.5}}
\multiput(0,1.25)(0.5,0.25){8}{\line(1,0){0.5}}
\multiput(0,1)(0.5,0.25){8}{\line(0,1){0.25}}
\multiput(0.5,1)(0.5,0.25){8}{\line(0,1){0.25}}
\thicklines
\put(0,1){\line(2,1){4}}
\end{picture}
\end{center}
\end{figure}
\end{multicols}

Mainittakoon vielä seuraava lause, jonka osittainen todistus jätetään harjoitustehtäväksi
(Harj.teht.\,\ref{H-uint-1: integroituvuus L-ehdolla}). Lauseen jatkuvuusoletus
(jota harjoitustehtävässä on vahvistettu) viittaa Määritelmään
\ref{jatkuvuus kompaktissa joukossa - Rn}.
\begin{*Lause} \label{jatkuvan funktion integroituvuus tasossa}
\index{Riemann-integroituvuus!b@jatkuvan funktion|emph}
\index{Riemann-integroituvuus!c@tasossa|emph}
Jos $f$ on jatkuva suorakulmiossa $T=[a,b]\times[c,d]$ ja $A \subset T$ on mitallinen, niin
$f$ on Riemann-integroituva yli $A$:n.
\end{*Lause} 

\subsection*{Siirto-, peilaus- ja kiertoinvarianssi}
\index{Jordan-mitta!a@pinta-alamitta|vahv}
\index{siirtoinvarianssi (mitan)|vahv}
\index{peilausinvarianssi (mitan)|vahv}
\index{kiertoinvarianssi (mitan)|vahv}

Jotta Jordan-mitta vastaisi tavanomaista geometriasta tunnettua pinta-alaa, on mitan oltava
\kor{siirto}-, \kor{peilaus} ja  \kor{kiertoinvariantti}. Tällä tarkoitetaan, että jos $A$ on
mitallinen ja $A'$ on $A$:n kanssa (geometrisesti)
\index{yhtenevyys}%
\kor{yhtenevä}, eli $A'=\mf(A)$, missä joillakin $a,b,\theta\in\R$ on
\[
\mf(x,y)=(a+x\cos\theta \mp y\sin\theta,\,b+x\sin\theta \pm y\cos\theta),
\]
\begin{figure}[H]
\begin{center}
\input{kuvat/kuvaUint-7.pstex_t}
\end{center}
\end{figure}
niin $\mu(A')=\mu(A)$. Kuvaus $\mf:A \kohti A'$ on affiinimuunnoksen erikoistapaus, joka
\index{siirto (translaatio)} \index{kierto!a@geom.\ kuvaus} \index{peilaus}%
koostuu \kor{siirrosta} (siirtovektori $\vec r_0=a\vec i + b\vec j\,$), \kor{kierrosta}  
(kiertokulma $\theta$) ja mahdollisesta \kor{peilauksesta} (merkin vaihtelu). Yhdistettyä
siirtoa ja kiertoa (kuva) sanotaan
\index{euklidinen liike}%
\kor{euklidiseksi liikkeeksi}.  

Siirto- ja peilausinvarianssi ovat mitan $\mu$ määritelmästä varsin ilmeisiä. Kierron suhteen 
sen sijaan voisi määritelmän koordinaattiriippuvuuden epäillä aiheuttavan ongelmia. Tällaisia
ei todellisuudessa ole, vaan pätee
(ks.\ Harj.teht.\,\ref{H-uint-1: kiertoinvarianssi a},\ref{H-uint-1: kiertoinvarianssi b})
\begin{Lause}
Jordanin pinta-alamitta on siirto-, peilaus- ja kiertoinvariantti, ts. geometrisesti yhtenevien
joukkojen mitat ovat samat.
\end{Lause}

\subsection*{Keskipistesääntö}
\index{keskipistesääntö|vahv}

Kuten yhden muuttujan integraaleja, myös tasointegraaleja voi laskea numeerisesti suoraan 
määritelmästä käsin, eli approksimoimalla integraalia Riemannin summalla:
\[ 
\int_A f\, d\mu = \int_T f_0\, d\mu 
    \,\approx\, \sum_{k=1}^m \sum_{l=1}^n f_0(\xi_{kl},\eta_{kl})\mu(T_{kl}), \quad T \supset A.
\]
Jos tässä valitaan $(\xi_{kl},\eta_{kl})=T_{kl}$:n keskipiste, niin approksimaatiota sanotaan
\index{yhdistetty!a@keskipistesääntö}%
\kor{yhdistetyksi keskipistesäännöksi}. Kuten yhdessä dimensiossa (vrt.\ Luku 
\ref{numeerinen integrointi}), tämä on Riemannin summiin perustuvista approksimaatioista
yleensä tarkin.
\begin{Exa} Jos $A = [a,b] \times [c,d]$, niin integraali $\int_A x\,d\mu$ saadaan lasketuksi
jakamalla $A\ $ $m \times n$ samankokoiseen suorakulmioon ja käyttämällä yhdistettyä
keskipistesääntöä:
\begin{align*}
\int_A x\,d\mu\ 
   &\approx\ \sum_{k=1}^m\sum_{l=1}^n 
    \left[a+\frac{(k-\tfrac{1}{2})(b-a)}{m}\right]\frac{b-a}{m}\cdot\frac{c-d}{n} \\
   &=\, (b-a)(c-d)\left[a + \frac{b-a}{m^2}\sum_{k=1}^m (k-\tfrac{1}{2})\right] \\
   &=\, (b-a)(c-d)[a+\tfrac{1}{2}(b-a)] \,=\, \tfrac{1}{2}(a+b)\mu(A).
\end{align*}
Koska tulos ei riipu parametreista $m,n$, on oltava $\int_A x\,d\mu = \tfrac{1}{2}(a+b)\mu(A)$.
Vastaavalla tavalla laskien saadaan $\int_A y\,d\mu = \tfrac{1}{2}(c+d)\mu(A)$. \loppu
\end{Exa}
Esimerkissä saatiin jo yhdellä suorakulmiolla ($m=n=1,\ T_{11}=A$) tarkka tulos. Yleisemmin
keskipistesääntö integroi suorakulmiossa tarkasti ensimmäisen asteen polynomin:
\begin{multicols}{2}
\begin{align*}
f(x,y) &= a_0 + a_1 x + a_2 y \quad (a_0,a_1,a_2 \in \R) \\
       &\impl \ \int_T f\, d\mu=f(x_0,y_0)\mu(T).
\end{align*}
\begin{figure}[H]
\setlength{\unitlength}{1cm}
\begin{center}
\begin{picture}(-3,0.5)(4,2)
\path(0,0)(4,0)(4,2)(0,2)(0,0)
\put(1.9,0.9){$\bullet$} \put(2.2,0.9){$(x_0,y_0)$}
\put(1.9,2.2){$T$}
\end{picture}
\end{center}
\end{figure}
\end{multicols}
Tämän ominaisuuden ja kahden muuttujan Taylorin kaavan
(Luku \ref{usean muuttujan taylorin polynomit}) perusteella keskipistesäännölle on johdettavissa 
virhearvio samaan tapaan kuin yhdessä dimensiossa (Harj.teht.\,\ref{H-uint-1: kp-virhearvio}).
\begin{Exa} \label{keskipistesääntö tasossa} Olkoon $A = [0,1] \times [0,1]$ ja laskettavana
integraali $\int_A x^2 y^2\,d\mu$. Kun $A$ jaetaan neliöihin kokoa $h \times h$ ja kussakin 
neliössä integroimispisteeksi valitaan (a) vasen alanurkka, (b) keskipiste, niin integraalille
saadaan seuraavat likiarvot (tarkka arvo $=1/9\,$; taulukkoon on merkitty myös summattavien 
termien lukumäärä $N=h^{-2}$)\,:
\begin{center}
\begin{tabular}{llll}
$h$    & $N$    & Likiarvo (a)  & Likiarvo (b) \\  \hline & \\
0.1    & $10^2$ & 0.081225000.. & 0.110556250.. \\
0.01   & $10^4$ & 0.107813722.. & 0.111105555.. \\
0.001  & $10^6$ & 0.110778138.. & 0.111111055.. \\
0.0001 & $10^8$ & 0.111077781.. & 0.111111110.. \qquad\qquad\loppu
\end{tabular}
\end{center}
\end{Exa}
Jos joukko $A$ on yleisempi joukko kuin suorakulmio, niin Riemannin summiin perustuva
numeerinen approksimaatio
\[ 
\int_A f\,d\mu = \int_T f_0\,d\mu 
                 \approx \sum_{k=1}^m \sum_{l=1}^n f_0(\xi_{kl},\eta_{kl})\mu(T_{kl}) 
\]
\begin{multicols}{2} \raggedcolumns
ei välttämättä ole kovin hyvä, syystä että $f$:n nollajatko $f_0$ on (yleensä) epäjatkuva 
sellaisissa suorakulmioissa $T_{kl}$, jotka leikkaavat reunaviivaa $\partial A$ (kuvio).
%\begin{multicols}{2} \raggedcolumns
Tällöin arvio
\[
\int_{T_{kl}} f_0\, d\mu\approx f_0(\xi_{kl},\eta_{kl})\mu(T_{kl})
\]
on pääsääntöisesti kehno, valittiinpa $(\xi_{kl},\eta_{kl}) \in T_{kl}$ miten hyvänsä.
\begin{figure}[H]
\begin{center}
\input{kuvat/kuvaUint-10.pstex_t}
\end{center}
\end{figure}
\end{multicols}

\begin{multicols}{2} \raggedcolumns
Jos halutaan parantaa algoritmia, niin on syytä käyttää tarkempia approksimaatioita
integraaleille
\[
\int_{T_{kl}} f_0\, d\mu=\int_{T_{kl}\cap A} f\, d\mu
\]
silloin kun $T_{kl}\cap\partial A\neq\emptyset$. Esimerkiksi jakoa voidaan paikallisesti
tihentää ja käyttää samaa algoritmia uudelleen ($T$:n tilalla $T_{kl}$, $A$:n tilalla
$A \cap T_{kl}$).
\begin{figure}[H]
\begin{center}
\input{kuvat/kuvaUint-11.pstex_t}
\end{center}
\end{figure}
\end{multicols}

\subsection*{$\R$:n pituusmitta}
\index{pituusmitta (Jordanin)|vahv}
\index{Jordan-mitta!b@$\R$:n pituusmitta|vahv}
\index{mitta, mitallisuus!a@Jordan-mitta|vahv}

Määriteltäessä $\R$:n (Jordanin) \kor{pituusmitta} kirjoitetaan aksiooman A1 tilalle
\begin{itemize}
\item[A1.] \kor{Suljetun välin mitta}: $\,\ A=[a,b]: \,\ \mu(A)=b-a$.
\end{itemize}
Muut aksioomat säilyvät ennallaan. Jos $A\subset\R$ on rajoitettu joukko, niin määritellään
\[
\int_A f\,d\mu = \int_T f_0\,d\mu, \quad T=[a,b] \supset A,
\]
missä $f_0$ on $f$:n nollajatko, ja tässä edelleen
\[
\int_T f_0\,d\mu = \Lim_{h \kohti 0} \sum_{k=1}^n f_0(\xi_k)\mu(T_k),
\]
missä $\,T_k=[x_{k-1},x_k]$, $\,\xi_k \in T_k\,$ ja $\,a=x_0<x_1 < \ldots <x_n=b$. Koska
$\mu(T_k)=x_k-x_{k-1}$, niin tässä $\int_T f_0\,d\mu=\int_a^b f_0(x)\,dx\,$ (Määritelmä
\ref{Riemannin integraali}). Määrätyssä (Riemannin) integraalissa on siis kyse integroimisesta
$\R$:n pituusmitan suhteen. Rajoitetun joukon $A\subset\R$ Jordan-mitta määritellään kuten
edellä, eli integraalin avulla:
\[
\mu(A)=\int_A d\mu.
\]
\begin{Exa} Jos $A=[0,1]\cup[2,4]$ niin integraalin additiivisuuden (tai suoraan määritelmän)
perusteella
\[
\int_A f\,d\mu = \int_{[0,1]} f\,d\mu+\int_{[2,4]} f\,d\mu 
               = \int_0^1 f(x)\,dx+\int_2^4 f(x)\,dx.
\]
Erityisesti on $\mu(A)=\int_A d\mu = 1+2=3$. \loppu
\end{Exa}

\Harj
\begin{enumerate}

\item \label{H-uint-1: T:n valinta}
Näytä, että jos $T_1$ ja $T_2$ ovat joukon $A\subset\R^2$ sisältäviä perussuorakulmioita, niin
samoin on $T_1 \cap T_2$. Päättele tästä, että määritelmä $\int_A f\,d\mu=\int_T f_0\,d\mu$
antaa saman tuloksen perussuorakulmion $T \supset A$ valinnasta riippumatta.

\item \label{H-uint-1: esimerkkejä}
a) Näytä, että joukko $A=([0,1]\cap\Q)\times[0,1]$ ei ole Jordan-mitallinen. \newline
b) Anna esimerkki joukosta $A$, jolle pätee $A\subset[0,1]\times[0,1]$, $\overline{\mu}(A)=1$
ja $\underline{\mu}(A)=0.99$.

\item \label{H-uint-1: väittämiä}
Todista seuraavat rajoitettuja joukkoja $A,B\subset\R^2$ koskevat
väittämät:  \vspace{1mm}\newline 
a) Jos $A \subset B$, niin $\underline{\mu}(A)\le\underline{\mu}(B)$ ja
$\overline{\mu}(A)\le\overline{\mu}(B)$. \vspace{1mm}\newline
b) Jos $\mu(B)=0 $, niin $\underline{\mu}(A \cup B)=\underline{\mu}(A)$ ja
$\overline{\mu}(A \cup B)=\overline{\mu}(A)$. \vspace{1mm}\newline
c) Jos $\mu(A \cap B)=0$, niin 
$\underline{\mu}(A \cup B)=\underline{\mu}(A)+\underline{\mu}(B)$ ja
$\overline{\mu}(A \cup B)=\overline{\mu}(A)+\overline{\mu}(B)$.

\item 
Laske joukon $A = \{(x,y)\in\R^2 \mid x \in [0, 1]\,\ja\, 0 \le y \le x^2\}$ Jordanin 
ulko-ja sisämitalle approksimaatiot jakamalla $T = [0, 1] \times [0, 1]$ suorakulmioihin kokoa
$h \times h^2\ (h^{-1}\in\N)$ ja näiden perusteella mitta $\mu(A)$.

\item
Olkoon $A\subset\R^2$ rajoitettu joukko ja $f$ ja $g$ määriteltyjä ja rajoitettuja $A$:ssa.
Näytä, että jos $f(x,y) \le g(x,y)\ \forall (x,y) \in A$, niin
$\overline{I}(f,\mu,A)\,\le\,\underline{I}(g,\mu,A)$. 

\item \label{H-uint-1: kiertoinvarianssi a}
a) Olkoon $A=$ suorakulmainen kolmio, jonka kärjet ovat pisteissä $(0,0)$, $(a, 0)$ ja $(0, b)$.
Näytä Jordan-mitan määritelmästä, että $\mu(A) = \frac{1}{2} |a| |b|$. \vspace{1mm}\newline
b) Vedoten a-kohdan tulokseen ja mitan additiivisuuteen päättele, että suorakulmion mitta on 
kiertoinvariantti.

\item 
Laske yhdistettyä keskipistesääntöä käyttäen integraali $\int_A (1+x-2 y)\,d\mu$, kun 
$A\subset\R^2$ on $T$:n muotoinen alue, jonka nurkkapisteet ovat $(2,0)$, $(3,0)$, $(3,3)$, 
$(5,3)$, $(5,4)$, $(0,4)$, $(0,3)$ ja $(2,3)$.

\item \label{H-uint-1: toisen asteen integraalit}
Olkoon $A=[-a/2,a/2]\times[-b/2,b/2]$. Jakamalla $A$ samankokoisiin suorakulmioihin ja 
käyttämällä yhdistettyä keskipistesääntöä näytä oikeaksi:
\[
\int_A x^2\,d\mu=\frac{1}{12}\,a^3b, \quad
\int_A y^2\,d\mu=\frac{1}{12}\,ab^3, \quad
\int_A \abs{xy}\,d\mu=\frac{1}{16}\,a^2b^2.
\]

\item 
Olkoon $A = [0,1] \times [0,1]$. Laske seuraavat integraalit jakamalla $A$ neliöihin kokoa 
$h \times h$ ($h^{-1}\in\N$), käyttämällä yhdistettyä keskipistesääntöä, ja laskemalla tuloksen
raja-arvo, kun $h \to 0$.
\[
\text{a)}\ \ \int_A xy^2\,d\mu \qquad
\text{b)}\ \ \int_A x^2y^2\,d\mu \qquad
\text{c)}\ \ \int_A e^{x+y}\,d\mu
\]

\item (*)
Luvussa \ref{pinta-ala ja kaarenpituus} väitettiin, että jos
\[
A = \{\,(x,y) \in \R^2 \mid x \in [a,b]\ \ja\ 0 \le y \le f(x)\,\},
\]
missä $f$ on rajoitettu, ei-negatiivinen ja Riemann-integroituva välillä $[a,b]$, niin
$\,\mu(A)=\int_a^b f(x)\,dx$. Todista väittämä uudelleen näyttämällä, että
\[ 
\underline{\mu}(A) = \underline{\int_a^b} f(x)\,dx, \quad 
\overline{\mu}(A) = \overline{\int_a^b} f(x)\,dx 
\]
ja vetoamalla Lauseeseen \ref{Riemann-integroituvuus}.

\item (*) \label{H-uint-1: integroituvuus L-ehdolla}
Todista Lauseen \ref{jatkuvan funktion integroituvuus tasossa} väittämä, kun jatkuvuuden
sijasta oletetaan, että $f$ toteuttaa Lipschitz-jatkuvuusehdon
\[
\abs{f(x_1,y_1)-f(x_2,y_2)} \le L(\abs{x_1-x_2}+\abs{y_1-y_2}), \quad 
                                              (x_1,y_1),\,(x_2,y_2) \in T.
\]
\kor{Vihje}: Vertaa ylä- ja alasummia $\overline{\sigma}(f_0,\mathcal{T}_h)$ ja
$\underline{\sigma}(f_0,\mathcal{T}_h)$ ($\mathcal{T}_h=T$:n ositus).

\item (*) \label{H-uint-1: fplus ja fmiinus}
Olkoon $A\subset\R^2$ rajoitettu joukko ja $f$ määritelty $A$:ssa. Määritellään
\[
f_+(x,y) = \begin{cases} 
            \,f(x,y), &\text{kun}\ f(x,y)>0, \\ 
            \,0,      &\text{muulloin}. 
           \end{cases}
\]
a) Näytä: $\ \overline{I}(f_+,A)-\underline{I}(f_+,A) \le \overline{I}(f,A)-\underline{I}(f,A)$.
\vspace{1mm}\newline
b) Päättele: Jos $f$ on Riemann-integroituva yli $A$:n, niin samoin ovat $f_+$, $f_-=f-f_+$ ja
$|f|$. 

\item (*) \label{H-uint-1: kiertoinvarianssi b}
Lähtien tehtävän \ref{H-uint-1: kiertoinvarianssi a} tuloksesta todista: Jos $A\subset\R^2$ on
rajoitettu, niin $\underline{\mu}(A)$ ja $\overline{\mu}(A)$ ovat kiertoinvariantteja.
Päättele, että jos $A$ on Jordan-mitallinen, niin $\mu(A)$ on kiertoinvariantti.

\item (*)
Integraalille $\int_A f\,d\mu$, missä $A$ on kolmio, jonka kärjet ovat $(0,0)$, $(1,0)$ ja
$(0,1)$, ja $f(x,y)=x+2y$, lasketaan likiarvo jakamalla $T=[0,1]\times[0,1] \supset A$
neliöihin kokoa $h \times h$ ja laskemalla $\int_T f_0\,d\mu$ yhdistetyllä keskipistesäännöllä.
a) Näytä, että tuloksena on ylälikiarvo, jonka virhe $=\tfrac{3}{2}h+\Ord{h^2}$. \newline
b) Mikä on integraalin tarkka arvo? 

\item (*) \label{H-uint-1: kp-virhearvio}
Olkoon $f$:n osittaisderivaatat toiseen kertalukuun asti jatkuvia suorakulmiossa
$A=[-a/2,a/2]\times[-b/2,b/2]$. Käyttämällä Taylorin lausetta, integraalien vertailuperiaatetta
ja tehtävän \ref{H-uint-1: toisen asteen integraalit} tuloksia näytä oikeaksi keskipistesäännön
virhearvio
\[
\Bigl|\int_A f\,d\mu - f(0,0)\,ab\,\Bigr|\,\le\,\frac{ab}{24}(M_{11}a^2+M_{22}b^2)
                                                +\frac{1}{16}\,M_{12}\,a^2b^2,
\]
missä $M_{11}$, $M_{22}$ ja $M_{12}$ ovat osittaisderivaattojen $f_{xx}$, $f_{yy}$ ja $f_{xy}$
itseisarvojen maksimiarvot $A$:ssa. Näytä edelleen, että jos $A$ jaetaan suorakulmioihin
kokoa $h_1 \times h_2\ (a/h_1\in\N,\ b/h_2\in\N)$, niin yhdistetylle keskipistesäännölle
pätee virhearvio
\[
\abs{E(f)}\,\le\,\frac{ab}{24}\left(M_{11}h_1^2+M_{22}h_2^2+\frac{3}{2}\,M_{12}\,h_1h_2\right).
\]
Mikä $E(f)$ on tarkasti, jos $f$ on toisen asteen polynomi ja tunnetaan osittaisderivaattojen
$f_{xx}$, $f_{yy}$ ja $f_{xy}$ (vakio)arvot $F_{11}$, $F_{22}$ ja $F_{12}\,$?

\end{enumerate}