\chapter{Eksponenttifunktio}  \label{eksponenttifunktio}
\index{eksponenttifunktio|vahv}

Tässä luvussa määriteltävää \kor{eksponenttifunktiota} voi syystä pitää
\pain{f}y\pain{siikan} yleisimpänä funktiona. Eksponenttifunktio esiintyy mitä
moninaisimmissa matemaattisissa malleissa, jotka kuvaavat loputonta kasvua tai 
vähenemistä/vaimenemista. Fysiikan ohella tällaisia \kor{eksponentiaalisen kasvun} tai 
\kor{eksponentiaalisen vaimenemisen} malleja on paljon biologiassa, taloustieteissä, ym.

Matemaattisten funktioiden paljoudessakin eksponenttifunktio erottuu siinä määrin 
erikoislaatuisena, että se ansaitsee oman lukunsa. Jatkossa tarkastellaan tämän funktion
ja sen käänteisfunktion, \kor{logaritmifunktion}, matemaattista määrittelyä ja ominaisuuksia
ensin reaalifunktioina (Luvut \ref{yleinen eksponenttifunktio}--\ref{exp(x) ja ln(x)}). Sen
jälkeen laajennetaan eksponenttifunktio kompleksifunktioksi ja tarkastellaan laajennuksen
synnyttämiä johdannaisfunktioita sekä ---  hieman yllättäviä --- yhteyksiä trigonometrisiin
funktioihin (Luku \ref{kompleksinen eksponenttifunktio}). Viimeisessä osaluvussa käydään
lyhyesti läpi eräitä reaalisen eksponenttifunktion sovellusesimerkkejä fysiikassa.