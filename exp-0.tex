\chapter{Eksponenttifunktio}  \label{eksponenttifunktio}
\index{eksponenttifunktio|vahv}

T�ss� luvussa m��ritelt�v�� \kor{eksponenttifunktiota} voi syyst� pit��
\pain{f}y\pain{siikan} yleisimp�n� funktiona. Eksponenttifunktio esiintyy mit�
moninaisimmissa matemaattisissa malleissa, jotka kuvaavat loputonta kasvua tai 
v�henemist�/vaimenemista. Fysiikan ohella t�llaisia \kor{eksponentiaalisen kasvun} tai 
\kor{eksponentiaalisen vaimenemisen} malleja on paljon biologiassa, taloustieteiss�, ym.

Matemaattisten funktioiden paljoudessakin eksponenttifunktio erottuu siin� m��rin 
erikoislaatuisena, ett� se ansaitsee oman lukunsa. Jatkossa tarkastellaan t�m�n funktion
ja sen k��nteisfunktion, \kor{logaritmifunktion}, matemaattista m��rittely� ja ominaisuuksia
ensin reaalifunktioina (Luvut \ref{yleinen eksponenttifunktio}--\ref{exp(x) ja ln(x)}). Sen
j�lkeen laajennetaan eksponenttifunktio kompleksifunktioksi ja tarkastellaan laajennuksen
synnytt�mi� johdannaisfunktioita sek� ---  hieman yll�tt�vi� --- yhteyksi� trigonometrisiin
funktioihin (Luku \ref{kompleksinen eksponenttifunktio}). Viimeisess� osaluvussa k�yd��n
lyhyesti l�pi er�it� reaalisen eksponenttifunktion sovellusesimerkkej� fysiikassa.