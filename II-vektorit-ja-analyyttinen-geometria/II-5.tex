\section{Funktion käsite. Trigonometriset funktiot} \label{trigonometriset funktiot}
\sectionmark{Trigonometriset funktiot}
\alku

Tässä luvussa esitellään matematiikassa keskeinen käsite \kor{funktio}, sen johdannainen
\kor{reaalifunktio} sekä ensimmäisinä reaalifunktioina \kor{trigonometriset funktiot}.
Osoittautuu, että riittävän yleisesti ja abstraktisti ymmärrettynä  funktioita on esiintynyt
jo aiemmin erinäisillä 'salanimillä'. Reaalifunktioiden osalta ei muita kuin trigonometrisia
funktioita toistaiseksi käsitellä; yleisempi reaalifunktioiden teoria esitetään jäljempänä
Luvuissa IV-VI.

\subsection*{Funktion käsite}
\index{funktio A!a@joukko-opillinen|vahv}

\kor{Funktio} l. \kor{kuvaus} (engl. function, map, mapping) ymmärretään matematiikassa
tavallisimmin kolmikkona muotoa
\[
\{\text{joukko} \ A, \ \text{sääntö} \ f, \ \text{joukko} \ B\}.
\]
Tällä tarkoitetaan, että on jokin sääntö $f$, jonka mukaan määräytyy y\pain{ksikäsitteinen}
$y \in B$ j\pain{okaisella} $x \in A$. Yhteys $x$:n $y$:n välillä merkitään $y=f(x)$ ja
lausutaan '$y$ on $f$ $x$'. Voidaan myös käyttää \pain{liittämisnuolta} ja merkitä 
$x \map f(x) (=y)$. Sanotaan, että $A$ on $f$:n
\index{mzyzy@määrittelyjoukko} \index{maalijoukko}%
\kor{määrittelyjoukko} (tai lähtöjoukko,
engl. domain) ja $B$ \kor{maalijoukko} ja merkitään $f:A \kohti B$. Funktion määrittelyjoukon
alkioihin viitataan yleisnimellä \kor{muuttuja} (engl.\ variable). Jos $y=f(x)$, niin sanotaan,
että $y$ on $f$:n \kor{arvo} (engl.\ value) \kor{$x$:ssä} (tai muuttujan arvolla $x$);
lyhyemmin lausutaan '$f$ $x$:ssä' tai usein myös '$f$ pisteessä $x$'. Maalijoukon osajoukko
\[
\{y \in B \ | \ y=f(x) \ \text{jollakin} \ x \in A\}
\]
\index{arvojoukko} \index{kuva (arvojoukko)}%
on nimeltään $f$:n \kor{arvojoukko} tai $A$:n \kor{kuva} $f$:ssä (engl.\ range, image). Tätä 
merkitään lyhyesti $f(A)$. 
\begin{figure}[H]
\begin{center}
\import{kuvat/}{kuvaII-4.pstex_t}
\end{center}
\end{figure}
\begin{Exa}
Jos $x$ ja $y$ ovat reaalilukuja ja kirjoitetaan $f(x,y)=x^2y-y^3$, tarkoitetaan (ellei 
määrittelyjoukosta toisin sovita) funktiota
\[
\{ \Rkaksi, \ f(x,y)=x^2y-y^3, \ \R \},
\]
joka siis liittää mihin tahansa lukupariin $(x,y)\in\R^2$ laskusäännön ilmaiseman reaaliluvun.
Funktion arvojen määräämisessä tarvitaan tässä tapauksessa ainoastaan reaalilukujen 
kerto- ja vähennyslaskuoperaatioita. Esim.
\begin{align*}
&f(3,-2)=3^2 \cdot (-2) - (-2)^3 = -10, \\
&f(2,\sqrt{2})=2^2\cdot\sqrt{2}-(\sqrt{2})^3=0, \\
&f(\pi,e)=\pi^2 e - e^3=6.74282937.. \loppu
\end{align*}
\end{Exa}

\subsection*{Funktio joukko-opissa}
\index{funktio A!a@joukko-opillinen|vahv}

Hyvin yleisesti ja abstraktisti ajatellen voidaan funktiota pitää 'säännön' sijasta enemmänkin
'luettelona', joka liittää kuhunkin alkioon $x \in A$ yksikäsitteisen ($x$:stä riippuvan)
alkion $y \in B$. Näin ajatellen funktiosta tulee puhtaasti joukko-opillinen käsite: Funktion
määrittelee joukon $A \times B = \{(x,y) \mid x \in A\, \ja\, y \in B\}$ (= $A$:n ja $B$:n
karteesinen tulo, vrt.\ alaviite Luvussa \ref{tasonvektorit}) mikä tahansa osajoukko
$F \subset A \times B$, joka toteuttaa ehdon
\[
\forall x \in A\ [\,(x,y) \in F\,\ \text{täsmälleen yhdellä}\,\ y \in B\,].
\]
Nimittäin tällaista joukkoa vastaa $A$:ssa määritelty funktio $f$, kun tulkitaan
\[
y=f(x)\ \ekv\ (x,y) \in F.
\]
Funktion käsitteen kannalta ei siis ole lopulta lainkaan merkitystä sillä, onko funktio
tunnettu 'lausekkeena' (laskusääntönä) vai pelkkänä 'datana'.\footnote[2]{Käytännössä
'funktiodata' voi perustua esim.\ suoriin mittauksiin tai se voi olla tietokonemalleihin
perustuva laskettu ennuste, kuten sääennustuksissa.} 
\begin{Exa} \pain{Reaaliluku}j\pain{ono} $\{a_1, a_2, \ldots\}$ on tulkittavissa funktioksi
tyyppiä $f: \ \N\kohti\R$. Nimittäin jono on ajateltavissa 'luettelona' $1 \map a_1$, 
$2 \map a_2$, $\ldots$ eli joukkona
\[
F= \{ (1,a_1), (2,a_2),\ldots \} \subset \N \times \Q. \loppu
\]
\end{Exa}
Kaikki tähän asti esiintyneet \pain{laskuo}p\pain{eraatiot} ovat itse asiassa funktioita. 
Funktion voi tällöin mieltää 'laskukoneeksi', joka antaa laskuoperaation lopputuloksen
annetuilla lähtötiedoilla.
\begin{Exa}
Reaalilukujen yhteenlasku ja kertolasku ovat funktioita tyyppiä $f:\ \Rkaksi \map \R$ 
laskusäännöillä
\[
\text{yhteenlasku:} \quad f(x,y)=x+y, \qquad \text{kertolasku:} \quad f(x,y)=xy.
\]
Jakolasku on funktio $f: \{(x,y) \in \Rkaksi \ | \ y \neq 0 \} \kohti \R$ säännöllä
$f(x,y)=x/y$. \loppu
\end{Exa}
\begin{Exa} \label{skalaaritulo funktiona} Edellisessä luvussa liitettiin euklidisen tason
kulmaan $\kulma(\vec a,\vec b)$ geometrisin keinoin määräytyvä reaaliluku, jota merkittiin 
$\cos\kulma(\vec a,\vec b)$. Koska ko.\ luku on yksikäsitteinen, niin kyseessä on funktio 
$\,\cos:\ A \kohti \R$, missä $A=$ kaikkien kulmien joukko. Funktion arvojoukko on 
$[-1,1]\subset\R$. Määrittelyjoukossa voidaan kulmat haluttaessa tulkita tason vektoripareina
$(\vec a,\vec b)$, missä $\vec a,\vec b \neq \vec 0$, tai yksikkövektorien pareina 
($\abs{\vec a}=\abs{\vec b}=1$).  \loppu
\end{Exa}

\subsection*{Reaalifunktio}
\index{funktio A!b@reaalifunktio|vahv}
\index{reaalifunktio|vahv}

'Funktioiden äiti' on \kor{reaalifunktio} eli reaalimuuttujan reaaliarvoinen funktio muotoa
$f: A \kohti \R$, missä $A \subset \R$. Myös termiä 'yhden muuttujan funktio' käytetään usein 
tässä rajatussa merkityksessä. Reaalimuuttujan funktio voidaan kätevästi havainnollistaa
\index{kuvaaja}%
\kor{kuvaajan} (engl.\ graph) avulla. Funktion $f: A \kohti \R$ kuvaaja joukossa
$B \subset A\ (A\subset\R)$ on euklidisen avaruuden $\Ekaksi$ pistejoukko
\[ 
G = \{P = (x,y) \mid y = f(x)\,\ja\, x \in B\},
\]
missä $P$:n koordinaatit $(x,y)$ viittaavat karteesiseen koordinaatistoon (ellei toisin sovita).
Huomattakoon, että jos $f$:n määrittelyjoukko rajataan joukoksi $B$, niin funktion
joukko-opillinen määritelmä on
\[ 
F = \{(x,y)\in\Rkaksi \mid y = f(x)\,\ja\, x \in B\}.
\]
Funktion tulkinta 'käyränä' (eli kuvaajana $G$) siis yksinkertaisesti geometrisoi funktion
abstraktin joukko-opillisen määritelmän.

Kun reaalilukujono tulkitaan funktioksi $f:\ \N\kohti\R$, niin $f$:n kuvaaja on $\Ekaksi$:n
pistejono. Tätä lukujonon kuvaustapaa on edellä jo käytetty Luvuissa \ref{jonon raja-arvo} ja
\ref{Cauchyn jonot}.
\begin{Exa} Alla on kuvattu funktio, jonka laskusääntö on $f(x)=x^2/(x+1)$,
joukoissa $B=[0,4]$ ja $B=[0,4]\cap\N$. Jälkimmäisessä tapauksessa kuvaaja esittää lukujonon
$\seq{a_n}=\{\,n^2/(n+1),\ n=1,2,\ldots\,\}$ alkupäätä. \loppu
\begin{figure}[H]
\setlength{\unitlength}{1cm}
\begin{center}
\begin{picture}(12.5,7.5)(0,-2)
\multiput(0,0)(7,0){2}{
\put(0,0){\vector(1,0){6}} \put(1,-1){\vector(0,1){6}}
\put(1.9,-0.5){$1$} \put(2.9,-0.5){$2$} \put(3.9,-0.5){$3$} \put(4.9,-0.5){$4$} 
\put(5.8,-0.5){$x$} \put(1.2,4.8){$y$}
\multiput(2,0)(1,0){4}{\line(0,-1){0.1}}}
\put(3.6,1.5){$G:\ y=f(x)$}
\put(9,0.5){$\scriptstyle{\bullet}$} \put(10,1.33){$\scriptstyle{\bullet}$}
\put(11,2.25){$\scriptstyle{\bullet}$} \put(12,3.2){$\scriptstyle{\bullet}$}
\put(2.5,-2){$f(x)=\dfrac{x^2}{x+1}$} \put(9.5,-2){$a_n=\dfrac{n^2}{n+1}$}
\put(1,0){
\curve( 
0, 0,
0.5, 0.17,
1, 0.5,
1.5, 0.9,
2, 1.33,
3, 2.25,
4, 3.2)}
\end{picture}
\end{center}
\end{figure}
\end{Exa}
Jos reaalifunktioon liittyy helposti ilmaistavissa oleva laskusääntö, niin funktio
ilmoitetaan tavallisimmin pelkkänä laskusääntönä. Tällöin oletetaan (ellei toisin sovita), että
määrittelyjoukko on suurin $\R$:n osa\-joukko, jossa laskusääntöä voi soveltaa. Yksittäistä
\index{funktioevaluaatio}%
laskusäännön käyttöä, eli laskuoperaatiota $x \map f(x)$ sanotaan \kor{funktioevaluaatioksi}.
Usein tämä onnistuu vain numeerisesti (eli likimäärin),  jolloin käytännössä pystytään vain
laskemaan äärellinen määrä termejä lukujonosta, jonka raja-arvo $=f(x)$.
\begin{Exa} Reaalifunktion $f(x)=x^2/(x^2-2)$ määrittelyjoukko on \newline
$A\ =\ \{x\in\R \mid x^2 \neq 2\}\ 
    =\ (-\infty,-\sqrt{2}) \cup (-\sqrt{2},\sqrt{2}) \cup (\sqrt{2},\infty)$. \loppu
\end{Exa}
\begin{Exa} \pain{Potenssisar}j\pain{a} $\sum_k a_k x^k$ voidaan tulkita reaalifunktioksi, jonka
arvo $x$:ssä on sarjan summa ja määrittelyjoukko on sarjan suppenemisväli. Esimerkiksi funktion
\[
f(x)=\sum_{k=1}^\infty \frac{x^k}{k}
\]
määrittelyjoukko on väli $[-1,1)$, vrt.\ Luku \ref{potenssisarja}. Ellei ole $x=0$, on
funktioevaluaatio $x \map f(x)$ tässä tapauksessa suoritettava numeerisesti, esim.\ 
laskemalla sarjan osasummia. \loppu
\end{Exa}

\pagebreak

\subsection*{Funktiot kosini ja sini}
\index{funktio C!a@$\sin$, $\cos$|vahv}
\index{trigonometriset funktiot|vahv}

Tarkastellaan yksikköympyrää, jonka keskipiste on karteesisen koordinaatiston origossa $O$ ja
jolta on valittu referenssipisteeeksi $A=(1,0)$. Pisteestä $A$ lähtien voidaan yksikköympyrää
kiertää joko \kor{positiiviseen kiertosuuntaan} eli vastapäivään tai
\kor{negatiiviseen kiertosuuntaan} eli myötäpäivään. Edellisessä tapauksessa asetetaan
\index{kulma!d@kiertokulma} \index{kiertokulma}%
\kor{kiertokulman} mitaksi kaarenpituus (= kehää pitkin kuljettu matka), jälkimmäisessä sen
vastaluku. Tällä tavoin kiertokulman mitta voi saada minkä tahansa reaaliarvon, ja jokaista
mittalukua $\alpha$ vastaa yksikäsitteinen kehän piste
\[
P(\alpha)=(x(\alpha),y(\alpha)), \quad \alpha \in \R.
\]
Tässä siis $P$ ei tarkoita yksittäistä pistettä, vaan yksikäsitteistä riippuvuutta 
$\alpha \map P(\alpha)$ eli 'pistearvoista' reaalimuuttujan funktiota. Vastaavasti $x$ ja $y$
tarkoittavat edellä funktioita tyyppiä
\[
x: \ \R \kohti [-1,1], \qquad y: \ \R \kohti [-1,1].
\]
\begin{figure}[H]
\begin{center}
\import{kuvat/}{kuvaII-8.pstex_t}
\end{center}
\end{figure}
Kulman kosinin aiemman määritelmän (ks.\ Luku \ref{skalaaritulo}) mukaisesti on
\[
\cos\kulma AOP(\alpha) = x(\alpha).
\]
Määritellään nyt reaalifunktio cos eli \kor{kosini} vaihtamalla tässä
'kulmamuuttujan' tilalle kiertokulman mittaluku $\alpha$\,:
\[
\cos\alpha = x(\alpha), \quad \alpha\in\R.
\]
Määritellään vastaavasti reaalifunktio sin eli \kor{sini} asettamalla
\[
\sin\alpha = y(\alpha), \quad \alpha \in \R. 
\]
Reaalifunktiot kosini ja sini (määrittelyjoukkona $\R$) ovat \kor{trigonometristen funktioiden}
perustyypit. Yhdenmuotoisten kolmioiden periaatteella nähdään, että määritelmät vastaavat 
tuttuja trigonometrian merkintöjä, kun $0 < \alpha < \pi/2$.
\begin{figure}[H]
\begin{center}
\import{kuvat/}{kuvaII-9.pstex_t}
\end{center}
\end{figure}
Muutamia määritelmistä suoraan seuraavia, tai symmetrian avulla helposti perusteltavia, kosinin
ja sinin ominaisuuksia on lueteltu alla. Nämä ovat voimassa $\forall \alpha \in \R$.
\begin{align}
&\cos(\alpha + 2 \pi) = \cos\alpha, \ \ \sin(\alpha + 2 \pi) = \sin\alpha. \label{trig1} \\
&\cos(-\alpha) = \cos\alpha, \ \ \sin(-\alpha) = -\sin\alpha. \label{trig2} \\
&\cos(\alpha + \pi) = -\cos\alpha, \ \ \sin(\alpha + \pi) = -\sin\alpha. \label{trig3} \\ 
&\sin\alpha = \cos(\pi/2 - \alpha), \ \ \cos\alpha = \sin(\pi/2 - \alpha). \label{trig4} \\
&\cos^2 \alpha + \sin^2 \alpha = 1. \label{trig5}
\end{align}
Näistä ominaisuus \eqref{trig1} (joka seuraa myös \eqref{trig3}:sta) kertoo, että cos ja sin
\index{jaksollinen funktio} \index{funktio B!c@jaksollinen}%
ovat \kor{jaksollisia} (periodisia) funktioita, jaksona $2\pi$ = yksikköympyrän kehän pituus. 
\index{parillinen, pariton!b@funktio} \index{funktio B!b@parillinen, pariton}%
Ominaisuus \eqref{trig2} kertoo, että kosini on \kor{parillinen} (symmetrinen) ja sini 
\kor{pariton} (antisymmetrinen) funktio. Ominaisuus \eqref{trig5} on Pythagoraan lause.
\begin{figure}[H]
\setlength{\unitlength}{1cm}
\begin{picture}(14,4)(-2,-2)
\put(-2,0){\vector(1,0){14}} \put(11.8,-0.4){$x$}
\put(0,-2){\vector(0,1){4}} \put(0.2,1.8){$y$}
%\linethickness{0.5mm}
\multiput(3.14,0)(3.14,0){3}{\drawline(0,-0.1)(0,0.1)}
\put(0.1,-0.4){$0$} \put(3.05,-0.4){$\pi$} \put(6.10,-0.4){$2\pi$} \put(9.20,-0.4){$3\pi$}
\curve(
   -1.5708,    0.0000,
   -1.0708,    0.4794,
   -0.5708,    0.8415,
   -0.0708,    0.9975,
    0.4292,    0.9093,
    0.9292,    0.5985,
    1.4292,    0.1411,
    1.9292,   -0.3508,
    2.4292,   -0.7568,
    2.9292,   -0.9775,
    3.4292,   -0.9589,
    3.9292,   -0.7055,
    4.4292,   -0.2794,
    4.9292,    0.2151,
    5.4292,    0.6570,
    5.9292,    0.9380,
    6.4292,    0.9894,
    6.9292,    0.7985,
    7.4292,    0.4121,
    7.9292,   -0.0752,
    8.4292,   -0.5440,
    8.9292,   -0.8797,
    9.4292,   -1.0000,
    9.9292,   -0.8755,
   10.4292,   -0.5366,
   10.9292,   -0.0663)
\curvedashes[1mm]{0,1,2}
\curve(
   -1.5708,   -1.0000,
   -1.0708,   -0.8776,
   -0.5708,   -0.5403,
   -0.0708,   -0.0707,
    0.4292,    0.4161,
    0.9292,    0.8011,
    1.4292,    0.9900,
    1.9292,    0.9365,
    2.4292,    0.6536,
    2.9292,    0.2108,
    3.4292,   -0.2837,
    3.9292,   -0.7087,
    4.4292,   -0.9602,
    4.9292,   -0.9766,
    5.4292,   -0.7539,
    5.9292,   -0.3466,
    6.4292,    0.1455,
    6.9292,    0.6020,
    7.4292,    0.9111,
    7.9292,    0.9972,
    8.4292,    0.8391,
    8.9292,    0.4755,
    9.4292,   -0.0044,
    9.9292,   -0.4833,
   10.4292,   -0.8439,
   10.9292,   -0.9978)
\put(1,1.2){$y=\sin x$}
\put(5.5,1.2){$y=\cos x$}
\end{picture} 
\end{figure}
Trigonometrisissa funktioevaluaatioissa $x\map\cos x$ ja $x\map\sin x$ on, poikkeuksellisia 
$x$:n arvoja lukuunottamaatta, turvauduttava laskimiin. Laskinta tarvitaan myös, kun halutaan
saada selville, missä kulmassa trigonometrinen funktio saa annetun arvon, ts.\ halutaan 
ratkaista $\alpha$ yhtälöstä $\,\cos\alpha=y\,$ tai $\,\sin\alpha=y\,$, kun tunnetaan
$y\in[-1,1]$. Laskimella saadaan yleensä yhden ratkaisun likiarvo (esim.\ komennolla 
$\cos^{-1} y$ tai $\arccos y$). Muut ratkaisut voidaan tämän jälkeen päätellä laskimeen 
enempää turvautumatta, sillä kosinin ja sinin määritelmien perusteella pätee
\[ \boxed{ \begin{aligned} \quad\ykehys
\cos\alpha=\cos\beta &\qekv \beta=\alpha + n \cdot 2\pi\ 
                            \tai\ \beta=-\alpha + n \cdot 2\pi, \qquad n\in\Z, \\
\sin\alpha=\,\sin\beta &\qekv \beta=\alpha + n \cdot 2\pi\ 
                              \tai\ \beta=\pi-\alpha + n \cdot 2\pi, \quad n\in\Z. \quad\akehys
\end{aligned} } \]
\begin{Exa} Millä kulman $\alpha$ asteluvuilla välillä $[0^{\circ},360^{\circ}]$ on
$\cos 2\alpha=-2/3$ (yhden desimaalin tarkkuus)?
\end{Exa}
\ratk Laskin antaa yhtälölle $\cos x=-2/3$ ratkaisun $x=2.300523.. \vastaa 131.8^{\circ}$,
joten mahdolliset $\alpha$:n arvot ovat
\[ \begin{cases}
\,2\alpha = 131.8^{\circ} + n \cdot 360^{\circ} 
               &\qimpl \alpha = \ 65.9^{\circ} + n \cdot 180^{\circ}, \quad n\in\Z, \\
\,2\alpha = -131.8^{\circ} + m \cdot 360^{\circ} 
               &\qimpl \alpha =  -65.9^{\circ} + m \cdot 180^{\circ}, \quad m\in\Z.
\end{cases} \]
Kysytyt $\alpha$:n arvot suuruusjärjestyksessä ($n=0,\ m=1,\ n=1,\ m=2$)\,:
\[
65.9^{\circ},\ 114.1^{\circ},\ 245.9^{\circ},\ 294.1^{\circ}. \loppu
\]

Perusominaisuuksien \eqref{trig1}--\eqref{trig5} ohella tarvitaan monesti nk.\
\kor{yhteenlaskukaavoja}
\begin{equation} \label{trig6} \boxed{ \begin{aligned}
\ykehys\quad &\cos(\alpha + \beta) 
                    = \cos\alpha\cos\beta - \sin\alpha\sin\beta, \quad  \\
     \akehys &\sin(\alpha + \beta) = \,\sin\alpha\cos\beta + \cos\alpha\sin\beta. 
\end{aligned} } \end{equation}
Nämä ovat voimassa $\forall \alpha, \beta \in \R$. Kaavat on helpointa perustella 
vektorilaskulla, ks.\ kuvio.
\begin{multicols}{2}
\begin{figure}[H]
\setlength{\unitlength}{1cm}
\begin{center}
\begin{picture}(4.5,4.5)(-1.5,0)
\put(2,2){\bigcircle{4}}
\put(2,2){\vector(3,1){1.9}}
\put(2,2){\vector(1,3){0.63}}
\put(2,2){\vector(1,0){2}} \put(2,2){\vector(0,1){2}}
\put(4,2.55){$\vec a$} \put(2.55,4){$\vec b$} \put(4.2,1.9){$\vec i$} \put(1.9,4.2){$\vec j$}
\put(2,2){\arc{2}{-0.32}{0}} \put(3.2,2.1){$\alpha$}
\put(2,2){\arc{1}{-1.25}{0}} \put(2.4,2.4){\begin{turn}{40}$\beta$\end{turn}}
\end{picture}
\end{center}
\end{figure}
\begin{align*}
\vec a &= \cos \alpha \, \vec i + \sin \alpha \, \vec j \\[2mm]
\vec b &= \cos \beta \, \vec i + \sin \beta \, \vec j
\end{align*}
\end{multicols}
Kuvion ja skalaaritulon määritelmän perusteella on
\[
\vec a \cdot \vec b = \cos\alpha\cos\beta + \sin\alpha\sin\beta = \cos(\beta-\alpha).
\]
Vaihtamalla $\alpha \rightarrow -\alpha$ ja käyttämällä symmetriaominaisuuksia \eqref{trig2}
seuraa kaavoista \eqref{trig6} ensimmäinen. Toinen seuraa tämän jälkeen ensimmäisestä sekä
kaavoista \eqref{trig4},\,\eqref{trig2}:
\begin{align*}
\sin(\alpha + \beta) &= \cos\left(\tfrac{\pi}{2} - \alpha - \beta\right) \\[1mm]
                     &= \cos\left(\tfrac{\pi}{2} - \alpha\right) \cos(-\beta) 
                            - \sin\left(\tfrac{\pi}{2} - \alpha\right) \sin(-\beta) \\[1mm]
                     &= \sin \alpha \cos \beta + \cos \alpha \sin \beta.
\end{align*}
Valitsemalla em.\ kaavoissa $\beta = \alpha$ saadaaan monesti kysytyt
\kor{kaksinkertaisen kulman kaavat}:
\begin{equation} \label{trig7} \boxed{ \begin{aligned}
\ykehys\quad \cos 2\alpha &= \cos^2\alpha-\sin^2\alpha \quad \\ 
                          &= 2\cos^2 \alpha - 1 \\
                          &= 1-2\sin^2\alpha, \quad \\
     \akehys \sin 2\alpha &= 2\sin \alpha \cos \alpha.
\end{aligned} } \end{equation}
\begin{Exa} \label{sin kolme alpha} Kaavojen \eqref{trig6},\,\eqref{trig7} ja \eqref{trig5}
perusteella
\begin{align*}
\sin 3\alpha &= \sin (\alpha + 2\alpha) \\
&= \sin \alpha \cos 2\alpha + \cos \alpha \sin 2\alpha \\
&= \sin \alpha(1-2\sin^2 \alpha) + 2\cos^2 \alpha \sin \alpha \\
&= \sin \alpha - 2\sin^3 \alpha + 2(1-\sin^2 \alpha) \sin \alpha \\
&= 3\sin\alpha - 4\sin^3\alpha. \loppu
\end{align*}
\end{Exa}
\begin{Exa} \label{kulman kolmijako}
Laske $\,x = \sin\frac{\pi}{18} = \sin 10\aste$.
\end{Exa}
\ratk Edellisen esimerkin perusteella $x$ toteuttaa kolmannen asteen yhtälön
\[
3x - 4x^3 = \sin 30\aste = \tfrac{1}{2}.
\]
Numeerisin keinoin löydetään ratkaisu $x=0.173648177669303 \ldots$\footnote[2]{Esimerkin luku
$x$ on algebrallinen mutta ei geometrinen, ts.\ $x \not\in\G$ (vrt.\ Luku \ref{geomluvut}). 
Kulmaa $\frac{\pi}{18} = 10\aste$ ei siten ole mahdollista konstruoida geometrisesti.
Yleisestikin gometrisesti konstruoitavissa olevat kulmat ovat melko harva joukko. Esimerkiksi
kulma $3\aste$ voidaan konstruoida (ks.\ Harj.teht. \ref{H-II-5: geometrinen kulma}), sen
sijaan kulmia $2\aste$ ja $1\aste$ ei voida.} \loppu

\subsection*{Tangentti ja kotangentti}
\index{funktio C!b@$\tan$, $\cot$|vahv}
\index{trigonometriset funktiot|vahv}

Muista trigonometrisista funktioista tärkeimmät ovat \kor{tangentti} ($\tan$) ja
\kor{kotangentti} ($\cot$), jotka määritellään
\begin{align*}
\tan\alpha = \frac{\sin\alpha}{\cos\alpha} \qquad 
                     &(\alpha \in \R, \ \alpha \neq (n + \tfrac{1}{2})\pi,\ n \in \Z), \\[3mm]
\cot\alpha = \frac{\cos\alpha}{\sin\alpha} \qquad 
                     &(\alpha \in \R, \ \alpha \neq n \pi, \ n \in \Z).
\end{align*}
\begin{figure}[H]
\setlength{\unitlength}{1cm}
\begin{picture}(14,4.5)(-2.5,-2.5)
\put(-2,0){\vector(1,0){13}} \put(10.8,-0.4){$x$}
\put(0,-2){\vector(0,1){4}} \put(0.2,1.8){$y$}
\multiput(3.14,0)(3.14,0){3}{\drawline(0,-0.1)(0,0.1)}
\put(0.1,-0.4){$0$} \put(3.05,-0.4){$\pi$} \put(6.10,-0.4){$2\pi$} \put(9.20,-0.4){$3\pi$}
\multiput(0,0)(3.14,0){4}{
\curve(
   -1.1,-2,     
   -1.0708,   -1.8305,
   -0.9708,   -1.4617,
   -0.8708,   -1.1872,
   -0.7708,   -0.9712,
   -0.6708,   -0.7936,
   -0.5708,   -0.6421,
   -0.4708,   -0.5090,
   -0.3708,   -0.3888,
   -0.2708,   -0.2776,
   -0.1708,   -0.1725,
   -0.0708,   -0.0709,
    0.0292,    0.0292,
    0.1292,    0.1299,
    0.2292,    0.2333,
    0.3292,    0.3416,
    0.4292,    0.4577,
    0.5292,    0.5848,
    0.6292,    0.7279,
    0.7292,    0.8935,
    0.8292,    1.0917,
    0.9292,    1.3386,
    1.0292,    1.6622,
        1.1,2)}
\curvedashes[1mm]{0,1,2}
\multiput(0,0)(3.14,0){3}{
\curve(
    0.5000,   1.8305,
    0.6000,    1.4617,
    0.7000,    1.1872,
    0.8000,    0.9712,
    0.9000,    0.7936,
    1.0000,    0.6421,
    1.1000,   0.5090,
    1.2000,   0.3888,
    1.3000,    0.2776,
    1.4000,   0.1725,
    1.5000,    0.0709,
    1.6000,  -0.0292,
    1.7000,  -0.1299,
    1.8000,   -0.2333,
    1.9000,   -0.3416,
    2.0000,   -0.4577,
    2.1000,   -0.5848,
    2.2000,   -0.7279,
    2.3000,   -0.8935,
    2.4000,   -1.0917,
    2.5000,   -1.3386,
    2.6000,   -1.6622,
    2.7000,   -2.1154)}
\put(-2.2,-0.4){$y=\tan x$}
\put(1.3,0.5){$y=\cot x$}
\end{picture}
\end{figure}
Tangentin määritelmästä voidaan päätellä, että yhtälöllä $\tan\alpha=y$ on ratkaisu jokaisella
$y\in\R$ ja että ratkaisu on yksikäsitteinen esim.\ lisäehdolla $\alpha \in (-\pi/2,\,\pi/2)$ 
tai $\alpha \in [0,\pi)$. Poikkeuksellisia $y$:n arvoja lukuunottamatta tarvitaan tällaisen
ratkaisun likiarvon hakemiseen laskin (komento $\tan^{-1} y$ tai $\arctan y$). Muut yhtälön 
ratkaisut saadaan perusratkaisusta lisäämällä $\pi$:n monikertoja, sillä tangentilla on
määritelmänsä perusteella ominaisuus
\[
\boxed{\quad\kehys \tan\alpha=\tan\beta \qekv \beta=\alpha + n \cdot \pi, \quad n\in\Z. \quad}
\]

Suoraan kaavasta \eqref{trig5} seuraa tangenttiin liittyvä, usein käyttöön tuleva kaava
\begin{equation} \label{trig8}
\boxed{\quad \frac{\ygehys 1}{\rule[-2mm]{0mm}{2mm} \cos^2 \alpha} = 1 + \tan^2 \alpha. \quad }
\end{equation}
Tangentin avulla saadaan myös joskus tarvittavia \kor{puolen kulman kaavoja}: Kun lähdetään
kaavoista \eqref{trig7} ja käytetään kaavaa \eqref{trig8}, saadaan
\begin{align*}
\sin \alpha &\,=\,2 \sin \frac{\alpha}{2} \cos \frac{\alpha}{2}
             \,=\, 2 \tan \frac{\alpha}{2} \cos^2 \frac{\alpha}{2}
             \,= \frac{2\tan \frac{\alpha}{2}}{1 + \tan^2 \frac{\alpha}{2}}\,, \\
\cos \alpha &\,=\,2\cos^2 \frac{\alpha}{2} - 1
             \,=\,\frac{2}{1+\tan^2 \frac{\alpha}{2}} - 1 
             \,=\,\frac{1 - \tan^2 \frac{\alpha}{2}}{1 + \tan^2 \frac{\alpha}{2}}\,.
\end{align*}
Näin saatiin kaavat \\
\begin{equation} \label{trig9}
\boxed{\quad\left.\begin{array}{ll}
\sin\alpha = \frac{\ykehys \D 2t}{\D 1+t^2} \\ \\
\cos\alpha = \frac{\D 1-t^2}{\D 1+t^2} \\ \\
\tan\alpha = \frac{\D 2t}{\akehys \D 1-t^2}
\end{array}\right\} \ t= \tan \frac{\alpha}{2}\,. \quad}
\end{equation} \\
Näiden mukaan trigonometristen funktioiden arvojen laskemiseen riittää suorittaa ainoastaan yksi
'aidosti trigonometrinen' laskuoperaatio $\alpha \map \tan \frac{\alpha}{2} = t$, minkä jälkeen
tarvitaan vain reaalilukujen kunnan laskuoperaatioita.\footnote[2]{Trigonometristen funktioiden
laskukaavoista \eqref{trig1}--\eqref{trig9} on syytä huomauttaa, että kaikki näissä kaavoissa
esiintyvät laskuoperaatiot, kuten $\alpha\map\sin\alpha$, $\alpha\map\pi/2-\alpha$,
$(\alpha,\beta)\map\alpha+\beta$ tai $\alpha\map\alpha/2$, voidaan toteuttaa geometrisesti, jos
lähtökohtana olevat kulmat tunnetaan geometrisina olioina. Näin ymmärrettynä kaavat 
\eqref{trig1}--\eqref{trig9} ovat siis päteviä, vaikkei kulman mittaa lainkaan määriteltäisi.}

Trigonometrisista funktioista maininnan arvoisia ovat vielä kosinin ja sinin johdannaisfunktiot
\index{funktio C!c@$\sec$, $\csc$}
\kor{sekantti} ($\sec$) ja \kor{kosekantti} ($\csc$):
\begin{align*}
\sec \alpha = \frac{1}{\cos \alpha} \qquad 
                     &(\alpha \in \R, \ \alpha \neq (n + \tfrac{1}{2}) \pi, \ n \in \Z), \\[3mm]
\csc \alpha = \frac{1}{\sin \alpha} \qquad 
                     &(\alpha \in \R, \ \alpha \neq n \pi, \ n \in \Z).
\end{align*}


\subsection*{Sovellusesimerkki: Harmoninen värähtely}
\index{zza@\sov!Harmoninen värähtely|vahv}

Sinin ja kosinin yhdistelmäfunktio $f(x) = A \sin x + B \cos x\ (A,B\in\R)$ on sovelluksissa 
yleinen. Tyypillisesti kyse on 
\index{harmoninen värähtely}%
\pain{harmonisesta} \pain{värähtel}y\pain{stä}, jossa jokin 
fysikaalinen suure $y$ (esim.\ sähköjännite) vaihtelee \pain{aika}muuttujan $t$ mukaan siten, 
että $y(t) = A \sin \omega t + B \sin \omega t$. Suuretta $\omega$  (yksikkö $= 1/$s) sanotaan 
värähtelyn \pain{kulmataa}j\pain{uudeksi}. Mainittuun funktioon $f(x)$ päädytään, kun otetaan 
käyttöön (dimensioton) muuttuja $x=\omega t$.

Tarkasteltava funktio saadaan selvempään muotoon, kun merkitään ensin
\[
R = \sqrt{A^2 + B^2}.
\]
Tällöin on $(A/R)^2+(B/R)^2=1$, eli piste $P=(A/R,B/R)$ on yksikköympyrällä
(ol.\ $(A,B)\neq(0,0)$). Näin ollen jollakin $\alpha\in\R$ pätee
\[
\frac{A}{R} = \cos\alpha, \quad \frac{B}{R} = \sin\alpha \qimpl \tan\alpha = \frac{B}{A}\,.
\]
Näistä viimeinen ehto määrää $\alpha$:n $\,\pi$:n monikertaa vaille 
yksikäsitteisesti. Huomioimalla myös $\sin\alpha$:n tai $\cos\alpha$:n merkki
(kahdesta muusta ehdosta) nähdään, että $\alpha$ on $2\pi$:n monikertaa vaille yksikäsitteinen.
Yhteenlaskukaavan \eqref{trig7} perusteella $f(x)$ voidaan nyt esittää muodossa
\begin{align*}
f(x) &= R(\cos\alpha\sin x + \sin\alpha\cos x) \\
     &= R\sin(x+\alpha) \\
     &= R\sin(x-\beta), \quad \beta=-\alpha.
\end{align*}
Sanotaan tällöin, että $R$ on (esim.\ värähtelyn)
\index{amplitudi} \index{vaihekulma}%
\kor{amplitudi} ja $\alpha$ 
(tai $\beta=-\alpha$) on \kor{vaihekulma} (vaihesiirtymä). Funktion $f$ kuvaaja saadaan siis 
skaalaamalla funktion $\sin x$ kuvaaja amplitudilla $R$ ja siirtämällä skaalattu kuvaaja 
vaihekulman verran $x$-akselin suunnassa.
\begin{figure}[H]
\setlength{\unitlength}{1cm}
\begin{center}
\begin{picture}(6.5,3)(0,-1)
\put(0,0){\vector(1,0){6}} \put(5.8,-0.5){$x$}
\put(1.7,-1){\vector(0,1){3}} \put(1.9,1.8){$y$}
\put(3,0){\line(0,-1){0.1}}   \put(3.1,-0.4){$\beta$}
\put(1.7,1){\line(1,0){0.1}}  \put(2,0.8){R}
\put(3,0){
\curve(
-2.17,-0.22,
   -2, 0.28,
-1.83, 0.70,
-1.67, 0.96,
 -1.5, 0.98,
-1.33, 0.76,
-1.17, 0.36,
   -1,-0.14,
-0.83, -0.6,
-0.67,-0.90,
 -0.5, -1.0,
-0.33,-0.84,
-0.17,-0.48,
    0,    0,
 0.17, 0.48,
 0.33, 0.84,
  0.5,  1.0,
 0.67, 0.90,
 0.83,  0.6,
    1, 0.14,
 1.17,-0.36,
 1.33,-0.76,
  1.5,-0.98,
 1.67,-0.96,
 1.83,-0.70,
    2,-0.28,
 2.17, 0.22)}
\end{picture}
\end{center}
\end{figure}
\begin{Exa} Kirjoita seuraavat funktiot perusmuotoon $f(x)=R\sin(x-\beta)$:
\[
\text{a)}\,\ f(x) = \sin x - \sqrt{3}\,\cos x \qquad \text{b)}\,\ f(x)=-3\sin x - 4\cos x
\]
\end{Exa}
\ratk \ a) \ Tässä on $R=2$, jolloin $\alpha$ ratkeaa ehdoista
\[
\cos\alpha = \frac{1}{2}\,, \quad \sin\alpha = -\frac{\sqrt{3}}{2} \qimpl \alpha 
                                             = -\frac{\pi}{3} + n\cdot 2\pi, \quad n\in\Z.
\]
Siis $f(x) = 2\sin(x-\tfrac{\pi}{3})$.

b) \ Tässä on $R=5$ ja $\tan\alpha=4/3$. Laskin antaa $\alpha=0.927295..\,$ eli 
$\alpha \approx 53.1\aste$, mutta ratkaisu ei ole käypä, koska on $\cos\alpha>0$. Valitaan 
$\alpha=-\pi+0.927295..$ $=-2.214297..\,$ eli $\alpha \approx 53.1\aste-180\aste=-126.9\aste$,
jolloin on edelleen $\tan\alpha=4/3$ ja $\cos\alpha<0$. Siis $f(x)=5\sin(x-\beta)$, missä 
$\beta=2.214297..\,\Vastaa\, 126.9\aste$. \loppu

\Harj
\begin{enumerate}

\item
Psykiatrin vastaanottoa voi kuvata matemaattisesti kolmikkona (psykiatri, potilas, diagnoosi).
Pohdi seuraavissa tapauksissa (kussakin erikseen), millaisilla oletuksilla kyseessä on
funktio: \newline
a) \ psykiatri : potilaat $\kohti$ diagnoosit \newline
b) \ potilas  :  psykiatrit $\kohti$ diagnoosit \newline
c) \ psykiatri : diagnoosit $\kohti$ potilaat \newline
d) \ potilas : diagnoosit $\kohti$ psykiatrit

\item
Funktiot $f$ ja $g$ määritellään laskusäännöillä $f(x,y)=x^5+2x^4y-2x^2y^3$ ja
$g(x,y,z)=(x+y+z)/(1+x^2+y^2+z^2)$, missä $x,y,z\in\R$. Laske $f(2,-3)$, $f(\sqrt{2},\sqrt{2})$,
$g(0,0,0)$ ja $g(1,-2,3)$. 

\item 
Mitkä ovat seuraavien funktioiden määrittelyjoukot ($x,y\in\R$)? 
\[
\text{a)}\,\ f(x)=x^{0} \quad\ \text{b)}\,\ f(x)=\frac{x^2}{x} \quad\ 
\text{c)}\,\ f(x,y)=\frac{x+y}{x^2-y^2}
\]
Voiko nämä funktiot ilmaista määrittelyjoukossaan jollakin yksinkertaisemmalla laskusäännöllä?

\item
Mitä yhteistä ja mitä eroa on seuraavilla reaalifunktioilla? \newline
a) \ $\D{f(x)=\frac{x^4}{x^2}\ \ \text{ja}\ \ g(x)=x^2 \quad\ }$
b) \ $\D{f(x)=\sum_{k=1}^\infty \left(\frac{x}{2}\right)^k\ \text{ja}\ \
               g(x)=\frac{x}{2-x}}$

\item
Mitkä seuraavista joukon $\Rkaksi=\R\times\R$ osajoukoista $F$ ovat funktioita? \newline
a) \ $F=\{(1,2),(2,1),(3,3),(\pi,e),(e,\pi)\}$ \newline
b) \ $F=\{(1,2),(2,3),(3,3),(3,2),(2,1)\}$ \newline
c) \ $F=\{(n,n^2) \mid n\in\Z\}$ \newline
d) \ $F=\{(n^2,n) \mid n\in\Z\}$ \newline
e) \ $F=\{(n^2,n) \mid n\in\N\}$

\item
Tulkitse funktioina (määrittely- ja arvojoukko!) tason vektorien laskuoperaatiot: yhteenlasku,
skalaarilla kertominen ja pistetulo.
 
\item
a) Onko olemassa funktio $f$, jonka määrittelyjoukko $=\R$ ja $\forall x\in\R$ pätee
$f(x)=\sum_{k=0}^\infty x^k$\,? Jos vastaus on myönteinen, niin määrittele $f$! \newline 
b) Yhden reaalimuuttujan sisältävän predikaatin (Luku \ref{logiikka}) voi tulkita funktioksi
ja jopa reaalifunktioksi. Miten?

\item
Olkoon $\sin\alpha=7/25$ ja $\cot\beta=-5/12$. Laske lausekkeen $\sin(\alpha-\beta)$ mahdolliset
arvot.

\item
Nelikulmion sivujen pituudet ovat $1$, $2$, $3$ ja $4$, ja yhden kulman mitta asteina on
$160\aste$. Laske kaikkien nämä ehdot täyttävien nelikulmioiden kolmen muun kulman mitat
$0.1$ asteen tarkkuudella. Piirrä kuviot! 

\item
Ratkaise seuraavat trigonometriset yhtälöt: \newline
a) \ $\sin 2x=\cos 7x\quad\,$ b) \ $\tan 2x=3\tan x\quad\,$ 
c) \ $4\sin^2 x=\tan x$ \newline
d) \ $\abs{\sin x+\abs{\sin x}}=\cos x+\abs{\cos x}\quad\,$ e) \ $\cos 2x=\sin x + \cos x$ 

\item
Ratkaise trigonometrinen epäyhtälö, eli määritä joukko $A\subset\R$ tai $A\subset\Rkaksi$
siten, että epäyhtälö toteutuu täsmälleen kun $x \in A$ tai $(x,y) \in A$\,: \newline
a) \ $\sin\abs{x}<\abs{\sin x}\quad$ b) \ $\abs{\sin 2x}\ge\abs{\sin 3x}\quad$
c) \ $\sin 4x>\cot x-\tan x$ \newline
d) \ $2\sin(x-y^2)>1 \quad$ e) \ $\sin(x-y)+\cos x>0$

\item
a) Johda tangentin yhteenlaskukaava $\ \displaystyle{\,\tan(\alpha + \beta)
   =\frac{\tan\alpha + \tan\beta}{1-\tan\alpha\tan\beta}\,}.$ \vspace{1mm}\newline
b) Tunnetaan $t=\cos\alpha$. Lausu $t$:n avulla $\cos n\alpha$, kun $n=2,3,4,5$. \newline
c) Johda käänteiset yhteenlaskukaavat, joissa $\,\cos x \cos y$, $\,\sin x\sin y\,$ ja \newline
$\,\cos x\sin y\,$ lausutaan $\,\cos{(x\pm y)}$:n ja $\,\sin{(x\pm y)}$:n avulla.

\item \label{H-II-5: trigtuloksia}
a) Näytä, että
\[
\sin\frac{\alpha}{2}=\pm\sqrt{\tfrac{1}{2}(1-\cos\alpha)}, \quad
\cos\frac{\alpha}{2}=\pm\sqrt{\tfrac{1}{2}(1+\cos\alpha)}.
\]
Millä väleillä on molemmissa kaavoissa voimassa etumerkki $+$\,? \newline
b) Johda kaavat
\[
\tan \frac{\alpha}{2}=\frac{\sin \alpha}{1+\cos \alpha}\,, \quad
  \cot \frac{\alpha}{2}=\frac{\sin \alpha}{1-\cos \alpha}\,.
\]

\item
Määritä amplitudi ja vaihekulma: \newline
a) \ $f(x)=3\cos x-4\sin x \quad\ \ \ $ b) \ $f(x)=-4\sin x+\cos x$ \newline
c) \ $f(x)=76\cos x+57\sin x \quad$     d) \ $f(x)\,=\,\sin 2x(\sec x-2\csc x)$

\item
Vaihtovirran kolmen eri vaiheen jännitteet ovat
\[
V_i(t)=V_0\sin(\omega t+\varphi_i),\ \ i=1,2,3, \quad \varphi_2=\varphi_1 + \frac{2\pi}{3}\,,\
                                                      \varphi_3=\varphi_1 + \frac{4\pi}{3}\,.
\]
Määritä vaiheiden 1 ja 2 välisen jännitteen $V_1-V_2$ amplitudi ja vaihekulma. Mikä on 
kaikkien vaiheiden jännitteiden summa?

\item
Saata funktio
\[
f(x) = \sin x + 2\sin(x+\frac{2\pi}{3})+3\sin(x+\frac{4\pi}{3})
\]
muotoon $\,f(x)=R\sin(x+\alpha)$.

\item (*) 
Funktioista puhuttaessa lausutaan $f(x)$ usein '$f$ pisteessä $x$'. Olkoon nyt $x\in\R$ annettu
'piste' ja määritellään 'pistefunktio' $x$ seuraavasti:
\[
x(f)=f(x)\text{ }\forall f\in A,
\]
missä $A=\{$reaalifunktiot, jotka on määritelty pisteessä $x\}$. Onko tällainen 
funktion määrittely todella mahdollinen ja jos on, miten $x(f)$ pitäisi lausua?

\item (*) \label{H-II-5: geometrinen kulma}
Tasakylkinen kolmio, jonka huippukulma $=36\aste$, voidaan jakaa kahteen kolmioon siten, että
molemmat osakolmiot ovat tasakylkisiä. Käyttäen tätä ideaa lähtökohtana laske $x=\sin 3\aste$
tarkasti juurilukujen avulla. Päättele, että kulma $3\aste$ on konstruoitavissa geometrisesti.

\item (*) \label{H-II-5: sinin raja-arvo}
Todista: $\ {\D \sin\frac{\pi}{2^{n+1}} 
               \,\le\, \left(\frac{1}{\sqrt{2}}\right)^n, \quad n=0,1,2, \ldots}$ 

\item (*) \label{H-II-5: minmax}
Määritä funktion $f(x)=\cos^2 x+4\sin x\cos x+3\sin^2 x$ pienin ja suurin arvo sekä minimi-
ja maksimikohdat saattamalla funktio muotoon $f(x)=R\sin(2x+\alpha)+C$.

\end{enumerate}