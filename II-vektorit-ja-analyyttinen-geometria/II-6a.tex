\section{Trigonometriset funktiot} \label{trigonometriset funktiot}
\alku
Tarkastellaan euklidisen tason origokeskistä \kor{yksikköympyrää}
\[
S=\{ P \in \Ekaksi \ \vert \ \abs{\overrightarrow{OP}}=1 \}.
\]
Olkoon $A,B \in S$ ja $\vec a = \overrightarrow{OA}$, $\vec b = \overrightarrow{OB}$, jolloin
$\abs{\vec a}=\abs{\vec b}=1$,  eli $\vec a$ ja $\vec b$ ovat yksikkövektoreita.
\begin{figure}[H]
\setlength{\unitlength}{1cm}
\begin{center}
\begin{picture}(4.5,4)(0,0.2)
\put(2,2){\bigcircle{4}}
\put(2,2){\vector(3,1){1.9}}
\put(2,2){\vector(1,3){0.63}}
\put(1.9,1.5){$O$} \put(3.6,2.1){$\vec a$}\put(2.2,3.5){$\vec b$} 
\put(4,2.55){$A$} \put(2.55,4){$B$}
\end{picture}
\end{center}
\end{figure}
Yksikkövektorit $\vec a$ ja $\vec b$ määrittelevät kulman, jota merkitään aiempaan tapaan
$\kulma(\vec a,\vec b)$ (vaihtoehtoinen merkintä $\kulma AOB$, ks.\ kuvio). Luvussa 
\ref{skalaaritulo} liitettiin kulmaan reaaliluku, kulman kosini. Tätä merkittiin 
$\cos\kulma(\vec a,\vec b)$, jolloin $\,\cos\,$ on tulkittavissa funktioksi, jonka 
määrittelyjoukkona ovat kulmat (vrt.\ edellinen luku):
%(ks. Esimerkki \ref{skalaaritulo funktiona} edellisessä luvussa).
\[
\kulma(\vec a,\vec b)\ \map\ \cos\kulma(\vec a,\vec b).
\]
Jatkossa kehitetään tätä funktioajatusta edelleen niin, että 'kulmamuuttuja' korvataan
reaalimuuttujalla, jolloin kosinista tulee reaalifunktio. Muunnos reaalifunktioksi
tapahtuu ottamalla muuttujaksi kulman sijasta kulman \pain{mittaluku}. Tätä varten 
tarvitaan siis jokin \kor{kulman mitta}. Myös mitta on reaaliarvoinen 'kulman funktio', ts.\ 
mitta liittää kulmaan yksikäsitteisen reaaliluvun. --- Vastaavasti on aiemmin liitetty janaan
(geometrisena oliona) janan mittaluku eli pituus.

Perinteisesti kulmaa on mitattu \kor{asteina} siten, että suora kulma $=90\aste$, oikokulma
$=180\aste$ ja täysi kulma $=360\aste$, jolloin (vrt.\ Luku \ref{skalaaritulo})
\[
\cos\,0\aste = \cos\,360\aste = 1, \quad \cos\,90\aste=0, \quad \cos 180\aste=-1.
\]
Tavanomaisia trigonometrian (kirjaimellisesti 'kolmion mittauksen') tuloksia ovat myös
\[
\cos 60\aste=\frac{1}{2}\,, \quad \cos 45\aste=\frac{1}{\sqrt{2}}\,, \quad
                                  \cos 30\aste=\frac{\sqrt{3}}{2}\,.
\]

Jatkossa pidetään astemittaa kulman mittauksen yhtenä vaihtoehtona. Pääsäännöksi oletetaan
kuitenkin toinen mitta, joka on matemaattisen analyysin kannalta luontevampi. Tässä 
mittaustavassa asetetaan kulman mitaksi funktio $\mu$, joka em.\ kuvioon viitaten määritellään
\[
\mu\bigl(\kulma(\vec a, \vec b)\bigr)=\alpha=\text{kaaren} \ AB \ \text{pituus}.
\]
Tässä mittauksessa on päätettävä, kumpaa kaarta (kahdesta vaihtoehdosta) mitataan, eli on
tehtävä ero \pain{sisäkulman} ja \pain{ulkokulman} välillä (samoin kuin eräissä kulman
geometrisissa tulkinnoissa, vrt.\ alaviite Luvussa \ref{geomluvut}). Mittayksikkö tässä 
mittauksessa on nimeltään \kor{radiaani}.

Kulman mitan ongelma on, että (ympyrä)kaaren pituutta ei ole käsitteenä määritelty.
Toistaiseksi tämä ongelma (joka kätkeytyy myös astemittaukseen) ohitetaan nojaamalla
poikkeuksellisen vahvasti intuitioon, eli olettamalla ympyräkaaren pituus yksinkertaisesti
'tutuksi'. (Tarvittava matemaattinen analyysi esitetään myöhemmin sopivammassa 
asiayhteydessä.) Samoin 'tutuksi' oletetaan reaaliluvun $\pi$ määritelmä yksikköympyrän
$S$ koko kehän pituuden avulla:
\[
\boxed{\kehys\quad 2 \pi = S\text{:n pituus} \quad}
\]
Astemitalla ja kaarenpituusmitalla on tällöin yhteys (= astemitan määritelmä)
\[
\boxed{\kehys\quad \text{kulma asteina}\ 
               =\ \frac{180}{\pi}\,\cdot\,\text{(kulma radiaaneina)} \quad}
\] 

Kun kulmalle on näin sovittu mittaluku radiaaneina, voidaan kosinin määritelmässä
kirjoittaa kulman tilalle yhtä hyvin mittaluku, jolloin tuloksena on reaalimuuttujan
funktio. Tälle käytetään samaa funktiomerkintää kuin geometriselle vastineelleen, eli
kirjoitetaan
\[
\cos\kulma(\vec a, \vec b)=\cos\alpha, \quad 
             \alpha =\mu\bigl(\kulma(\vec a,\vec b)\bigr).\footnote[1]{Muutamissa hyvin 
vakiintuneissa funktiomerkinnöissä, kuten $\cos\alpha$, ei muuttujaa ole tapana erottaa 
sulkeilla.}
\]

Jatkossa on mukavinta sitoa kulman mittaus karteesiseen koordinaatistoon, missä 
referenssipisteeksi em.\ kuviossa valitaan piste $A=(1,0)$ ($O=$ origo). Pisteestä $A$ lähtien
voidaan yksikköympyrää kiertää joko \kor{positiiviseen kiertosuuntaan} eli vastapäivään tai
\kor{negatiiviseen kiertosuuntaan} eli myötäpäivään. Edellisessä tapauksessa kiertokulman
mitta on kaarenpituus (= kehää pitkin kuljettu matka), jälkimmäisessä sen vastaluku. Tällä 
tavoin kulman mitta voi saada minkä tahansa reaaliarvon, ja jokaista mittalukua $\alpha$ vastaa
yksikäsitteinen kehän piste
\[
P(\alpha)=(x(\alpha),y(\alpha)), \ \alpha \in \R.
\]
Tässä siis $P$ ei tarkoita yksittäistä pistettä, vaan yksikäsitteistä riippuvuutta 
$\alpha \map P(\alpha)$ eli 'pistearvoista' funktiota
\[
P: \ \R \map S.
\]
Vastaavasti $x$ ja $y$ tarkoittavat yllä funktioita tyyppiä
\begin{align*}
x: \ \R \map [-1,1], \\
y: \ \R \map [-1,1].
\end{align*}
\begin{figure}[H]
\begin{center}
\import{kuvat/}{kuvaII-8.pstex_t}
\end{center}
\end{figure}

Kosini-funktion aiemman geometrisen määritelmän kanssa sopusoinnussa on sopimus
\[
\cos\alpha = x(\alpha), \ \alpha \in \R.
\]
Tämän mukaisesti kosini tulee määritellyksi koko $\R$:ssä. Määritellään vastaavasti funktio sin
eli \kor{sini} asettamalla
\[
\sin\alpha = y(\alpha), \ \alpha \in \R. 
\]
Kosini ja sini ovat \kor{trigonometristen funktioiden} perustyypit. Yhdenmuotoisten kolmioiden 
periaatteella nähdään, että asetetut määritelmät vastaavat tuttuja trigonometrian kaavoja, kun 
$0 < \alpha < \pi/2$.
\begin{figure}[H]
\begin{center}
\import{kuvat/}{kuvaII-9.pstex_t}
\end{center}
\end{figure}
Muutamia määritelmistä suoraan seuraavia, tai symmetrian avulla helposti perusteltavia, kosinin
ja sinin ominaisuuksia on lueteltu alla. Nämä ovat voimassa $\forall \alpha \in \R$.
\begin{align}
&\cos(\alpha + 2 \pi) = \cos\alpha, \ \ \sin(\alpha + 2 \pi) = \sin\alpha \label{trig1} \\
&\cos(-\alpha) = \cos\alpha, \ \ \sin(-\alpha) = -\sin\alpha \label{trig2} \\
&\cos(\alpha + \pi) = -\cos\alpha, \ \ \sin(\alpha + \pi) = -\sin\alpha \label{trig3} \\ 
&\sin\alpha = \cos(\pi/2 - \alpha), \ \ \cos\alpha = \sin(\pi/2 - \alpha) \label{trig4} \\
&\cos^2 \alpha + \sin^2 \alpha = 1 \label{trig5}
\end{align}
Näistä ominaisuus \eqref{trig1} (joka seuraa myös \eqref{trig3}:sta) kertoo, että cos ja sin ovat
\kor{jaksollisia} (periodisia) funktioita, jaksona $2\pi$ = yksikköympyrän kehän pituus. 
Ominaisuus \eqref{trig2} kertoo, että kosini on \kor{parillinen} (symmetrinen) ja sini 
\kor{pariton} (antisymmetrinen) funktio. Ominaisuus \eqref{trig5} on Pythagoraan lause.
\begin{figure}[H]
\setlength{\unitlength}{1cm}
\begin{picture}(14,4)(-2,-2)
\put(-2,0){\vector(1,0){14}} \put(11.8,-0.4){$x$}
\put(0,-2){\vector(0,1){4}} \put(0.2,1.8){$y$}
%\linethickness{0.5mm}
\multiput(3.14,0)(3.14,0){3}{\drawline(0,-0.1)(0,0.1)}
\put(0.1,-0.4){$0$} \put(3.05,-0.4){$\pi$} \put(6.10,-0.4){$2\pi$} \put(9.20,-0.4){$3\pi$}
\curve(
   -1.5708,    0.0000,
   -1.0708,    0.4794,
   -0.5708,    0.8415,
   -0.0708,    0.9975,
    0.4292,    0.9093,
    0.9292,    0.5985,
    1.4292,    0.1411,
    1.9292,   -0.3508,
    2.4292,   -0.7568,
    2.9292,   -0.9775,
    3.4292,   -0.9589,
    3.9292,   -0.7055,
    4.4292,   -0.2794,
    4.9292,    0.2151,
    5.4292,    0.6570,
    5.9292,    0.9380,
    6.4292,    0.9894,
    6.9292,    0.7985,
    7.4292,    0.4121,
    7.9292,   -0.0752,
    8.4292,   -0.5440,
    8.9292,   -0.8797,
    9.4292,   -1.0000,
    9.9292,   -0.8755,
   10.4292,   -0.5366,
   10.9292,   -0.0663)
\curvedashes[1mm]{0,1,2}
\curve(
   -1.5708,   -1.0000,
   -1.0708,   -0.8776,
   -0.5708,   -0.5403,
   -0.0708,   -0.0707,
    0.4292,    0.4161,
    0.9292,    0.8011,
    1.4292,    0.9900,
    1.9292,    0.9365,
    2.4292,    0.6536,
    2.9292,    0.2108,
    3.4292,   -0.2837,
    3.9292,   -0.7087,
    4.4292,   -0.9602,
    4.9292,   -0.9766,
    5.4292,   -0.7539,
    5.9292,   -0.3466,
    6.4292,    0.1455,
    6.9292,    0.6020,
    7.4292,    0.9111,
    7.9292,    0.9972,
    8.4292,    0.8391,
    8.9292,    0.4755,
    9.4292,   -0.0044,
    9.9292,   -0.4833,
   10.4292,   -0.8439,
   10.9292,   -0.9978)
\put(1,1.2){$y=\sin x$}
\put(5.5,1.2){$y=\cos x$}
\end{picture} 
\end{figure}
Trigonometrisissa funktioevaluaatioissa $x\map\cos x$ ja $x\map\sin x$ on, poikkeuksellisia 
$x$:n arvoja lukuunottamaatta, turvauduttava laskimiin. Laskinta tarvitaan myös, kun halutaan
saada selville, missä kulmassa trigonometrinen funktio saa annetun arvon, ts.\ halutaan 
ratkaista $\alpha$ yhtälöstä $\,\cos\alpha=y\,$ tai $\,\sin\alpha=y\,$, kun tunnetaan
$y\in[-1,1]$. Laskimella saadaan yleensä yhden ratkaisun likiarvo (esim.\ komennolla 
$\cos^{-1} y$ tai $\arccos y$). Muut ratkaisut voidaan tämän jälkeen päätellä laskimeen 
enempää turvautumatta, sillä kosinin ja sinin määritelmien perusteella pätee
\[ \boxed{ \begin{aligned} \quad\ykehys
\cos\alpha=\cos\beta &\qekv \beta=\alpha + n \cdot 2\pi\ \tai\ \beta=-\alpha + m \cdot 2\pi,
                                                        \qquad n,m\in\Z \\
\sin\alpha=\,\sin\beta &\qekv \beta=\alpha + n \cdot 2\pi\ \tai\ \beta=\pi-\alpha + m \cdot 2\pi,
                                                        \quad n,m\in\Z \quad\akehys
\end{aligned} } \]
\begin{Exa} Millä kulman $\alpha$ asteluvuilla välillä $[0^{\circ},360^{\circ}]$ on
$\cos 2\alpha=-2/3$ (yhden desimaalin tarkkuus)?
\end{Exa}
\ratk Laskin antaa yhtälölle $\cos x=-2/3$ ratkaisun $x=2.300523.. \vastaa 131.8^{\circ}$,
joten mahdolliset $\alpha$:n arvot ovat
\[ \begin{cases}
\,2\alpha = 131.8^{\circ} + n \cdot 360^{\circ} 
               &\qimpl \alpha = \ 65.9^{\circ} + n \cdot 180^{\circ}, \quad n\in\Z \\
\,2\alpha = -131.8^{\circ} + m \cdot 360^{\circ} 
               &\qimpl \alpha =  -65.9^{\circ} + m \cdot 180^{\circ}, \quad m\in\Z
\end{cases} \]
Kysytyt $\alpha$:n arvot suuruusjärjestyksessä ($n=0,\ m=1,\ n=1,\ m=2$)\,:
\[
65.9^{\circ},\ 114.1^{\circ},\ 245.9^{\circ},\ 294.1^{\circ} \loppu
\]

Perusominaisuuksien \eqref{trig1}--\eqref{trig5} ohella tarvitaan monesti nk.\
\kor{yhteenlaskukaavoja}
\begin{equation} \label{trig6} \boxed{ \begin{aligned}
\ykehys\quad &\cos(\alpha + \beta) 
                    = \cos\alpha\cos\beta - \sin\alpha\sin\beta \quad  \\
     \akehys &\sin(\alpha + \beta) = \,\sin\alpha\cos\beta + \cos\alpha\sin\beta 
\end{aligned} } \end{equation}
Nämä ovat voimassa $\forall \alpha, \beta \in \R$. Kaavat on helpointa perustella 
vektorilaskulla, ks.\ kuvio.
\begin{multicols}{2}
\begin{figure}[H]
\setlength{\unitlength}{1cm}
\begin{center}
\begin{picture}(4.5,4.5)(-1.5,0)
\put(2,2){\bigcircle{4}}
\put(2,2){\vector(3,1){1.9}}
\put(2,2){\vector(1,3){0.63}}
\put(2,2){\vector(1,0){2}} \put(2,2){\vector(0,1){2}}
\put(4,2.55){$\vec a$} \put(2.55,4){$\vec b$} \put(4.2,1.9){$\vec i$} \put(1.9,4.2){$\vec j$}
\put(2,2){\arc{2}{-0.32}{0}} \put(3.2,2.1){$\alpha$}
\put(2,2){\arc{1}{-1.25}{0}} \put(2.4,2.4){\begin{turn}{40}$\beta$\end{turn}}
\end{picture}
\end{center}
\end{figure}
\begin{align*}
\vec a &= \cos \alpha \, \vec i + \sin \alpha \, \vec j \\[2mm]
\vec b &= \cos \beta \, \vec i + \sin \beta \, \vec j
\end{align*}
\end{multicols}
Kuvion perusteella on
\[
\vec a \cdot \vec b = \cos\alpha\cos\beta + \sin\alpha\sin\beta = \cos(\beta-\alpha).
\]
Vaihtamalla $\alpha \rightarrow (-\alpha)$ ja käyttämällä symmetriaominaisuuksia \eqref{trig2}
seuraa kaavoista \eqref{trig6} ensimmäinen. Toinen seuraa tämän jälkeen ensimmäisestä sekä
kaavoista \eqref{trig4},\,\eqref{trig2}:
\begin{align*}
\sin(\alpha + \beta) &= \cos\left(\tfrac{\pi}{2} - \alpha - \beta\right) \\
                     &= \cos\left(\tfrac{\pi}{2} - \alpha\right) \cos(-\beta) 
                            - \sin\left(\tfrac{\pi}{2} - \alpha\right) \sin(-\beta) \\[1mm]
                     &= \sin \alpha \cos \beta + \cos \alpha \sin \beta.
\end{align*}
Valitsemalla em.\ kaavoissa $\beta = \alpha$ saadaaan monesti kysytyt
\kor{kaksinkertaisen kulman kaavat}:
\begin{equation} \label{trig7} \boxed{ \begin{aligned}
\ykehys\quad \cos 2\alpha &= \cos^2\alpha-\sin^2\alpha \quad \\ 
                          &= 2\cos^2 \alpha - 1 \\
                          &= 1-2\sin^2\alpha \quad \\
     \akehys \sin 2\alpha &= 2\sin \alpha \cos \alpha
\end{aligned} } \end{equation}
\begin{Exa} \label{sin kolme alpha} Kaavojen \eqref{trig6},\,\eqref{trig7} ja \eqref{trig5}
perusteella
\begin{align*}
\sin 3\alpha &= \sin (\alpha + 2\alpha) \\
&= \sin \alpha \cos 2\alpha + \cos \alpha \sin 2\alpha \\
&= \sin \alpha(1-2\sin^2 \alpha) + 2\cos^2 \alpha \sin \alpha \\
&= \sin \alpha - 2\sin^3 \alpha + 2(1-\sin^2 \alpha) \sin \alpha \\
&= 3\sin\alpha - 4\sin^3\alpha \loppu
\end{align*}
\end{Exa}
\begin{Exa} \label{kulman kolmijako}
Laske $\,x = \sin\frac{\pi}{18} = \sin 10^{\circ}$.
\end{Exa}
\ratk Edellisen esimerkin perusteella $x$ toteuttaa kolmannen asteen yhtälön
\[
3x - 4x^3 = \sin 30^{\circ} = \tfrac{1}{2}.
\]
Numeerisin keinoin löydetään ratkaisu $x=0.173648177669303 \ldots$\footnote[1]{Esimerkin luku
$x$ on algebrallinen mutta ei geometrinen, ts.\ $x \not\in\G$ (vrt.\ Luku \ref{geomluvut}). 
Kulmaa $\frac{\pi}{18} = 10^{\circ}$ ei siten ole mahdollista konstruoida geometrisesti lähtien
esim.\ suorasta kulmasta (= mittakulma). Gometrisesti konstruoitavissa olevat kulmat ovat 
yleisesti melko harva joukko. Esimerkiksi kulma $3^{\circ}$ on geometrisesti konstruoitavissa
(ks.\ Harj.teht. \ref{H-II-6: geometrinen kulma}), sen sijaan $2^{\circ}$ ja $1^{\circ}$ eivät
ole.} \loppu

\subsection*{Tangentti ja kotangentti}

Muista trigonometrisista funktioista tärkeimmät ovat \kor{tangentti} ($\tan$) ja
\kor{kotangentti} ($\cot$), jotka määritellään
\begin{align*}
\tan\alpha = \frac{\sin\alpha}{\cos\alpha} \qquad 
                     &(\alpha \in \R, \ \alpha \neq (n + \tfrac{1}{2})\pi,\ n \in \Z) \\[3mm]
\cot\alpha = \frac{\cos\alpha}{\sin\alpha} \qquad 
                     &(\alpha \in \R, \ \alpha \neq n \pi, \ n \in \Z)
\end{align*}
\begin{figure}[H]
\setlength{\unitlength}{1cm}
\begin{picture}(14,4)(-2.5,-2)
\put(-2,0){\vector(1,0){13}} \put(10.8,-0.4){$x$}
\put(0,-2){\vector(0,1){4}} \put(0.2,1.8){$y$}
\multiput(3.14,0)(3.14,0){3}{\drawline(0,-0.1)(0,0.1)}
\put(0.1,-0.4){$0$} \put(3.05,-0.4){$\pi$} \put(6.10,-0.4){$2\pi$} \put(9.20,-0.4){$3\pi$}
\multiput(0,0)(3.14,0){4}{
\curve(
   -1.1,-2,     
   -1.0708,   -1.8305,
   -0.9708,   -1.4617,
   -0.8708,   -1.1872,
   -0.7708,   -0.9712,
   -0.6708,   -0.7936,
   -0.5708,   -0.6421,
   -0.4708,   -0.5090,
   -0.3708,   -0.3888,
   -0.2708,   -0.2776,
   -0.1708,   -0.1725,
   -0.0708,   -0.0709,
    0.0292,    0.0292,
    0.1292,    0.1299,
    0.2292,    0.2333,
    0.3292,    0.3416,
    0.4292,    0.4577,
    0.5292,    0.5848,
    0.6292,    0.7279,
    0.7292,    0.8935,
    0.8292,    1.0917,
    0.9292,    1.3386,
    1.0292,    1.6622,
        1.1,2)}
\curvedashes[1mm]{0,1,2}
\multiput(0,0)(3.14,0){3}{
\curve(
    0.5000,   1.8305,
    0.6000,    1.4617,
    0.7000,    1.1872,
    0.8000,    0.9712,
    0.9000,    0.7936,
    1.0000,    0.6421,
    1.1000,   0.5090,
    1.2000,   0.3888,
    1.3000,    0.2776,
    1.4000,   0.1725,
    1.5000,    0.0709,
    1.6000,  -0.0292,
    1.7000,  -0.1299,
    1.8000,   -0.2333,
    1.9000,   -0.3416,
    2.0000,   -0.4577,
    2.1000,   -0.5848,
    2.2000,   -0.7279,
    2.3000,   -0.8935,
    2.4000,   -1.0917,
    2.5000,   -1.3386,
    2.6000,   -1.6622,
    2.7000,   -2.1154)}
\put(-2.2,-0.4){$y=\tan x$}
\put(1.3,0.5){$y=\cot x$}
\end{picture}
\end{figure}
Tangentin määritelmästä voidaan päätellä, että yhtälöllä $\tan\alpha=y$ on ratkaisu jokaisella
$y\in\R$ ja että ratkaisu on yksikäsitteinen esim.\ lisäehdolla $\alpha \in (-\pi/2,\,\pi/2)$ 
tai $\alpha \in [0,\pi)$. Poikkeuksellisia $y$:n arvoja lukuunottamatta tarvitaan tällaisen
ratkaisun likiarvon hakemiseen laskin (komento $\tan^{-1} y$ tai $\arctan y$). Muut yhtälön 
ratkaisut saadaan perusratkaisusta lisäämällä $\pi$:n monikertoja, sillä tangentilla on
määritelmänsä perusteella ominaisuus
\[
\boxed{\quad\kehys \tan\alpha=\tan\beta \qekv \beta=\alpha + n \cdot \pi, \quad n\in\Z \quad}
\]

Suoraan kaavasta \eqref{trig5} seuraa tangenttiin liittyvä, usein käyttöön tuleva kaava
\begin{equation} \label{trig8}
\boxed{\quad \frac{\ygehys 1}{\rule[-2mm]{0mm}{2mm} \cos^2 \alpha} = 1 + \tan^2 \alpha \quad }
\end{equation}
Tangentin avulla saadaan myös joskus tarvittavia \kor{puolen kulman kaavoja}: Kun lähdetään
kaavoista \eqref{trig7} ja käytetään kaavaa \eqref{trig8}, saadaan
\begin{align*}
\sin \alpha &\,=\,2 \sin \frac{\alpha}{2} \cos \frac{\alpha}{2}
             \,=\, 2 \tan \frac{\alpha}{2} \cos^2 \frac{\alpha}{2}
             \,= \frac{2\tan \frac{\alpha}{2}}{1 + \tan^2 \frac{\alpha}{2}} \\
\cos \alpha &\,=\,2\cos^2 \frac{\alpha}{2} - 1
             \,=\,\frac{2}{1+\tan^2 \frac{\alpha}{2}} - 1 
             \,=\,\frac{1 - \tan^2 \frac{\alpha}{2}}{1 + \tan^2 \frac{\alpha}{2}}
\end{align*}
Näin saatiin kaavat \\
\begin{equation} \label{trig9}
\boxed{\quad\left.\begin{array}{ll}
\sin\alpha = \frac{\ykehys \D 2t}{\D 1+t^2} \\ \\
\cos\alpha = \frac{\D 1-t^2}{\D 1+t^2} \\ \\
\tan\alpha = \frac{\D 2t}{\akehys \D 1-t^2}
\end{array}\right\} \ t= \tan \frac{\alpha}{2} \quad}
\end{equation} \\
Näiden mukaan trigonometristen funktioiden arvojen laskemiseen riittää suorittaa ainoastaan yksi
'aidosti trigonometrinen' laskuoperaatio $\alpha \map \tan \frac{\alpha}{2} = t$, minkä jälkeen
tarvitaan vain reaalilukujen kunnan laskuoperaatioita.\footnote[1]{Trigonometristen funktioiden
laskukaavoista \eqref{trig1}--\eqref{trig9} on syytä huomauttaa, että kaikki näissä kaavoissa
esiintyvät laskuoperaatiot, kuten $\alpha\map\sin\alpha$, $\alpha\map\pi/2-\alpha$,
$(\alpha,\beta)\map\alpha+\beta$ tai $\alpha\map\alpha/2$, voidaan toteuttaa geometrisesti, jos
lähtökohtana olevat kulmat tunnetaan geometrisina olioina. Näin ymmärrettynä kaavat 
\eqref{trig1}--\eqref{trig9} ovat siis päteviä, vaikkei kulman mittaa lainkaan määriteltäisi.}

Trigonometrisista funktioista maininnan arvoisia ovat vielä kosinin ja sinin johdannaisfunktiot
\kor{sekantti} ($\sec$) ja \kor{kosekantti} ($\csc$):
\begin{align*}
\sec \alpha = \frac{1}{\cos \alpha} \qquad 
                     &(\alpha \in \R, \ \alpha \neq (n + \tfrac{1}{2}) \pi, \ n \in \Z) \\[3mm]
\csc \alpha = \frac{1}{\sin \alpha} \qquad 
                     &(\alpha \in \R, \ \alpha \neq n \pi, \ n \in \Z)
\end{align*}


\subsection*{Funktio $f(x) = A \sin x + B \cos x$}

Sinin ja kosinin yhdistelmäfunktio $f(x) = A \sin x + B \cos x\ (A,B\in\R)$ on sovelluksissa 
yleinen. Kyse voi olla esim.\ \pain{harmonisesta} \pain{värähtel}y\pain{stä}, jossa jokin 
fysikaalinen suure $y$ (esim.\ sähköjännite) vaihtelee \pain{aika}muuttujan $t$ mukaan siten, 
että $y(t) = A \sin \omega t + B \sin \omega t$. Suuretta $\omega$  (yksikkö $= 1/$s) sanotaan 
\pain{kulmataa}j\pain{uudeksi}. Otsikon perustilanteeseen päädytään tällöin, kun otetaan 
käyttöön (dimensioton) muuttuja $x=\omega t$.

Tarkasteltava funktio saadaan selvempään muotoon, kun merkitään ensin
\[
R = \sqrt{A^2 + B^2}.
\]
Tällöin on $(A/R)^2+(B/R)^2=1$, eli piste $P=(A/R,B/R)$ on yksikköympyrällä (ol.\ $R>0$). 
Näinollen jollakin $\alpha\in\R$ pätee
\[
\frac{A}{R} = \cos\alpha, \quad \frac{B}{R} = \sin\alpha \qimpl \tan\alpha = \frac{B}{A}\,.
\]
Näistä viimeinen ehto määrää $\alpha$:n $\,\pi$:n monikertaa vaille 
yksikäsitteisesti. Huomioimalla myös $\sin\alpha$:n tai $\cos\alpha$:n merkki
(kahdesta muusta ehdosta) nähdään, että $\alpha$ on $2\pi$:n monikertaa vaille yksikäsitteinen.
Yhteenlaskukaavan \eqref{trig7} perusteella $f(x)$ voidaan nyt esittää muodossa
\begin{align*}
f(x) &= R(\cos\alpha\sin x + \sin\alpha\cos x) \\
     &= R\sin(x+\alpha) \\
     &= R\sin(x-\beta), \quad \beta=-\alpha.
\end{align*}
Sanotaan tällöin, että $R$ on (esim.\ värähtelyn) \kor{amplitudi} ja $\alpha$ 
(tai $\beta=-\alpha$) on \kor{vaihekulma} (vaihesiirtymä). Funktion $f$ kuvaaja saadaan siis 
skaalaamalla funktion $\sin x$ kuvaaja amplitudilla $R$ ja siirtämällä skaalattu kuvaaja 
vaihekulman verran $x$-akselin suunnassa.
\begin{figure}[H]
\setlength{\unitlength}{1cm}
\begin{center}
\begin{picture}(6.5,3)(0,-1)
\put(0,0){\vector(1,0){6}} \put(5.8,-0.5){$x$}
\put(1.7,-1){\vector(0,1){3}} \put(1.9,1.8){$y$}
\put(3,0){\line(0,-1){0.1}}   \put(3.1,-0.4){$\beta$}
\put(1.7,1){\line(1,0){0.1}}  \put(2,0.8){R}
\put(3,0){
\curve(
-2.17,-0.22,
   -2, 0.28,
-1.83, 0.70,
-1.67, 0.96,
 -1.5, 0.98,
-1.33, 0.76,
-1.17, 0.36,
   -1,-0.14,
-0.83, -0.6,
-0.67,-0.90,
 -0.5, -1.0,
-0.33,-0.84,
-0.17,-0.48,
    0,    0,
 0.17, 0.48,
 0.33, 0.84,
  0.5,  1.0,
 0.67, 0.90,
 0.83,  0.6,
    1, 0.14,
 1.17,-0.36,
 1.33,-0.76,
  1.5,-0.98,
 1.67,-0.96,
 1.83,-0.70,
    2,-0.28,
 2.17, 0.22)}
\end{picture}
\end{center}
\end{figure}
\begin{Exa} Kirjoita seuraavat funktiot perusmuotoon $f(x)=R\sin(x+\alpha)$:
\[
\text{a)}\,\ f(x) = \sin x - \sqrt{3}\,\cos x \qquad \text{b)}\,\ f(x)=-3\sin x - 4\cos x
\]
\end{Exa}
\ratk \ a) \ Tässä on $R=2$, jolloin $\alpha$ ratkeaa ehdoista
\[
\cos\alpha = \frac{1}{2}\,, \quad \sin\alpha = -\frac{\sqrt{3}}{2} \qimpl \alpha 
                                             = -\frac{\pi}{3} + n\cdot 2\pi, \quad n\in\Z.
\]
Siis $f(x) = 2\sin(x-\tfrac{\pi}{3})$.

b) \ Tässä on $R=5$ ja $\tan\alpha=4/3$. Laskin antaa $\alpha=0.927295..\,$ eli 
$\alpha \approx 53.1\aste$, mutta ratkaisu ei ole käypä, koska on $\cos\alpha>0$. Valitaan 
$\alpha=-\pi+0.927295..$ $=-2.214297..\,$ eli $\alpha \approx 53.1\aste-180\aste=-126.9\aste$,
jolloin on edelleen $\tan\alpha=4/3$ ja $\cos\alpha<0$. Siis $f(x)=5\sin(x-\beta)$, missä 
$\beta=2.214297..\,\Vastaa\, 126.9\aste$. \loppu

\Harj
\begin{enumerate}

\item
Olkoon $\sin\alpha=7/25$ ja $\cot\beta=-5/12$. Laske lausekkeen $\sin(\alpha-\beta)$ mahdolliset
arvot.

\item
Nelikulmion sivujen pituudet ovat $1$, $2$, $3$ ja $4$, ja yhden kulman mitta asteina on
$160\aste$. Laske kaikkien nämä ehdot täyttävien nelikulmioiden kolmen muun kulman mitat
$0.1$ asteen tarkkuudella. Piirrä kuviot! 

\item
Ratkaise seuraavat trigonometriset yhtälöt: \newline
a) \ $\sin 2x=\cos 7x\quad\,$ b) \ $\tan 2x=3\tan x\quad\,$ 
c) \ $4\sin^2 x=\tan x$ \newline
d) \ $\abs{\sin x+\abs{\sin x}}=\cos x+\abs{\cos x}\quad\,$ e) \ $\cos 2x=\sin x + \cos x$ 

\item
Ratkaise trigonometrinen epäyhtälö, eli määritä joukko $A\subset\R$ tai $A\subset\Rkaksi$
siten, että epäyhtälö toteutuu täsmälleen kun $x \in A$ tai $(x,y) \in A$\,: \newline
a) \ $\sin\abs{x}<\abs{\sin x}\quad$ b) \ $\abs{\sin 2x}\ge\abs{\sin 3x}\quad$
c) \ $\sin 4x>\cot x-\tan x$ \newline
d) \ $2\sin(x-y^2)>1 \quad$ e) \ $\sin(x-y)+\cos x>0$

\item
a) Johda tangentin yhteenlaskukaava $\ \displaystyle{\,\tan(\alpha + \beta)
   =\frac{\tan\alpha + \tan\beta}{1-\tan\alpha\tan\beta}\,}.$ \vspace{1mm}\newline
b) Tunnetaan $t=\cos\alpha$. Lausu $t$:n avulla $\cos n\alpha$, kun $n=2,3,4,5$. \newline
c) Johda käänteiset yhteenlaskukaavat, joissa $\,\cos x \cos y$, $\,\sin x\sin y\,$ ja \newline
$\,\cos x\sin y\,$ lausutaan $\,\cos{(x\pm y)}$:n ja $\,\sin{(x\pm y)}$:n avulla.

\item \label{H-II-6: trigtuloksia}
a) Näytä, että
\[
\sin\frac{\alpha}{2}=\pm\sqrt{\tfrac{1}{2}(1-\cos\alpha)}, \quad
\cos\frac{\alpha}{2}=\pm\sqrt{\tfrac{1}{2}(1+\cos\alpha)}.
\]
Millä väleillä on molemmissa kaavoissa voimassa etumerkki $+$\,? \newline
b) Johda kaavat
\[
\tan \frac{\alpha}{2}=\frac{\sin \alpha}{1+\cos \alpha}\,, \quad
  \cot \frac{\alpha}{2}=\frac{\sin \alpha}{1-\cos \alpha}\,.
\]

\item
Määritä amplitudi ja vaihekulma: \newline
a) \ $f(x)=3\cos x-4\sin x \quad\ \ \ $ b) \ $f(x)=-4\sin x+\cos x$ \newline
c) \ $f(x)=76\cos x+57\sin x \quad$     d) \ $f(x)\,=\,\sin 2x(\sec x-2\csc x)$

\item
Vaihtovirran kolmen eri vaiheen jännitteet ovat
\[
V_i(t)=V_0\sin(\omega t+\varphi_i),\ \ i=1,2,3, \quad \varphi_2=\varphi_1 + \frac{2\pi}{3}\,,\
                                                      \varphi_3=\varphi_1 + \frac{4\pi}{3}\,.
\]
Määritä vaiheiden 1 ja 2 välisen jännitteen $V_1-V_2$ amplitudi ja vaihekulma. Mikä on 
kaikkien vaiheiden jännitteiden summa?

\item
Saata funktio
\[
f(x) = \sin x + 2\sin(x+\frac{2\pi}{3})+3\sin(x+\frac{4\pi}{3})
\]
muotoon $\,f(x)=R\sin(x+\alpha)$.

\item (*) \label{H-II-6: geometrinen kulma}
Tasakylkinen kolmio, jonka huippukulma $=36\aste$, voidaan jakaa kahteen kolmioon siten, että
molemmat osakolmiot ovat tasakylkisiä. Käyttäen tätä ideaa lähtökohtana laske $x=\sin 3\aste$
tarkasti juurilukujen avulla. Päättele, että kulma $3\aste$ on konstruoitavissa geometrisesti.

\item (*) \label{H-II-6: sinin raja-arvo}
Todista: $\ {\D \sin\frac{\pi}{2^{n+1}} 
               \,\le\, \left(\frac{1}{\sqrt{2}}\right)^n, \quad n=0,1,2, \ldots}$ 

\item (*) \label{H-II-6: minmax}
Määritä funktion $f(x)=\cos^2 x+4\sin x\cos x+3\sin^2 x$ pienin ja suurin arvo sekä minimi-
ja maksimikohdat saattamalla funktio muotoon $f(x)=R\sin(2x+\alpha)+C$.

\end{enumerate}