\section{Osamurtokehitelmät. Sarjamenetelmä} \label{osamurtokehitelmät}
\alku
\index{osamurtokehitelmä|vahv}

Yleisen rationaalifunktion integraalifunktio on löydettävissä käyttäen nk. 
\kor{osamurtokehitelmää} yhdessä edellisen luvun tekniikoiden kanssa. Tarkastellaan yleistä
(reaalikertoimista) rationaalifunktiota
\[
f(x)=\frac{p(x)}{q(x)}=\frac{a_mx^m+\cdots +a_0}{b_nx^n+\cdots b_0},
\]
missä $m\geq 0$, $n\geq 1$ ja $a_m,b_n\neq 0$. Jos $m\geq n$, suoritetaan ensiksi jakolasku:
\[
\frac{p(x)}{q(x)}=p_0(x)+\frac{p_1(x)}{q(x)}\,,
\]
missä $p_0$ on polynomi astetta $m-n$,
\[
p_0(x)=c_{m-n}x^{m-n}+\cdots +c_0,
\]
ja jakojäännöksen $p_1(x)$ aste on $n-1$:
\[
p_1(x)=d_{n-1}x^{n-1}+\cdots +d_0.
\]
Jakolasku on suoritettavissa esimerkiksi yksinkertaisesti identifioimalla polynomien kertoimet
yhtälössä
\[
p(x)=p_0(x)q(x)+p_1(x),
\]
jolloin tuntemattomille kertoimille $c_{m-n},\ldots,c_0$ ja $d_{n-1},\ldots,d_0$ saadaan
(ratkeava) lineaarinen yhtälöryhmä. Seuraavassa esimerkissä käytetään vaihtoehtoista
\index{jakoalgoritmi (polynomien)} \index{polynomien jakoalgoritmi}%
(myös yleispätevää) menetelmää, \kor{polynomien jakoalgoritmia}, joka on etenkin käsinlaskussa
kätevä. Algoritmin yleisperiaate on esimerkistä arvattavissa.
\begin{Exa} Laske $p_0$ ja $p_1$, kun $\,\D{f(x)=\frac{x^3+2x^2}{x^2-2x+3}}\,$.
\end{Exa}
\ratk
\begin{align*}
x^3+2x^2                          &= [x(x^2-2x+3)+2x^2-3x]+2x^2 \\
                                  &= x(x^2-2x+3)+4x^2-3x \\
                                  &= x(x^2-2x+3)+[4(x^2-2x+3)+8x-12]-3x \\
                                  &= (x+4)(x^2-2x+3)+5x-12 \\[3mm]
\impl \ \frac{x^3+2x^2}{x^2-2x+3} &= x+4+\frac{5x-12}{x^2-2x+3}
                                   =p_0(x)+\frac{p_1(x)}{q(x)}\,. \loppu
\end{align*}

Kun jakolasku on suoritettu, saadaan integraalin lineaarisuuden nojalla
\[
\int\frac{p(x)}{q(x)}\, dx=\int p_0(x)\, dx+\int\frac{p_1(x)}{q(x)}\, dx.
\]
Tässä polynomiosan integraalifunktio on heti määrättävissä (polynomina), joten jatkossa
voidaan rajoittua tutkimaan jälkimmäistä termiä, jossa osoittajan aste on pienempi kuin
nimittäjän aste. Oletetaan, että tunnetaan kaikki $q$:n (reaaliset ja kompleksiset)
nollakohdat --- nämähän ovat ainakin numeerisesti laskettavissa. Jos nollakohdat
$x_1\ldots x_n$ ovat kaikki reaalisia ja yksinkertaisia, niin saadaan osamurtokehitelmä
(ks.\ Lause \ref{osamurtolause} alla)
\[
\frac{p_1(x)}{q(x)}=\frac{A_1}{x-x_1}+\cdots +\frac{A_n}{x-x_n}\,.
\]
Kertoimet $A_i$ voidaan määrätä lineaarisesta yhtälöryhmästä, joka syntyy, kun tämä yhtälö
kerrotaan puolittian $q(x)$:llä ja identifioidaan polynomien kertoimet vasemmalla ja oikealla.
Helpommin kertoimet kuitenkin lasketaan kirjoittamalla $q(x)=(x-x_i)q_i(x)$, jolloin
kertomalla yhtälö puolittain $(x-x_i)$:llä seuraa
\[
A_i=\lim_{x\kohti x_i} \left[(x-x_i)\frac{p_1(x)}{q(x)}\right] =\, \frac{p(x_i)}{q_i(x_i)}\,.
\]
Jos $q(x)$:llä on tekijänä $(x-a)^k$ vastaten $k$-kertaista reaalijuurta $x_i=a$, niin
osamurtokehitelmässä tätä tekijää vastaavat yleisemmin termit
\[
(x-a)^k\ \ext\ \frac{A_1}{x-a}+\frac{A_2}{(x-a)^2}+\cdots +\frac{A_k}{(x-a)^k}\,.
\]
Näistä kukin termi on suoraan integroitavissa. Jos lopuksi $q(x)$:llä on tekijänä
$(x^2+ax+b)^k$ vastaten $k$-kertaista kompleksista konjugaattijuuriparia 
(jolloin on $b-a^2/4>0$), niin tämä tekijä tuottaa osamurtokehitelmään termit
\[
(x^2+ax+b)^k\ \ext\ \frac{B_1 x+C_1}{x^2+ax+b}
             +\frac{B_2 x+C_2}{(x^2+ax+b)^2}+\cdots + \frac{B_k x+C_k}{(x^2+ax+b)^k}\,.
\]
Näiden integroimiseksi huomataan ensinnäkin, että
\[
\frac{Ax+B}{(x^2+ax+b)^l}\,
          =\,\frac{A}{2}\,\frac{2x+a}{(x^2+ax+b)^l}+\frac{B-\frac{1}{2}\,aA}{(x^2+ax+b)^l}\,,
\]
joten
\begin{align*}
\int\frac{Ax+B}{(x^2+ax+b)^l}\,dx\, 
        &=\,\frac{A}{2}\cdot\begin{cases}
                            \ln (x^2+ax+b),                                &\text{jos}\ l=1, \\
                            -\dfrac{1}{l-1}\,\dfrac{1}{(x^2+ax+b)^{l-1}}\,, &\text{jos}\ l>1
                            \end{cases} \\
        &+\,(B-\frac{1}{2}\,aA)\int\frac{1}{(x^2+ax+b)^l}\ dx.
\end{align*}
Tässä jäljelle jäänyt integraali on käsiteltävissä edellisen luvun menetelmin, sillä kun
kirjoitetaan
\[
x^2+ax+b = \left(x+\frac{a}{2}\right)^2 + D^2, \quad D=\sqrt{b-\frac{a^2}{4}}\,,
\]
niin sijoituksella 
\[
x+\frac{a}{2} = Dt, \quad dx = D dt
\]
saadaan
\[
\int\frac{1}{(x^2+ax+b)^l}\ dx 
        = D^{-2l+1}\int \frac{1}{(t^2+1)^l}\,dt, \quad t=D^{-1}\left(x+\frac{a}{2}\right).
\]

Kun em.\ tulokset yhdistetään ja huomioidaan edellisen luvun tulokset, voidaan päätellä, että
jokaisen rationaalifunktion integraalifunktio on muotoa
\[
\int \frac{p(x)}{q(x)}\, dx=R(x)+\sum A_iF_i(x)+C,
\]
missä $R$ on rationaalifunktio (tai polynomi), kertoimet $A_i$ ovat reaalilukuja ja kukin
$F_i$ (enintään $n$ kpl, $n=q$:n aste) on jokin seuraavista funktiomuodoista:
\[
\ln\abs{x+a_i},\quad\ln(x^2+a_ix+b_i),\quad\Arctan(a_ix+b_i).
\]

On siis päätelty, että rationaalifunktio voidaan aina integroida suljetussa muodossa, kun
nimittäjäpolynomin kaikki nollakohdat (kertalukuineen) tunnetaan. Päätelmä perustui
oletukseen, että osamurtokehitelmä edellä kuvatulla tavalla on aina mahdollinen, ts. että
pätee
\begin{Lause} (\vahv{Rationaalifunktion osamurtohajotelma}) \label{osamurtolause}
\index{rationaalifunktio!a@osamurtohajotelma|emph}
Olkoon $f(x)=p(x)/q(x)$ reaalinen rationaalifunktio, jossa osoittajapolynomin $p$ aste
$<q$:n aste. Olkoon nimittäjäpolynomin $q$ reaalijuuret $x_i,\ i=1 \ldots n_1$ ja
kompleksijuuret $c_j=a_j \pm ib_j,\ j=1 \ldots n_2\,$ ($a_j,b_j\in\R,\ b_j \neq 0$). Edelleen
olkoon näiden juurien kertaluvut $m_i$, $i=1 \ldots n_1$ ja $\nu_j$, $j=1 \ldots n_2$.
Tällöin on olemassa kertoimet $A_{ik},B_{jk},C_{jk}\in\R$ siten, että jokaisella $x\in\DF_f$
pätee
\[
f(x) = \sum_{i=1}^{n_1}\sum_{k=1}^{m_i} \frac{A_{ik}}{(x-x_i)^k}
     + \sum_{j=1}^{n_2}\sum_{k=1}^{\nu_j} \frac{B_{jk}x+C_{jk}}{[(x-a_j)^2+b_j^2]^k}\,.
\]
\end{Lause}
\tod Sivuutetaan, ks.\ Harj.teht.\,\ref{H-int-3: osamurtohajotelma}.
\begin{Exa}
$\D\int\frac{x^2}{x^3+x^2+x-3}\, dx\ =\ ?$
\end{Exa}
\ratk Nimittäjäpolynomin eräs juuri on $x_1=1:\ q(x)=x^3+x^2+x-3=(x-1)(x^2+2x+3)$. Muut kaksi
juurta ovat kompleksisia (konjugaattipari), joten osamurtokehitelmä on muotoa
\[
\frac{x^2}{x^3+x^2+x-3}=\frac{A}{x-1}+\frac{Bx+C}{x^2+2x+3}\,.
\]
Kertomalla tässä molemmat puolet $q(x)$:llä saadaan ensin määrätyksi kertoimet $A,B,C$, minkä
jälkeen integrointi onnistuu:
\begin{align*}
&\qquad\qquad x^2= A(x^2+2x+3)+(Bx+C)(x-1)\quad\forall x\in\R \\
&\qekv       (A+B-1)x^2+(2A-B+C)x+(3A-C)=0\quad\forall x\in\R \\
&\qekv       \begin{cases} A+B-1 &= 0 \\ 2A-B+C &= 0 \\ 3A-C &= 0 \end{cases} 
             \qekv \begin{cases} A &= 1/6 \\ B &= 5/6 \\ C &= 1/2 \end{cases} \\[1mm]
&\qimpl\int\frac{x^2}{x^3+x^2+x-3}\,dx \,=\, 
                \frac{1}{6}\int\frac{1}{x-1}\, dx+\frac{1}{6}\int\frac{5x+3}{x^2+2x+3}\, dx \\
&\qquad\qquad=\,\frac{1}{6}\ln\abs{x-1}+\frac{1}{6}\int\left(
                \frac{5}{2}\,\frac{2x+2}{x^2+2x+3}-\frac{2}{x^2+2x+3}\right)dx \\
&\qquad\qquad=\,\frac{1}{6}\ln\abs{x-1}+\frac{5}{12}\int\frac{2x+2}{x^2+2x+3}\,dx
               -\frac{1}{6}\int\frac{1}{(\frac{x+1}{\sqrt{2}})^2+1}\,dx \\
&\qquad\qquad=\,\underline{\underline{\frac{1}{6}\ln\abs{x-1}+\frac{5}{12}\ln(x^2+2x+3)
                            -\frac{\sqrt{2}}{6}\Arctan\left(\frac{x+1}{\sqrt{2}}\right)+C}}.
\end{align*}
Kerroin $A$ olisi voitu määrätä myös suoremmin laskemalla
\[
A\ =\ \lim_{x\kohti 1}\left[(x-1)\frac{x^2}{x^3+x^2+x-3}\right]\ 
   =\ \lim_{x\kohti 1}\frac{x^2}{x^2+2x+3}\ =\ \frac{1}{6}\,. \loppu
\]

\subsection*{Rationaalisiksi palautuvia integraaleja}

Jos $R(x,y)$ on kahden muuttujan rationaalifunktio, niin mm. seuraavat integraalit ovat
palautettavissa rationaalisiksi sopivilla sijoituksilla:
\begin{itemize}
\item[] a)\ $\int R(\sin x,\cos x)\,dx, \quad$ b)\ $\int R(e^x)\,dx$,
\item[] c)\ $\int R(x,\sqrt{1-x^2}\,)\,dx, \quad$ d)\ $\int R(x,\sqrt{x^2+1}\,)\,dx, \quad$
        e)\ $\int R(x,\sqrt{x^2-1}\,)\,dx$,
\item[] f)\ $\int R(x,\sqrt[m]{ax+b}\,)\,dx \quad (a\neq 0,\ m \in \N).$
\end{itemize}

\underline{Tapaus a)}\ \ Tässä toimii trigonometrinen 'gurusijoitus' 
(vrt.\ Luku \ref{trigonometriset funktiot})
\begin{align*}
\tan\frac{1}{2}x=t \qimpl &\begin{cases}
                        \,\sin x=\dfrac{2t}{1+t^2},\quad \cos x=\dfrac{1-t^2}{1+t^2}\,, \\[5mm] 
                        \,\dfrac{dt}{dx} = \dfrac{1}{2}\left(1+\tan^2\dfrac{1}{2}x\right) 
                                               \qimpl dx = \dfrac{2\, dt}{1+t^2}
                           \end{cases} \\[5mm]
\qimpl \int R(\sin x,\cos x)\,dx\, 
                          &=\, \int R\left(\frac{2t}{1+t^2},\frac{1-t^2}{1+t^2}\right)
                                                                   \frac{2}{1+t^2}\,dt \\
                          &=\, \int Q(t)\, dt,
\end{align*}
missä $Q$ on rationaalifunktio.

\underline{Tapaus b)}\ \ Sijoituksella (vrt.\ edellisen luvun Esimerkki 
\ref{rat-palautuva integraali 1})
\[
e^x = t \qimpl x=\ln t, \quad dx=\frac{1}{t}\,dt
\]
saadaan integraali muotoon 
\[
\int R(e^x)\,dx \,=\, \int R(t)\frac{1}{t}\,dt \,=\, \int Q(t)\,dt,
\]
missä $Q$ on jälleen rationaalifunktio. \\[5mm]
\underline{Tapaus c)}\ \ Tämä palautuu tapaukseen a) sijoituksilla
\[
x=\cos t \quad \text{tai} \quad x=\sin t.
\]

\underline{Tapaukset d) ja e)}\ \ Nämä palautuvat tapaukseen b) sijoituksilla
\[
\text{d)}\ x=\sinh t, \qquad \text{e)}\ x=\cosh t.
\]

\underline{Tapaus f)}\ \ Tässä toimii sijoitus 
(vrt.\ edellisen luvun Esimerkki \ref{rat-palautuva integraali 2}) 
\begin{align*}
&\sqrt[m]{ax+b}=t \qimpl x = \frac{1}{a}(t^m-b),\quad dx = \frac{m}{a}\,t^{m-1}\, dt \\[3mm]
&\impl \ \int R(x,\sqrt[m]{ax+b}\,)\, dx 
     = \int R\left(\frac{t^m-b}{a},\,t\right)\frac{m}{a}\,t^{m-1}\, dt=\int Q(t)\, dt.
\end{align*}
\begin{Exa}
$\D\int\frac{1}{\sin x}\, dx\ =\ ?$
\end{Exa}
\ratk Sijoituksella $\,\tan\frac{1}{2} x = t\,$ saadaan
\begin{align*}
\int\frac{1}{\sin x}\, dx 
  &= \int\frac{1+t^2}{2t}\cdot\frac{2}{1+t^2}\, dt =\int\frac{1}{t}\, dt \\
  &= \ln\abs{t}+C \\
  &= \ln\left|\tan\frac{1}{2}x\right|+C 
   = \underline{\underline{\ln\left|\frac{1-\cos x}{\sin x}\right|+C}}. \loppu
\end{align*}
\begin{Exa} $\D\int\frac{\sqrt{x^2+1}}{x}\,dx\ =\ ?$
\end{Exa}
\ratk Tehdään ensin sijoitus
\[
x=\sinh t,\quad dx=\cosh t\, dt,\quad \sqrt{x^2+1}=\cosh t
\]
\begin{align*}
\impl \ \int\frac{\sqrt{x^2+1}}{x}\,dx 
     &= \int\frac{\cosh^2 t}{\sinh t}\,dt = \int\frac{\sinh^2 t + 1}{\sinh t}\,dt \\
     &= \int \sinh t\,dt + \int\frac{1}{\sinh t}\,dt = \cosh t + \int \frac{2}{e^t-e^{-t}}\,dt.
\end{align*}
Tehdään uusi sijoitus 
\[
e^{t}=u,\quad t=\ln u,\quad \, dt=\frac{1}{u}\,du
\]
\begin{align*}
\impl \ \int \frac{2}{e^t-e^{-t}}\,dt 
               &= \int\frac{2}{u-u^{-1}}\cdot\frac{1}{u}\,du = \int\frac{2}{u^2-1}\,du \\
               &= \ln\left|\frac{u-1}{u+1}\right|+C = \ln\left|\frac{e^t-1}{e^t+1}\right|+C.
\end{align*}
Koska $\,x=\sinh t\ \ekv\ e^t = x+\sqrt{x^2+1}\,$, niin todetaan, että
\[
\int\frac{\sqrt{x^2+1}}{x}\,dx 
          = \sqrt{x^2+1}+\ln\left|\frac{x-1+\sqrt{x^2+1}}{x+1+\sqrt{x^2+1}}\right|+C.
\]
Huomioimalla vielä, että
(vrt.\ Harj.teht.\,\ref{jatkuvuuden käsite}:\ref{H-V-1: pelkistyvä funktio})
\[
\frac{x-1+\sqrt{x^2+1}}{x+1+\sqrt{x^2+1}} \,=\, \frac{x}{1+\sqrt{x^2+1}}\,, \quad x\in\R,
\]
saadaan lopputulokselle hiukan sievempi muoto:
\[
\int\frac{\sqrt{x^2+1}}{x}\,dx 
          = \underline{\underline{\sqrt{x^2+1}+\ln\frac{|x|}{1+\sqrt{x^2+1}}+C}}. \loppu
\]

\subsection*{Integroinnin sarjamenetelmä}
\index{sarjamenetelmä (integroinnin)|vahv}

Silloin kun funktio on esitettävissä suppenevan potenssisarjan (Taylorin sarjan, ks.\ Luku
\ref{taylorin lause}) summana, voidaan myös integraalifunktio esittää tässä muodossa.
Oletetaan, että
\[
f(x)=\sum_{k=0}^\infty a_k(x-x_0)^k,\quad \abs{x-x_0}<\rho \quad (\rho>0).
\]
Tällöin myös sarja
\[
F(x)=\sum_{k=0}^\infty \frac{a_k}{k+1}\,(x-x_0)^{k+1}
\]
suppenee, kun $\abs{x-x_0}<\rho$ (Lause \ref{potenssisarjan skaalaus}) ja sarja termeittäin 
derivoimalla (mikä Lauseen \ref{potenssisarja on derivoituva} mukaan on sallittua) todetaan,
että
$F$ on $f$:n integraalifunktio välillä $(x_0-\rho,x_0+\rho)$.
\begin{Exa}
$\D\int e^{x^2}\, dx\ =\ ?$
\end{Exa}
\ratk
\[
e^{x^2}=\sum_{k=0}^\infty \frac{1}{k!}\,x^{2k},\quad x \in (-\infty,\infty),
\]
joten
\begin{align*}
\int e^{x^2}\, dx 
      &= \sum_{k=0}^\infty \frac{1}{(2k+1)k!}\,x^{2k+1}+C \\
      &= C+x+\frac{1}{3}\,x^3+\frac{1}{10}\,x^5+\frac{1}{42}\,x^7+..\,,\quad x\in\R. \loppu
\end{align*}
\begin{Exa} Määritä sarjoja hyväksi käyttäen
\[
\text{a)}\ \int \frac{e^{-x}}{x}\,dx, \qquad
\text{b)}\ \int \frac{\sin x}{x^4}\,dx.
\]
\end{Exa} 
\ratk Lähtien Taylorin sarjoista
\[
e^{-x} = \sum_{k=0}^\infty \frac{(-1)^k}{k!}\,x^k, \quad\
\sin x = \sum_{k=0}^\infty \frac{(-1)^k}{(2k+1)!}\,x^{2k+1}
\]
saadaan
\begin{align*}
\int\frac{e^{-x}}{x}\,dx   
  &= \int \frac{1}{x}\left(1-x+\frac{1}{2!}\,x^2-\frac{1}{3!}\,x^3+\ldots\right)\,dx \\
  &= \int \left(x^{-1}-1+\frac{1}{2!}\,x-\frac{1}{3!}\,x^2+\ldots\right)\,dx \\
  &= \left(\ln\abs{x}-x+\frac{1}{2\cdot 2!}\,x^2-\frac{1}{3\cdot 3!}\,x^3+\ldots\right) + C \\
  &= \ln\abs{x} + C + \sum_{k=1}^\infty \frac{(-1)^k}{k\cdot k!}\,x^k,
\end{align*}
\begin{align*}
\int\frac{\sin x}{x^4}\,dx 
  &= \int \frac{1}{x^4}\left(x-\frac{1}{3!}\,x^3+\frac{1}{5!}\,x^5-\ldots\right)\,dx \\ 
  &= \int \left(x^{-3}-\frac{1}{3!}\,x^{-1}+\frac{1}{5!}\,x-\ldots\right)\,dx \\
  &= \left(-\frac{1}{2}\,x^{-2}-\frac{1}{6}\ln\abs{x}
                               +\frac{1}{2\cdot 5!}\,x^2-\ldots\right) +C \\
  &= -\frac{1}{2x^2}-\frac{1}{6}\ln\abs{x}+C
                    +\sum_{k=1}^\infty \frac{(-1)^{k+1}}{2k(2k+3)!}\,x^{2k}.
\end{align*}
Molemmat tulokset ovat päteviä väleillä $(-\infty,0)$ ja $(0,\infty)$. \loppu

\Harj
\begin{enumerate}

\item
Integroi seuraavat rationaalifunktiot:
\begin{align*}
&\text{a)}\ \ \frac{x^3+2x^2-x+5}{x+2} \qquad
 \text{b)}\ \ \frac{x^2}{x^2+x-2} \qquad
 \text{c)}\ \ \frac{1}{x^3-4x^2+3x} \\
&\text{d)}\ \ \frac{24}{x(x^2-1)(x^2-4)} \qquad
 \text{e)}\ \ \frac{1}{(1-x)^2(1+x)} \qquad
 \text{f)}\ \ \frac{3+x-2x^3}{1-x^3} \\
&\text{g)}\ \ \frac{x^8+1}{x^6+x^4} \qquad
 \text{h)}\ \ \frac{3x^2+1}{(x^2-1)^2} \qquad
 \text{i)}\ \ \frac{1}{x^4+3x^2+2} \qquad
 \text{j)}\ \ \frac{1}{x^4+1} \\
&\text{k)}\ \ \frac{1}{x(x^2+a^2)} \qquad
 \text{l)}\ \ \frac{1}{x^2(x^2-a^2)} \qquad
 \text{m)}\ \ \frac{x^3}{x^3-a^3} \qquad
 \text{n)}\ \ \frac{1}{x^4-a^4}
\end{align*}

\item
Muunna integraali 
\[
\int \frac{1}{x^4}{\sqrt{x^2+a^2}}\,dx, \quad a>0
\]
sijoituksilla \ a) $x=a\sinh t$, \ b) $x=a\tan t$, \ c) $x=a/t$. Valitse vaihtoehdoista
helpoin ja laske integraali suljetussa muodossa.

\item
Integroi sopivalla sijoituksella:
\begin{align*}
&\text{a)}\ \ \frac{e^{3x}}{e^x+2} \qquad
 \text{b)}\ \ \frac{1}{\cosh x} \qquad
 \text{c)}\ \ \frac{1}{1+\sqrt[3]{x}} \qquad
 \text{d)}\ \ \frac{1}{\sqrt{x+1}+\sqrt[4]{x+1}} \\
&\text{e)}\ \ x^2\sqrt{a^2-x^2} \qquad
 \text{f)}\ \ \frac{1}{x\sqrt{1-x^2}} \qquad
 \text{g)}\ \ \frac{1}{x(x^2+1)^{3/2}} \\
&\text{h)}\ \ \frac{1}{5+4\sin x} \qquad
 \text{i)}\ \ \frac{2-\sin x}{2+\cos x} \qquad
 \text{j)}\ \ \frac{1}{5-4\sin x+3\cos x}
\end{align*}

\item
Laske seuraavat integraalit sijoituksella $\,\tan x=t$\,:
\[
\text{a)}\ \ \int \frac{\tan x}{1+2\tan x}\,dx \qquad
\text{b)}\ \ \int \frac{1}{a\cos^2 x+b\sin^2 x}\,dx,\,\ (a,b) \neq (0,0)
\]

\item
Jos $R(x,y)$ on kahden muuttujan rationaalinen lauseke, niin millaisella sijoituksella
integraali
\[
\int R\left(x,\sqrt[m]{\frac{x+a}{x+b}}\right)dx, \quad m\in\N,\ m \ge 2,\ \ a,b\in\R
\]
muuntuu rationaaliseksi? Laske tällä tavoin $\D\int\frac{1}{x}\sqrt{\frac{x+1}{x-1}}\,dx$.


\item
Laske seuraavien funktioiden integraalifunktiot potenssisarjojen avulla:
\begin{align*}
&\text{a)}\ \ \frac{\sin x}{x} \qquad
 \text{b)}\ \ \frac{\cosh x}{x} \qquad\,\,
 \text{c)}\ \ \frac{1-\cos x}{x^3} \qquad
 \text{d)}\ \ \frac{\sin x}{x^6} \\
&\text{e)}\ \ x^3 e^{x^3} \qquad
 \text{f)}\ \ \sin x^2 \qquad\ \
 \text{g)}\ \ \frac{\sin x^2}{x^3} \qquad\quad\,\
 \text{h)}\ \ \frac{\cos x^3}{\sqrt{x}}
\end{align*}

\item (*)
Esitä alkeisfunktioina (jos mahdollista) tai potenssisarjoina funktiot $F$ ja $G$ siten,
että pätee
\begin{align*}
&\text{a)}\ \ \int e^x\ln x\,dx = F(x)\ln x + G(x)+C, \quad x\in(0,\infty), \\
&\text{b)}\ \ \int \frac{1}{\sqrt{x}}\,\ln x\,\sin x\,dx
                   = x\sqrt{x}\,[\,F(x)\ln x + G(x)\,]+C, \quad x\in(0,\infty), \\
&\text{c)}\ \ \int \frac{\cos x}{x^2+1}\,dx = F(x)\Arctan x+G(x), \quad x\in\R, \\
&\text{d)}\ \ \int \frac{\sin x}{x^2+1}\,dx = F(x)\ln\,(x^2+1)+G(x), \quad x\in\R.
\end{align*}

\item (*) \label{H-int-3: osamurtohajotelma} \index{rationaalifunktio!a@osamurtohajotelma}
Lauseen \ref{osamurtolause} todistamiseksi tarkastellaan kompleksimuuttujan
rationaalifunktiota $f(z)=p(z)/q(z)$, missä $q$ on reaalikertoiminen polynomi astetta
$n\in\N$ ja $p$ on reaalikertoiminen polynomi astetta $\le n-1$. \vspace{1mm}\newline
a) Olkoon $c\in\C$ polynomin $q$ yksinkertainen nollakohta ja olkoon 
$q(z)=(z-c)\,q_1(z)$ ja $g(z)=p(z)/q_1(z)$. Näytä, että
\[
f(z) \,=\, \frac{g(c)}{z-c}+\frac{r(z)}{q_1(z)}\,, \quad z\in\DF_f,
\]
missä $r$ on polynomi astetta $\le n-2$. Päättele, että jos $c=a\in\R$, niin $r$ on
reaalikertoiminen. \newline
b) Olkoon $c=a+ib$ ja $\overline{c}=a-ib$ polynomin $q$ yksinkertaiset kompleksijuuret
($a,b\in\R,\ b \neq 0$) ja kirjoitetaan $q(z)=(z-c)(z-\overline{c})\,q_1(z)$. Näytä, että on
olemassa $A\in\C$ ja $B,C\in\R$ siten, että pätee
\[
f(z) \ =\ \frac{A}{z-c}+\frac{\overline{A}}{z-\overline{c}} +\frac{r(z)}{q_1(z)}
     \ =\ \frac{Bz+C}{(z-a)^2+b^2}+\frac{r(z)}{p_1(z)}, \quad z\in\DF_f,
\]
missä $r$ on reaalikertoiminen polynomi astetta $\le n-3$. \newline
c) Olkoon $c\in\C$ $q$:n $m$-kertainen juuri ja olkoon $q(z)=(z-c)^mq_1(z)$ ja
$g(z)=p(z)/q_1(z)$. Näytä, että pätee
\[
f(z) \,=\, \sum_{k=1}^m \frac{A_k}{(z-c)^k}+\frac{r(z)}{p_1(z)}, \quad z\in\DF_f,
\]
missä $A_k=g^{(m-k)}(c)/(m-k)!$ ja $r$ on polynomi astetta $\le n-m-1$. Päättele, että jos
$c=a\in\R$, niin kertoimet $A_k$ ovat reaaliset ja $r$ on reaalikertoiminen. \newline
d) Olkoon $c=a+ib$ ja $\overline{c}=a-ib$ polynomin $q$ $m$-kertaiset kompleksijuuret
($a,b\in\R,\ b \neq 0$) ja kirjoitetaan $q(z)=(z-c)^m(z-\overline{c})^mq_1(z)$. Näytä, että
on olemassa kertoimet $A_k\in\C$ ja $B_k,C_k\in\R$ siten, että pätee
\begin{align*}
f(z) \,&=\, \sum_{k=1}^m \frac{A_k}{(z-c)^k}
           +\sum_{k=1}^m \frac{\overline{A}_k}{(z-\overline{c})^k}+\frac{r(z)}{q_1(z)} \\
       &=\, \sum_{k=1}^m \frac{B_kz+C_k}{[(z-a)^2+b^2]^k}+\frac{r(z)}{q_1(z)}, \quad z\in\DF_f,
\end{align*}
missä $r$ on reaalikertoiminen polynomi astetta $\le n-2m-1$. \newline
e) Perustuen kohtien a) ja b) tuloksiin ja Algebran peruslauseeseen, johda Lauseen
\ref{osamurtolause} väittämä siinä tapauksessa, että $q$:n kaikki juuret ovat
yksinkertaisia. \newline
f) Todista Lause \ref{osamurtolause} yleisessä tapauksessa perustuen kohtien c) ja d)
tuloksiin ja Algebran peruslauseeseen.


\end{enumerate}