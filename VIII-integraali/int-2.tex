\section{Osittaisintegrointi ja sijoitus} \label{osittaisintegrointi}
\alku
\index{osittaisintegrointi|vahv}

Integroimistekniikan, eli integraalifunktioiden etsimisen, kaksi keskeisintä yleistä metodia
ovat \kor{osittaisintegrointi} (engl. integration by parts, partial integration) ja 
\kor{sijoitus}(menettely) (engl.\ substitution). Näillä menetelmillä on 'matematiikan kaavoina'
yleisempääkin käyttöä.

Osittaisintegroinnin kaava on yksinkertaisesti tulon derivoimissääntö toisin kirjoitettuna:
\[
\frac{d}{dx}(fg)=f'g+fg' \qekv f'g = \frac{d}{dx}(fg)-fg'
\]
\[
\impl \quad \boxed{\quad \int f'(x)g(x)\, dx=f(x)g(x)-\int f(x)g'(x)\, dx. \quad}
\]
\begin{Exa}\ a) \ $\D\int xe^x\, dx=\,? \quad$ b) \ $\D\int e^x\sin x\, dx=\,?$
\end{Exa}
\ratk
\begin{align*}
\text{a)} \quad   \int xe^x\, dx\, &=\,\int\underbrace{e^x}_{f'(x)}\underbrace{x}_{g(x)}\, dx\,
                                    =\,e^x\cdot x-\int e^x\, dx
                                    =\,\underline{\underline{(x-1)e^x+C}}. \\
     \text{\underline{Tarkistus}:} &\quad \frac{d}{dx}((x-1)e^x)
                                              =e^x+(x-1)e^x=xe^x. \quad\text{OK!} \loppu
\intertext{$\qquad\ $b) \ Valitaan $f'(x)=f(x)=e^x$ ja integroidaan kahdesti osittain:}
F(x)        &= \int e^x\sin x\, dx=e^x\sin x-\int e^x\cos x\, dx \\
            &=e^x\sin x-e^x\cos x-\int e^x\sin x\, dx \\[2mm]
            &=e^x(\sin x -\cos x)-F(x) \\[2mm]
\impl \ F(x)&=\underline{\underline{\frac{1}{2}\,e^x(\sin x-\cos x)+C}}. \loppu
\end{align*}
\begin{Exa} \label{paha integraali} Määritä integraalifunktio 
$\displaystyle{\int \sqrt{a^2+x^2}\,dx}$, \ kun $a>0$. 
\end{Exa}
\ratk Osittain integroimalla \,($f'(x)=1,\ g(x)=\sqrt{a^2+x^2}$\,)\, saadaan ensin
\[
\int \sqrt{a^2+x^2}\,dx = x\sqrt{a^2+x^2} - \int \frac{x^2}{\sqrt{a^2+x^2}}\,dx.
\]
Tässä on
\begin{align*}
-\int\frac{x^2}{\sqrt{a^2+x^2}}\,dx\ 
               &=\ -\int\frac{(x^2+a^2)-a^2}{\sqrt{a^2+x^2}}\,dx \\
               &=\ -\int\sqrt{a^2+x^2}\,dx + \int\frac{a^2}{\sqrt{a^2+x^2}}\,dx,
\end{align*}
joten seuraa (ks.\ edellisen luvun kaavat (9) ja (I-2))
\begin{align*}
2\int \sqrt{a^2+x^2}\,dx\ &=\ x\sqrt{a^2+x^2} + \int \frac{a}{\sqrt{(x/a)^2+1}}\,dx \\
                          &=\ x\sqrt{a^2+x^2} + a^2\ln\bigl[(x/a)+\sqrt{(x/a)^2+1}\,\bigr]+C.
\end{align*}
Kun tässä kirjoitettaan vakion $C$ tilalle $a^2\ln a+C$ (mahdollista, koska $C$ on joka 
tapauksessa määräämätön), saadaan tulos muotoon
\[
\int \sqrt{a^2+x^2}\,dx\ 
  =\ \underline{\underline{\frac{1}{2}\bigl[x\sqrt{a^2+x^2} 
                + a^2\ln\bigl(x + \sqrt{a^2+x^2}\bigr)\bigr]+C}}. \loppu
\]

\subsection*{Reduktiokaavat}
\index{reduktiokaavat (integraalien)|vahv}

Osittain integroimalla voidaan johtaa \kor{reduktiokaavoja} (palautuskaavoja) koko joukolle
integraaleja, joissa on kokonaislukuparametri. Esimerkiksi seuraavat integraalit saadaan tällä
tavoin lasketuksi suljetussa muodossa:
\begin{align*}
&\int x^ne^x\,dx,\quad \int x^n\cos x\,dx,\quad x^n\sin x\,dx,\quad
 \int (1+x^2)^{-n}\, dx,\quad n\in\N, \\
&\int \cos^n x\, dx,\quad \int\sin^n x\, dx,\quad n\in\Z. \\
\end{align*}
Tarkastellaan esimerkkinä integraalia 
\[
I_n(x)=\int \cos^n x\,dx.
\]
Osittain integroimalla saadaan
\begin{align*}
I_n(x) &= \int \cos x\cdot \cos^{n-1} x\, dx \qquad[\,f'(x)=\cos x,\ g(x)=\cos^{n-1}x\,] \\
&=\sin x\cos^{n-1} x + (n-1)\int\sin^2 x\cos^{n-2} x\, dx \\
&=\sin x\cos^{n-1} x + (n-1)\int(1-\cos^2 x)\cos^{n-2}x\,dx \\[1mm]
&=\sin x\cos^{n-1} x + (n-1)I_{n-2}(x)-(n-1)I_n(x).
\end{align*}
Tämän perusteella integraalille $I_n(x)$ pätee reduktiokaava
\begin{equation} \label{reduktio 1}
nI_n(x)-(n-1)I_{n-2}(x)=\sin x\cos^{n-1} x.
\end{equation}
Kaava pätee itse asiassa kun $n\hookrightarrow\alpha$, $\alpha\in\R$, mutta siitä on hyötyä
lähinnä kun $n\in\Z$, jolloin $I_n$ voidaan palauttaa kaavan avulla tapauksiin $n=0,\pm 1$ ja
näin ollen integroida suljetussa muodossa (tapaukseen $n=-1$ soveltuu edellisen luvun 
taulukkokaava (14)).
\begin{Exa}
$\text{a)}\ \D\int\cos^3 x\, dx=\,? \quad\text{b)}\ \int\cos^{-3} x\, dx=\,?$
\end{Exa}
\ratk Indekseillä $n=3$ ja $n=-1$ reduktioaava \eqref{reduktio 1} antaa
\begin{align*}
I_3(x)    &= \frac{1}{3}\sin x\cos^2 x + \frac{2}{3}I_1(x), \\
I_{-3}(x) &= \frac{1}{2}\sin x\cos^{-2} x + \frac{1}{2}I_{-1}(x),
\end{align*}
joten (vrt.\ edellisen luvun Esimerkki \ref{cos-integraaleja})
\begin{align*}
\text{a)}\,\ \int \cos^3 x\, dx 
        &= \frac{1}{3}\sin x\cos^2 x + \frac{2}{3}\int\cos x\, dx \\
        &= \underline{\underline{-\frac{1}{3}\sin^3 x + \sin x+C}}
\intertext{ja edellisen luvun kaavan (14) perusteella}
\text{b)}\,\ \int \cos^{-3} x\, dx 
        &= \frac{1}{2}\sin x\cos^{-2} x + \frac{1}{2}\int\cos^{-1} x\,dx \\
        &=\underline{\underline{\frac{1}{2}\ln\left|
              \frac{\cos x}{1-\sin x}\right|+ \frac{1}{2}\,\frac{\sin x}{\cos^2 x}+C}}. \loppu
\end{align*}

Toisena esimerkkinä tarkasteltakoon integraalia
\[
I_n(x)=\int\frac{1}{(1+x^2)^n}\,dx,\quad n \ge 2.
\]
Pyritään palauttamaan tämä tapaukseen $n=1$, jossa integraalifunktio tunnetaan (edellisen
luvun taulukkokaava (11)). Valitaan $f'(x)=1,\ g(x)=(1+x^2)^{-n}$ ja integroidaan osittain:
\begin{align*}
I_n(x) &= \frac{x}{(1+x^2)^n}+2n\int\frac{x^2}{(1+x^2)^{n+1}}\, dx \quad [x^2=(x^2+1)-1] \\
&= \frac{x}{(1+x^2)^n}+2n I_n(x) - 2n I_{n+1}(x).
\end{align*}
Kirjoittamalla $n$:n tilalle $n-1$ päädytään reduktiokaavaan
\begin{equation} \label{reduktio 2}
I_n(x)=\frac{1}{2n-2}\,\frac{x}{(1+x^2)^{n-1}}+\frac{2n-3}{2n-2}\,I_{n-1}(x).
\end{equation}
\begin{Exa}
Kaavaa \eqref{reduktio 2} kahdesti soveltamalla saadaan
\begin{align*}
\int\frac{1}{(1+x^2)^3}\,dx 
&= \frac{1}{4}\frac{x}{(1+x^2)^2}+\frac{3}{4}\int\frac{1}{(1+x^2)^2}\,dx \\
&= \frac{1}{4}\,\frac{x}{(1+x^2)^2}+\frac{3}{4}
        \left(\frac{1}{2}\,\frac{x}{1+x^2}+\frac{1}{2}\int \frac{1}{1+x^2}\,dx\right) \\
&= \underline{\underline{\frac{1}{4}\,
         \frac{x}{(1+x^2)^2}+\frac{3}{8}\,\frac{x}{1+x^2}+\frac{3}{8}\Arctan x+C}}. \loppu
\end{align*}
\end{Exa}

\subsection*{Sijoitusmenettely}
\index{muuttujan vaihto (sijoitus)!b@integraalissa|vahv}

Sijoituksella tarkoitetaan integroinnissa \pain{muuttu}j\pain{an} \pain{vaihtoa}
(kuten raja-arvoja laskettaessa, vrt.\ Luku \ref{funktion raja-arvo}). Integroinnin 
sijoitusmenettely perustuu seuraavaan tulokseen, joka puolestaan perustuu yhdistetyn funktion
ja käänteisfunktion derivoimissääntöihin.
\begin{Prop} \label{sijoituspropositio}
Olkoon $u:(c,d)\Kohti (a,b)$ bijektio ja olkoon $u$ derivoituva välillä $(c,d)$ ja
käänteisfunktio $v=\inv{u}$ derivoituva välillä $(a,b)$. Tällöin, jos $G(t)$ on funktion
$g(t)=f(u(t))u'(t)$ integraalifunktio välillä $(c,d)$, ts.\
\[
\int f(u(t))u'(t)\, dt=G(t)+C,\quad t\in (c,d),
\]
niin funktion $f(x)$ integraalifunktio välillä $(a,b)$ on
\[
\int f(x)\, dx=G(v(x))+C,\quad x\in (a,b).
\]
\end{Prop}
\tod Jos $x\in (a,b)$ ja $t=v(x) \ \ekv \ x=u(t)$, niin yhdistetyn funktion derivoimissäännön
ja oletuksien perusteella
\begin{align*}
\frac{d}{dx}G(v(x)) &= G'(v(x))v'(x)=G'(t)v'(x) \\
&= f(u(t))u'(t)v'(x).
\end{align*}
Tässä on $u(t)=x$, joten $f(u(t))=f(x)$ ja käänteisfunktion derivoimissäännön
(Luku \ref{derivaatta}, kaava \eqref{D5}) mukaan
\[ 
u'(t)v'(x)=1 \qimpl \text{väite}. \loppu 
\]

Sovelluksissa sijoitus tehdään usein (ehkä useammin) muodossa $v(x)=t$. Mekaanisesti
sijoitusmenettely toimii siis seuraavalla tavalla:

\kor{Ongelma:} $\quad \D\int f(x)\,dx=\,?$
\begin{enumerate}
\item Tehdään sijoitus: $\quad x=u(t) \quad \text{tai} \quad v(x)=t$.
\item Ratkaistaan
      \[
      x=u(t)\ \impl\ t=v(x) \quad \text{tai} \quad v(x)=t \ \impl \ x=u(t).
      \]
\item Sijoitetaan integraaliin
      \[
      x=u(t)\,\ (\text{tai}\ v(x)=t) \quad \text{ja} \quad dx=u'(t)\,dt.
      \]
\item Kirjoitetaan $f(u(t))u'(t)=g(t)$ (mahdollinen sievennys) ja ratkaistaan muunnettu
      ongelma:
      \[
      \int g(t)\,dt = G(t)+C.
      \]
\item Sijoitetaan ratkaisuun $t=v(x)$, jolloin saadaan alkuperäisen ongelman ratkaisu:
      \[
      \int f(x)\, dx=G(v(x))+C.
      \]
\end{enumerate}
Vaiheessa 2 ei haittaa, vaikka käänteisfunktio $v=\inv{u}$ tai $u=\inv{v}$ olisi monihaarainen
(yhtälön $u(t)=x$ tai $v(x)=t$ ratkaisu monikäsitteinen), kunhan Proposition
\ref{sijoituspropositio} oletukset toteutuvat valitulla haaralla. On ainoastaan pidettävä
huolta, ettei laskun eri vaiheissa hypellä haaralta toiselle.
\begin{Exa} $\displaystyle{\int \frac{1}{\sqrt{x}+1}\, dx=\,?}$ 
\end{Exa}
\ratk Tässä on $(a,b) = (0,\infty)$.
\begin{enumerate}
\item Tehdään sijoitus: $\quad v(x) = \sqrt{x} = t \in (0,\infty)$.
\item Ratkaistaan: $\quad \sqrt{x}=t\ \impl\ x=u(t)=t^2$.
\item Sijoitetaan integraaliin: $\quad \sqrt{x}=t,\ dx=2t\,dt$.
\item Ratkaistaan muunnettu ongelma:
\[
\int \frac{2t}{t+1}\,dt \,=\, \int \left(2-\frac{2}{t+1}\right)dt \,=\, 2t - 2\ln(t+1) + C.
\]
\item Alkuperäisen ongelman ratkaisu sijoituksella $t=\sqrt{x}\,$:
\[
\int \frac{1}{\sqrt{x}+1}\,dx \,=\, \underline{\underline{2\sqrt{x} - 2\ln(\sqrt{x}+1) + C}}.
\]
\end{enumerate}
\begin{Exa} \label{rat-palautuva integraali 1}
$\D\int\frac{1}{e^x+1}\, dx=\,?$
\end{Exa}
\ratk Sijoittamalla $\ \D{e^x=t\in(0,\infty)\ \impl\ x=\ln t,\ dx=\frac{1}{t}\,dt}\,$ saadaan
\begin{align*}
\int\frac{1}{e^x+1}\,dx\,
&=\,\int\frac{1}{(t+1)t}\,dt\,
 =\,\ln\left|\frac{t}{t+1}\right|+C\quad 
               (\text{Esimerkki \ \ref{integraalifunktio}:\,\ref{E10.1.2}}) \\
&\impl \ \int\frac{1}{e^x+1}\,dx\,
 =\,\ln\left(\frac{e^x}{e^x+1}\right)+C\,
 =\,\underline{\underline{x-\ln(e^x+1)+C}}. \loppu
\end{align*}

\begin{Exa}
$\D\int\frac{1}{(1+x^2)^2}\,dx=\,?$
\end{Exa}
\ratk Tämän voi laskea reduktiokaavalla \eqref{reduktio 2}. Toinen vaihtoehto on
trigonometrinen sijoitus
\[
x=\tan t, \quad dx=\,\frac{1}{\cos^2 t}\,dt, \quad t\in (-\pi/2,\pi/2).
\]
Koska $\,1+\tan^2 t=1/\cos^2 t$, saadaan muunnetuksi integraaliksi
\begin{align*}
\int\frac{1}{(1+\tan^2 t)^2}\cdot\frac{1}{\cos^2 t}\,dt
       &= \int\cos^2 t\, dt = \frac{1}{2}(t+\sin t\cos t)+C \\
       &= \frac{1}{2}\left(t+\frac{\sin t}{\cos t}\cdot\cos^2 t\right)+C 
        = \frac{1}{2}\left(t+\frac{\tan t}{1+\tan^2 t}\right)+C.
\end{align*}
Ratkaisemalla lopuksi
\[
x=\tan t \ \ja \ t\in (-\pi/2,\pi/2) \qekv t=\Arctan x
\]
saadaan tulos
\[
\int\frac{1}{(1+x^2)^2}\, dx
        =\underline{\underline{\frac{1}{2}\left(\Arctan x + \frac{x}{1+x^2}\right) + C}}.
\]
\underline{Vertailu}: Reduktiokaava \eqref{reduktio 2} on menetelmänä suoraviivaisempi. \loppu
\begin{Exa} \label{rat-palautuva integraali 2} $\D\int\frac{1}{x+1+\sqrt[3]{x+1}}\, dx=\,?$ 
\end{Exa}
\ratk Sijoituksella
\begin{align*}
\sqrt[3]{x+1}=t \ \ekv \ x=t^3-1,\quad dx 
                     &= 3t^2\, dt\quad (x>-1,\ t>0)
\intertext{saadaan}
\int\frac{1}{x+1+\sqrt[3]{x+1}}\, dx = \int\frac{3t^2}{t^3+t}\,dt 
                     &=\int\frac{3t}{t^2+1}\, dt \\
                     &=\frac{3}{2}\ln (t^2+1)+C \\
                     &=\underline{\underline{\frac{3}{2}\ln[(x+1)^{2/3}+1]+C}}.
\end{align*}
Jos välillä $x\in(-\infty,-1)$ tulkitaan $\sqrt[3]{x+1}=-\sqrt[3]{\abs{x+1}}$, niin tulos on
pätevä myös tällä välillä. \loppu

\begin{Exa}
$\D\int \frac{e^x}{\sqrt{x}}\,dx\ =\ ?$ 
\end{Exa}
\ratk Sijoituksella $\ \D{\sqrt{x} = t,\ x = t^2,\ dx = 2t\,dt}\ $ tehtävä muuntuu muotoon
\[ 
\int 2e^{t^2}\,dt\ =\ ? 
\]
Tällä ei ole alkeisfunktioratkaisua (ainoastaan sarjaratkaisu, ks. seuraava luku), joten 
tyydytään tulokseen
\[ 
\int \frac{e^x}{\sqrt{x}}\,dx = 2G(\sqrt{x}), \quad G(t) = \int e^{t^2}\,dt. \loppu 
\]

\Harj
\begin{enumerate}

\item
Laske osittain integroimalla
\begin{align*}
&\text{a)}\ \ \int x\cos x\,dx \qquad
 \text{b)}\ \ \int x^3 e^{-x^2}\,dx \qquad
 \text{c)}\ \ \int x^3\cos(x^2)\,dx \\
&\text{d)}\ \ \int x^2 e^{ax}\,dx,\,\ a\in\R \qquad
 \text{e)}\ \ \int x^\alpha\ln x\,dx,\,\ \alpha\in\R \\
&\text{f)}\ \ \int \Arctan x\,dx \qquad
 \text{g)}\ \ \int \Arcsin x\,dx \qquad
 \text{h)}\ \ \int x\Arctan x\,dx
\end{align*}

\item
Johda osittain integroimalla reduktiokaava annetulle integraalille $I_n(x)$ ja laske kaavan
avulla $I_n(x)$ annetuilla $n$:n arvoilla. \vspace{1mm}\newline
a) \ $\int x^n e^x\,dx,\ n\in\N\cup\{0\},\ \ n=1,2,3$. \newline
b) \ $\int x^n \sin x\,dx,\ n\in\N\cup\{0\},\ \ n=1,2,3$. \newline
c) \ $\int x^n \cos x\,dx,\ n\in\N\cup\{0\},\ \ n=1,2,3$. \newline
d) \ $\int \sin^n x\,dx,\ n\in\Z,\ \ n=-3,4$. \newline
e) \ $\int (\ln x)^n\,dx,\ n\in\N\cup\{0\},\ \ n=1,2$. \newline
f) \ $\int (x^2+1)^{n-1/2}\,dx,\ n\in\Z,\ \ n=-1,2$. \newline
g) \ $\int (x^2-1)^{n-1/2}\,dx,\ n\in\Z,\ \ n=-1,2$. \newline
h) \ $\int (1-x^2)^{n-1/2}\,dx,\ n\in\Z,\ \ n=-1,2$.

\item
Laske annettua sijoitusta käyttäen:
\begin{align*}
&\text{a)}\ \ \int \frac{1}{\sqrt{5-x^2}}\,dx,\ \ x=\sqrt{5}\,t \qquad
 \text{b)}\ \ \int \frac{1}{x^2+2x+3}\,dx,\ \ x+1=\sqrt{2}\,t \\
&\text{c)}\ \ \int e^{\sqrt{x}}\,dx,\ \ x=t^2 \qquad\qquad\quad\,\
 \text{d)}\ \ \int e^{\sqrt[3]{x}}\,dx,\ \ \sqrt[3]{x}=t
\end{align*}

\item
Muunna seuraavat integraalit toiseen muotoon annetulla sijoituksella.
\begin{align*}
&\text{a)}\ \ \int \frac{1}{\sqrt{1+x^4}}\,dx,\ \ x^4=t \qquad
 \text{b)}\ \ \int e^{x^2}\,dx,\ \ x^2=t \\
&\text{c)}\ \ \int \sin(\ln x)\,dx,\ \ x=e^t \qquad\,
 \text{d)}\ \ \int \ln(\tan x)\,dx,\ \ \tan x=t
\end{align*}

\item
Funktioiden $x^\alpha\sin\beta x,\ x^\alpha\cos\beta x,\ x^\alpha e^{\beta x}$
($\alpha,\beta\in\R$) integraalifunktiot ovat alkeisfunktioita vain kun joko $\beta=0$ tai 
$\alpha\in\N\cup\{0\}$. Mitkä seuraavista ovat tämän tiedon perusteella alkeisfunktioita ja 
mitkä eivät?
\begin{align*}
&\text{a)}\ \ \int \cos(x^2)\,dx \qquad
 \text{b)}\ \ \int x^{13}\sqrt{x}\cos\sqrt{x}\,dx \qquad
 \text{c)}\ \ \int x^5 e^{x^4}\,dx \\
&\text{d)}\ \ \int x^7 e^{x^4}\,dx \qquad\,\ \
 \text{e)}\ \ \int \sqrt{\ln x}\,dx \qquad\
 \text{f)}\ \ \int \frac{1}{\ln x}\,dx \\
&\text{g)}\ \ \int \frac{1}{x\sqrt{\ln x}}\,dx \quad\,\ \ 
 \text{h)}\ \ \int \sin(e^x)\,dx \qquad
 \text{i)}\ \ \int \cos x\,\ln x\,dx
\end{align*}

\item (*)
Olkoon $n\in\N\cup\{0\}$ ja $a,b\in\R$. Johda reduktiokaavat integraaleille
\[
I_n(x)=\int x^n e^{ax}\cos bx, \quad J_n(x)=\int x^n e^{ax}\sin bx.
\]

\end{enumerate}