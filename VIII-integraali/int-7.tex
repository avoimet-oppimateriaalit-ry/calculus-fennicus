\section{Riemannin integraalin laajennukset} \label{integraalin laajennuksia}
\alku

Riemannin integraalin määritelmässä on perusoletuksena, että integroitava funktio $f$ on sekä 
\pain{määritelt}y että \pain{ra}j\pain{oitettu} koko integroimisvälillä, jonka on oltava 
(äärellinen) \pain{sul}j\pain{ettu} väli. Ilman näitä rajoituksia ei integraalia voi
yleisesti määritellä Riemannin summien (tai ylä- ja alasummien) raja-arvona, ts.\ rajoitukset
ovat välttämättömiä, jotta $f$ olisi Riemann-integroituva. Monien sovellusten kannalta
kuitenkin mainitut rajoitukset ovat turhan voimakkaita tai jopa keinotekoisia. Esimerkiksi
koska tiedetään, että integraalin arvo ei riipu $f$:n arvoista yksittäisessä (tai äärellisen
monessa) pisteessä, niin tuntuu turhalta  ylipäänsä vaatia, että $f$ on määritelty jokaisessa
integroimisvälin pisteessä. Tällaisten turhien rajoitusten poistamiseksi on tullut tavaksi
konstruoida Riemannin integraalin määritelmälle erilaisia laajennuksia, joita sanotaan
\index{epzyoi@epäoleellinen integraali}%
\kor{epäoleellisiksi} (engl.\ improper, kirjaimellisesti 'sopimaton' tai 'hyvien tapojen
vastainen') Riemannin
integraaleiksi. Laajennukset eivät johda kovin selkeään yleiseen integraalin määritelmään, vaan
niiden tarkoituksena on lähinnä 'paikata' alkuperäistä määritelmää niin, että tavallisimmat
sovelluksissa esiintyvät tapaukset tulevat katetuiksi. Jatkossa esitetään näistä laajennuksista
tavallisimmat.\footnote[2]{Riemannin integraalin ongelmat havaittiin jo 1800-luvun
jälkipuoliskolla, jolloin mittojen ja integraalien teoria kehittyi voimakkaasti. Lopullisen
ratkaisun ongelmaan toi ranskalaisen \hist{Henri Lebesgue}n (1875-1941) vuonna 1906 esittämä
määritelmä, joka on sittemmin tunnettu \kor{Lebesguen integraalina}. Lebesguen integraali
poistaa Riemannin integraalin kauneusvirheet, mutta sen määrittely vaatii syvällisempiä
mittateoreettisia pohdiskeluja. \index{Lebesgue, H.|av} \index{Lebesguen integraali|av}}

Jos $f$ on rajoitettu ja Riemann-integroituva välillä $[a,b]$, niin funktion
\[
F(x)=\int_a^x f(t)\, dt
\]
Lipschitz-jatkuvuuden (Lause \ref{Analyysin peruslause 2}) perusteella
\[
\int_{a+\eps}^b f(t)\, dt = F(b)-F(a+\eps) = \int_a^b f(t)\, dt + \ordoO{\eps},
\]
kun $0<\eps<b-a$, joten
\[
\int_a^b f(x)\, dx = \lim_{\eps \kohti 0^+} \int_{a+\eps}^b f(x)\, dx.
\]
Oikealla oleva raja-arvo ei riipu $f$:n arvoista pisteessä $a$ --- itse asiassa raja-arvo ei
riipu edes siitä, onko $f$ määritelty tässä pisteessä. Tämän perusteella on varsin luontevaa
määritellä integraali ko.\ raja-arvona (mikäli olemassa) silloinkin, kun $f$ ei ole pisteessä 
$a$ määritelty ja/tai $f$ ei ole rajoitettu ko. pistettä lähestyttäessä.
\begin{Exa} \label{epäoleellinen esim1} Jos $f(x)=1/\sqrt{x-a}$, niin $f$ ei ole välillä 
$[a,b]$ rajoitettu (eikä pisteessä $a$ määritelty), mutta em.\ raja-arvo on olemassa: 
\begin{align*}
\int_{a+\eps}^b \frac{1}{\sqrt{x-a}}\,dx\ 
                      =\ \sijoitus{x=a+\eps}{x=b} 2\sqrt{x-a}\
                     &=\ 2\sqrt{b-a}-2\sqrt{\eps} \\ 
                     &\kohti\ 2\sqrt{b-a}, \quad \text{kun}\,\ \eps \kohti 0^+. \loppu
\end{align*}
\end{Exa}
Kun em.\ raja-arvomenettelyä sovelletaan integroimisvälin kummassakin päätepisteessä, tullaan
seuraavaan määritelmään.
\begin{Def} \vahv{(Riemannin integraalin 1. laajennus)} \label{1. R-laajennus}
\index{Riemann-integroituvuus!a@laajennettu|emph} Olkoon $a<c<b$. Jos $f$ on rajoitettu ja
Riemann-integroituva väleillä $[a+\eps,c]$ ja $[c,b-\eps]$ aina kun $\eps>0$
($\eps<\min\{c-a,b-c\}$), niin määritellään
\[
\int_a^b f(x)\, dx = \lim_{\eps \kohti 0^+} \int_{a+\eps}^c f(x)\,dx
                   + \lim_{\eps \kohti 0^+} \int_c^{b-\eps} f(x)\,dx,
\]
sikäli kuin oikealla puolella olevat raja-arvot ovat olemassa. Sanotaan tällöin, että $f$ on
integroituva välillä $[a,b]$.
\end{Def}
Määritelmän voi esittää hieman lyhyemmin muodossa
\[
\int_a^b f(x)\,dx = \lim_{\eps_1,\eps_2 \kohti 0^+} \int_{a+\eps_1}^{b-\eps_2} f(x)\,dx,
\]
missä raja-arvomerkintä oikealla tarkoittaa, että $\eps_1 \kohti 0^+$ ja $\eps_2 \kohti 0^+$
toisistaan riippumatta. Jos oletetaan, että $f$:llä on välillä $(a,b)$ integraalifunktio $F$,
niin noudattaen lyhennettyä merkintätapaa on Määritelmän \ref{1. R-laajennus} mukaisesti
\[
\int_a^b f(x)\,dx = \lim_{\eps_1,\eps_2 \kohti 0^+}\sijoitus{a+\eps_1}{b-\eps_2} F(x)
                  = F(a^+)-F(b^-).
\]
Siis $f$ on tässä tapauksessa integroituva täsmälleen kun raja-arvot $F(a^+)$ ja $F(b^-)$ ovat
olemassa eli kun $F$ on jatkuva välillä $[a,b]$.
\jatko \begin{Exa} (jatko) Funktiolla $\,f(x)=1/\sqrt{x-a}\,$ on välillä $[a,b]$ jatkuva
integraalifunktio $F(x)=2\sqrt{x-a}$. Integraalin arvoa laskettaessa on käytännössä
turvallista (ja hyväksyttyä) ohittaa Määritelmän \ref{1. R-laajennus} raja-arvoprosessi eli
laskea suoraan
\[
\int_a^b \frac{1}{\sqrt{x-a}}\,dx \,=\, \sijoitus{x=a}{x=b} 2\sqrt{x-a} 
                                  \,=\, 2\sqrt{b-a}. \loppu
\]
\end{Exa}

Koska integraalin additiivisuutta on luonnollista pitää myös em.\ tavalla laajennettujen
(epäoleellisten) integraalien ominaisuutena, saadaan välittömästi seuraava laajennus.
\begin{Def}
\vahv{(Riemannin integraalin 2. laajennus)} \label{2. R-laajennus}
\index{Riemann-integroituvuus!a@laajennettu|emph} Jos $a=c_0<c_1<\ldots <c_n=b$ ja $f$ on
integroituva väleillä $[c_{k-1},c_k]$, $k=1\ldots n$, niin $f$ on integroituva välillä $[a,b]$
ja
\[
\int_a^b f(x)\, dx=\sum_{k=1}^m \int_{c_{k-1}}^{c_k} f(x)\, dx.
\]
\end{Def}
Tämän määritelmän mukaisesti on esimerkiksi välillä $[a,b]$ paloittain jatkuva funktio ko.\ 
välillä integroituva. Määritelmän mukaisesti integraalin arvo on tällöin riippumaton siitä, 
miten $f$ on määritelty (tai onko lainkaan määritelty) epäjatkuvuuspisteissä.
\begin{Exa} Jos $f(x)=k$, kun $k-1 < x < k$, $k=1 \ldots n\,$ ($f(k)$ joko määritelty tai ei,
kun $k=0 \ldots n$), niin Määritelmän \ref{2. R-laajennus} mukaisesti
\[
\int_0^n f(x)\, dx \,=\, \sum_{k=1}^n \int_{k-1}^k f(x)\, dx
                   \,=\, \sum_{k=1}^n k \,=\, \frac{1}{2}n(n+1). \loppu
\]
\end{Exa}
Lopuksi laajennetaan integraalin käsite koskemaan äärettömiä välejä $[a,\infty)$, \newline
$(-\infty,b]$ ja $(-\infty,\infty)\,$:
\begin{Def} \vahv{(Riemannin integraalin 3. laajennus)} \label{3. R-laajennus}
\index{Riemann-integroituvuus!a@laajennettu|emph} Jos $f$ on integroituva väleillä $[a,M]$
jokaisella $M>a$ ja on olemassa raja-arvo
\[
A=\lim_{M\kohti\infty} \int_a^M f(x)\,dx,
\]
niin sanotaan, että $f$ on integroituva välillä $[a,\infty)$, lukua $A$ sanotaan $f$:n
integraaliksi ko. välillä, ja merkitään
\[
A=\int_a^\infty f(x)\,dx.
\]
Vastaavasti määritellään
\begin{align*}
\int_{-\infty}^b f(x)\, dx 
                 &= \lim_{M\kohti\infty} \int_{-M}^b f(x)\, dx, \\
\int_{-\infty}^\infty f(x)\, dx 
                 &= \int_{-\infty}^c f(x)\, dx + \int_c^\infty f(x)\,dx, \quad c\in\R.
\end{align*}
\end{Def}
\begin{Exa}
Funktio $f(x)=1/(1+x^2)$ on integroituva välillä $(-\infty,\infty)$, sillä Määritelmän
\ref{3. R-laajennus} mukaisesti
\begin{align*}
\int_{-\infty}^\infty \frac{1}{1+x^2} 
        &\,=\, \int_{-\infty}^0\frac{1}{1+x^2}\,dx + \int_0^\infty \frac{1}{1+x^2}\,dx \\
        &\,=\, \lim_{M \kohti\infty} \int_{-M}^0\frac{1}{1+x^2}\,dx
              +\lim_{M\kohti\infty} \int_0^M \frac{1}{1+x^2}\,dx \\
        &\,=\, \lim_{M\kohti\infty}\sijoitus{-M}{0} \Arctan x
              +\lim_{M\kohti\infty}\sijoitus{0}{M} \Arctan x
         \,=\, \frac{\pi}{2}+\frac{\pi}{2}=\pi. \loppu
\end{align*}
%Symmetria huomioiden ja raja-arvoprosessi ohittaen laskusta tulee lyhyempi:
%\[
%\int_{-\infty}^\infty \frac{1}{1+x^2}\,dx 
%        \,=\, 2\int_0^\infty \frac{1}{1+x^2}\,dx
%        \,=\, 2\sijoitus{0}{\infty} \Arctan x
%        \,=\, 2\left(\frac{\pi}{2}-0\right) \,=\, \pi. \loppu
%\]
\end{Exa}
Jos $f$ on tarkasteltavalla (äärellisellä tai äärettömällä) välillä integroituva joko 
tavanomaisessa tai laajennetussa mielessä, niin sanotaan, että $f$:n integraali ko. välin yli
\index{suppeneminen!c@integraalin} \index{hajaantuminen!c@integraalin}%
\kor{suppenee}. Jos $f$ ei ole integroituva, niin sanotaan, että integraali
\kor{hajaantuu}. Kuten esimerkissä, integraalin suppeneminen selviää helposti
integraalifunktion avulla, sikäli kuin sellainen on integroimsvälillä löydettävissä. Nimittäin
Määritelmän \ref{3. R-laajennus} mukaan myös äärettömillä integroimisväleillä $[a,\infty)$,
$(-\infty,b]$ ja $(-\infty,\infty)$ pätee: jos $f$:n integraalifunktio välillä $(a,b)$ on $F$,
niin integraalin $\int_a^b f(x)\,dx$ arvo (sikäli kuin olemassa) on
\[
\int_a^b f(x)\,dx \,=\, \lim_{x \kohti b^+}F(x)-\lim_{x \kohti a^-}F(x),
\]
eli integraali suppenee täsmälleen kun molemmat raja-arvot oikealla puolella ovat olemassa.
Mikäli koko integroimisvälillä toimivaa integraalifunktiota ei löydy, voidaan väli aina jakaa
äärelliseen määrään osavälejä, käyttää integraalin additiivisuutta ja tutkia integraalin
suppeneminen kullakin osavälillä erikseen.
\begin{Exa}
Raja-arvoissa hieman 'oikaisten' voidaan laskea
\begin{align*}
\int_{-\infty}^\infty \frac{1}{1+x^2}\,dx 
           &= \sijoitus{-\infty}{\infty}\Arctan x 
            = \dfrac{\pi}{2}-\left(-\frac{\pi}{2}\right) = \pi \quad \text{(suppenee)}, \\
\int_0^1 \frac{1}{x}
           &= \sijoitus{0}{1} \ln x = \infty \quad \text{(hajaantuu)}.
%\int_0^\infty e^{-x} \, dx                &= \sijoitus{0}{\infty} (-e^{-x})=0-(-1)=1. \\
\intertext{Varomaton on sen sijaan lasku}
\int_{-\infty}^\infty \frac{1}{x^2}\,dx   &= \sijoitus{-\infty}{\infty} -\frac{1}{x} = 0,
\end{align*}
sillä huomaamatta jäi, että $F(x)=-1/x$ ei ole funktion $f(x)=1/x^2$ integraalifunktio koko
välillä $(-\infty,\infty)$. Jakamalla integraali kahteen osaan väleille $(-\infty,0]$ ja 
$[0,\infty)$ nähdään, että se hajaantuu. \loppu
\end{Exa} 

\subsection*{Epäoleellisten integraalien vertailu}

Silloin kun epäoleellisen integraalin suppenevuutta ei voida suoraan tutkia integraalifunktion
avulla, tai se on hankalaa, voidaan suppenevuus yleensä selvittää vertaamalla integraalia 
yksinkertaisempaan, laskettavissa olevaan integraaliin. Vertailussa lähdetään ennestään
tutusta vertailuperiaatteesta (Lause \ref{integraalien vertailuperiaate})
\[
f(x)\leq g(x)\quad\forall x\in [a,b] \ \impl \ \int_a^b f(x)\, dx\leq\int_a^b g(x)\,dx.
\]
Sovellettaessa tätä epäoleellisiin integraaleihin äärellisellä välillä $[a,b]$ riittää
olettaa, että $f(x) \le g(x)$ on voimassa avoimella välillä $(a,b)$
(Määritelmä \ref{1. R-laajennus}) tai avoimilla osaväleillä $\ (c_{k-1},c_k)\subset(a,b)$,
$\,k=1 \ldots n\,$ (Määritelmä \ref{2. R-laajennus}), sillä koska vertailuperiaate on pätevä
tavallisille Riemannin integraaleille, se pysyy voimassa myös määritelmien raja-arvoille.
Vertailtaessa integraaleja $\int_a^\infty f(x)\,dx$ ja $\int_a^\infty g(x)\,dx$
(Määritelmä \ref{3. R-laajennus}) riittää vastaavasti olettaa, että $f(x) \le g(x)$ väleillä
$\,(c_{k-1},c_k)$, $\,k=1,2,\ldots$ missä $\,a=c_0<c_1<\ldots$ ja $\,\lim_kc_k=\infty$.

Tarkastellaan esimerkkinä integraalia $\int_a^\infty f(x)\,dx$.
(Integraali $\int_{-\infty}^b f(x)\,dx$ palautuu tähän muuttujan vaihdolla $t=-x$.)
Funktioille, jotka
eivät vaihda merkkiään integroimisvälillä (lukuunottamatta mahdollisia erillisiä pisteitä,
jotka eivät vaikuta integraalin arvoon), voi integraalien suppenemistarkasteluissa käyttää
seuraavia vertailuperiaatteita, vrt.\ vastaavat periaatteet positiivitermisille sarjoille
(Lause \ref{sarjojen vertailu}). Yksinkertaisuuden vuoksi oletetetaan tässä funktiot
määritellyksi ja epäyhtälöt voimassa oleviksi koko välillä $[a,\infty)$. 
\begin{Lause} \label{integraalien majorointi ja minorointi}
\index{majoranttiperiaate|emph} \index{minoranttiperiaate|emph}% 
(\vahv{Majorantti- ja minoranttiperiaatteet integraaleille})\newline
Jos $f$ ja $g$ ovat määriteltyjä välillä $[a,\infty)$, integroituvia välillä $[a,M]$ jokaisella
$\,M>a\,$ ja $\,0 \le f(x) \le g(x)\ \forall x\in[a,\infty)$, niin pätee
\begin{align*}
&\int_a^\infty g(x)\,dx\ \text{suppenee} \,\ \qimpl \int_a^\infty f(x)\,dx\ \text{suppenee}, \\
&\int_a^\infty f(x)\,dx\ \text{hajaantuu} \qimpl \int_a^\infty g(x)\,dx\ \text{hajaantuu}.
\end{align*}
\end{Lause}
\tod Väittämät ovat loogisesti ekvivalentteja, joten riittää todistaa ensimmäinen.
Oletetaan siis, että $\int_a^\infty g(x)\,dx$ suppenee. Koska $f(x) \ge 0$ välillä
$[a,\infty)$, niin funktio $F(x)=\int_a^x f(t)\,dt$ on ko.\ välillä kasvava:
\[
F(x_2)-F(x_1) \,=\, \int_{x_1}^{x_2} f(t)\,dt \,\ge\, 0, \quad \text{kun}\ a \le x_1 < x_2.
\]
Samasta syystä $G(x)=\int_a^x g(t)\,dt$ on kasvava välillä $[a,\infty)$, ja koska
$f(x) \le g(x),\ x\in[a,\infty)\,$ ja integraali $\int_a^\infty g(t)\,dt\,$ suppenee, niin
voidaan päätellä:
\[
F(x)\,\le\,G(x)\,\le\,\lim_{x\kohti\infty}G(x)\,=\,\int_a^\infty g(t)\,dt, \quad x\in[a,\infty).
\]
Näin ollen $F(x)$ on välillä $[0,\infty)$ sekä kasvava että rajoitettu. Lauseen
\ref{monotonisen funktion raja-arvo} mukaan on tällöin olemassa raja-arvo
$\,\lim_{x\kohti\infty}F(x)=\int_a^\infty f(x)\,dx$. \loppu
\begin{Exa} Lauseen \ref{integraalien majorointi ja minorointi} perusteella integraali
$\,\int_1^\infty e^{-x}/x\,\,dx\,$ suppenee, sillä välillä $[1,\infty)\,$ on
$\,e^{-x}/x \le e^{-x}\,$ ja integraali $\,\int_0^\infty e^{-x}\,dx\,$ suppenee:
\[
\int_1^\infty e^{-x}\,dx\,=\,\sijoitus{1}{\infty} (-e^{-x})\,=\,1. \loppu
\]
\end{Exa}
Lauseen \ref{integraalien majorointi ja minorointi} majorantti- ja minoranttiperiaatteet
yleistyvät sopivin oletuksin koskemaan muitakin epäoleellisia integraaleja. Esimerkiksi
olettaen että $0 \le f(x) \le g(x)\ \forall x\in(a,b)$, Määritelmän \ref{1. R-laajennus}
mukaisille integraaleille pätee
\[
\int_a^b g(x)\,dx\ \text{suppenee} \qimpl \int_a^b f(x)\,dx\ \text{suppenee}.
\]
\begin{Exa} Integraalia $\,\int_0^1 e^{-x}/x\,\,dx$ voi tutkia muuttujan vaihdolla $t=1/x$,
jolloin Lause \ref{integraalien majorointi ja minorointi} soveltuu. Suorempi päättely kuitenkin
riittää: Integraali hajaantuu, koska $\,e^{-x}/x \ge e^{-1}x^{-1} > 0,\ x\in(0,1]\,$ ja
$\,\int_0^1 x^{-1}\,dx$ hajaantuu. \loppu
\end{Exa}

\subsection*{Cauchyn kriteeri integraaleille}

Seuraava yleinen suppenemiskriteeri toimii kaikille epäoleellisille integraaleille muotoa
$\,\int_a^\infty f(x)\,dx$, vrt.\ vastaava Cauchyn kriteeri sarjoille
(Lause \ref{Cauchyn sarjakriteeri}).
\begin{Lause} \label{Cauchyn integraalikriteeri} (\vahv{Cauchyn kriteeri integraaleille})
\index{Cauchyn!e@kriteeri integraaleille|emph}
Olkoon $f$ on integroituva välillä $[a,M]$ jokaisella $M>a$. Tällöin integraali
$\int_a^\infty f(x)\, dx$ suppenee täsmälleen kun jokaisella $\eps>0$ on olemassa $M>a$ siten,
että pätee
\[
\bigl|\int_{M_1}^{M_2} f(x)\,dx\,\bigr| \,<\, \eps, \quad \text{kun}\ M_1,M_2>M.
\]
\end{Lause}
\tod $\boxed{\impl}\quad$ Jos integraali suppenee ja $M_1,M_2>a$, niin
\begin{align*}
\int_{M_1}^{M_2} f(x)\, dx &= \int_a^{M_2} f(x)\, dx-\int_a^{M_1} f(x)\, dx \\
                            &\kohti \int_a^\infty f(x)\, dx-\int_a^\infty f(x)\, dx=0, \quad 
                                                      \text{kun}\,\ M_1,M_2\kohti\infty,
\end{align*}
joten Cauchyn kriteeri on täytetty.

$\boxed{\Leftarrow}\quad$ Oletetaan, että Cauchyn kriteeri on täytetty ja merkitään
\[
A_n=\int_a^n f(x)\, dx,\quad n\in\N, \ n>a.
\]
Oletuksen perusteella
\[
A_n-A_m=\int_m^n f(x) \, dx\kohti 0, \quad \text{kun}\,\ n,m\kohti\infty,
\]
joten $\{A_n\}$ on Cauchyn jono, ja siis (Lause \ref{Cauchyn kriteeri})
\[
A_n\kohti A\in\R, \quad \text{kun}\ n\kohti\infty.
\]
Käyttämällä tätä tulosta ja vetoamalla uudelleen oletukseen todetaan, että jos $\eps>0$, niin
valitsemalla $M\in\R$ ja $n\in\N$ riittävän suuriksi on
\begin{align*}
\bigl|\int_a^M f(x)\, dx-A\,\bigr|\ &\le\ \bigl|\int_a^M f(x)\,dx-A_n\,\bigr|+|A_n-A| \\
                                    &= \bigl|\int_n^M f(x)\,dx\,\bigr|+|A_n-A|<\eps.
\end{align*}
Näin ollen $f$ on integroituva välillä $[a,\infty)$:
\[
\lim_{M\kohti\infty}\int_a^M f(x)\,dx = A. \loppu
\]
\begin{Exa}
Suppeneeko vai hajaantuuko integraali $\D\,\int_0^\infty \frac{\cos x}{\sqrt{x}}\,dx$\,?
\end{Exa}
\ratk Jaetaan integraali kahteen osaan:
\[
\int_0^\infty \frac{\cos x}{\sqrt{x}}\,dx = \int_0^{\pi/2} \frac{\cos x}{\sqrt{x}}\,dx
                                          + \int_{\pi/2}^\infty \frac{\cos x}{\sqrt{x}}\,dx.
\]
Integroimalla osittain ensimmäisessä osassa todetaan
\[
\int_0^{\pi/2} \frac{\cos x}{\sqrt{x}}\,dx  
            = \sijoitus{0}{\pi/2} 2\sqrt{x}\cos x + \int_0^{\pi/2} 2\sqrt{x}\sin x\,dx
            = 2 \int_0^{\pi/2} \sqrt{x}\sin x\,dx.
\]
Funktio $f(x)=\sqrt{x}\sin x$ on jatkuva välillä $[0,\pi/2]$, joten tämä osa integraalista
suppenee. Integraalin loppuosassa integroidaan osittain toisin päin:
\[
\int_{\pi/2}^M \frac{1}{\sqrt{x}}\cos x \, dx 
                   = \sijoitus{\pi/2}{M}\frac{1}{\sqrt{x}}\sin x
                            +\frac{1}{2}\int_{\pi/2}^M x^{-3/2}\sin x\,dx.
\]
Tässä
\[
\sijoitus{\pi/2}{M}\frac{1}{\sqrt{x}}\sin x \kohti -\sqrt{\frac{2}{\pi}}, \quad
                                                        \text{kun}\,\ M\kohti\infty,
\]
joten suppenemiskysymys siirtyy koskemaan integraalia
\[
\int_{\pi/2}^\infty x^{-3/2}\sin x \, dx.
\]
Kolmioepäyhtälön (Luku \ref{määrätty integraali}, epäyhtälö \eqref{int-4: kolmioepäyhtälö})
avulla arvioiden todetaan, että jos $\,\pi/2<M_1<M_2$, niin
\begin{align*}
\left|\int_{M_1}^{M_2} x^{-3/2}\sin x\,dx\right| 
           &\le \int_{M_1}^{M_2} x^{-3/2}\abs{\sin x\,}\,dx \\
           &\le \int_{M_1}^{M_2} x^{-3/2} \, dx \\
           &= -2\sijoitus{M_1}{M_2} x^{-1/2}\kohti 0, \quad \text{kun}\,\ M_1,M_2\kohti\infty,
\end{align*}
joten Cauchyn kriteerin perusteella myös tämä osa integraalista suppenee. Siis vastaus on:
Suppenee! \loppu

\subsection*{Integraalit ja sarjat}
\index{sarjaoppi (klassinen)|vahv}

Edellä on jo nähty, että integraalien $\int_a^\infty f(x)\,dx$ ja sarjojen suppenemisteoriat
muistuttavat toisiaan. Laskennallisemmaltakin kannalta integraalien ja sarjojen yhteys on
merkittävä, sillä osoittautuu, että integraaalien avulla voidaan sekä yksinkertaistaa
sarjojen suppenemistarkasteluja että tehostaa sarjojen summien numeerista laskemista.

Tarkastellaan esimerkkinä sarjaa $\,\sum_{k=1}^\infty f(k)$, missä oletetaan, että $f$ on
määritelty välillä $\,[1,\infty)\,$ ja lisäksi jollakin $n\in\N$ pätee:
\begin{itemize}
\item[(i)]  $f$ on vähenevä välillä $\,[n,\infty)$ ja $\,\lim_{x\kohti\infty}f(x)=0$.
\end{itemize}
Oletuksesta (i) seuraa, että $f(x) \ge 0$ välillä $[n,\infty)$ ja jokaisella $k \ge n$ on
voimassa
\[
f(k+1)\leq f(x)\leq f(k), \quad \text{kun}\,\ x\in [k,k+1].
\]
Näin ollen
\[
\int_k^{k+1} f(k+1)\, dx\ \le\ \int_k^{k+1} f(x)\, dx\ \leq\ \int_{k}^{k+1} f(k)\, dx,
\]
eli
\[
f(k+1)\ \leq\ \int_k^{k+1} f(x)\, dx\ \leq\ f(k).
\]
\begin{figure}[H]
\setlength{\unitlength}{1cm}
\begin{center}
\begin{picture}(11,5)(-1,-1)
\put(-1,0){\vector(1,0){11}} \put(9.8,-0.5){$x$}
\put(0,-1){\vector(0,1){5}} \put(0.2,3.8){$y$}
\curve(4,4,6,2.5,8,1.8)
\dashline{0.1}(5,0)(5,3.13) \dashline{0.1}(6,0)(6,3.13) \dashline{0.1}(6,2.5)(5,2.5)
\dashline{0.1}(6,3.13)(5,3.13)
\put(4.95,3.08){$\scriptstyle{\bullet}$} \put(5.93,2.43){$\scriptstyle{\bullet}$}
\put(2,0){\line(0,-1){0.1}} \put(5,0){\line(0,-1){0.1}} \put(6,0){\line(0,-1){0.1}}
\put(1.9,-0.5){$n$} \put(4.9,-0.5){$k$} \put(5.9,-0.5){$k+1$}
\put(8.2,1.7){$y=f(x)$}
\end{picture}
\end{center}
\end{figure}
Summaamalla em.\ tulos yli indeksien $\,k=n\ldots N-1$ ($N>n$) ja käyttämällä integraalin
additiivisuutta saadaan
\[
\sum_{k=n+1}^N f(k) \,\le\, \int_n^N f(x)\,dx \ \le\ \sum_{k=n}^{N-1} f(k)
                                             \ =\, \sum_{k=n+1}^{N-1} f(k) + f(n).
\]
Koska tässä $\lim_nf(n)=0$ oletuksen (i) mukaan, niin Cauchyn sarja- ja integraalikriteerien
(Lauseet \ref{Cauchyn sarjakriteeri} ja \ref{Cauchyn integraalikriteeri}) perusteella
päätellään, että sarja $\sum_{k=1}^\infty f(k)$ suppenee täsmälleen kun integraali
$\int_n^\infty f(x)\,dx$ suppenee. Lisäksi rajalla $N\kohti\infty$ saadaan arviot
\begin{align*}
           &\sum_{k=n+1}^\infty f(k) 
              \,\le\, \int_n^\infty f(x)\,dx \,\le\, \sum_{k=n+1}^\infty f(k) + f(n) \\
\ekv \quad &\int_n^\infty f(x)\,dx - f(n) 
              \,\le\, \sum_{k=n+1}^\infty f(k) \,\le\, \int_n^\infty f(x)\,dx.
\end{align*}
Tästä seuraa sarjan summan laskemisen kannalta mielenkiintoinen tulos:
\begin{align*}
\sum_{k=1}^\infty f(k) &\,=\, \sum_{k=1}^{n} f(k) + \int_n^\infty f(x)\,dx\,-\eps_n \\
                      &\,=\, \sum_{k=1}^{n-1} f(k) + \int_n^\infty f(x)\,dx\,+\delta_n,
\end{align*}
missä $\,0 \le \eps_n \le f(n)\,$ ja $\,0 \le \delta_n \le f(n)\,$ ($\delta_n=f(n)-\eps_n$).
Tämä tarkoittaa, että kun sarjan 'häntä' (indeksistä $n$ tai $n+1$ alkaen) lasketaan 
integraalina, niin tällä tavoin saatavista approksimaatioista
\begin{align}
\sum_{k=1}^\infty f(k)\ 
         &\approx\ \sum_{k=1}^{n-1} f(k) + \int_n^\infty f(x)\,dx \label{appr a} \tag{a} \\
         &\approx\ \sum_{k=1}^{n} f(k) + \int_n^\infty f(x)\,dx \label{appr b} \tag{b}
\end{align}
\eqref{appr a} antaa sarjan summalle alalikiarvon ja \eqref{appr b} ylälikiarvon, ja kummankin
virhe on enintään $f(n)=$ sarjan $n$:s termi (= approksimaatioiden erotus). Tulos on siis
pätevä, mikäli em.\ oletus (i) on voimassa.
%Seuraavaa esimerkkiä voi verrata aiempiin suppenemistarkasteluihin Luvussa \ref{potenssisarja}
%(Lause \ref{harmoninen sarja}).
\begin{Exa} \label{sarja vs integraali}
Millä $\alpha$:n arvoilla ($\alpha\in\R$) sarja $\,\sum_{k=1}^\infty 1/k^\alpha\,$ suppenee?
\end{Exa}
\ratk Sarja epäilemättä hajaantuu, jos $\alpha \le 0$. Jos $\alpha>0$, niin oletus (i) on
voimassa, kun $n=1$. Integraali $\,\int_1^\infty x^{-\alpha}\,dx\,$ suppenee täsmälleen kun
$\alpha>1$, joten sarja suppenee täsmälleen samalla ehdolla. \loppu

Esimerkin kysymys ratkaistiin jo aiemmin (Lause \ref{harmoninen sarja}) mutta vaivalloisemmin.
Paitsi yksinkertaistaa suppenemistarkastelun, integraaliin vertaaminen helpottaa usein myös
sarjan summan numeerista laskemista.
\jatko \begin{Exa} (jatko) Esimerkin suppenevan (yliharmonisen) sarjan summalle pätee
approksimaation (b) perusteella
\[
\sum_{k=1}^\infty\frac{1}{k^\alpha}\ 
   \approx\ \sum_{k=1}^n\frac{1}{k^\alpha}+\frac{1}{\alpha-1}\,n^{1-\alpha}, \quad \alpha>1.
\]
Tämä on ylälikiarvo, jota voi verrata pelkkään sarjan katkaisuun eli alalikiarvoon
\[
\sum_{k=1}^\infty\frac{1}{k^\alpha}\ \approx\ \sum_{k=1}^n\frac{1}{k^\alpha}\,.
\]
Edellisen virhe on enintään $\,n^{-\alpha}$, jälkimmäisen 
$n^{1-\alpha}/(\alpha-1)+\mathcal{O}(n^{-\alpha})$. Esim.\ jos $\alpha=5/4$, on tarkkuusero
käytännössä dramaattinen, ks.\ Harj.teht.\,\ref{H-int-7: hidas sarja}. \loppu
\end{Exa}
Huomautettakoon, että esimerkissä vieläkin tehokkaampi algoritmivaihtoehto on
approksimaatioiden (a) ja (b) keskiarvo eli
\[
\sum_{k=1}^\infty \frac{1}{k^\alpha} \,\approx\,
\sum_{k=1}^n\frac{1}{k^\alpha}+\frac{1}{\alpha-1}\,n^{1-\alpha}-\frac{1}{2}\,n^{-\alpha}\,.
\]
Tämän virhe on suuruusluokkaa $\mathcal{O}(n^{-1-\alpha})$
(ks.\ Harj.teht.\,\ref{numeerinen integrointi}:\ref{H-int-9: hidas sarja}).

\pagebreak

\Harj
\begin{enumerate}

\item
Laske tai osoita hajaantuvaksi kohdissa a)--o). Kohdissa p)--ö) määritä $k$:n arvot ($k\in\R$),
joilla integraali suppenee.
\begin{align*}
&\text{a)}\ \ \int_0^6 \frac{2x}{x^2-4}\,dx \qquad
 \text{b)}\ \ \int_0^6 \frac{2x}{\abs{x^2-4}^{2/3}}\,dx \qquad
 \text{c)}\ \ \int_0^1 \frac{x^5}{(1-x^2)^{3/2}}\,dx \\
&\text{d)}\ \ \int_0^1 x^{-4/5}\ln x\,dx \qquad
 \text{e)}\ \ \int_0^{\pi/2} \tan x\,dx \qquad
 \text{f)}\ \ \int_0^\pi \frac{x}{\cos^2 x}\,dx \\
&\text{g)}\ \ \int_0^2 \frac{x^2}{\sqrt{2x-x^2}}\,dx \qquad
 \text{h)}\ \ \int_{-\infty}^\infty \frac{1}{x^2+7x+12}\,dx \qquad
 \text{i)}\ \ \int_0^\infty \frac{1}{x^3+1}\,dx \\
&\text{j)}\ \ \int_{-\infty}^\infty \frac{1}{x^2+7x+13}\,dx \qquad
 \text{k)}\ \ \int_0^\infty \frac{1}{x^3-1}\,dx \qquad
 \text{l)}\ \ \int_1^\infty \frac{1+\sqrt{x}}{x^2+x}\,dx \\
&\text{m)}\ \ \int_1^\infty \frac{1+\sqrt{x}}{x\sqrt{x}+2x}\,dx \qquad
 \text{n)}\ \ \int_0^\infty \frac{1}{\sqrt{e^x-1}}\,dx \qquad
 \text{o)}\ \ \int_{-\infty}^\infty e^{-\abs{x}}\cos x\,dx \\
&\text{p)}\ \ \int_0^1 \frac{1+x+x^2}{x^k}\,dx \qquad
 \text{q)}\ \ \int_1^\infty \frac{1+x+x^2}{x^k}\,dx \qquad
 \text{r)}\ \ \int_0^\infty \frac{1}{kx^2+1}\,dx \\
&\text{s)}\ \ \int_0^1 \frac{x^{k-1}+x^{-k}}{1+x}\,dx \qquad
 \text{t)}\ \ \int_0^{\pi/2} \sin^k x\,dx \qquad
 \text{u)}\ \ \int_0^\infty x^k e^{-x}\,dx \\
&\text{v)}\ \ \int_0^\infty x^k\sin x\,dx \qquad
 \text{x)}\ \ \int_0^\infty \frac{\abs{k}^x}{\sqrt{e^x+1}}\,dx \qquad
 \text{y)}\ \ \int_0^\infty x^k\ln x\,dx \\
&\text{z)}\ \ \int_0^1 \frac{e^x}{(1-x^2)^k}\,dx \qquad
 \text{å)}\ \ \int_0^\infty \abs{x^2-1}^k e^{-x}\,dx \qquad
 \text{ä)}\ \ \int_{-\infty}^\infty \frac{\ln\abs{x}}{\abs{x^3+1}^k}\,dx \\
&\text{ö)}\ \ \int_{-\infty}^\infty\left[\sqrt[4]{|\sin x\cos x|}\ e^{|x|}\,(\ln|x|)^8\right]^kdx
\end{align*}

\item
Johda palautuskaava seuraaville integraaleille, kun $n\in\N\cup\{0\}$.
\[
\text{a)}\ \ \int_0^\infty x^n e^{-x}\,dx \qquad
\text{b)}\ \ \int_0^\infty \frac{1}{(x^2+1)^{n+1}}\,dx \qquad
\text{c)}\ \ \int_0^1 (\ln x)^n\,dx
\]

\item
Todista suppeneminen sijoituksella $t=1/x$\,:
\[
\text{a)}\ \ \int_0^1 e^{-1/x}\,dx \qquad
\text{b)}\ \ \int_0^1 \cos\frac{1}{x}\,dx \qquad
\text{c)}\ \ \int_0^\infty \frac{1}{x}\sin\frac{1}{x}\,dx
\]

\item
Sievennä ja piirrä kuvaaja:
\[
f(x)=\int_0^\pi \frac{x\sin t}{\sqrt{1+2x\cos t+x^2}}\,dx.
\]

\item \label{H-int-7: Gamma} \index{Gammafunktio ($\Gamma$-funktio)}
\kor{Gammafunktio} ($\Gamma$-funktio) määritellään 
\[
\Gamma(x)=\int_0^{\infty} t^{x-1} e^{-t}\,dt.
\]
Näytä tosiksi seuraavat väittämät ja hahmottele näiden perusteella
$\Gamma$-\-funktion kulku. (Kohdassa e) on $\Gamma(1/2)=\sqrt{\pi} \approx 1.77$, ks.\
Propositio \ref{Gamma(1/2)}.) \vspace{1.5mm}\newline
a) \ $\Gamma$ on määritelty välillä $(0,\infty)$. \newline
b) \ $\Gamma(x+1) = x \Gamma(x)\ \forall x>0$. \newline
c) \ $\Gamma(n)=(n-1)!\,\ \forall n\in\N$. \newline
d) \ $\lim_{x \kohti 0^+} \Gamma(x)=\lim_{x\kohti\infty} \Gamma(x)=\infty$. \newline
e) \ $\Gamma(1/2)=2\,\Gamma(3/2)=\int_{-\infty}^\infty e^{-x^2}\,dx$.
\item
a) Halutaan selvittää, millä $\alpha$:n arvoilla ($\alpha\in\R$) sarja
\[
\sum_{k=2}^\infty \frac{1}{k(\ln k)^\alpha}
\]
suppenee. Ratkaise kysymys integraaliin vertaamalla. \vspace{1mm}\newline
b) Olkoon $n\in\N$ suuri luku. Perustele integraalin avulla arviot
\[
\sum_{k=1}^n k^\alpha \,=\,
    \begin{cases}
    \,\frac{1}{\alpha+1}\,n^{\alpha+1}+\mathcal{O}(n^\alpha), &\text{kun}\ \alpha>-1, \\
    \,\ln n+\mathcal{O}(n^{-1}),                               &\text{kun}\ \alpha=-1.
                        \end{cases}
\]
Mikä on jäännöstermi $\mathcal{O}(n^\alpha)$ tarkasti tapauksissa $\alpha=0$ ja $\alpha=1$\,?

\item \label{H-int-7: hidas sarja}
Käytettävissä on tietokone, joka laskee summan $s_n=\sum_{k=1}^n k^{-5/4}$ ajassa $10^{-10}n$
sekuntia. Summien $s_n$ avulla halutaan laskea raja-arvo $s=\lim_ns_n$ kuuden desimaalin
tarkkuudella (virhe < $5 \cdot 10^{-7}$). Arvioi laskenta-aika \newline
a) vuosimiljardeina approksimaatioilla $\,s \approx s_n$, \newline
b) mikrosekunteina approksimaatioilla $\,s \approx s_n+4/\sqrt[4]{n}$.

\item (*) \index{Eulerin!b@vakio}
\kor{Eulerin vakio} $\gamma=0.5772156649..\,$ määritellään
\[
\gamma\,=\,\lim_{n\kohti\infty} \left(\sum_{k=1}^n \frac{1}{k}\,-\,\ln n\right)\,
        =\,\sum_{k=1}^\infty a_k, \quad a_k=\frac{1}{k}-\ln\left(1+\frac{1}{k}\right).
\]
a) Perustele jälkimmäinen laskukaava sekä ko.\ sarjan suppeneminen. \newline
b) Näytä, että suurilla $n\in\N$ pätee
   $\,\sum_{k=1}^na_k + \frac{1}{2n} = \gamma + \mathcal{O}(n^{-2})$.
        
\item (*)
Arvioi luku $n!$ sekä ylhäältä että alhaalta vertaamalla lukua 
$\ln(n!) = \ln 1 + \ln 2 + \dots + \ln n$ integraaliin. Kuinka moninumeroinen luku on $1000!$
(kymmenjärjestelmässä)\,?

\end{enumerate}