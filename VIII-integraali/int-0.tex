
\chapter{Integraali}
\label{Integraali}

Yhden reaalimuutujan analyysissä on \kor{integraalin} käsitteellä kaksi olomuotoa:
\kor{määräämätön} ja \kor{määrätty} integraali. Edellisellä tarkoitetaan
annetun funktion \kor{integraalifunktiota}, eli kyse on derivoinnin käänteisoperaatiosta,
johon on jo alustavasti tutustuttu Luvussa \ref{väliarvolause 2}. Tässä luvussa, tarkemmin
luvuissa \ref{integraalifunktio}--\ref{osamurtokehitelmät}, esitellään aiempaa
systemaattisemmin integraalifunktioiden etsimisessä käytetyt  menetelmät ja funktiotyypit,
jotka näillä menetelmillä ovat hallittavissa. Kyseessä on hyvin perinteinen matematiikan
taitolaji, jossa keskeistä roolia näyttelevät erilaiset funktioalgebran keinot, kuten
\kor{osittaisintegrointi}, muuttujan vaihto eli \kor{sijoitus} ja \kor{osamurtokehitelmät}.

Luvuissa \ref{määrätty integraali}--\ref{analyysin peruslause} tarkastellaan määrätyn
integraalin, tarkemmin \kor{Riemann\-integraalin}, eri määrittelytapoja ja ominaisuuksia
sekä liittymistä määräämättömään integraaliin \kor{Analyysin peruslauseen} kautta.
Luvussa \ref{integraalin laajennuksia} tarkastellaan Riemannin integraalin käsitteen
laajennuksia, \kor{epäoleellisia} integraaleja, ja näihin liittyen integraalien ja sarjojen
vertailua. Kahdessa viimeisessä osaluvussa tarkastelun kohteena ovat määrättyyn integraaliin
perustuvat pinta-alan ja kaarenpituuden yksinkertaisimmat laskukaavat sovelluksineen
(Luku \ref{pinta-ala ja kaarenpituus}) ja lopuksi \kor{numeerisen integroinnin} menetelmät
(Luku \ref{numeerinen integrointi}). 