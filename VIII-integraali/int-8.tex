\section{Pinta-ala ja kaarenpituus} \label{pinta-ala ja kaarenpituus}
\alku
\index{pinta-ala!c@tasoalueen|vahv}

Tässä luvussa tarkastellaan määrätyn integraalin käyttöä kahdessa tasogeometrisessa tehtävässä:
käyrän viivan rajoittaman \kor{tasoalueen} ($\Rkaksi$:n tai $E^2$:n osajoukon) pinta-alan 
määrittämisessä ja yksinkertaisen käyränkaaren kaarenpituuden laskemisessa.

Olkoon $f(x) \ge 0$ välillä $[a,b]$. Tarkastellaan käyrän $y=f(x)$ ja suorien $y=0$, $x=a$ ja
$x=b$ rajaamaa $\Rkaksi$:n osajoukkoa
\[
A = \{\,(x,y) \in \Rkaksi \mid x \in [a,b]\ \ja\ 0 \le y \le f(x)\,\}.
\]
\begin{figure}[H]
\setlength{\unitlength}{1cm}
\begin{center}
\begin{picture}(10,6)(-1,-1)
\put(-1,0){\vector(1,0){10}} \put(8.8,-0.4){$x$}
\put(0,-1){\vector(0,1){6}} \put(0.2,4.8){$y$}
\put(1.5,0){\line(0,1){3.765}} \put(7,0){\line(0,1){3.3179}}
\put(1.4,-0.5){$a$} \put(6.9,-0.5){$b$}
\put(3,1.5){$A$} \put(8,3.7){$y=f(x)$}
\curve(
    1.0000,    3.0000,
    1.5000,    3.7635,
    2.0000,    4.0000,
    2.5000,    3.8848,
    3.0000,    3.5679,
    3.5000,    3.1733,
    4.0000,    2.8000,
    4.5000,    2.5211,
    5.0000,    2.3843,
    5.5000,    2.4117,
    6.0000,    2.6000,
    6.5000,    2.9202,
    7.0000,    3.3179,
    7.5000,    3.7130,
    8.0000,    4.0000)
\end{picture}
\end{center}
\end{figure}
Halutaan määrittää $A$:n pinta-ala. Tämä on ei-negatiivinen reaaliluku, jota merkitään
jatkossa symbolilla $\mu(A)$. --- Kyseessä on itse asiassa funktio tyyppiä
$\mu:\ \mathcal{M} \kohti [0,\infty)$, missä $\mathcal{M}$ koostuu $\Rkaksi$:n osajoukoista,
tarkemmin sanoen sellaisista 
\index{mitta, mitallisuus}%
nk.\ \kor{mitallisista} joukoista, joiden pinta-ala on
määriteltävissä. Lukua $\mu(A)$ sanotaan tämän mukaisesti myös $A$:n (pinta-ala)\kor{mitaksi}
(engl.\ measure)\footnote[2]{Käsitteet 'mitta' ja 'mitallisuus' viittaavat matematiikan lajiin
nimeltä \kor{mittateoria}. Yleisemmän mittateorian perusideoita ei tässä yhteydessä
tarkastella, vaan asiaan palataan myöhemmin usean muuttujan analyysin yhteydessä.}. Ajatellen
toistaiseksi vain oletettua muotoa olevia tasoalueita asetetaan pinta-alamitalle $\mu$ seuraavat
kaksi aksioomaa:
\index{pinta-alamitta!a@tason|(}%
\begin{itemize}
\item[A1.] \kor{Vertailuperiaate}: Jos $0 \le m \le f(x) \le M\ \forall x\in[a,b]$,
           niin \index{vertailuperiaate!b@mittojen}%
           \[ 
           m(b-a) \le \mu(A) \le M(b-a). 
           \]
\item[A2.] \kor{Additiivisuus}: Jos $A_1=\{\,(x,y) \in A \mid x\in[a,c]\,\}$ ja
           $A_2=\{\,(x,y) \in A \mid x\in[b,c]\,\}$, missä $a<c<b$, niin $A$ on mitallinen
           täsmälleen kun $A_1$ ja $A_2$ ovat molemmat mitallisia, ja pätee
           \index{additiivisuus!b@mitan}%
           \[ 
            \mu(A) = \mu(A_1)+\mu(A_2). 
           \]
\end{itemize} 
\begin{figure}[H]
\setlength{\unitlength}{1cm}
\begin{center}
\begin{picture}(10,6)(-1,-1)
\put(-1,0){\vector(1,0){10}} \put(8.8,-0.4){$x$}
\put(0,-1){\vector(0,1){6}} \put(0.2,4.8){$y$}
\put(1.5,0){\line(0,1){3.765}} \put(7,0){\line(0,1){3.3179}} \put(4,0){\line(0,1){2.8}}
\put(1.4,-0.5){$a$} \put(6.9,-0.5){$b$} \put(3.9,-0.5){$c$}
\put(2.6,1.3){$A_1$} \put(5.3,1.3){$A_2$} \put(8,3.7){$y=f(x)$}
\curve(
    1.0000,    3.0000,
    1.5000,    3.7635,
    2.0000,    4.0000,
    2.5000,    3.8848,
    3.0000,    3.5679,
    3.5000,    3.1733,
    4.0000,    2.8000,
    4.5000,    2.5211,
    5.0000,    2.3843,
    5.5000,    2.4117,
    6.0000,    2.6000,
    6.5000,    2.9202,
    7.0000,    3.3179,
    7.5000,    3.7130,
    8.0000,    4.0000)
\end{picture}
\end{center}
\end{figure}
Oletetaan nyt $A$ sellaiseksi, että aksioomien A1--A2 mukainen pinta-alamitta $\mu(A)$ on 
määriteltävissä, ja tutkitaan, mitä tästä seuraa. Ensinnäkin todetaan additiivisuusaksioomaa
A2 toistuvasti soveltamalla, että jos $\X=\{x_k,\ k=0 \ldots n\}$ on välin $[a,b]$ jako 
(ts.\ $a=x_0<x_1< \ldots <x_n=b$), ja merkitään
\[
A_k = \{\,(x,y) \in A \mid x_{k-1} \le x \le x_k\,\}, \quad k = 1 \ldots n,
\]
niin jokainen $A_k$ on mitallinen ja
\[
\mu(A) = \sum_{k=1}^n \mu(A_k).
\]
Vertailuperiaatteen A1 mukaan pätee tässä
\begin{align*}
m_k \le f(x) &\le M_k\ \forall x\in[x_{k-1},x_k] \\[2mm]
             &\qimpl m_k(x_k-x_{k-1}) \le \mu(A_k) \le M_k(x_k-x_{k-1}) \\
             &\qimpl \mu(A_k) = \eta_k (x_k-x_{k-1}), \quad \eta_k\in[m_k,M_k].
\end{align*}
Jos $f$ on jatkuva välillä$[a,b]$, niin Weierstrassin lauseen 
(Lause \ref{Weierstrassin peruslause}) mukaan voidaan tässä valita
\[
m_k = \min_{x\in[x_{k-1},x_k]}\{f(x)\}, \quad M_k = \max_{x\in[x_{k-1},x_k]}\{f(x)\},
\]
jolloin Ensimmäisen väliarvolauseen (Lause \ref{ensimmäinen väliarvolause}) mukaan on olemassa 
$\xi_k\in[x_{k-1},x_k]$ siten, että $f(\xi_k)=\eta_k$.

Yhdistämällä em.\ päätelmät todetaan, että jos $f$ on jatkuva välillä $[a,b]$, niin jokaiseen
välin $[a,b]$ jakoon $\X$ liittyen $\mu(A)$ on ilmaistavissa eräänä ko.\ jakoon liittyvänä 
Riemannin summana:
\[
\mu(A) = \sum_{k=1}^n f(\xi_k)(x_k-x_{k-1}), \quad \xi_k\in[x_{k-1},x_k], \quad k = 1 \ldots n.
\]
Koska Analyysin peruslauseen ja Riemannin integraalin määritelmän mukaan tämän kaavan oikealla
puolella olevat summat lähestyvät määrättyä integraalia $\int_a^b f(x)\,dx$ jaon tihetessä,
riippumatta pisteiden $\xi_k$ valinnasta, niin seuraa, että $\mu(A)$ on määriteltävä kaavalla
\begin{equation} \label{pinta-alakaava}
\boxed{\quad\kehys \mu(A) = \int_a^b f(x)\,dx. \quad}
\end{equation}
Näin määritellylle mitalle aksioomat A1--A2 myös toteutuvat integraalin vastaavien
ominaisuuksien (Lauseet \ref{integraalien vertailuperiaate} ja \ref{integraalin additiivisuus})
perusteella. On siis päätelty, että jos $f$ on jatkuva välillä $[a,b]$, niin $A$ on aksioomien
A1--A2 mukaisesti mitallinen jos (ja vain jos!) mitta määritellään kaavalla
\eqref{pinta-alakaava}. \index{pinta-alamitta!a@tason|)}
\begin{Exa} \label{Arkhimedeen kaava} \index{paraabelin segmentti} \kor{Paraabelin segmentin}
\[
A = \{\,(x,y) \in \Rkaksi \mid x\in[0,2]\ \ja\ 0 \le y \le 2x-x^2\,\}
\]
pinta-ala on kaavan \eqref{pinta-alakaava} mukaisesti\footnote[2]{Paraabelin segmentin
pinta-alan laski ensimmäisenä antiikin huomattavin matemaatikko \hist{Arkhimedes}
(287-212 eKr). Arkhimedeen menetelmä perustui segmenttiä approksimoiviin monikulmioihin, ks.\
Harj.teht.\,\ref{numeerinen integrointi}:\ref{H-int-9: Arkhimedes}. \index{Arkhimedes|av}}
\[
\mu(A) = \int_0^2 (2x-x^2)\,dx = \sijoitus{0}{2}\left(x^2-\frac{1}{3}\,x^3\right) 
                               = \underline{\underline{\frac{4}{3}}}\,. \loppu
\]
\end{Exa}

Laskukaava \eqref{pinta-alakaava} seuraa myös pelkästään olettamalla, että $f$ on välillä
$[a,b]$ rajoitettu ja Riemann-integroituva. Nimittäin jos $\X$ on välin $[a,b]$ jako, niin 
em.\ päätelmien mukaisesti voidaan $\mu(A)$ (sikäli kuin olemassa) arvioida Riemannin
ylä- ja alasummilla (ks.\ Luku \ref{riemannin integraali}):
\[
\underline{\sigma}(f,\X) \le \mu(A) \le \overline{\sigma}(f,\X).
\]
Tällöin seuraa Lauseesta \ref{Riemann-integroituvuus}, että mitta on määriteltävä
kaavalla \eqref{pinta-alakaava}, jolloin aksioomat A1--A2 myös toteutuvat. On päädytty
seuraavaan tulokseen. 
\begin{Lause} Jos $f$ on määritelty, rajoitettu ja Riemann-integroituva välillä $[a,b]$ ja
$f(x) \ge 0,\ x\in[a,b]$, niin tasoalueen
\[
A = \{\,(x,y) \in \Rkaksi \mid x \in [a,b]\ \ja\ 0 \le y \le f(x)\,\}
\]
pinta-ala $\mu(A)$ määräytyy aksioomista A1--A2 yksikäsitteisesti kaavalla
\eqref{pinta-alakaava}.
\end{Lause}
\begin{Exa} Laske ympyräneljänneksen
\[
A = \{\,(x,y) \in \Rkaksi \mid x,y \ge 0\ \ja\ x^2+y^2 \le R^2\,\}
\]
pinta-ala. \end{Exa}
\ratk Kaavan \eqref{pinta-alakaava} mukaan laskien saadaan
\begin{align*}
\mu(A) &= \int_0^R \sqrt{R^2-x^2}\,dx \\
       &\qquad [\ \text{sijoitus}\,\ x=R\sin t,\,\ dx=R\cos t\,dt\ ] \\
       &= \int_0^{\pi/2} R^2\cos^2 t\, dt \\
       &= \sijoitus{0}{\pi/2} \frac{1}{2}\,R^2(t+\sin t \cos t) 
        = \underline{\underline{\frac{1}{4}\,\pi R^2}}. \loppu
\end{align*}
\begin{Exa} Määritä $\mu(A)$, kun
\[
A = \{\,(x,y) \in \Rkaksi \mid x\in[0,2]\ \ja\ 0 \le y \le \max\{x,2e^{-x}\}\,\}.
\]
\end{Exa}
\begin{multicols}{2} \raggedcolumns
\ratk Tässä on ensin ratkaistava
\[
x = 2e^{-x}\ \impl\ x=c \approx 0.852606,
\]
jolloin on (vrt.\ kuva)
\[
f(x) = \begin{cases} 
       \,2e^{-x}, &\text{kun}\ x\in[0,c], \\ x, &\text{kun}\ x\in[c,2].
\end{cases}
\]
\begin{figure}[H]
\setlength{\unitlength}{1.5cm}
\begin{center}
\begin{picture}(10,3)(-0.5,0)
\put(-0.5,0){\vector(1,0){4}} \put(3.3,-0.35){$x$}
\put(0,-0.5){\vector(0,1){3}}
\put(0.8526,0.8526){\line(1,1){1.1474}}
\put(0.8526,0){\line(0,1){0.8526}} \put(2,0){\line(0,1){2}}
\curve(
0.0000, 2.0000,
0.2000, 1.6375,
0.4000, 1.3406,
0.6000, 1.0976,
0.8526, 0.8526)
\put(-0.3,1.9){$2$}
\put(1.9,-0.4){$2$} \put(0.7526,-0.4){$c$}
\end{picture}
\end{center}
\end{figure}
\end{multicols}
Kaavan \eqref{pinta-alakaava} ja integraalin additiivisuuden perusteella saadaan
\begin{align*}
\mu(A) = \int_0^2 f(x)\,dx 
             &= \int_0^c 2e^{-x}\,dx + \int_c^2 x\,dx \\
             &= \sijoitus{0}{c}(-2e^{-x}) + \sijoitus{c}{2}\tfrac{1}{2}x^2 \\
             &= 4-2e^{-c}-\tfrac{1}{2}c^2 \approx \underline{\underline{2.78393}}. \loppu
\end{align*}

Sopimalla pinta-alamitalle lisää ominaisuuksia voidaan integraalien avulla laskea myös 
yleisempien tasoalueiden pinta-aloja. Tarkasteltakoon tässä ainoastaan esimerkkinä kahden
käyrän väliin jäävän alueen pinta-alaa. Olkoon
$0 \le g(x) \le f(x)\ \forall x\in[a,b]$. Halutaan märätä $\mu(A)$, kun
\[
A = \{(x,y)\in\Rkaksi \mid x\in[a,b]\ \ja\ g(x) \le y \le f(x)\}.
\]
\begin{figure}[H]
\setlength{\unitlength}{1cm}
\begin{center}
\begin{picture}(10,6)(-1,-1)
\put(-1,0){\vector(1,0){10}} \put(8.8,-0.4){$x$}
\put(0,-1){\vector(0,1){6}} \put(0.2,4.8){$y$}
\put(1.5,0.9375){\line(0,1){2.8275}} \put(5.5,1.5375){\line(0,1){0.8742}}
\put(1.5,0){\line(0,1){0.1}} \put(5.5,0){\line(0,1){0.1}}
\put(1.4,-0.5){$a$} \put(5.4,-0.5){$b$}
\put(3,2){$A$} \put(6,1.6){$y=g(x)$} \put(3,3.8){$y=f(x)$}
\curve(
    1.0000,    3.0000,
    1.5000,    3.7635,
    2.0000,    4.0000,
    2.5000,    3.8848,
    3.0000,    3.5679,
    3.5000,    3.1733,
    4.0000,    2.8000,
    4.5000,    2.5211,
    5.0000,    2.3843,
    5.5000,    2.4117,
    6.0000,    2.6000,
    6.5000,    2.9202)
\curve(
    1.0000,    1.2000,
    1.2500,    1.0534,
    1.5000,    0.9375,
    1.7500,    0.8344,
    2.0000,    0.7500,
    2.2500,    0.6844,
    2.5000,    0.6375,
    2.7500,    0.6034,
    3.0000,    0.6000,
    3.2500,    0.6034,
    3.5000,    0.6375,
    3.7500,    0.6844,
    4.0000,    0.7500,
    4.2500,    0.8344,
    4.5000,    0.9375,
    4.7500,    1.0534,
    5.0000,    1.2000,
    5.5000,    1.5375,
    6.0000,    1.9500,
    6.5000,    2.4375)
\end{picture}
\end{center}
\end{figure}
Kun merkitään
\begin{align*}
A_1 &= \{(x,y)\in\Rkaksi \mid x\in[a,b]\ \ja\ 0 \le y \le g(x)\}, \\
A_2 &= \{(x,y)\in\Rkaksi \mid x\in[a,b]\ \ja\ 0 \le y \le f(x)\},
\end{align*}
niin olettamalla additiivisuuslaski $\mu(A_2)=\mu(A_1)+\mu(A)$ (pinta-alamitan uusi aksiooma!)
seuraa laskukaava
\[
\mu(A)=\mu(A_2)-\mu(A_1)=\int_a^b [f(x)-g(x)]\,dx.
\]
Tätä voidaan pitää pätevänä aina kun $f$ ja $g$ ovat integroituvia välillä $[a,b]$ ja
$f(x) \ge g(x)\ \forall x\in[a,b]$. Ellei jälkimmäinen oletus ole voimassa, sovelletaan kaavaa 
asettamalla $f$:n tilalle $\max\{f,g\}$ ja $g$:n tilalle $\min\{f,g\}$, jolloin yleiseksi
laskukaavaksi kahden käyrän väliin jäävän alueen pinta-alalle välillä $[a,b]$ tulee
\begin{equation} \label{pinta-alakaava 2}
\boxed{\kehys\quad \mu(A)=\int_a^b \abs{f(x)-g(x)}\,dx. \quad}
\end{equation}
\begin{Exa} Laske käyrien $y=x$ ja $y=2e^{-x}$ väliin jäävän alueen pinta-ala välillä $[0,2]$.
\end{Exa}
\ratk Kaavan \eqref{pinta-alakaava 2} ja edellisen esimerkin perusteella
\begin{align*}
\mu(A) = \int_0^2 \abs{2e^{-x}-x}\,dx &= \int_0^c (2e^{-x}-x)\,dx+\int_c^2(x-2e^{-x})\,dx \\
                                      &= \sijoitus{0}{c}\left(-2e^{-x}-\frac{1}{2}\,x^2\right)
                                       + \sijoitus{c}{2}\left(\frac{1}{2}\,x^2+2e^{-x}\right) \\
                                      &=4+2e^{-2}-4e^{-c}-c^2 
                                       \approx \underline{\underline{1.83852}}. \loppu
\end{align*}                     

\subsection*{Kaarenpituus}
\index{kaarenpituus|vahv}

Luvussa \ref{kaarenpituus} esitetyn määritelmän mukaisesti voidaan kaaren
\[
S = \{\,(x,y) \in \Rkaksi \mid x\in[a,b]\ \ja\ y=f(x)\,\}
\]
\index{kaarenpituusmitta}%
\kor{kaarenpituusmitta} $\mu(S)$ määritellä välin $[a,b]$ jakoihin $\X$ liittyvien summien
\[
\sigma(f,\X) = \sum_{k=1}^n \sqrt{(x_k-x_{k-1})^2+[f(x_k)-f(x_{k-1})]^2}
\]
pienimpänä ylärajana
\[
\mu(S) = \sup_\X \sigma(f,\X),
\]
sikäli kuin summien joukko on rajoitettu. Jos oletetaan, että $f$ on jatkuva välillä $[a,b]$ ja
derivoituva välillä $(a,b)$, niin Differentiaalilaskun väliarvolauseen mukaan on olemassa 
$\,\xi_k\in(x_{k-1},x_k),\ k=1 \ldots n\,$ siten, että
\[
f(x_k)-f(x_{k-1}) = f'(\xi_k)(x_k-x_{k-1}), \quad k=1 \ldots n,
\]
jolloin
\[
\sigma(f,\X) = \sum_{k=1}^n \sqrt{1+[f'(\xi_k)]^2}\,(x_k-x_{k-1}).
\]
Mainituin oletuksin summa $\sigma(f,\X)$ on siis eräs välin $[a,b]$ jakoon $\X$  ja funktioon
$g(x)=\sqrt{1+[f'(x)]^2}\,$ liittyvä Riemannin summa. Päätellään, että sikäli kuin $g$ on
Riemann-integroituva välillä $[a,b]$ (Analyysin peruslauseen mukaan riittää, että $f$ on 
jatkuvasti derivoituva välillä $[a,b]$), on kaarenpituusmitta määriteltävissä ja pätee 
integraalikaava
\begin{equation} \label{kaarenpituuskaava}
\boxed{\kehys\quad \mu(S) = \int_a^b \sqrt{1+[f'(x)]^2}\,dx. \quad}
\end{equation}
\begin{Exa} \label{ympyrän kaari: pituus} Laske kaarenpituus puoliympyrän kaarelle
\[
S = \{\,(x,y) \in \Rkaksi \mid -R \le x \le R\ \ja\ y=\sqrt{R^2-x^2}\,\}.
\]
\end{Exa}
\ratk Tässä on $\,f(x)=\sqrt{R^2-x^2}$, joten kaavan \eqref{kaarenpituuskaava} mukaan on
\[
\mu(S) \,=\, \int_{-R}^R \sqrt{1+\frac{x^2}{R^2-x^2}}\,dx 
       \,=\, \int_{-R}^R \frac{R}{\sqrt{R^2-x^2}}\,dx.
\]
Tämä on suppeneva (epäoleellinen) integraali. Sijoituksella $x=R\cos t,\ t\in[0,\pi]$ saadaan
\[
\mu(S) = \int_\pi^0 (-R)\,dt = \int_0^\pi R\,dt = \underline{\underline{\pi R}}. \loppu
\]
\begin{Exa} Laske käyrän $y=x^2$ kaarenpituus välillä $[0,1]$. \end{Exa}
\ratk Kaavan \eqref{kaarenpituuskaava} ja Luvun \ref{osittaisintegrointi} Esimerkin
\ref{paha integraali} perusteella (sijoitus $2x=t$)
\begin{align*}
\mu(S) = \int_0^1 \sqrt{1+4x^2}\,dx 
      &= \sijoitus{t=0}{t=2}
        \frac{1}{4}\left[t\sqrt{1+t^2}+\ln\left(t+\sqrt{1+t^2}\right)\right] \\
      &= \frac{1}{2}\sqrt{5}+\frac{1}{4}\ln(2+\sqrt{5}) \approx \underline{\underline{1.47894}}. 
                                                                                       \loppu
\end{align*}
\begin{Exa} Käyrän $y=\sin x$ kaarenpituus välillä $[0,\pi]$ on
\[
\mu(S)=\int_0^\pi \sqrt{1+\cos^2 x}\,dx.
\]
Tämä on nk.\ 
\index{elliptinen integraali}%
\kor{elliptinen integraali}, jota ei voi laskea suljetussa muodossa. \loppu
\end{Exa} 
Kuten esimerkeistä nähdään, kaarenpituusintegraali voi olla melko yksinkertaisissakin
tapauksissa hankala tai peräti mahdoton suljetussa muodossa integroimisen kannalta.
Vaihtoehtona on tällöin integraalin laskeminen suoraan numeerisin menetelmin
(ks.\ seuraava luku).

\Harj
\begin{enumerate}

\item
Laske seuraavien tasoalueiden pinta-alat (tarkasti tai likiarvona).
\begin{align*}
&\text{a)}\ \ 0 \le y \le \abs{\cos 3x},\,\ x\in[0,\pi] \qquad
 \text{b)}\ \ 0 \le \min\{x,\cos x\},\,\ x\in[0,\pi/2] \\[1mm]
&\text{c)}\ \ \min\{\cos x,\sin 2x\} \le y \le \max\{\cos x,\sin 2x\},\,\ x\in[0,\pi] \\
&\text{d)}\ \ \abs{x} \le y \le \sqrt{x^2+1}\,,\,\ \abs{x} \le \sqrt{a^2-1}\,,\,\ a>1
\end{align*}

\item
Laske seuraavien käyrien väliin jäävän alueen pinta-ala annetulla välillä 
(tarkasti tai likiarvona). \vspace{1mm}\newline
a) \ $y=\sinh x$, $y=\cosh x$, $[-\ln 2,\ln 2] \qquad$
b) \ $y=\ln x$, $y=e^{x}$, $[1/2,2]$ \newline
c) \ $y=\cos x$, $y=x$, $[0,\pi] \hspace{3.1cm}$
d) \ $y=\sin x$, $y=x^2$, $[0,1]$

\item
Laske likiarvo käyrien $y=e^x$ ja $y=3-x^2$ väliin jäävän rajoitetun taso\-alueen pinta-alalle.

\item \label{H-int-8: ellipsin pinta-ala}
Näytä, että ellipsin
\[
S: \quad \frac{x^2}{a^2}+\frac{y^2}{b^2}=1 \quad (a,b>0)
\]
sisään jäävän alueen pinta-ala on $\mu(A)=\pi ab$.

\item
Laske seuraavien käyrien kaarenpituudet annetulla välillä. \vspace{1mm}\newline
a) \ $y=2\sqrt{x},\,\ [0,4] \qquad\qquad\,$
b) \ $y=\ln x,\,\ [1,e]$ \newline
c) \ $y=\ln\cos x,\,\ [0,\pi/4] \qquad$
d) \ $y=(1-x/3)\sqrt{x}\,,\,\ [0,3]$

\item
Näytä, että ellipsin puolikaaren $\,S:\,(x/a)^2+(y/b)^2=1\ \ja\ y \ge 0\,$ pituus saadaan
(elliptisenä) integraalina
\[
\mu(S)=\int_{-\pi/2}^{\pi/2} \sqrt{a^2\cos^2 t+b^2\sin^2 t}\,dt.
\]

\item (*)
Laske käyrien $y=e^{-|x|}\cos x$ ja $y=e^{-|x|}\sin x$ väliin jäävän tasoalueen pinta-ala.

\item (*)
Laske seuraavien alueiden pinta-alat ($a>0$): \vspace{1mm}\newline
a) \ Sykloidin $\,x=a(t-\sin t),\ y=a(1-\cos t)$ ja $x$-akselin väliin jäävä alue välillä
$x\in[0,2\pi a]$. \newline
b)\, Asteroidin $\,\abs{x}^{2/3}+\abs{y}^{2/3}=a^{2/3}$ sisään jäävä alue.

\item (*)
Laske seuraavien kaarien kaarenpituudet ($a>0$): \vspace{1mm}\newline
a) \ Sykloidin kaari $\,S:\ x=a(t-\sin t),\ y=a(1-\cos t),\ t\in[0,2\pi]$. \newline
b)\, Asteroidin kaari $\,S:\ \abs{x}^{2/3}+\abs{y}^{2/3}=a^{2/3},\ x,y \ge 0$.

\item (*) \label{H-int-8: ketjuviiva} \index{ketjuviiva}
Painovoiman vaikuttaessa suunnassa $-\vec j\,$ noudattaa $xy$-tasolla vapaasti riippuva ketju
\kor{ketjuviivaa}
\[
y=a\cosh\left(\frac{x-b}{a}\right)+c,
\]
missä $a>0$ ja $b,c\in\R$. \vspace{1mm}\newline
a) Laske vakiot $a,b,c$ ja hahmottele kejuviivan kulku, kun ketjun päät ovat pisteissä $(-2,0)$
ja $(2,0)$ ja ketjun pituus $=8$. \vspace{1mm}\newline
b) Jos ketjun päät ovat pisteissä $(-2,4)$ ja $(4,4)$ ja ketju kulkee origon kautta, niin mikä
on ketjun pituus?

\end{enumerate}
