\section{Kunta} \label{kunta}
\alku

Rationaaliluvut laskuoperaatioineen ovat esimerkki (toistaiseksi ainoa) algebrallisesta
rakennelmasta nimeltä \kor{kunta} (engl.\ field, ruots.\ kropp, saks.\ Körper). Jos \K\ on jokin
lukujoukko (voisi olla yleisempikin joukko), niin sanotaan, että $(\K,+,\cdot )$ on kunta, jos
ensinnäkin
\index{laskuoperaatiot!ab@kunnan}% 
\begin{itemize}
\item Laskuoperaatiot $\,x,y \map x+y$ ja $\,x,y \map x \cdot y$ on määritelty yksikäsitteisesti
$\forall x,y \in \K$ ja pätee $x+y \in \K$ ja $x \cdot y \in \K$,
\end{itemize}
ja lisäksi seuraavat \vahv{kunta-aksioomat} ovat voimassa:
\index{kunta-aksioomat}%
\begin{itemize}
\item Vaihdanta-, liitäntä- ja osittelulait
   \begin{itemize}
   \item[(K1)] $\quad x+y = y+x$
   \item[(K2)] $\quad x \cdot y = y \cdot x$
   \item[(K3)] $\quad x+(y+z) = (x+y)+z$
   \item[(K4)] $\quad x \cdot (y \cdot z) = (x \cdot y) \cdot z$
   \item[(K5)] $\quad x \cdot (y+z) = x \cdot y + x \cdot z$
   \end{itemize}
\item Nolla--alkion ja vasta--alkion olemassaolo
   \begin{itemize}
   \index{nolla-alkio (kunnan)}%
   \item[(K6)] $\quad \text{On olemassa \kor{nolla--alkio}}\ 0 \in \K,\ 
                          \text{jolle pätee}\ \ x+0 = x\ \ \forall\ x \in \K.$
   \index{vasta-alkio (kunnan)}%   
   \item[(K7)] $\quad \text{Jokaisella}\ x \in \K\ \text{on \kor{vasta--alkio}}\ -x \in \K,\ 
                          \text{jolle pätee}\ \ x+(-x) = 0.$
   \end{itemize}
\item Ykkösalkion ja käänteisalkion olemassaolo
   \begin{itemize}
   \index{ykkösalkio (kunnan)}%
   \item[(K8)] $\quad \text{On olemassa \kor{ykkösalkio}}\ 1 \in \K,\ \ 
                          \text{jolle pätee}\ \ x \cdot 1 = x \ \ \forall x \in \K$.
   \index{kzyzy@käänteisalkio (kunnan)}%
   \item[(K9)] $\quad \text{Jokaisella}\ x \in \K,\ x \neq 0\ 
                          \text{on \kor{käänteisalkio}}\ x^{-1} \in \K,\
                          \text{jolle pätee}$\newline
               \phantom{ai} $x \cdot x^{-1} = 1.$
   \end{itemize}
\item Erotteluaksiooma
   \begin{itemize}
   \item[(K10)] $\quad 0 \neq 1$
   \end{itemize}
\end{itemize}
Aksioomista erillinen alkuoletus katsotaan yleensä $\K$:n laskuoperaatioiden määritelmään
sisältyväksi. Erillisenä tämä oletus on huomioitava lähinnä silloin, kun laskuoperaatiot on
alunperin määritelty jossakin suuremmassa joukossa $\A\supset\K$, tai kun tarkasteltavaa joukkoa
$\K$ halutaan laajentaa. Tällöin on varmistettava, että laskuoperaatioiden tulos pysyy joukossa
$\K$, jota tarkastellaan, ks.\ esimerkit jäljempänä.
Peruslaskuoperaatioiden $(+,\cdot)$ lisäksi kunnassa voidaan aina määritellä
yhdistetyt laskuoperaatiot $x,y \map x-y = x+(-y)$ (vähennyslasku) ja
$x,y \map x/y = x \cdot y^{-1}$ (jakolasku, $y \neq 0$), jolloin on $-x=0-x$ ja $x^{-1}=1/x$.
Koska aksiooma (K10) tarkoittaa 'ei päde $0=1$', niin kunnassa tarvitaan yleisesti vain
samastusrelaatio (\K:n alkioiden erotteluperiaate), ei
järjestysrelaatiota.
\index{samastus '$=$'!c@kunnan|av}%
\footnote[2]{Samastusrelaatiolta '$=$' edellytetään aina aksioomat \
(S1) $x=x\ \forall x$, \ (S2) $x=y \ \impl\ y=x$ ja (S3) $x=y \,\ja\, y=z \ \impl\ x=z$. Kunnan
samastusrelaatiolta vaaditaan lisäksi yhteensopivuus laskuoperaatioiden kanssa siten, että
laskuoperaation tulos on aina yksikäsitteinen. Vaihdantalakien (K1)--(K2) ja aksiooman (S3)
perusteella tämä vaatimus toteutuu olettamalla:\newline
(SK1) $x=y \ \impl\ x+z=y+z\ \forall z\,$ ja
(SK2) $x=y \ \impl\ x \cdot z = y \cdot z\ \forall z$.}
\begin{Exa} \label{yksinkertaisin kunta}
Yksinkertaisin mahdollinen kunta saadaan, kun valitaan $\K = \{0,1\}$ (missä $0\neq 1$) ja 
sovitaan laskusäännöstä $1+1 = 0$ (!). Tällöin $-1 = 1$ ja muut laskusäännöt ovat pääteltävissä
aksioomista (K1),\,(K2), (K5),\,(K6) ja (K8):
\begin{align*}
0+0 &= 0, \quad 0+1 = 1+0 = 1, \quad 1\,\cdot\,1 = 1, \quad 0\,\cdot\,1 = 1\,\cdot\,0 = 0, \\
0\,\cdot\,0 &= 0\,\cdot\,(1+1) = 0\,\cdot\,1 + 0\,\cdot\,1 = 0.
\end{align*}
Näillä säännöillä aksioomat (K1)-(K9) ovat kaikki voimassa (samoin oletus (K0)), joten kyseessä
on kunta. 
\loppu
\end{Exa}
\begin{Lause} \label{kuntatuloksia} Jokaisessa kunnassa $(\K,+,\cdot)$ pätee
\begin{itemize}
\item[(a)] $\quad \text{Nolla--alkio, vasta--alkio, ykkösalkio ja käänteisalkio ovat
yksikäsitteisiä.}$
\item[(b)] $\quad -(-x) = x \quad \forall x \in \K, \quad\quad 
                  (x^{-1})^{-1} = x \quad \forall x \in \K,\ x \neq 0.$
\item[(c)] $\quad 0 \cdot x = 0 \quad \forall x \in \K$.
\item[(d)] $\quad -x = (-1) \cdot x \quad \forall x \in \K$.
\end{itemize}
\end{Lause}

\tod (a) Oletetaan, että $0 \in \K$ ja $\theta \in \K$ ovat kaksi nolla--alkiota. Tällöin
aksiooman (K6) mukaan on
\[
x = x + 0 \quad \forall x \in \K, \quad \quad y + \theta = y \quad \forall y \in \K.
\]
Kun valitaan $x = \theta$ ja $y = 0$ ja käytetään aksioomaa (K1), seuraa
\[
\theta = \theta + 0 = 0 + \theta = 0,
\]
eli $\theta = 0$. Siis nolla--alkio on yksikäsitteinen. Vasta--alkion yksikäsitteisyyden 
toteamiseksi oletetaan, että $a\in\K$ ja $b\in\K$ ovat saman alkion $x \in \K$ 
vasta--alkioita. Tällöin aksioomien (K6),\,(K3) ja (K1) perusteella on
\[
a=a+0=a+(x+b)=(a+x)+b=(x+a)+b=0+b=b+0=b,
\]
eli $a=b$. Siis vasta--alkiokin on yksikäsitteinen. Muut väitetyt yksikäsitteisyydet 
seuraavat samanlaisella päättelyllä.

(b) Väittämän ensimmäinen osa todistettiin edellisessä luvussa (Lause \ref{Z-tuloksia}(b)).
Toisen osan todistamiseksi sovelletaan aksioomia (K2) ja (K9):
\[
x^{-1} \cdot x = x \cdot x^{-1} = 1\ \ \impl\ \ (x^{-1})^{-1}=x.
\]

(c) Kun merkitään $0 \cdot x = y\in\K$, niin aksioomien (K6),\,(K2) ja (K5) perusteella
\begin{equation}
y = 0 \cdot x = (0 + 0) \cdot x = 0 \cdot x + 0 \cdot x = y+y.  \tag{$\star$}
\end{equation}
Käyttäen tätä tulosta ja kunta--aksioomia päätellään:
\begin{align}
0 &= y+(-y)      \tag{K7} \\
  &= (y+y)+(-y)  \tag{$\star$} \\
  &= y+[y+(-y)]  \tag{K3} \\
  &= y+0         \tag{K7} \\
  &= y.          \tag{K6}
\end{align}

(d) Tuloksen (c) ja kunta--aksioomien perusteella
\begin{align}
x + (-1) \cdot x &= x \cdot 1 + x \cdot (-1)  \tag*{(K8),\,(K2)} \\
                 &= x \cdot [1 + (-1)]        \tag{K5} \\
                 &= x \cdot 0                 \tag{K7} \\
                 &= 0. \loppu                 \tag*{(K2),\,(c)}
\end{align}

Yleisessä kunnassa voidaan potenssiin korotus määritellä samalla tavoin kuin rationaalilukujen 
kunnassa, vrt.\ edellinen luku.
\begin{Exa} Johda binomikaavat lausekkeille $\ (x+y)^2\ $ ja $\ (x+y)^3\ $ sellaisessa kunnassa
$(\K,+,\cdot)$, jossa pätee $1+1 = \spadesuit$, $\spadesuit+1 = \clubsuit$. \end{Exa}
\ratk Potenssiin korotuksen määritelmän ja kunta-aksioomien perusteella
\begin{align*}
(x+y)^2\ =\ (x+y) \cdot (x+y)\ &=\ x \cdot x + x \cdot y + y \cdot x + y \cdot y \\
                               &=\ x^2 + (x \cdot y + x \cdot y) + y^2.
\end{align*}
Tässä on edelleen kunta--aksioomien ja oletuksen perusteella
\[
x \cdot y + x \cdot y\ =\ (x \cdot y) \cdot (1+1)\ 
                       =\ (x \cdot y) \cdot \spadesuit\ =\ \spadesuit \cdot x \cdot y,
\]
joten pyydetty ensimmäinen binomikaava on
\[
(x+y)^2\ =\ x^2 + \spadesuit \cdot x \cdot y + y^2.
\]
Vastaavalla päättelyllä saadaan toiseksi kaavaksi
\[
(x+y)^3\ =\ x^3 + \clubsuit \cdot x^2 \cdot y + \clubsuit \cdot x \cdot y^2 + y^3. \loppu
\]
%\begin{Exa} Päättele, että kunnan nolla--alkiolla ei ole käänteisalkiota. \end{Exa}
%\ratk Lauseen \ref{kuntatuloksia} väittämän (c) ja erotteluaksiooman (K10) perusteella on 
%$0 \cdot x = 0 \neq 1\ \forall x \in \K$, eli mikään $x \in \K$ ei täytä $0$:n käänteisalkiolle
%asetettavaa vaatimusta $0 \cdot x = 1$. \loppu

Seuraava kunta--algebran tulos osoittautuu jatkossa hyödylliseksi
(Harj.teht.\,\ref{H-I-2: kuntakaavat}a).
\begin{Prop} \label{kuntakaava} Jokaisessa kunnassa pätee
\begin{align*}
x^n-y^n\, &=\, (x-y)\left(x^{n-1}+x^{n-2}y + \ldots + y^{n-1}\right) \\
          &=\, (x-y)\sum_{k=0}^{n-1} x^{n-1-k}y^k, \quad n\in\N.
\end{align*} \end{Prop}
 
\subsection*{Järjestetty kunta}
\index{jzy@järjestetty kunta|vahv}%

Jos kunnassa on määritelty järjestysrelaatio edellisen luvun aksioomien (J1)-(J4) mukaisesti,
niin sanotaan, että kyseessä on \kor{järjestetty kunta} (engl. ordered field). Järjestetyssä
kunnassa jokainen alkio on aksiooman (J1) mukaisesti joko positiivinen ($x>0$), negatiivinen
($x<0$), tai $=0$. Toistaiseksi ainoa esimerkki järjestetystä kunnasta on rationaalilukujen
kunta. 
\addtocounter{Thm}{-2}
\begin{Lause} (jatko) Jokaisessa järjestetyssä kunnassa $(\K,+,\cdot,<)$ pätee
\begin{itemize}
\item[(e)] $\quad x>0\ \ \impl\ \ -x<0, \quad\quad x>0\ \ \impl\ \ x^{-1} > 0$.
\item[(f)] $\quad x>0\ \ \&\ \ y>0 \ \ \impl\ \ x+y>0$.
\item[(g)] $\quad 0 < 1$.
\end{itemize} \end{Lause}
\addtocounter{Thm}{1}
\tod (e) \ Oletetaan, että $0<x$. Tällöin on aksiooman (J3) mukaan myös 
\[
0+(-x)\ <\ x + (-x), 
\]
mikä kunta-aksioomien mukaan pelkistyy ensimmäiseksi väittämäksi $-x < 0$. Väittämän toisen 
osan todistamiseksi suljetaan pois aksiooman (J1) jättämät muut vaihtoehdot. Jos $x^{-1} = 0$,
niin aksioomien (K8),\,(K2) ja tuloksen (c) mukaan on $1 = x \cdot x^{-1} = x \cdot 0 = 0$ eli
$0=1$, mikä on ristiriidassa aksiooman (K10) kanssa. Siis mahdollisuus $x^{-1} = 0$ on pois
suljettu. Jos $x^{-1} < 0$, niin tuloksen (d) ja väittämän (e) jo todistetun ensimmäisen osan
mukaan on $(-1) \cdot x^{-1} > 0$. Tällöin aksiooman (J4) ja oletuksen $x>0$ mukaan on myös 
$(-1) \cdot x^{-1} \cdot x > 0$, mikä sievenee kunta-aksioomien ja mainittujenen tulosten 
perusteella muotoon $1<0$. Tämäkin on mahdotonta aksiooman (J1) ja vielä todistamatta olevan 
väittämän (g) mukaan, joten sikäli kuin (g) on tosi, jää ainoaksi vaihtoehdoksi $x^{-1} > 0$.   

(f) Päätellään ensin aksioomien (J3), (K1) ja (K6) perusteella:
\[
0<y\ \ \impl\ \ x<x+y.
\]
Jatketaan tästä soveltaen aksioomaa (J3):
\[
0<x\ \ \&\ \ x<x+y\ \ \impl\ \ 0<x+y.
\]

(g) Aksioomien (J1) ja (K10) mukaan on oltava joko $0<1$ tai $1<0$. Jos oletetaan jälkimmäinen
vaihtoehto, niin aksiooman (J3) mukaan on siinä tapauksessa
\[
1 + (-1) < 0 + (-1),
\]
mikä sievenee aksioomien (K7),\,(K1) ja (K6) perusteella muotoon
\[
0 < -1.
\]
Tällöin on aksiooman (J4) perusteella oltava myös
\[
0 < (-1) \cdot (-1).
\]
Mutta väittämien (b),\,(d) perusteella on $1 = -(-1) = (-1) \cdot (-1)$, joten on päätelty,
että oletetussa vaihtoehdossa $1<0$ pätee myös $0<1$. Aksiooman (J1) mukaan tämä on kuitenkin 
mahdotonta, joten ainoaksi vaihtoehdoksi jää $0<1$. Päättelyssä ei tarvittu vielä avoimena 
olevaa väittämää (e), joten myös tämän väittämän todistus tuli samalla loppuun viedyksi. 
\loppu

Järjestetyssä kunnassa voidaan jokaiseen kunnan alkioon liittää 'itseisalkio', eli ko.\ alkion
\index{itseisarvo}%
\kor{itseisarvo} (engl.\ absolute value) määritelmällä
\[
\abs{x}\ =\ \max \{x,-x\}\ =\ \begin{cases}
                                 \ x,  &\text{jos $x \ge 0$,} \\
                                  -x,   &\text{jos $x < 0$.}
                              \end{cases}
\]       
Itseisarvon avulla on edelleen määriteltävissä kahden luvun välinen \kor{etäisyys}
\[ d(x,y)\ =\ \abs{x-y}, \]
jolloin sellaiset sanonnat kuin '$y$ on lähempänä $x$:ää kuin $z$' saavat (algebrallisen) 
sisällön. Seuraavaa itseisarvoon liittyvää tulosta tarvitaan matemaattisessa analyysissä hyvin
usein.
\begin{Lause} \label{kolmioepäyhtälö} (\vahv{Kolmioepäyhtälö})
\index{kolmioepäyhtälö!a@järjestetyn kunnan|emph} Järjestetyssä kunnassa $(\K,+,\cdot,<)$ pätee
\[
\boxed{\kehys \quad \abs{\,\abs{x} - \abs{y}\,}\ \ 
             \le\ \ \abs{x+y}\ \ \le\ \ \abs{x} + \abs{y}, \quad x,y \in \K. \quad}
\] 
\end{Lause}
\tod Päätulos on epäyhtälöistä jälkimmäinen, sillä edellinen seuraa kunta--aksioomista,
jälkimmäisestä epäyhtälöstä ja itseisarvon määritelmästä päättelyllä
\begin{align*}
&\begin{cases}
 \,\abs{x}\ =\ \abs{\,(x+y) + (-y)\,}\ \le\ \abs{x+y} + \abs{-y}\ =\ \abs{x+y} + \abs{y} \\
 \,\abs{y}\ =\ \abs{\,(x+y) + (-x)\,}\ \le\ \abs{x+y} + \abs{-x}\ =\ \abs{x+y} + \abs{x}
\end{cases} \\
& \quad\ \ \qimpl \pm\,(\abs{x}-\abs{y})\ \le\ \abs{x+y}
           \qimpl  \abs{\abs{x}-\abs{y}}\ \le\ \abs{x+y}.
\end{align*}
Kolmioepäyhtälön jälkimmäisessä osassa väitetään itseisarvon määritelmän perusteella, että
\[
\pm(x+y)\ \le\ \abs{x} + \abs{y}.
\]
Koska saman määritelmän mukaan on
\[
\pm x \le \abs{x} = a, \quad \pm y \le \abs{y} =b,
\]
niin nähdään, että väitetty epäyhtälö seuraa väittämästä
\begin{Lem} Järjestetyssä kunnassa pätee 
\[  
x \le a\ \ \ja\ \ y \le b \qimpl x+y \le a+b. 
\] 
\end{Lem}
\tod Väittämässä on neljä vaihtoehtoa:\ a) $x<a,\ y<b$,\ b) $x<a,\ y=b$,\ c) $x=a,\ y<b$,\ 
d) $x=a,\ y=b$. Vaihtoehdossa a) on $\,x<a\ \ekv\ a-x>0\,$ ja $\,y<b\ \ekv\ b-y>0$, jolloin
Lauseen \ref{kuntatuloksia} väittämästä (f) ja kunta-aksioomista seuraa
$(a-x)+(b-y)>0\ \ekv\ x+y<a+b$. Myös vaihtoehdoissa b) ja c) väite toteutuu tässä muodossa ja
vaihtoehdossa d) muodossa '$=$', kuten nähdään helposti. \loppu

\subsection*{Alikunta ja kuntalaajennus}
\index{alikunta|vahv} \index{kuntalaajennus|vahv}%

Sanotaan, että kunta $(\A,+,\cdot)$ on kunnan $(\K,+,\cdot)$ \kor{alikunta} (engl.\ subfield),
jos $\A \subset \K$ ja kunnilla on samat laskuoperaatiot, ts.\ kunnan $(\A,+,\cdot)$ 
laskuoperaatiot $=$ kunnan $(\K,+,\cdot)$ operaatiot osajoukkoon $\A$ rajoitettuina. Tällöin 
kunnilla on myös yhteiset nolla- ja ykkösalkiot, eli kunnan $(\K,+,\cdot)$ nolla- ja 
ykkösalkioille pätee $\ 0,1 \in \A$. Nimittäin jos \K:n nolla--alkio on $0 \in \K$ ja \A:n 
nolla--alkio on $\theta \in \A$, niin jokaisella $x \in \A$ (esim.\ $x=\theta$) voidaan päätellä
\[
x = x + \theta \ \ \impl \ \ -x + x = -x + x + \theta \ \ 
                     \impl \ \ 0 = 0+\theta = \theta+0 = \theta.
\]
Tässä $-x \in \K$ on $x$:n vasta--alkio \K:ssa, jolloin päättely perustui ensin kunnan 
$(\A,+,\cdot)$ aksioomaan (K6) ja sen jälkeen kunnan $(\K,+,\cdot)$ aksioomiin (K1),\,(K3), (K6)
ja (K7). Vastaavaan tapaan nähdään, että ykkösalkio on kunnissa sama. 

Jos kunta $(\A,+,\cdot)$ on kunnan $(\K,+,\cdot)$ alikunta, sanotaan vastaavasti, että 
jälkimmäinen on edellisen \kor{kuntalaajennus}. Myöhemmissä luvuissa tehdään useita lukualueen
laajennuksia tyyppiä $\Q \ext \K$. Näille on yhteistä, että laajennuksissa syntyy uusia kuntia,
joiden kaikkien yhteinen alikunta on $(\Q,+,\cdot)$. Seuraavassa hyvin varovainen esimerkki
tällaisesta laajennuksesta.

\begin{Exa} \label{muuan kunta} Sovitaan, että on olemassa luku $a$ jolla on ominaisuus
\[
a^2 = a \cdot a = 2.
\]
(Pidetään tunnettuna, että $a\not\in\Q$.) Luvun $a$ ja rationaalilukujen välisistä
laskuoperaatioista sovitaan, että $1 \cdot a = a$ ja $0 \cdot a = 0\in\Q$ ja lisäksi
sovitaan, että vaihdanta-, liitäntä- ja osittelulait (K1)--(K5) ovat voimassa myös, kun
luku $a$ on operaatioissa mukana. Näiden sopimusten nojalla voidaan muodostaa lukujoukko
\[
\K = \{\, z = x + ya \mid x,y \in \Q\,\}.
\]
Tämä on $\Q$:n aito laajennus, sillä $x\in\Q\ \impl x=x+0a\in\K$, mutta $0+1a=a\not\in\Q$.
Tehtyjen sopimusten mukaan on erityisesti $0+0a=0$. Luvulla $0$ ei ole $\K$:ssa muita
esitysmuotoja, sillä jos $x+ya=0$, niin on oltava $y=0\ \impl\ x=0$, koska muuten olisi
$a=-x/y\in\Q$. (Yleisemmin voidaan päätellä, että jokaisen luvun $z\in\K$ esitysmuoto $z=x+ya$
on yksikäsitteinen.)

Jos $z_1 = x_1 + y_1a \in \K$ ja $z_2 = x_2 + y_2a \in \K$ ($x_1,x_2,y_1,y_2 \in \Q$), niin
oletettujen aksioomien (K1)--(K5) ja laskusäännön $a^2=2$ nojalla
\begin{align*}
z_1+z_2 &= (x_1+x_2) + (y_1+y_2)a, \\
z_1z_2  &= (x_1x_2 + y_1y_2a^2) + (x_1y_2 + x_2y_1)a = (x_1x_2+2y_2y_2) + (x_1y_2 + x_2y_1)a,
\end{align*}
joten $z_1+z_2\in\K$ ja $z_1z_2\in\K$. Laskuoperaatioita koskeva perusoletus (ensimmäinen
oletus edellä) on siis voimassa. Edelleen nähdään, että kunta--aksioomista ovat oletettujen
lisäksi voimassa myös (K6), (K8) ja (K10) ($0,1\in\Q$) ja (K7) (jos $z = x + ya$, niin 
$-z = -x -ya$). Lopuksi päätellään, että myös (K9) on voimassa. Nimittäin $z z^{-1} = 1$, kun
määritellään 
\[
z^{-1} = (x^2-2y^2)^{-1}\,(x-ya) = \dfrac{x}{x^2-2y^2} - \dfrac{y}{x^2-2y^2}\,a. 
\]
Tässä on $x^2-2y^2=0$ vain kun $x=y=0$ (koska $(x/y)^2 \neq 2$ kun $x,y\in\Q$ ja $y\neq 0$),
joten jokaisella $z\in\K,\ z \neq 0$ on käänteisluku $z^{-1}$. On päätelty, että $(\K,+,\cdot)$
on kunta ja siis rationaalilukujen kunnan $(Q,+,\cdot)$ aito laajennus.

Kunta $(\K,+,\cdot)$ voidaan myös järjestää. Sovitaan ensinnäkin, että $a>0$ (mahdollista,
koska $a^2=(-a)^2$). Yleisemmin nähdään, että vertailu $z_1<z_2$ ($z_1,z_2\in\K$) palautuu
kunta--algebran avulla aina viime kädessä luvun $a$ ja rationaaliluvun vertailuksi.
(Esim.\ $7-2a>1-a\ \ekv\ 6-3a>0\ \ekv\ 3/2>a$.) Jos $x\in\Q$, niin ilmeisesti on $x<a$ aina kun
$x \le 0$. Jos taas on $x>0$, niin on myös $x+a>0\ \ekv\ (x+a)^{-1}>0$, jolloin vertailukysymys
$x\,?\,a$ ratkeaa käyttämällä järjestysrelaation aksioomaa (J4)\,:
\[
x>a \qekv x-a\,=\,\frac{x^2-a^2}{x+a}\,>\,0 \qekv x^2-a^2>0 \qekv x^2>2.
\]   
Järjestetyssä kunnassa $(\K,+,\cdot,<)$ luku $a$ sijoittuu siis rationaalilukujen 'väliin'\,:
Jokaisen luvun $x \in \Q$ kohdalla voidaan ratkaista, onko $a<x$ vai $a>x$, ja tämä vertailu
perustuu vain rationaalilukujen $0,x,2$ ja $x^2$ vertailuun. \loppu
\end{Exa}

\subsection*{Juuriluvut ja murtopotenssit}

Esimerkin \ref{muuan kunta} kuntaa huomattavasti käyttökelpoisempi kuntalaajennus saadaan
aikaan, kun rationaalilukujen joukko laajennetaan lukujoukoksi, jossa kuntaoperaatioiden
$x,y \map x+y,\,x-y,\,xy,\,x/y$ lisäksi sallitaan laskuoperaatiot
\[
x \map \sqrt[m]{x}, \quad x>0, \quad m\in\N,\ m \ge 2.
\]
\index{juuriluku}%
Tässä luvulla $\sqrt[m]{x}$ tarkoitetaan \kor{juurilukua} $y$, joka toteuttaa ehdot
\[
y^m = x\ \ \ja\ \ y>0.
\]
Sopimuksella $\sqrt[1]{x}=x$ tulee juuriluku $\sqrt[m]{x}$ määritellyksi $\forall m\in\N$.
Sanotaan, että $\sqrt[m]{x}$ on luvun $x$ \kor{neliöjuuri}, jos $m=2$
(lyhennysmerkintä $\sqrt{x}$), \kor{kuutiojuuri}, jos $m=3$, tai  yleisemmin $m$:\kor{s juuri},
luetaan '$m$:s juuri $x$'. Tällainen luku voidaan yksinkertaisesti sopia 'olemassa olevaksi'
samaan tapaan kuin meneteltiin Esimerkissä \ref{muuan kunta} luvun $a\ (=\sqrt{2})$ kohdalla.
Lisäehto $y>0$ (joka jo viittaa järjestysrelaatioon) tarvitaan takaamaan luvun $y$
yksikäsitteisyys, sillä $m$:n ollessa parillinen on $(-y)^m=y^m$. Yksikäsitteisyys mainitulla
lisäehdolla seuraa identiteetistä (ks.\ Propositio \ref{kuntakaava})
\[
(y_1-y_2)(y_1^{m-1}+y_1^{m-2}y_2+ \ldots + y_2^{m-1}) = y_1^m-y_2^m.
\]
Jos tässä oletetetaan, että $y_1^m=y_2^m=x$ ja $y_1,\,y_2>0$, niin oikea puoli $=0$ ja vasemmalla
puolella on jälkimmäinen tekijä $>0$, joten on oltava $y_1-y_2=0$.  

Laskuoperaatioilla $x \map \sqrt[m]{x}$ laajennettua lukujoukkoa, lähtökohtana rationaaliluvut,
merkittäköön symbolilla $\J$. Joukko $\J$ määritellään yksinkertaisesti koostuvaksi luvuista,
jotka saadaan äärellisellä määrällä kuntaoperaatioita ja operaatioita $x \map \sqrt[m]{x}$
lähtien luvuista $0,1\in\Z$. Tällöin on ilmeistä, että jos $x,y\in\J$, niin myös
$x \pm y,\,xy\,,x/y\in\J\ (y \neq 0)$, joten laskuoperaatioita koskeva perusoletus on voimassa.
Lukujen $x\in\J$ 'ulkonäkö' voi kylläkin olla konstikas.
\begin{Exa}
\[
\left(3+\sqrt{2}
-\sqrt[8]{\frac{\sqrt[6]{5-\sqrt[4]{3}}}{\sqrt[4]{3+\sqrt[3]{2}}}+\frac{1}{\sqrt[7]{7}}}\right)/
\left(2+\sqrt{3}
+\sqrt[8]{\frac{\sqrt[6]{3-\sqrt[4]{5}}}{\sqrt[4]{2+\sqrt[3]{3}}}+\sqrt[7]{7}}\right)
                                                        \ \in\ \J. \loppu
\]
\end{Exa}
Juuriluvuilla laskettaessa oletetaan kunta--algebran laskusäännöt päteviksi, ts.\ oletetaan,
että $(\J,+,\cdot)$ on kunta. Tästä oletuksesta sekä juuriluvun määritelmästä voidaan johtaa
seuraavat yleiset laskusäännöt (ol.\ $x,y>0,\ m,n\in\N$):
\[
\text{a)}\ \ \sqrt[m]{x} = \sqrt[mn]{x^n}. \qquad
\text{b)}\ \ \sqrt[m]{x}\sqrt[m]{y} = \sqrt[m]{xy}. \qquad
\text{c)}\ \ \left(\sqrt[m]{x}\right)^{-1} = \sqrt[m]{x^{-1}}.
\]
Sääntöjen perustelemiseksi merkitään $a=\sqrt[m]{x}$, $b=\sqrt[m]{y}\,$, jolloin on
$a,b>0\ \impl ab,a^{-1}>0$. Juuriluvun määritelmään ja kunta--algebraan vedoten voidaan
tällöin päätellä:
\begin{align*}
&\text{a)}\ \ a^m=x\ \ \impl\ \ \ (a^m)^n = a^{mn} = x^n\ \  \impl\ \ a=\sqrt[mn]{x^n}. \\[1mm]
&\text{b)}\ \ a^m=x\ \ja\ b^m=y\ \ \impl\ \  a^mb^m =(ab)^m = xy\ \ \impl\ \ ab=\sqrt[m]{xy}. \\
&\text{c)}\ \ a^m=x\ \ \impl\ \ (a^m)^{-1} = (a^{-1})^m=x^{-1}\ \ \impl\ \ a^{-1}=\sqrt[m]{x^{-1}}.
\end{align*}

Juuriluvuilla laskemisen säännöt saadaan kätevästi yhdistetyksi potenssien laskusääntöihin
\index{murtopotenssi}%
(vrt.\ edellinen luku), kun määritellään luvun $x>0$ \kor{murtopotenssi} asettamalla
\[
x^{p/q} = \sqrt[q]{x^p}, \quad p\in\Z,\ q\in\N,\ q \ge 2.
\]
Tämä määrittelee yksiselitteisesti luvun $x^r$ jokaisella $r\in\Q$, sillä jos $r$ esitetään
perusmuodossa $r=p/q,\ q\in\N$, niin määritelmän ja säännön a) perusteella on
$x^{(pn)/(qn)}=\sqrt[qn]{x^{pn}}=\sqrt[nq]{(x^p)^n}=\sqrt[q]{x^p}=x^{p/q}\ \forall n\in\N$, eli
$x^r$ ei riipu $r$:n esitysmuodosta. Säännöistä a)--c) voidaan nyt johtaa murtopotensseille
samat laskusäännöt kuin kokonaislukupotensseille (Harj.teht.\,\ref{H-I-2: murtopotenssit})\,:
\[
\boxed{\kehys \quad x^rx^s=x^{r+s}, \quad x^ry^r=(xy)^r, \quad (x^r)^s=x^{rs}, 
                                                         \quad x,y>0,\ r,s\in\Q. \quad} 
\]

Luvut $x\in\J,\ x\not\in\Q$ on myös mahdollista sijoitella rationaalilukujen 'väleihin' niin,
että syntyy järjestetty kunta $(\J,+,\cdot,<)$. Järjestysrelaation määrittely yleisessä
tapauksessa vaatisi kuitenkin tähänastista vakavampia lukuteoreettisia pohdiskeluja, siksi
asiaan ei toistaiseksi puututa. Todettakoon ainoastaan, että suotuisissa erikoistapauksissa
vertailu on mahdollista palauttaa suoraan rationaalilukujen vertailuksi samalla tavoin kuin
Esimerkissä \ref{muuan kunta} edellä. Ideana on tällöin purkaa juurilausekkeet päättelyllä: Jos
$x,y>0$ ja $m\in\N$, niin $x<y\ \ekv\ x^m<y^m$ (Harj.teht.\,\ref{H-I-2: potenssien järjestys}).
\begin{Exa} Kumpi on suurempi, $\,x=\sqrt{\sqrt{2}+5}\,$ vai $\,y=\sqrt[4]{41}$\,? \end{Exa}
\ratk Koska $\,x<y\ \ekv\ x^4<y^4\ (x,y>0)$, niin päätellään
\[
x<y \qekv (\sqrt{2}+5)^2 < 41 \qekv 10\sqrt{2}+27 < 41 \qekv 10\sqrt{2} < 14.
\]
Jatkamalla tästä päättelyllä $\,a<b\ \ekv\ a^2<b^2\ (a,b>0)$ nähdään, että
\[
10\sqrt{2} < 14 \qekv 200 < 196.
\]
Koska viimeinen epäyhtälö ei toteudu, vaihdetaan epäyhtälöiden suunnat ja päätellään:
$x>y\ \ekv\ 200>196$. Siis $x$ on suurempi. \loppu

\Harj
\begin{enumerate}

\item
Lukujoukossa $\K$ on vain luvut 0,1 ja $\diamondsuit=1+1$ (kaikki keskenään eri suuria).
Määrittele (jos mahdollista) $\K$:n laskusäännöt (yhteenlasku, kertolasku, vastaluvut, 
käänteisluvut) siten, että $(\K,+,\cdot)$ on kunta.

\item
\index{nollasääntö!a@tulon}%
Vedoten kunta--aksioomiin tai Lauseen \ref{kuntatuloksia} väittämiin näytä, että jokaisessa
kunnassa pätee: \ a) \ $x \cdot y = 0$ $\ \ekv\ $ $x=0$ $\tai$ $y=0$
(\kor{tulon nollasääntö}), \ \ b) \ $(-x)\cdot (y)=-(x\cdot y)$, \ 
c) \ $(-x)\cdot (-y)=x\cdot y$, \ d) $0$:lla ei ole käänteisalkiota.

\item \label{H-I-2: kuntakaavat}
Näytä, että jokaisessa kunnassa pätee \vspace{1mm}\newline
a)\,\ $x^n-y^n = (x-y)\sum_{k=0}^{n-1} x^{n-1-k}y^k,\,\ n\in\N$ \vspace{1mm}\newline
b)\,\ $x^n+y^n = (x+y)\sum_{k=0}^{n-1} (-1)^kx^{n-1-k}y^k,\,\ $ 
      jos $n\in\N$ ja $n$ on pariton \vspace{1mm}\newline
c)\,\ $1+x + \ldots + x^n = (x^{n+1}-1)(x-1)^{-1},\,\ n\in\N,\ x \neq 1$

\item \label{H-I-2: järjestysaksioomat}
Näytä, että järjestetyn kunnan aksiooma (J3) voidaan korvata aksioomalla
(J3') $x<y \,\ekv\, x-y<0$.

\item \label{H-I-2: järjestetyn kunnan väitteitä}
Näytä, että jokaisessa järjestetyssä kunnassa pätee: \newline
a) \ $x>1\,\ \impl\,\ 0 < x^{-1} < 1$ \newline
b) \ $x>0\ \ja\ y>1\ \impl\ 0 < x/y < x$ \vspace{0.2mm}\newline
c) \ $\abs{x \cdot y}=\abs{x}\cdot\abs{y}$ \vspace{0.2mm}\newline
d) \ $\abs{x^{-1}}=\abs{x}^{-1}$ \newline
e) \ $x^2<y^2\ \ekv\ \abs{x} < \abs{y}$ \newline
f) \ $0 \le x \le a\ \ja\ 0 \le y \le b\ \impl\ 0 \le xy \le ab$

\item \label{H-I-2: potenssien järjestys}
Lähtien tehtävän \ref{H-I-2: kuntakaavat} kaavasta a) näytä, että järjestetyssä kunnassa pätee:
Jos $x>0$, $y>0$ ja $n\in\N,\ n \ge 2$, niin $x<y\ \ekv\ x^n<y^n$.

\item
Aseta luvut $a$, $17/12$ ja $72/51$ suuruusjärjestykseen Esimerkin \ref{muuan kunta} kunnassa.
Suorita vertailut tarkasti!

\item \label{H-I-2: murtopotenssit}
Perustele murtopotenssien laskusäännöt \,\ a) $x^rx^s=x^{r+s}$,\,\ b) $x^ry^r=(xy)^r$,
c) $(x^r)^s=x^{rs}$ \ ($x,y>0,\ r,s\in\Q$).

\item
Selvitä ilman laskinta lukujen $x,y\in\J$ suuruusjärjestys palauttamalla vertailu
rationaalilukujen vertailuksi\,:\vspace{1mm}\newline 
a)\,\ $x=\sqrt[3]{1020},\ \ y=\sqrt{102}, \quad$
b)\,\ $x=\sqrt[3]{3},\ \ y=\sqrt[4]{43/10},$\vspace{1mm}\newline
c)\,\ $x=2-\sqrt{3},\ \ y=1/\sqrt{4\sqrt{3}+7}.$

\item
Olkoon $(\K,+,\cdot,<)$ järjestetty kunta. Näytä: \ a) \ $x+1>x\ \forall x\in\K$. \newline
b) Joukossa $\K$ on äärettömän monta eri alkiota. 

\item (*)
Näytä käsinlaskulla, että pätee
\[
\text{a)}\ \ \frac{577}{408}-\frac{1}{400000}\ <\ \sqrt{2}\ <\ \frac{577}{408}\,, \qquad
\text{b)}\ \ \sqrt{2}\ <\ \frac{665857}{470832}\,.
\]

\item (*)
Näytä, että järjestetyn kunnan aksiooma (J2) voidaan korvata aksioomalla
(J2') $x>0 \,\ja\, y>0 \,\impl\, x+y>0$.

\item (*)
Halutaan määritelllä pienin mahdollinen rationaalilukujen kunnan laajennus $(\K,+,\cdot)$ siten,
että $\sqrt{2}\in\K$ ja $\sqrt{3}\in\K$. Näytä, että \ a) $\K$ koostuu luvuista muotoa
$\,x+y\sqrt{2}+z\sqrt{3}+u\sqrt{6},\ x,y,z,u\in\Q$, \ b) kunnassa $(\K,+,\cdot)$ on
määriteltävissä järjestysrelaatio, joka perustuu vain rationaalilukujen vertailuun.

\item(*) \label{H-I-2: Big Ben} \index{zzb@\nim!Big Ben}
(Big Ben) Tarkastellaan joukkoa $\K=$\{kellon viisarit\}. Jokainen viisari $v\in\K$ on 
\kor{lukupari} $(r,\theta)$, missä $r=$ viisarin pituus (yksikkö m) ja $\theta =$ viisarin 
suunta asteina, mitattuna klo 12:sta positiivisena myötäpäivään tai negatiivisena vastapäivään.
Sovitaan, että \ $(r_1,\theta _1)=(r_2,\theta _2)$, jos joko (i) $r_1=r_2=0$ tai
(ii) $r_1=r_2>0$ ja $\theta _1-\theta _2=k \cdot 360,\ k\in\Z$. Jos $r=0$, sanotaan viisaria
$(r,\theta)$ \kor{nollaviisariksi}, merkitään\ $0_v$. Määritellään viisareiden kertolasku
seuraavasti:
\[
(r_1,\theta _1)\cdot (r_2, \theta _2) = (r_1r_2,\,\theta _1+\theta _2).
\]
a) Näytä, että kertolaskulle pätee sekä vaihdantalaki että liitäntälaki.
\newline
b) Määrittele ykkösviisari sekä viisarin $(r, \theta) \neq 0_v$ käänteisviisari.
\newline
c) Big Ben, jonka minuuttiviisarin pituus on 4 ja tuntiviisarin pituus on 2,
näyttää aikaa noin klo 3. Mikä kellonaika on tarkemmin kyseessä (sekunnin tarkkuus\,!), kun
tiedetään, että myös Big Benin käänteiskello Small Ben näyttää samaan aikaan aivan järkevää
(vaikkakin toista) kellonaikaa?  --- Huomaa, että myös Small Benin minuuttiviisari on pidempi\,!

\end{enumerate}