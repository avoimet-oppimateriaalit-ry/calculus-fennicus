\chapter{Kompleksiluvut}

Siirtyminen reaaliluvista kompleksilukuihin on matemaattisen analyysin merkittävimpiä ja samalla
merkillisimpiä aluevaltauksia. Kyse on lukualueen laajennuksesta, ts. siirtymisestä jälleen 
uudelle 'todellisuuden' tasolle. Vaikka laajennusta voi pitää vain kuvitelmana, niin tämä 
kuvitelma on yksinkertaistanut matemaattista ajattelua siinä määrin, että sillä on lopulta ollut
syvällinen vaikutus kaikkeen matematiikkaan, myös käytännön laskentamenetelmiin. Matemaattisen
analyysin perinteessä lukualueen laajennus korostuu käsitteissä \kor{reaalianalyysi} ja 
\kor{kompleksianalyysi}. Molemmat ovat nykyään hyvin laajoja (ja hieman epämääräisiä) 
matematiikan alueita. Kompleksianalyysin osa-alueista maininnan arvoinen on kompleksimuuttujan
funktioiden teoria eli \kor{funktioteoria}\footnote[2]{Funktioteorian tutkimusperinne on 
Suomessa vahva. Tätä matematiikan suuntausta edusti myös Suomen historian tunnetuin 
matemaatikko, akateemikko \hist{Rolf Nevanlinna} (1895-1980). \index{Nevanlinna, R.|av}}.
