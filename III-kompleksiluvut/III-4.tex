\section{*Kompleksikertoiminen vektoriavaruus}
\alku
\index{vektoric@vektoriavaruus!b@kompleksikertoiminen|vahv}

Luvussa \ref{tasonvektorit} esiteltiin algebra nimeltä (lineaarinen)
vektoriavaruus, symbolisesti $(U,\K)$, missä $U$ on vektorien muodostama joukko ja $\K$ on
nk.\ skalaarien muodostama vektoriavaruuden kertojakunta. Vektoriavaruudessa on määritelty
kaksi laskuoperaatiota, vektorien yhteenlasku ja skalaarilla kertominen, jotka ovat
funktioita tyyppiä

\begin{tabular}{ll}
yhteenlasku: &$U \times U \kohti U$, \\
skalaarilla kertominen: &$\K \times U \kohti U$.
\end{tabular}

Sisätuloavaruudeksi sanotaan vektoriavaruutta, jossa on määritelty myös skalaaritulo
(eli sisätulo, ks.\ Luku \ref{abstrakti skalaaritulo}) funktiona tyyppiä

\begin{tabular}{ll}
skalaaritulo: &$\qquad\qquad\,\ U \times U \kohti \K$.
\end{tabular}

Tähän asti on käsitelty lähinnä tapausta $\K = \R$, jolloin puhutaan \kor{reaalikertoimisesta} 
avaruudesta. Laajennus reaaliluvuista kompleksilukuihin mahdollistaa nyt myös 
\kor{kompleksikertoimisen} vektoriavaruuden, jossa $\K = \C$. Kompleksikertoiminen 
vektoriavaruus ei algebrana oleellisesti poikkea reaalikertoimisesta niin kauan kuin vain 
vektorien yhteenlasku ja skalaarilla kertominen on määritelty. Sen sijaan skalaaritulon 
määrittely funktiona tyyppiä $U \times U \kohti \C$ aiheuttaa teoriaan lisäpiirteitä, jotka on
otettava huomioon. Kompleksikertoimisen sisätuloavaruuden $(U, \C)$ sisätulon 
$\scp{\cdot}{\cdot}$ on toteutettava seuraavat ehdot (vrt.\ Luvun \ref{abstrakti skalaaritulo}
ehdot reaalikertoimiselle tapaukselle)\,:
\begin{enumerate}
\item \kor{Puolisymmetrisyys}
\begin{itemize}
\item[] $\scp{\mpu}{\mpv}=\overline{\scp{\mpv}{\mpu}} \quad \forall \ \mpu,\mpv \in U$.
\end{itemize}
\item \kor{Sekvilineaarisuus} \index{sekvilineaarisuus}
\begin{itemize}
\item[(a)] $\scp{\alpha \mpu + \beta \mpv}{\mw} = \alpha \scp{\mpu}{\mw} 
                                                                + \beta \scp{\mpv}{\mw}$,
\item[(b)] $\scp{\mpu}{\alpha \mpv + \beta \mw} = \overline{\alpha} \scp{\mpu}{\mpv} 
                                                        + \overline{\beta} \scp{\mpu}{\mw}$
\item[] \qquad\qquad $\forall \ \mpu, \mpv, \mw \in U,\ \alpha, \beta \in \C$.
\end{itemize}
\item \kor{Positiividefiniittisyys} \index{positiividefiniittisyys!a@skalaaritulon}
\begin{itemize}
\item[(a)] $\scp{\mpu}{\mpu} \geq 0 \ \forall \ \mpu \in U$,
\item[(b)] $\scp{\mpu}{\mpu} = 0 \ \ekv \ \mpu = \mathbf{0}$.
\end{itemize}
\end{enumerate}
Havaitaan, että reaalikertoimisesta tapauksesta poikkeavat vain symmetriaehto (1) sekä 
sekvilineaarisuusehto (2b) (sekvilineaarinen = $1\frac{1}{2}-$lineaarinen, engl.\ sesquilinear 
-- vrt. bilineaarinen = kaksoislineaarinen, engl.\ bilinear). Itse asissa vain symmetriaehdon
ero on olennainen, sillä (1) \& (2a) $\impl$ (2b) (Harj.teht.\,\ref{H-III-4: skalaaritulo}).
Huomattakoon, että symmetriaehto (1) pitää erikoistapauksena sisällään ehdon
$\scp{\mpu}{\mpv} = \scp{\mpv}{\mpu}$, kun $\scp{\mpu}{\mpv} \in \R$, joten aiemmin asetettuja
reaalikertoimisen tapauksen ehtoja ei erillisinä enää tarvita. Koska ehto (1) myös takaa, että 
$\scp{\mpu}{\mpu}\in\R\,\ \forall \mpu \in U$, niin ehto (3a) on mielekäs. Uusista ehdoista
aksioomina tarpeellisia ovat (1), (2a), (3a) ja (3b):n osa $[\impl]$, sillä (1) \& (2a)
$\impl$ (2b), kuten sanottu, ja (2a) $\impl$ (3b):n osa $[\Leftarrow]$.

Skalaaritulon keskeinen ominaisuus on Cauchyn--Schwarzin epäyhtälö, joka vapautettiin 
geometriasta Luvussa \ref{abstrakti skalaaritulo}. Varmistetaan nyt, että epäyhtälö pätee myös
kompleksikertoimisessa tapauksessa (vrt. Lause \ref{schwarzR}).
\begin{Lause} \label{schwarzC} \index{Cauchyn!f@--Schwarzin epäyhtälö|emph}
Jokaiselle aksioomat 1--3 toteuttavalle kompleksikertoimisen vektoriavaruuden $U$
skalaaritulolle pätee Cauchyn--Schwarzin epäyhtälö
\[
\abs{\scp{\mpu}{\mpv}} \leq \abs{\mpu} \abs{\mpv},
\]
missä
\[
\abs{\mpu}=\scp{\mpu}{\mpu}^{1/2}.
\]
\tod
Tapaus $\mpu = \mathbf{0}$ on jälleen selvä, joten voidaan olettaa $\mpu \neq \mathbf{0}$, 
jolloin on $\scp{\mpu}{\mpu}>0$ aksioomien (3a,b) perusteella. Lähdetään jälleen epäyhtälöstä
\[
\scp{\beta \mpu + \mpv}{\beta \mpu + \mpv} \geq 0,
\]
joka nyt on voimassa $\forall \beta \in \C$ (aksiooma (3a)).

Käyttämällä sekvilineaarisuusehtoja (2a,b), symmetriaehtoa (1) ja Luvun 
\ref{kompleksiluvuilla laskeminen} kaavoja (2),\,(3),\,(7) tämä purkautuu muotoon
\begin{align*}
\scp{\beta \mpu}{\beta \mpu} + \scp{\beta \mpu}{\mpv} + \scp{\mpv}{\beta \mpu} 
                                                      + \scp{\mpv}{\mpv} &\geq 0 \\
\ekv \ \abs{\beta}^2 \scp{\mpu}{\mpu} + 2 \; \text{Re} \; \{ \beta \scp{\mpu}{\mpv} \} 
                                      + \scp{\mpv}{\mpv} &\geq 0.
\end{align*}
Koska tämä on voimassa $\forall \beta \in \C$ ja koska $\scp{\mpu}{\mpu} > 0$, voidaan valita
\[
\beta = - \frac{\overline{\scp{\mpu}{\mpv}}}{\scp{\mpu}{\mpu}}\,,
\]
jolloin Luvun \ref{kompleksiluvuilla laskeminen} kaavojen (3),\,(6) perusteella seuraa
\begin{align*}
&-\frac{\abs{\scp{\mpu}{\mpv}}^2}{\scp{\mpu}{\mpu}} + \scp{\mpv}{\mpv} \geq 0 \\
& \ekv \ \abs{\scp{\mpu}{\mpv}}^2 \leq \scp{\mpu}{\mpu} \scp{\mpv}{\mpv} \\[2mm]
& \ekv \ \abs{\scp{\mpu}{\mpv}} \leq \abs{\mpu} \abs{\mpv}. \quad\loppu
\end{align*}
\end{Lause}
\begin{Exa} Yksinkertaisin esimerkki kompleksikertoimisesta sisätuloavaruudesta on $\C$ itse,
ts.\ $(U, \C)$, missä $U=\C$. Nimittäin koska kompleksilukujen yhteenlasku on samanlainen 
operaatio kuin tason vektoreiden yhteenlasku, ja kertolasku voidaan tulkita myös skalaarilla
kertomiseksi, niin $(\C,\C)$ on vektoriavaruus. Kun tässä avaruudessa määritellään skalaaritulo
\[
\scp{u}{v} = u \overline{v}, \quad u,v \in \C,
\]
toteutuvat em. ehdot 1--3 (vrt. Luvun \ref{kompleksiluvuilla laskeminen} kaavat). 
Cauchyn--Schwarzin epäyhtälö pelkistyy tässä erikoistapauksessa yhtälöksi:
\[
\abs{\scp{u}{v}} = \abs{u \overline{v}} = \abs{u}\abs{v}, \quad u,v \in \C. \loppu
\]
\end{Exa}
\begin{Exa} Kompleksikertoimisia sisätuloavaruuksia ovat myös avaruuksien $(\Rkaksi, \R)$ ja
$(\R^3,\R)$ laajennukset $(\C^2,\C)$ ja $(\C^3,\C)$. Esimerkiksi $(\C^3,\C)$:ssä skalaaritulo
määritellään
\begin{align*}
&\mpu =(u_1,u_2,u_3) \in \C^3, \quad \mpv =(v_1,v_2,v_3) \in \C^3\,: \\
&\scp{\mpu}{\mpv} = \sum_{i=1}^3 u_i \overline{v_i}.
\end{align*}
\end{Exa}
Avaruuden $(\C^n, \C),\ (n=2,3)$ 
\index{euklidinen!b@normi} \index{normi!euklidinen}%
\kor{euklidinen normi} on
\[
\abs{\mpu} = \scp{\mpu}{\mpu}^{1/2} = ( \sum_{i=1}^n \abs{u_i}^2 )^{1/2}. \loppu
\]

\Harj
\begin{enumerate}

\item \label{H-III-4: skalaaritulo}
Näytä, että skalaaritulon sekvilineaarisuusominaisuus (2b) seuraa ehdoista (1) ja (2a).

\item
Ovatko avaruuden $(\C^2,\C)$ vektorit 
\[
\mpu=(3-4i,\,7-i), \ \ \mpv=(2-11i,\,13-9i)
\]
lineaarisesti riippumattomat?
 
\item
Jaa sisätuloavaruuden $(\C^2,\C)$ vektori $\mpu=(1+i,2-3i)$ kahteen komponenttiin siten, että
toinen komponentti on vektorin $\mpv=(1-i,2+i)$ suuntainen ja toinen $\mpv$:tä vastaan
kohtisuora.

\end{enumerate}