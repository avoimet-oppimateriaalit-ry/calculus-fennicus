\section{Usean muuttujan Taylorin polynomit} \label{usean muuttujan taylorin polynomit}
\alku \sectionmark{Usean muuttujan Taylor}
\index{Taylorin polynomi!a@usean muuttujan|vahv}

Tarkastellaan kahden muuttujan funktiota $f=f(x,y)$ pisteitä $(x_0,y_0)$ ja $(x,y)$
yhdistävällä janalla $S$. Tällä janalla $f$ voidaan tulkita yhden muuttujan funktiona
\[
g(t)=f(x_0+h_xt,y_0+h_yt),\quad t\in [0,1],
\]
missä on merkitty
\[
h_x=x-x_0,\quad h_y=y-y_0.
\]
\begin{figure}[H]
\setlength{\unitlength}{1cm}
\begin{center}
\begin{picture}(6,3)
\path(0,0)(6,3) \put(3,1.5){\vector(2,1){0.1}}
\put(0.9,0.4){$\bullet$} \put(0.55,0.8){$t=0$} \put(0.8,0){$(x_0,y_0)$}
\put(4.9,2.4){$\bullet$} \put(4.55,2.8){$t=1$} \put(4.8,2){$(x,y)$}
\put(3,1.1){$t$} \put(2.5,1.5){$S$} \curve(2.4,1.6,2.2,1.4,2,1)
\end{picture}
\end{center}
\end{figure}
Oletetaan, että funktion $f$ osittaisderivaatat $\partial^\alpha$ ovat kertalukuun 
$\abs{\alpha}=n+1$ asti olemassa ja jatkuvia janalla $S$. Tällöin voidaan osittaisderivoinnin
ketjusäännöistä päätellä, että funktio $g$ on $n+1$ kertaa jatkuvasti derivoituva välillä 
$[0,1]$, jolloin Taylorin lauseen mukaan pätee jokaisella $t\in (0,1]$
\[
g(t)=\sum_{k=0}^n \frac{1}{k!}g^{(k)}(0)\,t^k+\frac{1}{(n+1)!}g^{(n+1)}(\xi)\,t^{n+1},
\]
missä $\xi\in (0,t)$. Kun erityisesti valitaan $t=1$, on tulos
\begin{equation} \label{udif-9: eq1}
f(x,y)=g(1)=\sum_{k=0}^n\frac{1}{k!}g^{(k)}(0)+\frac{1}{(n+1)!}g^{(n+1)}(\xi). \tag{$\star$}
\end{equation}
Tässä voidaan derivaatta $g'(t)$ laskea ketjusäännön perusteella muodossa
\begin{align*}
g'(t) &= h_x f_x(x_0+h_xt,y_0+h_yt)+h_y f_y(x_0+h_xt,y_0+h_yt) \\
      &=[\,(h_x\partial_x+h_y\partial_y)f\,](x_0+h_xt,y_0+h_yt).
\end{align*}
Soveltamalla tätä sääntöä uudelleen nähdään, että
\[
g^{(k)}(t)=[\,(h_x\partial_x+h_y\partial_y)^kf\,](x_0+h_xt,y_0+h_yt). 
\]
Tässä $h_x$ ja $h_y$ tulkitaan derivoitaessa vakioiksi (koska derivointi kohdistuu muuttujaan
$t$). Tuloksen perusteella $f(x,y)$:n lauseke \eqref{udif-9: eq1} voidaan tulkita
\kor{kahden muuttujan Taylorin kaavana}
\[
\boxed{\begin{aligned}\ykehys
f(x,y)   &= T_n(x,y,x_0,y_0)+R_n(x,y), \quad \text{missä} \\
\quad T_n(x,y,x_0,y_0) 
         &= \sum_{k=0}^n \frac{1}{k!}\,[\,(h_x\partial_x+h_y\partial_y)^kf\,](x_0,y_0), \\
R_n(x,y) &=\frac{1}{(n+1)!}\,[\,(h_x\partial_x + h_y\partial_y)^{n+1}f\,](x(\xi),y(\xi)),\\[2mm]
h_x      &= x-x_0,\quad h_y=y-y_0, \\
x(\xi)   &= x_0+h_x\xi,\quad y(\xi)=y_0+h_y\xi,\quad \xi\in (0,1). \akehys\quad
\end{aligned}}
\]
Tässä $T_n(x,y,x_0,y_0)$ on $f$:n \kor{Taylorin polynomi} astetta $n$ pisteessä $(x_0,y_0)$ ja
\index{jzyzy@jäännöstermi (Lagrangen)}%
$R_n$ on \kor{jäännöstermi}. Huomioimalla, että (binomikaavan perusteella, vrt.\
Luku \ref{osittaisderivaatat})
\[
(h_x\partial_x+h_y\partial_y)^k f 
        = \sum_{l=0}^k \binom{k}{l}h_x^{k-l}h_y^l\partial_x^{k-l}\partial_y^l f, \quad
          \binom{k}{l}=\frac{k!}{(k-l)!\,l!}\,,
\]
saadaan polynomille $T_n(x,y,x_0,y_0)$ konkreettisempi esitysmuoto
\[
\boxed{\quad T_n(x,y,x_0,y_0)=\sum_{k=0}^n\sum_{l=0}^k \frac{1}{(k-l)!\, l!}\,
   \frac{\partial^k f}{\partial x^{k-l}\partial y^l}(x_0,y_0)\,(x-x_0)^{k-l}(y-y_0)^l. \quad}
\]
Tästä ja helposti todennettavasta derivointikaavasta
\[
\left[\frac{\partial^{i+j}}{\partial x^i\partial y^j}(x-x_0)^m(y-y_0)^n\right]_
                                  {\left|\begin{array}{l} 
                                   \scriptstyle{x=x_0} \\ 
                                   \scriptstyle{y=y_0} \end{array}\right.}
=\begin{cases} m!\,n!, &\text{jos $i=m$ ja $j=n$}, \\ 0, &\text{muulloin} \end{cases}
\]
nähdään, että polynomi $p(x,y)=T_n(x,y,x_0,y_0)$ toteuttaa ehdot
\[
\partial^\alpha p(x_0,y_0)=\partial^\alpha f(x_0,y_0),\quad \abs{\alpha}\leq n.
\]
Taylorin polynomi myös määräytyy yksikäsitteisesti näistä ehdoista, kuten yhden muuttujan
tapauksessa (vrt.\ Luku \ref{taylorin lause}).
\begin{Exa}
Määrää funktion
\[
f(x,y)=e^{-x}+xy+y^2+\cos x\,\cos y
\]
Taylorin polynomi $T_2(x,y,0,0)$.
\end{Exa}
\ratk Derivoimalla todetaan
\begin{align*}
&f(0,0)=2, \quad f_x(0,0)=-1, \quad f_y(0,0)=0, \\
&f_{xx}(0,0)=0, \quad f_{xy}(0,0)=f_{yx}(0,0)=f_{yy}(0,0)=1,
\end{align*}
joten
\begin{align*}
T_2(x,y,0,0) &= 2 +(-1) \cdot x + 0 \cdot y
                  +\frac{1}{2!0!} \cdot 0 \cdot x^2 
                  +2\cdot\frac{1}{1!1!} \cdot 1 \cdot xy
                  +\frac{1}{0!2!} \cdot 1 \cdot y^2 \\
             &= 2-x+xy+\frac{1}{2}y^2. \loppu
\end{align*}

Taylorin kaava siis pätee, kunhan $f$:n osittaisderivaatat ovat kertalukuun $n+1$ asti jatkuvia
tarkastelupisteen $(x,y)$ ja pisteen $(x_0,y_0)$ välisellä yhdysjanalla. Taylorin kaava
on siten kahdenkin muuttujan  tapauksessa johtoajatukseltaan yksiulotteinen. Jos halutaan, että
kaava pätee tarkastelupisteen liikkuessa jossakin joukossa $A$ pisteen $(x_0,y_0)$
ympäristössä, niin onkin oletettava, $f$:n riittävän säännöllisyyden lisäksi, että $A$ on
\index{tzy@tähden muotoinen}%
pisteen $(x_0,y_0)$ suhteen nk. \kor{tähden muotoinen} (engl.\ star-shaped). Tämä tarkoittaa,
että jokaiseen pisteeseen $(x,y)\in A$ on pisteestä $(x_0,y_0)$ 'näköyhteys', eli pisteiden
välinen yhdysjana on kokonaisuudessaan joukossa $A$. Näin ajatellen voidaan Taylorin kaavasta
johtaa seuraava yleinen polynomiapproksimaatiotulos kahden muuttujan funktiolle.
\begin{Lause} \label{Taylor-R2} \index{Taylorin lause!c@kahden muuttujan|emph}
Jos funktion $f=f(x,y)$ osittaisderivaatat ovat kertalukuun $n+1$ asti jatkuvia pisteen 
$(x_0,y_0)$ suhteen tähden muotoisessa ja kompaktissa joukossa $A$, niin jokaisella 
$(x,y)\in A$ pätee
\[
f(x,y)=T_n(x,y,x_0,y_0)+\Ord{h^{n+1}}, \quad h=\sqrt{(x-x_0)^2+(y-y_0)^2}.
\]
Jos $f$:n osittaisderivaatat ovat $A$:ssa jatkuvia  kertalukuun $n$ asti, niin 
\[
f(x,y)=T_n(x,y,x_0,y_0)+\ord{h^n}, \quad (x,y) \in A.
\]
\end{Lause}
\tod Jos $f$:n osittaisderivaatat ovat kertalukuun $n+1$ jatkuvia kompaktissa joukossa $A$,
niin ne ovat myös rajoitettuja (ks.\ Lause \ref{weierstrass - Rn}):
\[
\abs{\partial^\alpha f(x,y)}\leq C_\alpha\quad\forall\,(x,y) \in A,\quad \abs{\alpha}\leq n+1.
\]
Valitsemalla $\,C=\max_{\abs{\alpha}=n+1} C_\alpha\,$ seuraa Taylorin kaavan jäännöstermille arvio
\begin{align*}
\abs{R_n(x,y)} &\le \frac{C}{(n+1)!}\sum_{l=0}^{n+1} 
                                    \binom{n+1}{l}\abs{x-x_0}^{n+1-l}\abs{y-y_0}^l \\
               &=\frac{C}{(n+1)!}\,\left(\abs{x-x_0}+\abs{y-y_0}\right)^{n+1} \\
               &\le \frac{C(\sqrt{2})^{n+1}}{(n+1)!}\,h^{n+1}=\Ord{h^{n+1}}, \quad (x,y) \in A.
\end{align*}
Astetta heikompien säännöllisyysoletusten ollessa voimassa päätellään vastaavasti
(vrt.\ Lause \ref{Taylorin approksimaatiolause})
\begin{align*}
f(x,y) &= T_{n-1}(x,y,x_0,y_0)+R_{n-1}(x,y) = T_n(x,y,x_0,y_0)+\ord{h^n}. \loppu
\end{align*}

Tapauksessa $n=1$ Lauseen \ref{Taylor-R2} jälkimmäinen väittämä on: Jos $f$, $f_x$,
$f_y$ ovat jatkuvia pisteen $(x_0,y_0)$ (tähden muotoisessa) ympäristössä, niin
\begin{align*}
f(x,y) &= T_1(x,y,x_0,y_0)+\ord{h} \\
&= f(x_0,y_0)+f_x(x_0,y_0)(x-x_0)+f_y(x_0,y_0)(y-y_0)+\ord{h}.
\end{align*}
Tulos on ennestään tuttu, sillä tehdyin oletuksin $f$ on differentioituva pisteessä $(x_0,y_0)$
(vrt.\ Luku \ref{gradientti}). Lauseen \ref{Taylor-R2} jälkimmäistä väittämää voidaan käyttää
hyväksi myös määrättäessa $f$:n Taylorin polynomia: Väittämän mukaan riittää löytää mikä
tahansa polynomi $p(x,y)$, jolle pätee
\[
f(x,y)-p(x,y)=\ord{h^{n}},\quad h=\sqrt{(x-x_0)^2+(y-y_0)^2}.
\]
Tällainen polynomi on välttämättä yksikäsitteinen --- siis $\ p(x,y)=T_n(x,y,x_0,y_0)$
(vrt.\ Propositio \ref{Taylor-prop}).
\jatko \begin{Exa} (jatko) Esimerkin funktiolle pätee
\begin{align*}
f(x,y)&=1-x+\frac{1}{2}x^2+\Ord{\abs{x}^3}+xy+y^2
                          +(1-\frac{1}{2}x^2)(1-\frac{1}{2}y^2)+\ordoO{x^4+y^4} \\
&= 2-x+xy+\frac{1}{2}y^2+\Ord{\abs{x}^3+\abs{y}^3} \\
&= 2-x+xy+\frac{1}{2}y^2+\Ord{h^3},
\end{align*}
joten myös tällä perusteella on
\[
T_2(x,y,0,0)=2-x+xy+\frac{1}{2}y^2. \loppu
\]
\end{Exa}

\subsection*{Monen muuttujan Taylorin polynomit}

Em.\ tarkastelut yleistyvät helposti kolmen ja useamman muuttujan funktioihin. Kolmen muuttujan
funktion $f=f(x,y,z)$ Taylorin polynomi määritellään
\begin{align*}
T_n(x,y,z,x_0,y_0,z_0) 
    &= \sum_{k=0}^n \frac{1}{k!}(h_x\partial_x+h_y\partial_y+h_z\partial_z)^kf(x_0,y_0,z_0), \\
h_x &= x-x_0,\quad h_y=y-y_0,\quad h_z=z-z_0.
\end{align*}
Yleisemmin jos $f=f(\mx)=f(x_1,\ldots,x_d)$, niin $f$:n Taylorin polynomi astetta $n$ pisteessä 
$\ma=(a_1,\ldots,a_d)\in\R^d$ on
\[ \boxed{
\quad T_n(\mx,\ma)
  =\sum_{k=0}^n\frac{1}{k!}\bigl[\bigl(\sum_{i=1}^d h_i\partial_i\bigr)^kf\bigr](\ma),\quad
                               h_i=x_i-a_i,\quad i=1\ldots d. \quad}
\]
Mainittakoon ilman tarkempia perusteluja, että tämä lauseke on purettavissa seuraavaan muotoon,
joka ei ulkonäöltään juuri poikkea yhden muuttujan tilanteesta
(vrt.\ myös tapaus $d=2$ edellä):
\[ \boxed{ \begin{aligned}
\quad &T_n(\mx,\ma)=\sum_{\abs{\alpha}\le n}\frac{\ygehys 1}{\alpha !}\,
                             \partial^\alpha f(\ma)\,(\mx-\ma)^\alpha, \quad \\
      &\alpha ! = \prod_{i=1}^d \alpha_i\,!\,, \quad 
                  \mx^\alpha = \prod_{i=1}^d x_i^{\alpha_i}, \\
      &\alpha   = (\alpha_1, \ldots, \alpha_d), \quad \mx = (x_1, \ldots, x_d). \akehys
\end{aligned} } \]
Kaavassa summaus käy yli kaikkien erilaisten moni-indeksien 
(eli järjestettyjen indeksijoukkojen), joille $\abs{\alpha} = \alpha_1 + \ldots \alpha_d \le n$.

Taylorin polynomi  $p(\mx)=T_n(\mx,\ma)$ määräytyy aina myös yksikäsitteisesti polynomina 
astetta $n$, joka toteuttaa ehdot
\[
\partial^\alpha p(\ma)=\partial^\alpha f(\ma),\quad \abs{\alpha}\leq n.
\]
Sikäli kuin $f$ ja $f$:n osittaisderivaatat kertalukuun $n$ asti ovat jatkuvia pisteen $\ma$
ympäristössä, niin riittää myös löytää (keinolla millä hyvänsä) polynomi $p$ astetta $n$,
jolle pätee
\[ 
f(\mx) = p(\mx) + \ord{\abs{\mx-\ma}^n}.
\]
Tällöin on $p(\mx)=T_n(\mx,\ma)$.

\subsection*{Polynomi $T_2(\mx,\ma)$ -- Hessen matriisi}

Monen muuttujan Taylorin polynomeja lasketaan harvemmin astelukua $n=2$ pidemmälle.
Em.\ kaavan mukaan $f$:n toisen asteen Taylorin polynomi on
\[
T_2(\mx,\ma) 
= f(\ma)+\sum_{i=1}^d \partial_i f(\ma)(x_i-a_i)
        + \frac{1}{2} \sum_{i=1}^d\sum_{j=1}^d \partial_i\partial_j f(\ma)(x_i-a_i)(x_j-a_j).
\]
Tämän voi esittää matriisimuodossa
\[
T_2(\mx,\ma)=f(\ma)+[\Nabla f(\ma)]^T(\mx-\ma)+\frac{1}{2}(\mx-\ma)^T\mH(\ma)(\mx-\ma),
\]
missä $\Nabla f(\ma)$ on gradientti ja $\mH(\ma)$ on $f$:n toisen kertaluvun
osittaisderivaatoista koostuva nk.\
\index{Hessen matriisi}%
\kor{Hessen matriisi} (engl.\ Hessian)\,:
\[
[\,\mH(\ma)\,]_{ij}=\frac{\partial^2 f}{\partial x_i\partial x_j}(\ma),\quad i,j=1\ldots d.
\]
\begin{Exa}
Määrää funktion
\[
f(x_1,\ldots,x_d)=\prod_{i=1}^d e^{c_i x_i}
\]
Taylorin polynomi $T_2(\mx,\mo)$.
\end{Exa}
\ratk Koska
\[
f(\mo)=1,\quad \frac{\partial f}{\partial x_i}(\mo)=c_i\,,\quad i=1\ldots d, \quad
\frac{\partial^2 f}{\partial x_i\partial x_j} = c_i c_j\,,\quad i,j=1\ldots d,
\]
niin
\begin{align*}
T_2(\mx,\mo) &= 1+\left(\sum_{i=1}^d x_i\,
                \frac{\partial}{\partial x_i}\right)f(\mo)+\frac{1}{2}\left(\sum_{i=1}^d 
                         x_i\,\frac{\partial}{\partial x_i}\right)^2f(\mo) \\
             &= 1 + \sum_{i=1}^d c_i x_i+\frac{1}{2}\sum_{i,j=1}^d c_i c_j x_i x_j\,.
\end{align*}
Kun merkitään $\mc=[c_1,\ldots,c_d]^T$ ja $\mH=(c_ic_j)=\mc\mc^T$ 
($=f$:n Hessen matriisi origossa), niin tulos saadaan muotoon
\begin{align*}
T_2(\mx,\mo) \,&=\, 1 + \mc^T\mx + \frac{1}{2} \mx^T\mH\mx \\
             \,&=\, 1 + \mc^T\mx + \frac{1}{2} (\mc^T\mx)^2.
\end{align*}
Lyhyempää tietä samaan tulokseen tullaan, kun huomataan käyttää reaalifunktion
$\exp(x)$ ominaisuuksia:
\begin{align*}
f(\mx) = \exp({\mc^T\mx}) &= 1+\mc^T\mx+\frac{1}{2}(\mc^T\mx)^2+\Ord{\abs{\mc^T\mx}^3} \\
                          &= 1+\mc^T\mx+\frac{1}{2}(\mc^T\mx)^2+\Ord{\abs{\mx}^3}. \loppu
\end{align*}

\Harj
\begin{enumerate}

\item
Laske seuraavien funktioiden Taylorin polynomit annettua astetta $n$ annetussa pisteessä.
Käytä tilaisuuden tullen apuna yhden muuttujan Taylorin polynomeja.
\vspace{3mm} \newline
a) \ $\D f(x,y)=2x^4-5y^3+2xy^2,\,\ (0,0),\,\ n=3$ \vspace{3mm}\newline
b) \ $\D f(x,y)=\sqrt{x^3y}\,,\,\ (1,4),\,\ n=4$ \vspace{1.5mm}\newline
c) \ $\D f(x,y)=\frac{1}{2+x-2y}\,,\,\ (2,1),\,\ n=3$ \vspace{1.5mm}\newline
d) \ $\D f(x,y)=y^2\ln x,\,\ (1,0),\,\ n=4$ \vspace{3mm}\newline
e) \ $\D f(x,y)=\sin(xy),\,\ (0,0),\,\ n\in\N$ \vspace{3mm}\newline
f) \ $\D f(x,y)=e^{xy}\sin y,\,\ (0,0),\,\ n=3$ \vspace{3mm}\newline
g) \ $\D f(x,y)=\cos(x+\sin y),\,\ (0,0),\,\ n=4$ \vspace{1.5mm}\newline
h) \ $\D f(x,y)=\int_0^{x+y^2-1} e^{-t^2}\,dt+\int_0^{\pi/2} \cos(xt^2+yt)\,dt,\,\ 
                                             (0,1),\,\ n=2$ \vspace{1.5mm}\newline
i) \ $\D x^2+2y^2+3z^2=6\ \ekv\ z=f(x,y),\,\ (1,1,1),\,\ n=2$ \vspace{3mm}\newline
j) \ $\D x+xy^2+2z-\sin z=0\ \ekv\ z=f(x,y),\,\ (0,0,0),\,\ n=2$ \vspace{3mm}\newline
k) \ $\D f(x,y,z)=e^{xyz},\,\ (0,0,0),\,\ n\in\N$ \vspace{1.5mm}\newline
l) \ $\D f(x,y,z)=\frac{1}{x+y+z}\,,\,\ (1,0,0),\,\ n=3$ \vspace{1.5mm}\newline
m) \ $\D f(\mx)=\cos(x_1 + \ldots + x_n),\,\ \mo,\,\ n=2$ \vspace{1.5mm}\newline
n) \ $\D f(\mx)=\frac{1}{x_1 + \ldots + x_n}\,,\,\ (1,\ldots,1),\,\ n=2$

\item
Laske funktion $f(x,y)=\ln(1+x^2+y^2)$ osittaisderivaatta $f_{xxxxxxyy}(0,0)$ Taylorin
polynomien avulla.

\item
Arvioi, kuinka suuri on seuraavien approksimaatioiden virhe enintään kiekossa
$A:\,x^2+y^2 \le a^2\ (a>0)$.
\begin{align*}
&\text{a)}\ \ e^{x+2y} \approx 1+x+2y \qquad\quad\,
 \text{b)}\ \ e^{x+2y} \approx 1+x+2y+\frac{1}{2}\,x^2+2xy+2y^2 \\
&\text{b)}\ \ \cos(xy)\approx 1-\frac{1}{2}\,x^2y^2 \qquad
 \text{d)}\ \ 2xy+2\cos(x-y) \approx 2+x^2+y^2
\end{align*}

\item
Määritä funktion toisen asteen Taylorin polynomi ja sen avulla Hessen matriisi
origossa:
\begin{align*}
&\text{a)}\ \ x\sin(x+y) \qquad
 \text{b)}\ \ xy\cos(x-y) \qquad\quad
 \text{c)}\ \ e^{x+y}+e^{x-y} \\
&\text{d)}\ \ \frac{1-x+x^2}{1+y-y^2} \qquad\
 \text{e)}\ \ \cos(x+y-2z) \qquad
 \text{f)}\ \ \frac{e^{xy}+e^{yz}+e^{xz}}{1+x+y+z}
\end{align*}

\item
Yhtälö $x^y = y^x\ \ekv\ y \ln x = x \ln y$ määrittelee käyrän, mahdollisesti useampia, pisteen
$(a, a)$ $(a > 0)$ ympäristössä. Tutki asiaa approksimoimalla funktiota 
$f(x, y) =  y \ln x - x \ln y$ ensimmäisen, toisen ja kolmannen asteen Taylorin polynomilla 
pisteessä $(a, a)$.

\end{enumerate}