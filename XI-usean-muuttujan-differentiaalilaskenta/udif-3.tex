\section{Gradientti} \label{gradientti}
\alku
\index{gradientti|vahv}
\index{differentiaalioperaattori!d@gradientti (nabla $\nabla$)|vahv}

Palautettakoon mieliin Luvusta \ref{differentiaali}, että jos yhden muuttujan funktio $f$ on
derivoituva pisteessä $x$, niin pätee
\[
f(x+\Delta x)=f(x)+df(x,\Delta x) + o(\abs{\Delta x}),
\]
missä $df(x,\Delta x)= f'(x)\Delta x$ on $f$:n differentiaali. Vastaava tulos kahden muuttujan
funktiolle $f(x,y)$ on kehitelmä muotoa
\[
f(x+\Delta x,y+\Delta y)=f(x,y)+df(x,y,\Delta x,\Delta y)+\ord{h}, \quad 
                       h=\sqrt{(\Delta x)^2+(\Delta y)^2}.
\]
Sikäli kuin tämä on pätevä jossakin pisteen $(x,y)$ ympäristössä, sanotaan jälleen, että
$df(x,y,\Delta x,\Delta y)$ on $f$:n
\index{differentiaali}%
\kor{differentiaali} pisteessä $(x,y)$. Differentiaalin
lausekkeen johtamiseksi oletetaan, kuten ketjusäännössä edellä (Lause \ref{ketjusääntö}), että
$f$ on jatkuva pisteen $(x,y)$ ympäristössä $U_\delta(x,y)$ ja osittaisderivaatat $f_x$ ja $f_y$
ovat olemassa ko.\ ympäristössä  ja jatkuvia pisteessä $(x,y)$. Tällöin pätee
(vrt.\ Lauseen \ref{ketjusääntö} todistus) 
\begin{align*}
f(x+\Delta x,&y+\Delta y)-f(x,y) \\
             &= [f(x+\Delta x,y+\Delta y)-f(x,y+\Delta y)] + [f(x,y+\Delta y)-f(x,y)] \\
             &= f_x(\xi,y+\Delta y)\Delta x+f_y(x,\eta)\Delta y \\
             &= f_x(x,y)\Delta x + f_y(x,y)\Delta y + r(x,y,\Delta x,\Delta y),
\end{align*}
missä
\[
r(x,y,\Delta x,\Delta y) = [f_x(\xi,y+\Delta y)-f_x(x,y)]\Delta x 
                         + [f_y(x,\eta)-f_y(x,y)]\Delta y.
\]
Koska tässä on $\abs{\xi-x}\le\abs{\Delta x}$ ja $\abs{\eta-y}\le\abs{\Delta y}$ ja $f_x$ ja
$f_y$ ovat jatkuvia pisteessä $(x,y)$, niin $f_x(\xi,y+\Delta y)-f_x(x,y)=\ord{1}$ ja
$f_y(x,\eta)-f_y(x,y)=\ord{1}$, kun $h \kohti 0$. Siis 
(vrt.\ suuruusluokka-algebran säännöt Luvussa \ref{taylorin polynomien sovelluksia})
\[
r(x,y,\Delta x,\Delta y) = \ord{1}\Delta x + \ord{1}\Delta y = \ord{h}, \quad
                                           h=\sqrt{(\Delta x)^2+(\Delta y)^2}.
\]
On päätelty, että pätee kehitelmä
\begin{equation} \label{grad-2}
\boxed{\kehys\quad 
f(x+\Delta x,y+\Delta y)=f(x,y)+f_x(x,y)\Delta x+f_y(x,y)\Delta y+o(h). \quad}
\end{equation}
Etsitty differentiaalin lauseke (tehdyin oletuksin) on näin muodoin
\[ 
df(x,y,\Delta x,\Delta y)) = f_x(x,y)\Delta x + f_y(x,y)\Delta y.  
\]
\begin{Exa} \label{grad-ex1} Arvioi $f(1.01,1.98)$ differentiaalin
$df(1,2,0.01,-0.02)$ avulla, kun $f(x,y)=xy-2y^3$.
\end{Exa}
\ratk Tässä on $f_x(x,y) = y$ ja $f_y(x,y)=x-6y^2$, joten
\begin{align*}
f(1.01,1.98) &= f(1+0.01,2-0.02) \\
             &\approx f(1,2) + f_x(1,2)\cdot 0.01 + f_y(1,2)\cdot(-0.02) \\
             &= -14 + 2\cdot 0.01 + (-23)\cdot (-0.02) = -13.52.
\end{align*}
Tarkka arvo on $f(1.01,1.98)=-13.524984$. \loppu

Kun differentiaalikehitelmässä \eqref{grad-2} otetaan käyttöön vektorimerkintä
\[ 
\Delta \vec r = \Delta x\,\vec i + \Delta y\,\vec j, 
\]
ja käytetään myös muuttujien $x,y$ tilalla vektorimerkintää $\vec r = x\vec i + y\vec j$, niin
kehitelmä voidaan kirjoittaa muodossa
\begin{equation} \label{grad-3}
f(\vec r + \Delta \vec r) 
         = f(\vec r) + \nabla f(\vec r)\cdot\Delta\vec r + o(\abs{\Delta\vec r}),
\end{equation}
missä $\nabla f$ on $f$:n \kor{gradientti}, joka määritellään
\[
\boxed{\kehys\quad \nabla f(x,y)=f_x(x,y)\vec i+f_y(x,y)\vec j. \quad}
\]
Myös kolmen muuttujan funktiolle on johdettavissa differentiaalikehitelmä muotoa \eqref{grad-3}
samalla tavoin kuin edellä. Tällöin on 
$\Delta\vec r = \Delta x\,\vec i + \Delta y\,\vec j + \Delta z\,\vec k$, ja gradientti 
määritellään
\[
\boxed{\kehys\quad \nabla f(x,y,z)=f_x(x,y,z)\vec i+f_y(x,y,z)\vec j+f_z(x,y,x)\vec k. \quad}
\]

Ym.\ tulosten perusteella gradienttia voi pitää 'yleistettynä derivaattana'. Yhden muuttujan
derivaattaoperaattoria $\dif=d/dx$ vastaa siis kahden ja kolmen muuttujan tapauksessa 
vektorimuotoinen differentiaalioperaattori
\[
\nabla=\begin{cases}
\vec i\,\partial_x+\vec j\,\partial_y &(d=2), \\
\vec i\,\partial_x+\vec j\,\partial_y+\vec k\,\partial_z &(d=3).
\end{cases}
\]
Gradientille käytetään myös symbolia 'grad'. Lukutapoja ovat 'gradientti', 'grad' ja 'nabla'.
\begin{Exa} Funktion $f(x,y,z)=x^2+3y^2+xyz$ gradientti on
\begin{align*}
\nabla f &= (\vec i\,\partial_x + \vec j\,\partial_y + \vec k\,\partial_z)(x^2+3y^2+xyz) \\
         &= (2x+yz)\vec i +(6y+xz)\vec j+xy\vec k. \loppu
\end{align*}
\end{Exa}

\subsection*{Monen muuttujan gradientti}
\index{gradientti!a@monen muuttujan|vahv}

Edellä johdetut differentiaalikehitelmät ovat helposti yleistettävissä koskemaan yleisempää
$n$ muuttujan funktioita. Tällöin on kätevintä siirtyä matriisialgebran merkintätapoihin, eli
indeksoidaan muuttujat,
\[
x \ext x_1,\quad y \ext x_2,\quad \ldots,
\]
ja kirjoitetaan $\vec r\,$:n paikalle pystyvektori $\mx\,$:
\[
\vec r \ext \begin{bmatrix} x_1 \\ x_2 \\ \vdots \\ x_n \end{bmatrix}=\mx,\quad 
\Delta\vec r \ext \Delta\mx.
\]
Näillä merkinnöillä differentiaalikehitelmä \eqref{grad-3} saa muodon
\[
f(\mx+\Delta\mx)
     =f(\mx)+\sum_{k=1}^n \frac{\partial f}{\partial x_k}(\mx)\Delta x_k+o(\abs{\Delta\mx}),
\]
missä $\abs{\cdot}$ tarkoittaa $\R^n$:n euklidista normia. Tulos voidaan kirjoittaa 
matriisialgebran kielellä muotoon
\begin{equation} \label{grad-4}
\boxed{\begin{aligned}
\ykehys\quad f(\mx+\Delta\mx)&=f(\mx)+\scp{\Nabla f(\mx)}{\Delta\mx}
                              +o(\abs{\Delta\mx}), \\[2mm]
             \Nabla f(\mx)   &=\left[\frac{\partial f}{\partial x_1}(\mx),
                                     \frac{\partial f}{\partial x_2}(\mx),\ldots,
                                     \frac{\partial f}{\partial x_n}(\mx)\right]^T.\Akehys\quad
       \end{aligned} }
\end{equation}

Kehitelmässä \eqref{grad-4} on termi
$df(\mx,\Delta\mx) = \scp{\Nabla f(\mx)}{\Delta\mx} = (\Nabla f(\mx))^T\Delta\mx$ 
\index{differentiaali}%
jälleen \kor{differentiaali}. Muuttujien lukumäärästä riippumatta voi differentiaalia käyttää 
funktion arvon muutosten arviointiin samaan tapaan kuin Esimerkissä \ref{grad-ex1} edellä.
\begin{Exa} Ristikkorakenteen (ks.\ Luku \ref{yhtälöryhmät}) kuormitus aiheuttaa yksittäisen 
sauvan päätepisteiden $P$ ja $Q$ siirtymät
\begin{align*}
&\vec u = \overrightarrow{PP'}=u_1\vec i+u_2\vec j+u_3\vec k, \\
&\vec v = \overrightarrow{QQ'}=v_1\vec i + v_2\vec j+v_3\vec k.
\end{align*}
Olettaen, että $\abs{\vec u},\abs{\vec v}\ll\abs{\vec a}$, $\vec a=\overrightarrow{QP}$,
määritä sauvan venymä likimäärin differentiaalin avulla.
\end{Exa}
\ratk Venymä on kuuden muuttujan funktio
\[
f(u_1,u_2,u_3,v_1,v_2,v_3) = \abs{\vec a+\vec u-\vec v}-\abs{\vec a}.
\]
Tässä on
\begin{align*}
\abs{\vec a+\vec u-\vec v\,}^2\,
             &=\,(\vec a+\vec u-\vec v)\cdot (\vec a+\vec u-\vec v) \\
             &=\,\abs{\vec a\,}^2+2\vec a\cdot(\vec u-\vec v)+\abs{\vec u-\vec v\,}^2,
\end{align*}
joten
\[
f(u_1,\ldots,v_3) 
 =\left[\abs{\vec a}^2+\sum_{k=1}^3 2a_k(u_k-v_k)+\sum_{k=1}^3 (u_k-v_k)^2\right]^{1/2}
                                                 -\ \abs{\vec a}.
\]
Tällöin on $f(\mo)=0$ ja
\[
\frac{\partial f}{\partial u_k}(\mo)
              =\frac{a_k}{\abs{\vec a}}\,,\quad \frac{\partial f}{\partial v_k}(\mo)
              =-\frac{a_k}{\abs{\vec a}}\,, \quad k=1,2,3,
\]
joten differentiaalikehitelmän \eqref{grad-4} mukaan
\[ 
f(u_1,\ldots,v_3) \approx \sum_{k=1}^3 \frac{a_k}{\abs{\vec a}}\,(u_k-v_k) 
                        = \underline{\underline{\vec t\cdot (\vec u-\vec v)}},
\]
missä $\vec t=\vec a/\abs{\vec a\,}$ on $\vec a$:n suuntainen yksikkövektori. \loppu

\subsection*{Differentioituvuus}
\index{gradientti!b@differentioituvuus|vahv}

Kehitelmä \eqref{grad-4} on pätevä olettaen, että $f$ on jatkuva $\mx$:n ympäristössä
\[ 
U_\delta(\mx) = \{\mx' \in \R^n \mid |\mx'-\mx|<\delta\}, \quad \delta>0 
\]
ja lisäksi osittaisderivaatat $\partial f/\partial x_k,\ k=1 \ldots n\,$ ovat olemassa ko.\
ympäristössä ja jatkuvia pisteessä $\mx$. Tällaisista taustaoletuksista päästään eroon, kun
asetetaan seuraava yleisempi \kor{differentioituvuuden} ja samalla gradientin määritelmä.
\begin{Def} \label{differentioituvuus} \index{differentioituvuus|emph}
Funktio $f:D_f\kohti\R$, $D_f\subset\R^n$, on \kor{differentioituva} pisteessä $\mx\in D_f$, 
jos $f$ on määritelty pisteen $\mx$ jossakin ympäristössä ja on olemassa pystyvektori 
$\mathbf{h}\in\R^n$ siten, että pätee 
\[
f(\mx+\Delta\mx)= f(\mx)+\mathbf{h}^T\Delta\mx+o(\abs{\Delta\mx}), \quad 
                      \text{kun}\,\ \abs{\Delta\mx}\kohti 0.
\]
Tällöin $\mathbf{h}$ on $f$:n \kor{gradientti} pisteessä $\mx$, merkitään
$\mathbf{h}=\Nabla f(\mx)$.
\end{Def}
Jos $f$ on differentioituva, niin gradientti on laskettavissa osittaisderivaattojen avulla
kaavan \eqref{grad-4} mukaisesti:
\begin{Prop} Jos $f$ on differentioituva pisteessä $\mx \in \R^n$, niin osittaisderivaatat
$\partial f/\partial x_k,\ k=1\ldots n\,$ ovat olemassa pisteesä $\mx$, ja pätee
$[\Nabla f(\mx)]_k=\partial f(\mx)/\partial x_k,\ k=1 \ldots n$.
\end{Prop}
\tod Kun kehitelmässä \eqref{grad-4} asetetaan $\Delta\mx=\Delta x_k\me_k$ 
($\me_k=$ euklidinen yksikkövektori), niin kehitelmä voidaan kirjoittaa muotoon
\[
f(x_1,\ldots,x_{k-1},x_k+\Delta x_k,x_{k+1},\ldots,x_n) - f(x_1,\ldots,x_n) 
                  = [\Nabla f(\mx)]_k\Delta x_k+o(\abs{\Delta x_k}).
\]
Jakamalla $(\Delta x_k)$:lla ja antamalla $\Delta x_k\kohti 0$ seuraa
$[\Nabla f(\mx)]_k=\partial f(\mx)/\partial x_k$. \loppu

Yhteenvetona edellä esitetystä voidaan todeta, että funktion jatkuvuus pisteen $\mx$
ympäristössä, osittaisderivaattojen olemassaolo ko.\ ympäristössä ja jatkuvuus pisteessä $\mx$
rittävät yhdessä takaamaan differentioituvuuden; toisaalta differentioituvuus pisteessä $\mx$
takaa vain osittaisderivaattojen olemassaolon kyseisessä pisteessä, ei muualla. Vertaamalla
edellisen luvun tarkasteluihin nähdäänkin nyt, että differentioituvuus on usean muuttujan
derivoituvuuden luontevin määritelmä. Nimittäin ensinnäkin kehitelmän \eqref{grad-4}
perusteella on ilmeistä, että differentioituvuudesta seuraa välittömästi jatkuvuus ko.\
pisteessä, kuten yhden muuttujan tapauksessa (vrt.\ Luku \ref{derivaatta})). Toiseksi nähdään,
että myös osittaisderivoinnin ketjusääntöjä perusteltaessa voidaan ulkofunktiota koskevat
jatkuvuus- ym. oletukset korvata differentioituvuudella. Nimittäin jos tarkastellaan Lauseen
\ref{ketjusääntö} funktiota $F(t)=f(x(t),y(t))$, merkitään $x(t)=x$, $y(t)=y$,
$x(t+\Delta t)=x+\Delta x$, $y(t+\Delta t)=y+\Delta y$ ja oletetaan, että $f$ on 
differentioituva pisteessä $(x,y)$, niin kehitelmän \eqref{grad-2} perusteella
\[
\frac{F(t+\Delta t)-F(t)}{\Delta t} = f_x(x,y)\frac{\Delta x}{\Delta t}
                                    + f_y(x,y)\frac{\Delta y}{\Delta t} 
                                    + (\Delta t)^{-1}\ord{h}.
\]
Jos edelleen derivaatat $x'(t)$ ja $y'(t)$ ovat olemassa, niin tässä on
\[
h=\sqrt{(\Delta x)^2+(\Delta y)^2} = \sqrt{[x'(t)]^2+[y'(t)]^2}\Delta t + \ord{\abs{\Delta t}},
\]
joten
\[
(\Delta t)^{-1}\ord{h} = (\Delta t)^{-1}\ord{\abs{\Delta t}} \kohti 0, \quad 
                                                           \text{kun}\ \Delta t \kohti 0.
\]
Näin muodoin seuraa ketjusääntö $F'(t)=f_x(x,y)x'(t)+f_y(x,y)y'(t)$.

Yhden muuttujan derivoituvuuden ja useamman muuttujan differentioituvuuden vastaavuudesta 
kertoo myös seuraava Differentiaalilaskun väliarvolauseen yleistys. Todistus jätetään
harjoitustehtäväksi (Harj.teht.\,\ref{H-udif-3: differentioituvuus}c).
\begin{Lause} (\vahv{$\R^n$:n väliarvolause}) \label{väliarvolause-Rn}
\index{vzy@väliarvolauseet!d@$\R^n$:n|emph}
Olkoon $\ma,\mb\in\R^n,\ \ma\neq\mb$ ja olkoon $f(\mx)$ jatkuva pisteessä $\mx(t)=t\ma+(1-t)\mb$ 
jokaisella $t\in[0,1]$ ja differentioituva pisteessä $\mx(t)$ jokaisella $t\in(0,1)$. Tällöin
jollakin $\xi\in(0,1)$ pätee 
\[
f(\mb)-f(\ma)=\scp{\Nabla f\bigl(\mx(\xi)\bigr)}{\mb-\ma}.
\]
\end{Lause}

\subsection*{Suunnatut derivaatat}
\index{osittaisderivaatta!f@suunnattu derivaatta|vahv}
\index{gradientti!c@suunnattu derivaatta|vahv}
\index{suunnattu derivaatta|vahv}

Gradienttia käytetään usean muuttujan funktioita tutkittaessa samaan tapaan kuin derivaattaa
yhden muuttujan tapauksessa. Gradientin avulla voidaan erityisesti laskea, millä tavoin
funktion arvot muuttuvat lähellä annettua pistettä $\mx$, kun ko.\ pisteestä siirrytään
jonkin $\R^n$:n yksikkövektorin osoittamaan suuntaan. Nimittäin jos $\me$ on tällainen vektori,
eli
\[ 
\me = (e_i), \quad \abs{\me}^2 = \sum_{i=1}^n e_i^2 = 1, 
\]
niin asettamalla $\Delta\mx = \Delta t\,\me$ kehitelmässä \eqref{grad-4} saadaan tulos
\begin{equation} \label{grad-6}
f(\mx+\Delta t\,\me)=f(\mx)+\partial_{\me}f(\mx)\Delta t+\ord{\abs{\Delta t}},
\end{equation}
missä 
\[
\boxed{\quad \partial_{\me}f(\mx)
          =\sum_{k=1}^n e_k\frac{\partial f}{\partial x_k}(\mx)=\scp{\Nabla f(\mx)}{\me}. \quad}
\]
Näin määritelty $\partial_{\me}f$ voidaan nimetä $f$:n \kor{suunnatuksi derivaataksi} suuntaan
$\me$, sillä määritelmän \eqref{grad-6} mukaan on
\[
\partial_{\me}f(\mx)=\lim_{\Delta t\kohti 0}\frac{f(\mx+\Delta t\,\me)-f(\mx)}{\Delta t} 
                    = g'_{\me}(0), \quad g_{\me}(t) = f(\mx + t\me). 
\]
Erityisesti jos valitaan $\me=\me_k$ (euklidinen yksikkövektori), niin 
$\partial_{\me}f = \partial f/\partial x_k$. Siis osittaisderivaatat ovat suunnattujen 
derivaattojen erikoistapauksia.

Kahden tai kolmen muuttujan tapauksessa voidaan kehitelmä \eqref{grad-6} kirjoittaa haluttaessa
muotoon
\[
f(\vec r+\Delta t\,\vec e\,)=f(\vec r\,)+\vec e\cdot\nabla f(\vec r\,)\Delta t+\ord{\abs{\Delta t}},
\]
jolloin suunnattu derivaatta on differentiaalioperaattori
\[
\partial_e = \vec e\cdot\nabla=\begin{cases}
\,e_x\partial_x+e_y\partial y &(d=2), \\
\,e_x\partial_x+e_y\partial y+e_z\partial z &(d=3). 
\end{cases}
\]

Jos monen muuttujan tapauksessa on $\Nabla f(\mx) \neq \boldsymbol{0}$, niin suunnaksi $\me$ 
voidaan valita $\me = \pm [\Nabla f(\mx)]/\abs{\Nabla f(\mx)}$, jolloin on vastaavasti 
$\partial_{\me}f(\mx) = \pm \abs{\Nabla f(\mx)}$. Koska toisaalta Cauchyn-Schwarzin epäyhtälön
perusteella on
\[ 
\abs{\partial_{\me}f(\mx)} =   \left|\,\sum_{k=1}^n e_k \frac{\partial f}{x_k}\,\right| 
                           \le \abs{\me}\abs{\Nabla f(\mx)} = \abs{\Nabla f(\mx)}, 
\]
niin on päätelty, että suunnatun derivaatan suurin arvo $=\abs{\Nabla f(\mx)}$ gradientin
suuntaan ja pienin arvo $=-\abs{\Nabla f(\mx)}$ vastakkaiseen suuntaan. Siis\,:
\[ \boxed{\begin{aligned}
\quad &\text{Funktio kasvaa voimakkaimmin gradientin suuntaan, vähenee}\quad\ykehys \\
      &\text{voimakkaimmin negatiivisen gradientin suuntaan, ja muuttuu} \\
      &\text{vähiten suuntiin $\me$, joille $\scp{\Nabla f}{\me}=0$}.\akehys
\end{aligned} } \] 
\begin{Exa} Laske funktion $f(x,y,z)=x^3-2x^2yz+4yz^2-2x-y+z$ derivaatta suuntaan 
$\vec e = (\vec i - 2\vec j - \vec k)/\sqrt{6}$ pisteessä $(1,-1,2)$ sekä määritä suunta
$\vec n$, johon funktio kasvaa nopeimmin ko.\ pisteessä.
\end{Exa}
\ratk Gradientti on
\begin{align*}
&\nabla f(x,y,z) = (3x^2-4xyz-2)\vec i + (-2x^2z+4z^2-1)\vec j + (-2x^2y+8yz+1)\vec k \\
&\impl\quad \nabla f(1,-1,2) = 9\vec i + 11\vec j - 13\vec k.
\end{align*}
Derivaatta suuntaan $\vec e$ on
\[ \vec e\cdot\nabla f(1,-1,2) = \frac{1}{\sqrt{6}}\,[1 \cdot 9 -2 \cdot 11 - 1 \cdot (-13)] 
                               = \underline{\underline{0}}. 
\]
Nopeimman kasvun suunta on
\begin{align*}
\vec n = \frac{\nabla f(1,-1,2)}{\abs{\nabla f(1,-1,2)}} 
      &= \frac{1}{\sqrt{371}}\,(9\vec i + 11\vec j -13\vec k) \\
      &\approx \underline{\underline{0.47\vec i + 0.57\vec j - 0.67\vec k}}. \loppu
\end{align*}
\begin{Exa} Kolmion kärjet ovat pisteissä $P_1=(1,0)$, $P_2=(0,2)$ ja $P_3=(2,3)$. Kolmion 
kutakin kärkeä liikutetaan sama (pieni) matka $\Delta s$ siten, että $P_1$ siirtyy suuntaan 
$\vec j$, $P_2$ suuntaan $-\vec i$ ja $P_3$ suuntaan $\vec e$. Miten suunta $\vec e$ on 
valittava, jotta kolmion pinta-ala \ a) muuttuisi mahdollisimman vähän, \ 
b) kasvaisi mahdollisimman paljon, \ c) pienenisi mahdollisimman paljon\,? 
\end{Exa}
\ratk Jos kolmion kärjet muutoksen jälkeen ovat $P_1=(x_1,y_1)$, $P_2=(x_2,y_2)$ ja 
$P_3=(x_3,y_3)$, niin pinta-ala on (vrt.\ Luku \ref{ristitulo})
\begin{align*}
f(\mx)=f(x_1,y_1,x_2,y_2,x_3,y_3) 
    &= \frac{1}{2}\begin{vmatrix} x_3-x_1 & y_3-y_1 \\ x_2-x_1 & y_2-y_1 \end{vmatrix} \\ 
    &= \frac{1}{2}\,(x_3-x_1)(y_2-y_1)-\frac{1}{2}\,(x_2-x_1)(y_3-y_1).
\end{align*}
Jos $\,\vec e = \cos\varphi\vec i + \sin\varphi\vec j$, niin kärkien siirtymää kuvaa vektori
\[
\Delta \mx = \Delta s\,[0,1,-1,0,\cos\varphi,\sin\varphi]^T.
\]
Funktion $f$ gradientti on
\begin{align*}
           &\Nabla f(\mx) = \frac{1}{2}\,[y_3-y_2,x_2-x_3,y_1-y_3,x_3-x_1,y_2-y_1,x_1-x_2]^T \\
\impl\quad &\Nabla f(1,0,0,2,2,3) = \frac{1}{2}\,[1,-2,-3,1,2,1]^T.
\end{align*}
Differentiaalin avulla arvioitu pinta-alan muutos on siis
\[
\Delta f \approx \scp{\Nabla f(1,0,0,2,2,3)}{\Delta\mx} 
         = \frac{\Delta s}{2}\,(1+2\cos\varphi+\sin\varphi).
\]
Pinta-ala muuttuu vähiten, kun $1+2\cos\varphi+\sin\varphi=0$, eli kun joko $\vec e=-\vec j$
($\varphi=3\pi/2$) tai $\vec e=(-4\vec i+3\vec j\,)/5$. Pinta-ala muuttuu eniten, kun 
\[
D_\varphi(1+2\cos\varphi+\sin\varphi)=-2\sin\varphi+\cos\varphi=0,
\]
eli kun $\vec e=\pm(2\vec i + \vec j\,)/\sqrt{5}$. Näihin suuntiin on vastaavasti
$\Delta f \approx \frac{\Delta s}{2}(1\pm\sqrt{5})$.

Vastaus:
\[
\text{a)}\,\ \vec e = -\vec j\,\ \text{tai}\,\ \vec e = \frac{1}{5}(-4\vec i+3\vec j), \,\
\text{b)}\,\ \vec e = \frac{1}{\sqrt{5}}(2\vec i + \vec j), \,\
\text{c)}\,\ \vec e = -\frac{1}{\sqrt{5}}(2\vec i + \vec j). \loppu
\]

\subsection*{Tasa-arvopinnat}
\index{tasa-arvopinta|vahv}

Samaan tapaan kuin derivaatan avulla voidaan määrätä käyrän tangentti ja normaali 
(vrt.\ Luku \ref{derivaatta geometriassa}), voidaan osittaisderivaattojen avulla määrätä 
\index{tangenttitaso} \index{normaali(vektori)!c@pinnan}%
\kor{pinnan tangenttitaso} ja tämän tason normaali, eli \kor{pinnan normaali}
(suora tai vektori). Jos pinnan $S\subset\Ekolme$ yhtälö on 
\[ 
S: \ F(x,y,z)=0, 
\]
niin pinnan normaali (ja normaalin avulla tangenttitaso) annetussa pisteessä voidaan laskea 
nopeimmin ajattelemalla, että pinta on funktion $F$ \pain{tasa-arvo}p\pain{inta}. On ilmeistä,
että yksikkövektori $\vec t$ on $F$:n tasa-arvopinnan tangenttivektori pisteessä
$(x_0,y_0,z_0)$ täsmälleen, kun $F$:n suunnattu derivaatta $\vec t\,$:n suuntaan häviää ko.\
pisteessä, eli kun $\vec t \cdot \nabla F(x_0,y_0,z_0)=0$. Tämän perusteella on
$\vec n = \nabla F(x_0,y_0,z_0)$ pinnan normaalivektori ko.\ pisteessä. Samaan tulokseen
tullaan, jos tarkasteltava pinta on pisteen $(x_0,y_0,z_0)$ kautta kulkeva $F$:n
tasa-arvopinta, jonka yhtälö on $\,F(x,y,z) = c = F(x_0,y_0,z_0)$. On siis päätelty, että
\[ \boxed{ \begin{aligned}
           \ykehys\quad &\vec n = \nabla F(x_0,y_0,x_0)\ 
                            \text{on funktion}\ F\ \text{tasa-arvopinnan} \quad \\
                        &\text{normaalivektori pisteessä}\ (x_0,y_0,z_0). \akehys
           \end{aligned} } 
\]
\begin{Exa} Määritä pinnan
\[ 
S:\ F(x,y,z) = x^2+y^2+z^2-xy-yz-xz = 1. 
\]
tangenttitason yhtälö pisteessä $(1,1,2)$.
\end{Exa}
\ratk Funktion $F$ gradientti on
\[ 
\nabla F(x,y,z) = (2x-y-z)\vec i + (2y-x-z)\vec j + (2z-y-x)\vec k, 
\]
joten pinnan normaalivektori pisteessä $(1,1,2)$ on
\[ 
\vec n = \nabla F(1,1,2) = -\vec i - \vec j + 2\vec k. 
\]
Tangenttitason yhtälö pisteessä $(1,1,2)\vastaa\vec r_0$ on
\[
T:\ (\vec r-\vec r_0)\cdot\vec n = 0\ \ekv\ \underline{\underline{x+y-2z+2=0}}. \loppu
\]
\begin{Exa} \label{udif-3: pintaesim} Jos pinnan yhtälö on annettu muodossa $S:\ z=f(x,y)$,
niin kirjoittamalla yhtälö muotoon $F(x,y,z)=z-f(x,y)=0$ saadaan normaalivektoriksi pisteessä
$(x,y,z) \in S\,$:
\[
\vec n = \nabla F(x,y,z) = -f_x(x,y)\vec i-f_y(x,y)\vec j+\vec k. \loppu
\]
\end{Exa}
\begin{Exa} Määritä käyrän
\[
\begin{cases} \,x^2-y^2+z^2=1, \\ \,x^2+y^2-2z^2=0 \end{cases}
\]
tangenttivektori pisteessä $(1,1,1)$. \end{Exa}
\ratk Kyseessä on kahden pinnan leikkauskäyrä. Pinnat ovat funktioiden $F_1(x,y,z)=x^2-y^2+z^2$
ja $F_2(x,y,z)=x^2+y^2-2z^2$ tasa-arvopintoja, joiden normaalivektorit pintojen yhteisessä 
pisteessä $(x,y,z)=(1,1,1)$ ovat
\begin{align*}
\vec n_1\,&=\,\nabla F_1(x,y,z)\,
           =\,2x\,\vec i - 2y\,\vec j + 2z\,\vec k\,=\,2(\vec i-\vec j+\vec k), \\
\vec n_2\,&=\,\nabla F_2(x,y,z)\,
           =\,2x\,\vec i + 2y\,\vec j - 4z\,\vec k\,=\,2(\vec i+\vec j-2\vec k).
\end{align*}
Kysytty tangenttivektori $\vec t$ on näitä vastaan kohtisuora, eli esimerkiksi
\[
\vec t = \frac{1}{4}\,\vec n_1\times\vec n_2 
       = \underline{\underline{\vec i+3\vec j+2\vec k}}. \loppu
\]

\subsection*{Parametrisen pinnan normaali}
\index{normaali(vektori)!c@pinnan|vahv}
\index{parametrinen pinta|vahv}

Tarkastellaan $\R^3$:n parametrisoitua pintaa (vrt.\ Luku \ref{parametriset käyrät})
\[
S:\ \vec r=\vec r(u,v)\ \ekv\ 
                        \begin{cases} \,x=x(u,v),\\ \,y=y(u,v), \\ \,z=z(u,v). \end{cases}
\]
Jos oletetaan, että tässä funktiot $x$, $y$ ja $z$ ovat differentioituvia pisteessä $(u,v)$,
niin on ilmeistä, että vektorit
\begin{align*}
\vec t_u &= \lim_{\Delta u\kohti 0} \frac{\vec r(u+\Delta u,v)-\vec r(u,v)}{\Delta u}
          =\frac{\partial\vec r}{\partial u}(u,v), \\
\vec t_v &= \lim_{\Delta v\kohti 0} \frac{\vec r(u,v+\Delta v)-\vec r(u,v)}{\Delta v}
          =\frac{\partial\vec r}{\partial v}(u,v)
\end{align*}
\index{tangenttivektori (pinnan)}%
ovat pinnan tangenttitason suuntaisia eli \kor{pinnan tangenttivektoreita} pisteessä 
$P\vastaa\vec r(u,v)$. Parametrisaatiolta on sopivaa odottaa, että nämä vektorit
ovat lineaarisesti riippumattomat, jolloin pinnan normaalivektori pisteessä $P$ on
\[
\boxed{\quad\kehys 
  \vec n=\frac{\partial \vec r}{\partial u}\times\frac{\partial \vec r}{\partial v} \quad
         \text{(parametrisen pinnan normaali)}. \quad}
\]
\begin{Exa} (Vrt.\ Esimerkki \ref{udif-3: pintaesim}.) Jos pinnan yhtälö on $S:\ z=f(x,y)$,
niin parametrisoidusta esityksestä
\[ 
S:\ \vec r(x,y)\,=\,x\,\vec i+y\,\vec j+f(x,y)\,\vec k
\]
saadaan $S$:n tangentti- ja normaalivektoreiksi pisteessä $(x,y,f(x,y))\,$:
\begin{align*}
\vec t_x\,&=\,\partial_x\vec r(x,y)\,=\,\vec i+f_x(x,y)\,\vec k, \\
\vec t_y\,&=\,\partial_y\vec r(x,y)\,=\,\vec j+f_y(x,y)\,\vec k, \\
\vec n\   &=\,\vec t_x \times \vec t_y\,=\,-f_x(x,y)\,\vec i -f_y(x,y)\,\vec j+\vec k. \loppu
\end{align*}
\end{Exa}
\begin{Exa} (Vrt.\ Esimerkki \ref{parametriset käyrät}:\,\ref{jäähdytystorni}.) Määritä
parametrisen pinnan
\[
S:\ \begin{cases} \,x=(2-2v)\cos u-v\sin u, \\ \,y=v\cos u+(2-2v)\sin u, \\ \,z=3v \end{cases}
\]
normaali pisteessä $(x,y,z)=(2,0,0)$.
\end{Exa}
\ratk Koska $(2,0,0)\vastaa \vec r=2\vec i=\vec r(0,0)$, niin lasketaan
\begin{align*}
\vec t_u(0,0) &\,=\, \frac{\partial\vec r}{\partial u}(0,0)
               \,=\, \frac{\partial x}{\partial u}(0,0)\vec i
                     +\frac{\partial y}{\partial u}(0,0)\vec j
                     +\frac{\partial z}{\partial u}(0,0)\vec k=2\vec j, \\
\vec t_v(0,0) &\,=\, \frac{\partial\vec r}{\partial v}(0,0)
               \,=\, -2\vec i +\vec j +3\vec k \\[2mm]
\impl\ \vec n &\,=\, \vec t_u\times\vec t_v
               \,=\,6\vec i +4\vec k 
               \,\uparrow\uparrow\, \underline{\underline{3\vec i+2\vec k}}. \loppu
\end{align*}

\Harj
\begin{enumerate}

\item
Laske funktion arvolle likiarvo annetussa pisteessä käyttäen differentiaalia.
\begin{align*}
&\text{a)}\ \ x^2y^3, \quad (3.1,0.9) \qquad\qquad\qquad\qquad\
 \text{b)}\ \ \Arctan(y/x), \quad (3.01,2.98) \\[0.5mm]
&\text{c)}\ \ 24\,(x^2+xy+y^2)^{-1}, \quad (2.1,1.9) \qquad
 \text{d)}\ \ \sin(\pi xy+\ln y), \quad (0.01,1.05) \\
&\text{e)}\ \ \sqrt{x+2y+3z}\,, \quad (1.9,1.8,1.1) \qquad\,\
 \text{f)}\ \ xe^{y+z^2}, \quad (2.05,-3.92,1.97)
\end{align*}

\item
Suorakulmaisen särmiön muotoisen laatikon särmät on mitattu $1\%$:n tarkkuudella. Arvioi,
kuinka suuri vaikutus ($\%$) mittausvirheillä on enintään mittausten perusteella
laskettuun laatikon a) tilavuuteen, b) pohjan pinta-alaan, c) lävistäjään.

\item
Pisteen $A$ etäisyys tarkkailupisteestä $B$ on pyritty selvittämään kolmiomittauksella
valitsemalla toinen tarkkailupiste $C$ ja mittaamalla kulmat $ABC$ ja $ACB$. Mittaustulokset
ovat: $\kulma ACB=45 \pm 0.1\aste$, $\kulma ABC=105 \pm 0.1\aste$ ja pisteiden $B$ ja $C$
välimatkaksi on mitattu $265 \pm 0.5$ m. Laske kysytty etäisyys $d$ ja arvioi virhe $\Delta d$
differentiaalin avulla (maksimivirhe).

\item
Tornin korkeus mitataan kulmamittauksella kahdesta pisteestä $A$ ja $B$, jotka ovat
samassa suunnassa tornista katsoen. Mitatut kulmat ovat $50 \pm 1\aste$ ja $35 \pm 1\aste$.
Lisäksi pisteiden $A$ ja $B$ välimatkaksi on mitattu $100 \pm 1$ m. Mikä on tornin
korkeuden laskettu arvo ja kuinka suuri voi mittausvirhe korkeintaan olla differentiaalin
perusteella arvioiden?

\item
Suunnistaja, joka on kartan origossa, on matkalla rastille, joka on kartan mukaan suunnassa
$\varphi=30\aste$ (polaarikoordinaatti). Sateen ja hikoilun vuoksi karttapaperi on kutistunut
$x$-suunnassa $1\%$ ja $y$-suunnassa $0.6\%$. Arvioi, kuinka monta $\%$ suunnistajan on syytä
korjata rastin suuntaa ja kartalta mitattua eäisyyttä päästäkseen oikeisiin arvoihin.

\item Nelikulmion pinta-ala on funktio $f(\mx) = f(x_1, y_1, x_2, y_2, x_3, y_3, x_4, y_4)$,
missä $(x_i, y_i)$, $i = 1,\ldots,4$ ovat nelikulmion kärkipisteet. Laske $\nabla f$ 
pisteessä ${\bf x} = (0,0,1,0,1,1,0,1)$ ja differentiaalin avulla likimäärin sen nelikulmion
pinta-ala, jonka kärjet ovat pisteissä $\,(0.03, -0.05)$, $(1.02,-0.02)$, $(0.96, 1.01)$, 
$(-0.01, 0.98)$.

\item \label{H-udif-3: differentioituvuus}
a) Olkoon $f(0,0)=0$ ja $f(x,y)=xy\sin(1/\sqrt{x^2+y^2}\,)$ muulloin. Näytä, että $f$ on
origossa differentioituva mutta $f_x$ ja $f_y$ eivät ole origossa jatkuvia.\vspace{1mm}\newline
b) Olkoon $f(x,y)=0$ kun $x=0$ tai $y=0$ ja $f(x,y)=\abs{xy}^\alpha(x^2+y^2)^\beta$ muulloin 
($\alpha,\beta\in\R$). Määritä $A\subset\R^2$ ja $B\subset\R^2$ siten, että $f$ on
differentioituva origossa täsmälleen kun $(\alpha,\beta) \in A$ ja differentioituva jokaisessa
pisteessä $(x,y)\in\R^2$ täsmälleen kun $(\alpha,\beta) \in B$. \vspace{1mm}\newline
c) Todista Lause \ref{väliarvolause-Rn} tarkastelemalla funktiota
$F(t)=f\bigl(t\ma+(1-t)\mb\bigr)$.

\item
Laske suunnattu derivaatta annetussa pisteessä sekä annettuun että funktion nopeimman kasvun
suuntaan: \vspace{1mm}\newline
a) \ $e^{xy+y},\ \ (0,0),\ \ \vec i+\vec j$ \vspace{0.8mm}\newline
b) \ $\sin(\pi xy)\cos(\pi y^2),\ \ (1,2),\ \ \vec i-2\vec j$ \vspace{0.3mm}\newline
c) \ $xy^2z^3,\ \ (-3,2,1),\ \ 6\vec i-2\vec j+3\vec k$ \vspace{0.8mm}\newline
d) \ $ xy^2+yz^3,\ \ (3,-2,-1),\ \ \vec 2i+\vec 2j-3\vec k$

\item
Seuraavassa on annettu funktio $f(x,y)$ ja piste $(x_0,y_0)$. Määritä kussakin tapauksessa \,
(i) gradientti $\nabla f(x_0,y_0)$, \, (ii) käyrän $f(x,y)=f(x_0,y_0)$ tangentin yhtälö
pisteessä $(x_0,y_0)$, \, (iii) pinnan $z=f(x,y)$ tangenttitason yhtälö pisteessä
$(x_0,y_0,z_0),\ z_0=f(x_0,y_0)$. \vspace{2mm}\newline
a) \ $ f(x,y)=x^2-y^2,\,\ (2,-1)$ \quad
b) \ $f(x,y)=x^2/(x+y),\,\ (1,1)$ \vspace{1mm}\newline
c) \ $f(x,y)=e^{x^2y+x^2},\,\ (2,0)$ \qquad
d) \ $f(x,y)=\ln(x^2+y^2),\,\ (1,-2)$

\item 
Määritä seuraavien pintojen tangenttitaso annetussa pisteessä $P$ sekä ne pintojen pisteet,
joissa tangenttitaso on $xy$-tason suuntainen. \vspace{2mm}\newline
a) \ $x^2-2xy-y^2+z^2-2x-2y-3z= 0, \quad P=(0,0,0)$ \vspace{1mm}\newline
b) \ $xy+yz-2xz+z^3=1, \quad P=(1,1,1)$ \vspace{1mm}\newline
c) \ $e^{x+2y-z}=x+y^2+2z+1, \quad P=(0,0,0)$

\item
Määritä seuraavien funktioiden tasa-arvopinnan normaali ja ko.\ pinnan tangenttitason yhtälö
annetussa pisteessä. \vspace{2mm}\newline
a) \ $f(x,y,z)=x^2y+y^2z+z^2x,\ \ (1,-1,1)$ \vspace{1mm}\newline
b) \ $f(x,y,z)=\cos(x+2y+3z),\ \ (\pi/2,\pi,\pi)$ \vspace{1mm}\newline
c) \ $f(x,y,z)=ye^{-x^2},\ \ (0,1,1)$ \vspace{1mm}\newline
d) \ $f(x,y,z)=\sin x\sin 2y\sin 3z\sin(3x-2y+z),\ \ (\pi/6,\pi/6,\pi/6)$

\item
Määritä avaruuskäyrän tangenttivektori annetussa pisteessä $P\,$:
\begin{align*}
&\text{a)}\ \ \begin{cases} 
              \,x^2-y^2+2z^2=13, \\ \,3x+2y-z=3,
              \end{cases} \quad P=(-2,3,-3) \\
&\text{b)}\ \ \begin{cases} 
              \,x^3-xyz^3+yz^5=1, \\ \,xy^2+yz+2xz^3=0,
              \end{cases} \quad P=(1,2,-1)
\end{align*}

%\item (*) \label{H-DD-2: tulon ja osamäärän gradientti}
%a) Näytä, että jos $n$ muuttujan funktiot $f$ ja $g$ ovat differentioituvia pisteessä 
%$\mx\in\R^n$, niin myös $u=fg$ on differentioituva $\mx$:ssä ja pätee tulon derivoimissääntö 
%$\Nabla u(\mx)=(g\Nabla f+f\Nabla g)(\mx)$. \ b) Näytä, että lisäehdolla $g(\mx) \neq 0$
%myös $v=f/g$ on differentioituvua $\mx$:ssä. Mikä on $\Nabla v(\mx)$:n lauseke?

\item (*)
Kepin päät ovat pisteissä $A=(1,1,-2)$ ja $B=(-1,-2,4)$. Keppiä liikutetaan niin, että sen päät
liukuvat pitkin käyriä 
\[
S_1: \begin{cases} \,x^3+y^2+z=0, \\ \,x+y^2+z^3=-5 \end{cases} \ \text{ja} \quad
S_2: \begin{cases} \,x^3+y^2+z=7, \\ \,x+y^2+z^3=67 \end{cases}
\]
(kepin pituus säilyy). Olkoon kepin päät pienen siirron jälkeen pisteissä $A' \in S_1$ ja
$B' \in S_2$. Laske suhteen $|AA'|/|BB'|$ raja-arvo, kun $|AA'| \kohti 0$.
 
\item (*)
Suora, jonka eräs piste on $A$, liikkuu siten, että suora on koko ajan $xy$-tason suuntainen,
piste $A$ liikkuu vakionopeudella $z$-akselia pitkin ylöspäin, ja samalla suora pyörii
vakiokulmanopeudella $z$-akselin ympäri vastapäivään, positiivisen $z$-akselin suunnasta 
katsoen. Suora on $x$ akselin suuntainen täsmälleen kun $A$ on jossakin pisteistä 
$(0,0,3n),\ n\in\Z$. Määritä suoran liikkuessaan piirtämän (viivoitin)pinnan parametrinen
esitys sekä ko. pinnan normaali ja tangenttitaso siinä pinnan ja suoran $x=1,\, y=2$ 
leikkauspisteessä, joka on lähinnä $xy$-tasoa.

\item (*) \index{zzb@\nim!Jyrkin lasku}
(Jyrkin lasku) Tasosta kohoaa tunturi. Sektorissa $\,A:\,0 \le y \le 2x$ on tunturin korkeus
tasosta (yksiköt km)
\[
h(x,y) = e^{(-x^2-2xy+y^2)/100}, \quad (x,y) \in A.
\]
Laskettelija Jyrki lähtee pisteestä $(3,6)$ ja laskee sukset suunnattuina joka hetki niin, että
lasku on jyrkin mahdollinen. Laske Jyrkin laskureitti!

\end{enumerate}