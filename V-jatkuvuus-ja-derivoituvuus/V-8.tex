\section{Analyyttiset kompleksifunktiot} \label{analyyttiset funktiot}
\alku \sectionmark{Analyyttiset funktiot}
\index{funktio A!e@kompleksifunktio|vahv}

Kompleksifunktiolla tarkoitetaan kompleksimuuttujan kompleksiarvoista funktiota eli
funktiota tyyppiä $f: \DF_f\kohti\C,\ \DF_f\subset\C$. Tällainen on esimerkiksi Luvussa
\ref{III-3} tarkasteltu (koko $\C$:ssä määritelty) polynomi. Koska $z$ on
tulkittavissa tason (kompleksitason) pisteeksi, niin kompleksifunktio voidaan ymmärtää
myös kahden reaalimuuttujan kompleksiarvoisena funktiona kirjoittamalla (vrt.\ Luvun
\ref{III-3} Esimerkki \ref{kompleksipolynomin juuret})
\[
f(x+iy) = u(x,y)+iv(x,y),
\]
missä $u(x,y)=\text{Re}\,f(x+iy)$ ja $v(x,y)=\text{Im}\,f(x+iy)$. Huolimatta tästä erosta
suhteessa yhden reaalimuuttujan funktioihin voidaan kompleksifunktioille määritellä käsitteet
jatkuvuus, raja-arvo ja derivoituvuus aivan samalla tavoin kuin reaalifunktioille. Esimerkiksi
$f$ on jatkuva pisteessä $z\in\DF_f$ täsmälleen kun kaikille kompleksilukujonoille $\seq{z_n}$
pätee (vrt.\ Määritelmä \ref{funktion jatkuvuus})
\[ 
z_n\in\DF_f\ \ja\ z_n \kohti z \qimpl f(z_n) \kohti f(z).
\]
Tässä lukujonojen $\seq{z_n}$ ja $\seq{f(z_n)}$ suppeneminen viittaa Määritelmään
\ref{jonon raja}, joka toimii sellaisenaan myös kompleksilukujen jonoille, kunhan merkinnän
$\abs{\cdot}$ tulkitaan tarkoittavan kompleksiluvun itseisarvoa. Määritelmän mukaan pätee
erityisesti (kuten reaalilukujonoillekin)
\[ 
z_n \kohti z \qekv \abs{z_n - z} \kohti 0.
\]
Jos tässä on $z=0$, niin pätee siis
\[
z_n=r_n\vkulma{\varphi_n} \kohti 0 \qekv r_n \kohti 0.
\]
Tämän mukaan lähestyminen kohti kompleksitason pistettä (tässä origoa) voi tapahtua
äärettömän monesta eri suunnasta ($\varphi_n=\varphi\in[0,2\pi)\ \forall n$) tai suuntia
vaihdellen (esim.\ spiraalimainen lähestyminen). Jatkossa nähdään, että tämä kvalitatiivinen
ero suhteessa reaalilukujonoon (jolla mahdollisia lähestymissuuntia kohti raja-arvoa on vain
kaksi) voi tuottaa kompleksifunktoiden raja-arvotarkasteluissa yllätyksiä.
\begin{Exa} Funktio $f(z)=\overline{z}$ (eli $f(x+iy)=x-iy$) on jatkuva $\C$:ssä, sillä
kompleksialgebran (ks.\ Luku \ref{kompleksiluvuilla laskeminen}) mukaan
\[
\abs{f(z_n)-f(z)} = \abs{\overline{z_n}-\overline{z}} 
                  = \abs{\overline{z_n-z}} = \abs{z_n-z} \kohti 0
\]
aina kun $z_n \kohti z$. \loppu
\end{Exa}
Myös jatkuvuuden vaihtoehtoinen määritelmä (Määritelmä \ref{vaihtoehtoinen jatkuvuus})
toimii sellaisenaan kompleksifunktioille, samoin funktion raja-arvon määritelmä.
Raja-arvon käsitteeseen perustuva funktion derivaatta määritellään myös samoin kuin
reaalifunktioille:
\[ 
f'(z) = \lim_{\Delta z \kohti 0} \frac{f(z + \Delta z) - f(z)}{\Delta z}. 
\]
\index{derivoituvuus}%
Jos raja-arvo on olemassa, niin sanotaan, että $f$ on \kor{derivoituva} pisteessä $z$. 
\begin{Exa} \label{kompleksifunktioiden derivoituvuus} Tutki kompleksifunktioiden
\[ 
\text{a)}\ f(z) = \bar{z}, \quad \text{b)}\ f(x+iy) = x^2 + iy^2, \quad 
\text{c)}\ f(x+iy) = x^2-y^2 + 2ixy 
\]
derivoituvuutta. \end{Exa}
\ratk a) Jos $\Delta z_n = \Delta r_n\vkulma{\varphi_n}
= \Delta r_n(\cos\varphi_n + i\sin\varphi_n) \neq 0$ ($\Delta r_n \neq 0$), niin
\[ 
\frac{f(z + \Delta z_n) - f(z)}{\Delta z_n} 
             = \frac{\cos\varphi_n - i\sin\varphi_n}{\cos\varphi_n + i\sin\varphi_n}. 
\]
Nähdään, ettei vaadittua raja-arvoa ole, eli $f$ ei ole missään pisteessä derivoituva.

b) Jos tässä merkitään $\Delta z_n = \Delta x_n + i\Delta y_n$, niin
\[ 
\frac{f(z + \Delta z_n) - f(z)}{\Delta z_n} 
    = 2\,\frac{x\Delta x_n + iy\Delta y_n}{\Delta x_n + i\Delta y_n}
                           + \frac{(\Delta x_n)^2 + (\Delta y_n)^2}{\Delta x_n + i\Delta y_n}. 
\]
Siirtymällä polaariesitykseen nähdään, että jälkimmäinen termi oikealla $\kohti 0$ aina kun 
$\abs{\Delta z_n} = r_n \kohti 0$. Ensimmäisellä termillä sen sijaan on vain lähestymissuunnasta
riippuvia suunnattuja raja-arvoja, ellei ole $x=y=t$, jolloin ko.\ termi yksinkertaistuu muotoon
$2t$. Päätellään siis, että $f$ on derivoituva ainoastaan kompleksitason suoralla
$z = t(1+i),\ t \in \R$, ja tällä suoralla $f'(z) = 2Rez$. 

c) Tässä esimerkissä ollaan onnekkaampia, sillä $f(x+iy) = (x+iy)^2$, jolloin
\[ 
\frac{f(z + \Delta z) - f(z)}{\Delta z} = \frac{(z+\Delta z)^2 - z^2}{\Delta z} 
                                        = 2z + \Delta z, \quad \Delta z \neq 0. 
\]
Siis $f$ on derivoituva jokaisessa pisteessä $z \in \C$ ja $f'(z) = 2z$. \quad \loppu

Esimerkki kertoo, että kompleksifunktion derivoituvuus on kaikkea muuta kuin 'läpihuutojuttu'
siinäkin tapauksessa, että funktion reaali- ja imaginaariosat ovat säännöllisiä funktioita
(kuten esimerkissä polynomeja). Derivoituvuus onkin kom\-pleksifunktiolle paljon vaativampi
ominaisuus kuin reaalifunktiolle.

Seuraavassa määritelmässä laajennetaan derivoituvuusehto koskemaan kompleksitason
\kor{avointa} osajoukkoa $G\subset\C$.
\begin{Def} (\vahv{Analyyttinen kompleksifunktio}) \label{analyyttinen funktio}
\index{analyyttinen kompleksifunktio|emph} \index{avoin joukko|emph}
\index{ympzy@($\delta$-)ympäristö} 
Kompleksitason osa\-joukko $G\subset\C$ on \kor{avoin}, jos jokaisella $z_0 \in G$ on
\kor{ympäristö}
\[ 
U_{\delta}(z_0) = \{z \in \C \mid \abs{z-z_0}<\delta\},\ \ \delta>0
\]
siten, että $U_{\delta} \subset G$. Kompleksifunktio, joka on derivoituva ei-tyhjässä,
avoimessa joukossa $G\subset\C$, on \kor{analyyttinen} $G$:ssä.
\end{Def}
Analyyttiset funktiot muodostavat kompleksifunktioiden tärkeän ja paljon tutkitun
'vähemmistön'.\footnote[2]{Analyyttisten kompleksifunktioiden teorian tarkempi esittely
kuuluu toisiin yhteyksiin. Mainittakoon teorian tuloksista kuitenkin, että jos $f$ on
analyyttinen $G$:ssä, niin samoin on $f'$, jolloin $f$ on itse asiassa mielivaltaisen monta
kertaa derivoituva $G$:ssä. --- Tämä tulos kertoo osaltaan, että kompleksifunktion analyytisyys
on paljon voimakkaampi ominaisuus kuin reaalifunktion derivoituvuus.}
\jatko \begin{Exa} (jatko) Esimerkissä c-kohdan funktio $f(z) = z^2$ on analyyttinen koko 
kompleksitasossa ($G = \C$), kun taas a- ja b- kohtien funktiot eivät ole analyyttisiä
missään. \loppu 
\end{Exa}
\index{kokonainen funktio}%
Jos $f$ on analyyttinen koko kompleksitasossa, niin sanotaan, että $f$ on  \kor{kokonainen} 
(engl.\ entire) funktio. Kaikki polynomit $p(z)$ (myös kompleksikertoimiset) ovat kokonaisia 
funktioita, sillä polynomin derivoimissääntö perustuu vain kunta-algebraan, joka ei lainkaan 
muutu siirryttäessä reaalimuuttujasta kompleksimuuttujaan (vrt.\ Esimerkki 
\ref{kompleksifunktioiden derivoituvuus}, c-kohta). Algebraan perustuvat myös funktioiden
summan, tulon, osamäärän ja yhdistetyn funktion derivoimissäännöt reaalifunktioille, joten
niissäkin voidaan reaalimuuttujan $x$ tilalle vaihtaa kompeksimuuttuja $z$ säännön 
\index{kompleksimuuttujan!b@rationaalifunktio} \index{rationaalifunktio}%
muuttumatta. Näiden sääntöjen perusteella voidaan erityisesti jokainen (kompleksikertoiminen)
\kor{kompleksimuuttujan rationaalifunktio} derivoida samalla tavoin kuin reaalimuuttujan 
tapauksessa. Rationaalifunktiot ovatkin Määritelmän \ref{analyyttinen funktio} mukaisesti
analyyttisiä koko määrittelyjoukossaan, sillä jos $f(z) = p(z)/q(z)$ ($p$ ja $q$ polynomeja)
ja $z_0\in\DF_f$, niin $q(z_0) \neq 0$, jolloin $q(z) \neq 0$ myös jossakin ympäristössä
$U_{\delta}(z_0)$. Tällöin $U_{\delta}(z_0)\subset\DF_f$, eli $\DF_f$ on avoin joukko.
\begin{Exa} Rationaalifunktiot
\[
f(z) = \frac{1}{z^2 + 1}\,, \quad g(z) = \frac{i}{z^2 + iz}
\]
on määritelty koko kompleksitasossa lukuunottamatta pisteitä $\pm i$ ($f$) ja $0,-i$ ($g$).
Määrittelyjoukot ovat avoimia, ja molemmat funktiot ovat määrittelyjoukossaan derivoituvia,
siis analyyttisiä. Derivaatat lasketaan kuten reaalimuuttujan tapauksessa:
\[ 
f'(z) = -\frac{2z}{(z^2 + 1)^2}\,, \quad g'(z) = -\frac{i(2z+i)}{(z^2 + iz)^2}\,. \loppu 
\] 
\end{Exa}

Jos kompleksifunktio on analyyttinen nollakohtansa ympäristössä, niin nollakohtaa voidaan
etsiä Newtonin iteraatiolla
\[
z_{n+1} = z_n - \frac{f(z_n)}{f'(z_n)}\,, \quad n=0,1,\ldots
\]
Esimerkiksi polynomin yksinkertaista nollakohtaa etsittäessä algoritmi toimii erinomaisesti,
kunhan alkuarvaus on riittävän hyvä
(ks.\ Harj.teht.\,\ref{H-V-8: Newton 1}--\ref{H-V-8: Newton 3}).

\Harj
\begin{enumerate}

\item
Näytä suoraan kompleksifunktion derivaatan määritelmästä, että \newline
a) \ $\dif z^{-1}=-z^{-2},\quad$ b) \ $\dif (z+i)^{-2}=-2(z+i)^{-3}.$

\item
Missä kompleksitason osajoukossa seuraavat funktiot ovat analyyttisiä?
\begin{align*}
&\text{a)}\ \ f(z)=z^3+iz \qquad \text{b)}\ \ f(z)=\frac{z+1}{z^2+z+1} \qquad
 \text{c)}\ \ f(z)=\frac{1}{z^3-8} \\
&\text{d)}\ \ f(z)=\frac{z}{\abs{z}^2} \qquad 
 \text{e)}\ \ f(z)=\frac{\overline{z}}{\abs{z}^2} \qquad
 \text{f)}\ \ f(z)=(z+\overline{z})^2-2\abs{z}^2-\overline{z}^2
\end{align*}

\item \label{H-V-8: Newton 1}
Näytä, että jos $a\in\C,\ a \neq 0$, niin Newtonin iteraatio
\[
z_0\in\C, \quad z_{n+1} = \frac{1}{2}\left(z_n + \frac{a}{z_n}\right), \quad n=0,1,\ldots
\]
(vrt.\ Luku \ref{kiintopisteiteraatio}, Esimerkki \ref{neliöjuuri a}) suppenee kvadraattisesti
kohti funktion $f(z)=z^2-a$ nollakohtaa, jos alkuarvaus $z_0$ on nollakohtaa (kumpaa tahansa)
riittävän lähellä. Kokeile iteraation toimivuuttaa tapauksessa $a=i$ valinnoilla a) $z_0=1$, \
b) $z_0=i$, \ c) $z_0=-1-i$.

\item \label{H-V-8: Newton 2}
Etsi polynomin $f(z)=z^4+z+4$ nollakohdat likimäärin Newtonin menetelmällä. Huomaa, että
Newtonin iteraatio ei tässä tapauksessa suppene reaalisilla alkuarvauksilla --- miksei?

\item (*) \label{H-V-8: Newton 3}
Todista Lauseen \ref{Newtonin konvergenssi} vastine analyyttiselle kompleksifunktiolle
$f(z)$ tapauksessa, jossa $f$ on polynomi.

\end{enumerate}