\chapter{Jatkuvuus ja derivoituvuus}  \label{jatkuvat funktiot}

Funktion \kor{jatkuvuus} ja \kor{derivoituvuus} ovat käsitteellisiä peruslähtökohtia 
matematiikan suuntauksessa, jota kutsutaan väljästi \kor{analyysiksi}. Jatkuvuudesta, tai 
yleisemmin funktion \kor{säännöllisyydestä} puhuttaessa on kyse funktion arvojen
ennustettavuudesta muuttujan (muuttujien) arvojen vaihdellessa. Jatkuvuuden ja derivoituvuuden
käsitteet tuovat matemaattisten funktioiden teoriaan kokonaan oman 'makunsa' verrattuna
tähän asti tarkasteltuun funktioiden algebraan.

Tässä luvussa tarkastelun kohteena ovat pääosin vain yhden reaalimuuttujan funktiot. Näille
funktioille määritellään ensin jatkuvuus peruskäsitteenä ja jatkuvuuteen läheisesti liittyvä
\kor{funktion raja-arvon} käsite. Pelkkää jatkuvuutta vahvempina säännöllisyyden lajeina
määritellään mm.\ derivoituvuus (Luku \ref{derivaatta}) ja derivoituvuutta vahvemmat 
\kor{sileyden} lajit (Luku \ref{ääriarvot}). Näiden käsitteiden pohjalta luonnehditaan
funktioita, mm.\ esitetään kaksi reaalimuuttujan analyysin keskeistä \kor{väliarvolausetta} ja
tarkastellaan näiden lauseiden seuraamuksia yhtälöiden ja myös yksinkertaisten 
\kor{differentiaaliyhtälöiden} ratkeavuusteoriassa. 

Derivaatta tuo mukanaan myös derivoimissäännöt eli uuden ulottuvuuden funktioalgebraan.
Luvuissa \ref{derivaatta}--\ref{kaarenpituus} johdetaan kaikki keskeiset derivoimissäännöt,
mukaanlukien implisiittifunktiot, potenssisarjan summana määritellyt funktiot ja
(Luvussa \ref{kaarenpituus} erikseen) trigonometriset funktiot. 

Luvussa \ref{kiintopisteiteraatio} tarkastelun kohteena ovat yhtälöitten likimääräisessä
ratkaisussa yleisesti käytettävien algoritmien, \kor{kiintopisteiteraatioiden},
toimintaperiaatteet ja suppenevuusteoria. Luvussa \ref{analyyttiset funktiot} tarkastellaan
lyhyesti jatkuvuuden ja derivoituvuuden käsitteiden laajentamista kompleksifunktioihin ja
määritellään tähän liittyen \kor{analyyttisen} kompleksifunktion käsite. Viimeisessä
osaluvussa todistetaan jatkuvien funktioiden teorian keskeisimpiä väittämiä kuten 
\kor{Weierstrassin lause}. Tässä yhteydessä esitetään myös lyhyesti, miten Algebran peruslause
on todistettavissa kompleksialgebran ja analyysin tuloksia yhdistelemällä.
